\documentclass{scrreprt}

\usepackage{aligned-overset}
\usepackage{amsmath}
\usepackage{amssymb}
\usepackage{bm}
\usepackage[shortlabels]{enumitem}
\usepackage{hyperref}
\usepackage[utf8]{inputenc}
\usepackage{multicol}
\usepackage{mathtools}
\usepackage{physics}
\usepackage{tabularx}
\usepackage[table]{xcolor}
\usepackage{titling}
\usepackage{fancyhdr}
\usepackage{xfrac}
\usepackage{pgfplots}
\usepackage{tikz-3dplot}

\pgfplotsset{compat = newest}
\usetikzlibrary{intersections}
\usetikzlibrary{patterns}
\usepgfplotslibrary{fillbetween}

\author{Karsten Lehmann}
\date{SoSe 2022}
\title{Übungsblatt 05\\Lineare Algebra - Weiterführende Konzepte}

\setlength{\headheight}{26pt}
\pagestyle{fancy}
\fancyhf{}
\lhead{\thetitle}
\rhead{\theauthor}
\lfoot{\thedate}
\rfoot{Seite \thepage}

\newcommand\Grad{\text{Grad}}
\newcommand\Kern{\text{Kern}}

\begin{document}

\paragraph{Aufgabe 2} Entscheiden Sie für jede der folgenden Matrizen
$X \in \qty\big{A, B, C}$, ob $X$ diagonalisierbar ist.
Falls $X$ diagonalisierbar ist, dann finden Sie eine Basis von $K^2$
(wobei $K$ der jeweils zugrundeliegende Körper sei), die aus Eigenvektoren
von $X$ besteht und eine invertierbare Matrix
$S \in \mathcal{M}_{2 \times 2}\qty\big(K)$, sodass $S^{-1}XS$ eine
Diagonalmatrix ist.

\begin{enumerate}[(a)]
\item $A \coloneqq \begin{pmatrix}
    5 & 4 \\
    -3 & -2 \\
  \end{pmatrix} \in \mathcal{M}_{2 \times 2}\qty\big(\mathbb{R})$

  \subparagraph{Lsg.} Das charakteristische Polynom der Matrix ist
  \begin{flalign*}
    \det\qty(XI_n - A) &= \det\qty(
      \begin{pmatrix}
        X & 0 \\
        0 & X \\
      \end{pmatrix} - \begin{pmatrix}
        5 & 4 \\
        -3 & -2 \\
      \end{pmatrix}
    ) & \\
    &= \det\begin{pmatrix}
      X - 5 & -4 \\
      3 & X + 2 \\
    \end{pmatrix} \\
    &= X^2 - 3X + 2 = \qty\big(X - 1)\qty\big(X - 2)
  \end{flalign*}
  Nach Satz 10.41 (a) der Vorlesung (\emph{``Wenn das charakteristische
    Polynom von $A$ komplett in verschiedene Linearfaktoren zerfällt, so ist
    $A$ diagonalisierbar''}), ist die Matrix $A$ diagonalisierbar.

  Seien nun $B_{\lambda_1} = 1 \cdot I_n - A = \begin{pmatrix}
    -4 & -4 \\
    3 & 3 \\
  \end{pmatrix}$ und
  $B_{\lambda_2} = 2 \cdot I_n - A = \begin{pmatrix}
    -3 & -4 \\
    3 & 4 \\
  \end{pmatrix}$.
  Nach Anwendung des Gauß-Verfahrens bleiben übrig:
  \[
    \begin{pmatrix}
      -4 & -4 \\
      3 & 3 \\
    \end{pmatrix}
    \overset{
      \substack{
        Z.1 + \frac{4}{3} \cdot Z.2 \\
        Z.2 / 3
      }
    }\leadsto
    \begin{pmatrix}
      0 & 0 \\
      1 & 1 \\
    \end{pmatrix}
  \]
  und
  \[
    \begin{pmatrix}
      -3 & -4 \\
      3 & 4 \\
    \end{pmatrix}
    \overset{Z.1 + Z.2}
    \leadsto
    \begin{pmatrix}
      0 & 0 \\
      3 & 4 \\
    \end{pmatrix}
  \]

  Nun ist $V_{\lambda_1} = \qty{
    \begin{pmatrix}x\\y\end{pmatrix} \in \mathbb{R}^2
    \:{\Big |}\:
    x - y = 0
  } = \qty{
    \begin{pmatrix}1\\-1\end{pmatrix} \cdot x
    \:{\Big |}\:
    x \in \mathbb{R}
  }$ und

  $V_{\lambda_2} = \qty{
    \begin{pmatrix}x\\y\end{pmatrix} \in \mathbb{R}^2
    \:{\Big |}\:
    3 \cdot x - 4 \cdot y = 0
  } = \qty{
    \begin{pmatrix}4\\-3\end{pmatrix} \cdot x
    \:{\Big |}\:
    x \in \mathbb{R}
  }$.

  Jetzt bilden $\qty{
    \begin{pmatrix}1\\-1\end{pmatrix},
    \begin{pmatrix}4\\-3\end{pmatrix}
  }$ ein Basis in $\mathbb{R}^2$ und
  $S = \begin{pmatrix}
    1 & 4 \\
    -1 & -3 \\
  \end{pmatrix}$ ist nach Satz 7.38 der Vorlesung (\emph{``Es sind äquivalent:
    $A$ ist invertierbar und die Spalten von $A$ bilden eine Basis von $K^n$''})
  invertierbar mit $S^{-1} = \begin{pmatrix}
    -3 & -4 \\
    1 & 1 \\
  \end{pmatrix}$

  Schließlich ist
  \[
    S^{-1} \cdot A \cdot S =
    \begin{pmatrix}
      -3 & -4 \\
      1 & 1 \\
    \end{pmatrix} \cdot \begin{pmatrix}
      5 & 4 \\
      -3 & -2 \\
    \end{pmatrix} \cdot \begin{pmatrix}
      1 & 4 \\
      -1 & -3 \\
    \end{pmatrix} = \begin{pmatrix}
      1 & 0 \\
      0 & 2 \\
    \end{pmatrix}
  \]
  ein Diagonalmatrix, deren Diagonaleneinträge die Eigenwerte von $A$ sind.

\newpage
\item $B \coloneqq \begin{pmatrix}
    0 & 1 \\
    -1 & -1 \\
  \end{pmatrix} \in \mathcal{M}_{2 \times 2}\qty\big(\mathbb{F}_3)$, wobei
  $\mathbb{F}_3 \coloneqq \mathbb{Z}/3$

  \subparagraph{Lsg.} Das charakteristische Polynom der Matrix ist
  \begin{flalign*}
    \det\qty(XI_n - A) &= \det\qty(
      \begin{pmatrix}
        X & 0 \\
        0 & X \\
      \end{pmatrix} - \begin{pmatrix}
        0 & 1 \\
        -1 & -1 \\
      \end{pmatrix}
    ) & \\
    &= \det\begin{pmatrix}
      X & 2 \\
      1 & X + 1 \\
    \end{pmatrix} \\
    &= X^2 + X + 1 = \qty\big(X - 1)\qty\big(X - 1)
  \end{flalign*}
  Nun ist $\lambda = 1$ ein Eigenwert von $B$.
  Sei nun $C_\lambda = 1 \cdot I_n - B = \begin{pmatrix}
    1 & 2 \\
    1 & 2 \\
  \end{pmatrix}$.
  Nach Anwendung des Gauß-Verfahrens bleiben übrig:
  \[
    \begin{pmatrix}
      1 & 2 \\
      1 & 2 \\
    \end{pmatrix}
    \overset{Z.1 + 2 \cdot Z.2}\leadsto
    \begin{pmatrix}
      0 & 0 \\
      1 & 2 \\
    \end{pmatrix}
  \]
  Daher ist $V_{\lambda} = \qty{
    \begin{pmatrix}x\\y\end{pmatrix} \in \mathbb{F}_3^2
    \:{\Big |}\:
    x -2 \cdot y = 0
  } = \qty{
    \begin{pmatrix}1\\1\end{pmatrix} \cdot x
    \:{\Big |}\:
    x \in \mathbb{F}_3
  }$.

  Allerdings bilden Elemente aus $V_\lambda$ keine Basis von $\mathbb{F}_3$,
  daher ist $A$ nach Korollar 10.38 der Vorlesung (\emph{``Es sind äquivalent:
    $A$ ist diagonalisierbar und $K^n$ besitzt eine Basis aus Eigenvektoren von
    $A$''}) nicht diagonalisierbar.

\item $C \coloneqq \begin{pmatrix}
    0 & 1 \\
    -1 & -1 \\
  \end{pmatrix} \in \mathcal{M}_{2 \times 2}\qty\big(\mathbb{F}_5)$, wobei
  $\mathbb{F}_5 \coloneqq \mathbb{Z}/5$

  Das charakteristische Polynom der Matrix ist
  \begin{flalign*}
    \det\qty(XI_n - C) &= \det\qty(
      \begin{pmatrix}
        X & 0 \\
        0 & X \\
      \end{pmatrix} - \begin{pmatrix}
        0 & 1 \\
        -1 & -1 \\
      \end{pmatrix}
    ) & \\
    &= \det\begin{pmatrix}
      X & 4 \\
      1 & X + 1 \\
    \end{pmatrix} \\
    &= X^2 + X + 1
  \end{flalign*}
  Dieses Polynom hat keine Nullstellen auf $\mathbb{F}_5$, daher ist die Matrix
  $C$ nicht diagonalisierbar.
\end{enumerate}

\newpage
\paragraph{Aufgabe 3} Seien $v_1 \coloneqq \begin{pmatrix}1\\1\end{pmatrix} \in
\mathbb{R}^2$ und $v_2 \coloneqq \begin{pmatrix}2\\0\end{pmatrix} \in
\mathbb{R}^2$.
Seien weiterhin $\lambda_1 \coloneqq 2$ und $\lambda_2 \coloneqq -4$.
Finden Sie eine Matrix $A$, sodass $v_i$ für $i \in \qty\big{1, 2}$ ein
Eigenvektor von $A$ mit Eigenwert $\lambda_i$ ist.

\subparagraph{Lsg.} Es ist $D = \begin{pmatrix}
  2 & 0 \\
  0 & -4 \\
\end{pmatrix}$ die Diagonalmatrix mit den Eigenwerten von $A$ als
Diagonaleneinträge.
Weiter sind $v_1, v_2$ linear unabhängig in $\mathbb{R}^2$.
Damit ist $S = \begin{pmatrix}
  1 & 2 \\
  1 & 0 \\
\end{pmatrix}$ nach Satz 7.38 der Vorlesung (\emph{``Es sind äquivalent:
  $A$ ist invertierbar und die Spalten von $A$ bilden eine Basis von $K^n$''})
invertierbar.
Aus dem Gauß-Jacobi Verfahren folgt:

\begin{flalign*}
  \qty(
  \begin{array}{cc|cc}
    1 & 2 & 1 & 0 \\
    1 & 0 & 0 & 1 \\
  \end{array})
  \overset{Z.2 + Z.1}&\leadsto
  \qty(
  \begin{array}{cc|cc}
    1 & 2 & 1 & 0 \\
    2 & 2 & 1 & 1 \\
  \end{array}) & \\
  \overset{-1 \cdot Z.1 + Z.2}&\leadsto
  \qty(
  \begin{array}{cc|cc}
    1 & 0 & 0 & 1 \\
    2 & 2 & 1 & 1 \\
  \end{array}) \\
  \overset{Z.2 - 2 \cdot Z.1}&\leadsto
  \qty(
  \begin{array}{cc|cc}
    1 & 0 & 0 & 1 \\
    0 & 2 & 1 & -1 \\
  \end{array}) \\
  \overset{\frac{1}{2} \cdot Z.2}&\leadsto
  \qty(
  \begin{array}{cc|cc}
    1 & 0 & 0 & 1 \\
    0 & 1 & \frac{1}{2} & -\frac{1}{2} \\
  \end{array})
\end{flalign*}
$S^{-1} = \frac{1}{2} \cdot \begin{pmatrix}
  0 & 2 \\
  1 & -1 \\
\end{pmatrix}$.
Schließlich ist
\[
  A = SDS^{-1} = \begin{pmatrix}
    1 & 2 \\
    1 & 0 \\
  \end{pmatrix} \cdot \begin{pmatrix}
    2 & 0 \\
    0 & -4 \\
  \end{pmatrix} \cdot \frac{1}{2} \cdot \begin{pmatrix}
    0 & 2 \\
    1 & -1 \\
  \end{pmatrix} = \begin{pmatrix}
    -4 & 6 \\
    0 & 2 \\
  \end{pmatrix}
\]

\paragraph{Aufgabe 4} Seien $K$ ein Körper, $r \in \mathbb{N}$ und
$V_1, \ldots, V_r$ Vektorräume über $K$.
Dann heißt
\[
  \bigoplus_{i = 1}^r V_i \coloneqq \qty{
    \qty(v_1, \ldots, v_r) \:{\Big |}\:
    v_i \in V_i \text{ für alle $1 \leq i \leq r$}
  }
\]
zusammen mit der durch
\[
  \qty\big(v_1, \ldots, v_r) + \qty\big(w_1, \ldots, w_r) \coloneqq
  \qty\big(v_1 + w_1, \ldots, v_r + w_r)
\]
für alle $\qty\big(v_1, \ldots, v_r), \qty\big(w_1, \ldots, w_r) \in
\oplus_{i = 1}^r V_i$ definierten Addition und der durch
\[
  \lambda \cdot \qty\big(v_1, \ldots, v_r) \coloneqq
  \qty\big(\lambda \cdot v_1, \ldots, \lambda \cdot v_r)
\]
für alle $\lambda \in K$ und alle
$\qty\big(v_1, \ldots, v_r) \in \oplus_{i = 1}^r V_i$ definierten
Skalarmultiplikation die \emph{externe direkte Summe} der Vektorräume
$V_1, \ldots, V_r$.

\noindent
Für $v_1 \in V_1$ und $v_r \in V_r$ seien
$\overline{v_1} \coloneqq \qty\big(v_1, 0_{V_2}, \ldots, 0_{V_r})$
und
$\overline{v_r} \coloneqq \qty\big(0_{V_1}, 0_{V_2}, \ldots, v_r)$.

\noindent
Für $1 < i < r$ und $v_i \in V_i$ sei $\overline{v_i} \coloneqq \qty\big(
  0_{V_1}, \ldots, 0_{V_{i - 1}}, v_1, 0_{V_{i + 1}}, \ldots, 0_{v_r}
)$ und für $1 \leq i \leq r$ sei $\overline{V_i} \coloneqq \qty\big{
  \overline{v_i} \:{\big |}\: v_i \in V_i
}$.
\newpage
\begin{enumerate}[(i)]
\item Zeigen Sie, dass $\oplus_{i = 1}^r V_i$ ein Vektorraum über $K$ ist.

  \subparagraph{Lsg.} $\oplus_{i = 1}^r V_i$ ist ein Vektorraum über $K$, falls
  folgenden Eigenschaften erfüllt sind:
  \begin{enumerate}[(1)]
  \item $\oplus_{i = 1}^r V_i$ ist zusammen mit der oben definierten Addition
    eine abelsche Gruppe.
    Das heißt für beliebige $a, b, c \in \oplus_{i = 1}^r V_i$ gilt:
    \begin{itemize}
    \item $a + (b + c) = (a + b) + c$.
      Für die einzelnen Elemente gilt dabei
      $a_i + (b_i + c_i) = (a_i + b_i) + c_i$, da $a_i, b_i, c_i \in V_i$
      und $V_i$ ein Vektorraum ist. $\checkmark$
    \item $a + b = b + a$.
      Auf die einzelnen Komponenten bezogen gilt $a_i + b_i = b_i + a_i$,
      da $a_i, b_i, c_i \in V_i$ und $V_i$ ein Vektorraum ist. $\checkmark$
    \item Es existiert ein neutrales Element $0 \in \oplus_{i = 1}^r V_i$,
      so dass gilt $a + 0 = 0 + a = a$.
      Sei $0 = \qty\big(0_{V_1}, \ldots, 0_{V_r})$.
      $\checkmark$
    \item Es existiert $-a \in \oplus_{i = 1}^r V_i$, so dass $a + (-a) = 0$.
      Sei $-a = \qty\big(-a_0, \ldots, -a_r)$.
      $\checkmark$
    \end{itemize}
  \item Die Skalarmultiplikation ist assoziativ, dass heißt für
    $\lambda, \mu \in K$ und $a \in \oplus_{i = 1}^r V_i$ gilt
    \[
      \lambda \cdot \qty\big(\mu \cdot a)
      = \qty\big(\lambda \cdot \mu) \cdot a
    \]
    Für die einzelnen Komponenten gilt dabei wieder
    $\lambda \cdot \qty\big(\mu \cdot a_i) =
    \qty\big(\lambda \cdot \mu) \cdot a_i$, da $a_i \in V_i$
    und $V_i$ nach Voraussetzung ein Vektorraum über $K$ ist.
    $\checkmark$
  \item Es gilt das Distributivgesetz für Vektoren und Skalare.
    Für $a, b \in \oplus_{i = 1}^r V_i$ und $\lambda, \mu \in K$ gilt
    \[
      \lambda \cdot a + b = \lambda \cdot a + \lambda \cdot b
    \]
    und
    \[
      (\lambda + \mu) \cdot a = \lambda \cdot a + \mu \cdot a
    \]
    Beide Gleichungen lassen sich wieder durch die Betrachtung der einzelnen
    Komponenten zeigen. $\checkmark$
  \item Für alle $a \in \oplus_{i = 1}^r V_i$ gilt $a \cdot 1_K = a$.
    Nun gilt wieder für jeden Komponenten $a_i \cdot 1_K = a_i$.
    $\checkmark$
  \end{enumerate}

\item Sei $1 \leq i \leq r$.
  Zeigen Sie, dass $\overline{V_i}$ ein Unterraum von $\oplus_{i = 1}^r V_i$ ist
  und dass $V_i \cong \overline{V_i}$.

  \subparagraph{Lsg.} Es ist leicht zu erkennen, dass
  $\overline{V_i} \ne \emptyset$ und
  $\overline{V_i} \subseteq \oplus_{i = 1}^r V_i$ gilt.

  Seine nun $a, b \in \overline{V_i}$ und $\lambda \in K$ beliebig.
  Dann ist
  $a = \qty\big(
    0_{V_1}, \ldots, 0_{V_{i - 1}}, a_i, 0_{V_{i + 1}}, \ldots, 0_{V_r}
  )$,
  $b = \qty\big(
    0_{V_1}, \ldots, 0_{V_{i - 1}}, b_i, 0_{V_{i + 1}}, \ldots, 0_{V_r}
  )$,
  und
  \begin{flalign*}
    \qty\big(
      0_{V_1},
      \ldots,
      0_{V_{i - 1}},
      a_i,
      0_{V_{i + 1}},
      \ldots,
      0_{V_r}
    )
    &+ \lambda \cdot
    \qty\big(
      0_{V_1},
      \ldots,
      0_{V_{i - 1}},
      b_i,
      0_{V_{i + 1}},
      \ldots,
      0_{V_r}
    ) \\
    &= \qty\big(
      0_{V_1},
      \ldots,
      0_{V_{i - 1}},
      a_i + \lambda \cdot b,
      0_{V_{i + 1}},
      \ldots,
      0_{V_r}
    )
  \end{flalign*}
  Nun ist $a_i + \lambda \cdot b_i \in V_i$ und damit
  $\qty\big(
    0_{V_1},
    \ldots,
    0_{V_{i - 1}},
    a_i + \lambda \cdot b,
    0_{V_{i + 1}},
    \ldots,
    0_{V_r}
  ) \in \overline{V_i}$.
  Nach Lemma 3.17 der Vorlesung (Untervektorraumkriterium) folgt die Behauptung.

  Sei weiter
  $\gamma_i \colon V_i \to \overline{V_i},
  v_i \mapsto \qty\big(0, 0, \ldots, v_i, \ldots, 0)$.
  Dann gilt $\gamma_i(a + \lambda \cdot b) = \gamma_i(a) + \lambda \cdot i(b)$
  für $a, b \in V_i, \lambda \in K$.
  Aus Lemma 6.2 der Vorlesung (\emph{``Seien $V$ und $W$ zwei Vektorräume über
    demselben Körper $K$ und sei $\varphi \colon V \to W$ eine Abbildung.
    Dann sind äquivalent: Die Funktion $\varphi$ ist eine lineare Abbildung und
    Es gilt $\varphi(v_1 + \lambda v_2) = \varphi(v_1) + \lambda\varphi(v_2)$
    für alle $v_1, v_2 \in V$ und $\lambda \in K$''})
  folgt, dass $\gamma_i$ eine lineare Abbildung ist.

  Weiter ist $\Kern(\gamma_i) = \qty\big{0_{V_i}}$, damit folgt aus Lemma 6.21 der
  Vorlesung (\emph{``Es ist $\varphi$ genau dann ein Monomorphismus. wenn
    $\Kern\qty\big(\varphi) = \qty\big{0_V}$ gilt.''}) die Injektivität von
  $\gamma_i$.

  Sei nun $v \in \overline{V_i}$ beliebig.
  Per Definition von $\overline{V_i}$ hat $v$ die Form
  $\qty\big(0_{V_1}, 0_{V_2}, \ldots, v_i, \ldots, 0_{V_r})$
  und $\gamma_i(v_i) = v$.

  $\Rightarrow \gamma_i$ ist surjektiv

  $\Rightarrow \gamma_i$ ist ein Isomorphismus.

  $\Rightarrow V_i \cong \overline{V_i}$

\item Zeigen Sie, dass $\oplus_{i = 1}^r V_i$ die interne direkte Summe der
  Unterräume $\overline{V_1}, \ldots, \overline{V_r}$ ist.

\item Sei nun $V$ ein Vektorraum über $K$ und seien $V_1, \ldots, V_r$
  Unterräume von $V$, sodass $V$ die innere direkte Summe $V_1, \ldots, V_r$
  ist.
  Zeigen Sie, dass ein Isomorphismus $\varphi \colon V \to \oplus_{i = 1}^r V_i$
  existiert, sodass $\varphi\qty\big(V_i) = \overline{V_i}$ für alle
  $1 \leq i \leq r$.

  \subparagraph{Lsg.} Sei $\varphi \colon V \to \oplus_{i = 1}^r V_i,
  \sum_{i = 1}^r v_i \mapsto \qty\big(v_1, \ldots, v_r)$ und seien
  $v, w \in V$ sowie $\lambda \in K$ beliebig.
  Dann ist $v + \lambda \cdot w = \sum_{i = 1}^r v_i + \lambda \cdot w_i$ und
  $\varphi\qty\big(v + \lambda \cdot w) = \qty\big(
    v_1 + \lambda \cdot w_1,
    v_2 + \lambda \cdot w_2,
    \ldots,
    v_r + \lambda \cdot w_r,
    )$.
    Weiter ist
    \begin{flalign*}
      \varphi\qty\big(v) + \lambda \cdot \varphi\qty\big(w)
      &= \qty\big(
        v_1,
        v_2,
        \ldots,
        v_r
      ) + \lambda \cdot \qty\big(
        w_1,
        w_2,
        \ldots,
        w_r
      ) & \\
      &= \qty\big(
        v_1,
        v_2,
        \ldots,
        v_r
      ) + \qty\big(
        \lambda \cdot w_1,
        \lambda \cdot w_2,
        \ldots,
        \lambda \cdot w_r
      ) \\
      &= \qty\big(
        v_1 + \lambda \cdot w_1,
        v_1 + \lambda \cdot w_2,
        \ldots,
        v_1 + \lambda \cdot w_r
      )
    \end{flalign*}
    Aus Lemma 6.2 der Vorlesung folgt $\varphi$ ist eine lineare Abbildung.
    Offensichtlich ist $\Kern\qty\big(\varphi) = \qty\big{0_V}$.

    $\Rightarrow \varphi$ ist nach Lemma 6.21 ein Monomorphismus.

    Nun ist $\dim\qty\big(V) = \dim\qty(\oplus_{i = 1}^r V_i)$, aus Korollar
    6.25 der Vorlesung  (\emph{``Seien $V$ und $W$ zwei endlich-dimensionale
      Vektorräume über demselben Körper $K$ mit $\dim\qty(V) = \dim\qty(W)$.
      Sei $\varphi \colon V \to W$ eine lineare Abbildung.
      Dann sind äquivalent: $\varphi$ ist ein Monomorphismus und $\varphi$
      ist ein Isomorphismus''}) folgt die Behauptung.

  \item Erläutern Sie unter Verwendung von (iii) und (iv), warum man sagen
    könnte, dass interne und externe direkte Summen von Vektorräumen
    ``das Gleiche'' sind.

    \subparagraph{Lsg.} Ja, eigentlich schon.
    Allerdings werden bei der internen direkten Summe Summen betrachtet
    und bei der externen direkten Summe Tupel.
    Dennoch sind die beiden zueinander isomorph.
\end{enumerate}
\end{document}
