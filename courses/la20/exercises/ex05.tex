\documentclass{scrreprt}

\usepackage{aligned-overset}
\usepackage{amsmath}
\usepackage{amssymb}
\usepackage{bm}
\usepackage[shortlabels]{enumitem}
\usepackage{hyperref}
\usepackage[utf8]{inputenc}
\usepackage{multicol}
\usepackage{mathtools}
\usepackage{physics}
\usepackage{tabularx}
\usepackage[table]{xcolor}
\usepackage{titling}
\usepackage{fancyhdr}
\usepackage{xfrac}
\usepackage{pgfplots}
\usepackage{tikz-3dplot}

\pgfplotsset{compat = newest}
\usetikzlibrary{intersections}
\usetikzlibrary{patterns}
\usepgfplotslibrary{fillbetween}

\author{Karsten Lehmann}
\date{SoSe 2022}
\title{Übungsblatt 05\\Lineare Algebra - Weiterführende Konzepte}

\setlength{\headheight}{26pt}
\pagestyle{fancy}
\fancyhf{}
\lhead{\thetitle}
\rhead{\theauthor}
\lfoot{\thedate}
\rfoot{Seite \thepage}

\newcommand\Grad{\text{Grad}}
\newcommand\Kern{\text{Kern}}

\begin{document}

\paragraph{Aufgabe 2} Entscheiden Sie für jede der folgenden Matrizen
$X \in \qty\big{A, B, C}$, ob $X$ diagonalisierbar ist.
Falls $X$ diagonalisierbar ist, dann finden Sie eine Basis von $K^2$
(wobei $K$ der jeweils zugrundeliegende Körper sei), die aus Eigenvektoren
von $X$ besteht und eine invertierbare Matrix
$S \in \mathcal{M}_{2 \times 2}\qty\big(K)$, sodass $S^{-1}XS$ eine
Diagonalmatrix ist.

\begin{enumerate}[(a)]
\item $A \coloneqq \begin{pmatrix}
    5 & 4 \\
    -3 & -2 \\
  \end{pmatrix} \in \mathcal{M}_{2 \times 2}\qty\big(\mathbb{R})$

  \subparagraph{Lsg.} Das charakteristische Polynom der Matrix ist
  \begin{flalign*}
    \det\qty(XI_n - A) &= \det\qty(
      \begin{pmatrix}
        X & 0 \\
        0 & X \\
      \end{pmatrix} - \begin{pmatrix}
        5 & 4 \\
        -3 & -2 \\
      \end{pmatrix}
    ) & \\
    &= \det\begin{pmatrix}
      X - 5 & -4 \\
      3 & X + 2 \\
    \end{pmatrix} \\
    &= X^2 - 3X + 2 = \qty\big(X - 1)\qty\big(X - 2)
  \end{flalign*}
  Nach Satz 10.41 (a) der Vorlesung (\emph{``Wenn das charakteristische
    Polynom von $A$ komplett in verschiedene Linearfaktoren zerfällt, so ist
    $A$ diagonalisierbar''}), ist die Matrix $A$ diagonalisierbar.

  Seien nun $B_{\lambda_1} = 1 \cdot I_n - A = \begin{pmatrix}
    -4 & -4 \\
    3 & 3 \\
  \end{pmatrix}$ und
  $B_{\lambda_2} = 2 \cdot I_n - A = \begin{pmatrix}
    -3 & -4 \\
    3 & 4 \\
  \end{pmatrix}$.
  Nach Anwendung des Gauß-Verfahrens bleiben übrig:
  \[
    \begin{pmatrix}
      -4 & -4 \\
      3 & 3 \\
    \end{pmatrix}
    \overset{
      \substack{
        Z.1 + \frac{4}{3} \cdot Z.2 \\
        Z.2 / 3
      }
    }\leadsto
    \begin{pmatrix}
      0 & 0 \\
      1 & 1 \\
    \end{pmatrix}
  \]
  und
  \[
    \begin{pmatrix}
      -3 & -4 \\
      3 & 4 \\
    \end{pmatrix}
    \overset{Z.1 + Z.2}
    \leadsto
    \begin{pmatrix}
      0 & 0 \\
      3 & 4 \\
    \end{pmatrix}
  \]

  Nun ist $V_{\lambda_1} = \qty{
    \begin{pmatrix}x\\y\end{pmatrix} \in \mathbb{R}^2
    \:{\Big |}\:
    x - y = 0
  } = \qty{
    \begin{pmatrix}1\\-1\end{pmatrix} \cdot x
    \:{\Big |}\:
    x \in \mathbb{R}
  }$ und

  $V_{\lambda_2} = \qty{
    \begin{pmatrix}x\\y\end{pmatrix} \in \mathbb{R}^2
    \:{\Big |}\:
    3 \cdot x - 4 \cdot y = 0
  } = \qty{
    \begin{pmatrix}4\\-3\end{pmatrix} \cdot x
    \:{\Big |}\:
    x \in \mathbb{R}
  }$.

  Jetzt bilden $\qty{
    \begin{pmatrix}1\\-1\end{pmatrix},
    \begin{pmatrix}4\\-3\end{pmatrix}
  }$ ein Basis in $\mathbb{R}^2$ und
  $S = \begin{pmatrix}
    1 & 4 \\
    -1 & -3 \\
  \end{pmatrix}$ ist nach Satz 7.38 der Vorlesung (\emph{``Es sind äquivalent:
    $A$ ist invertierbar und die Spalten von $A$ bilden eine Basis von $K^n$''})
  invertierbar mit $S^{-1} = \begin{pmatrix}
    -3 & -4 \\
    1 & 1 \\
  \end{pmatrix}$

  Schließlich ist
  \[
    S^{-1} \cdot A \cdot S =
    \begin{pmatrix}
      -3 & -4 \\
      1 & 1 \\
    \end{pmatrix} \cdot \begin{pmatrix}
      5 & 4 \\
      -3 & -2 \\
    \end{pmatrix} \cdot \begin{pmatrix}
      1 & 4 \\
      -1 & -3 \\
    \end{pmatrix} = \begin{pmatrix}
      1 & 0 \\
      0 & 2 \\
    \end{pmatrix}
  \]
  ein Diagonalmatrix, deren Diagonaleneinträge die Eigenwerte von $A$ sind.

\newpage
\item $B \coloneqq \begin{pmatrix}
    0 & 1 \\
    -1 & -1 \\
  \end{pmatrix} \in \mathcal{M}_{2 \times 2}\qty\big(\mathbb{F}_3)$, wobei
  $\mathbb{F}_3 \coloneqq \mathbb{Z}/3$

  \subparagraph{Lsg.} Das charakteristische Polynom der Matrix ist
  \begin{flalign*}
    \det\qty(XI_n - A) &= \det\qty(
      \begin{pmatrix}
        X & 0 \\
        0 & X \\
      \end{pmatrix} - \begin{pmatrix}
        0 & 1 \\
        -1 & -1 \\
      \end{pmatrix}
    ) & \\
    &= \det\begin{pmatrix}
      X & 2 \\
      1 & X + 1 \\
    \end{pmatrix} \\
    &= X^2 + X + 1 = \qty\big(X - 1)\qty\big(X - 1)
  \end{flalign*}
  Nun ist $\lambda = 1$ ein Eigenwert von $B$.
  Sei nun $C_\lambda = 1 \cdot I_n - B = \begin{pmatrix}
    1 & 2 \\
    1 & 2 \\
  \end{pmatrix}$.
  Nach Anwendung des Gauß-Verfahrens bleiben übrig:
  \[
    \begin{pmatrix}
      1 & 2 \\
      1 & 2 \\
    \end{pmatrix}
    \overset{Z.1 + 2 \cdot Z.2}\leadsto
    \begin{pmatrix}
      0 & 0 \\
      1 & 2 \\
    \end{pmatrix}
  \]
  Daher ist $V_{\lambda} = \qty{
    \begin{pmatrix}x\\y\end{pmatrix} \in \mathbb{R}^2
    \:{\Big |}\:
    x -2 \cdot y = 0
  } = \qty{
    \begin{pmatrix}1\\1\end{pmatrix} \cdot x
    \:{\Big |}\:
    x \in \mathbb{F}_3
  }$.

  Allerdings bilden Elemente aus $V_\lambda$ keine Basis von $\mathbb{F}_3$,
  daher ist $A$ nach Korollar 10.38 der Vorlesung (\emph{``Es sind äquivalent:
    $A$ ist diagonalisierbar und $K^n$ besitzt eine Basis aus Eigenvektoren von
    $A$''}) nicht diagonalisierbar.

\item $C \coloneqq \begin{pmatrix}
    0 & 1 \\
    -1 & -1 \\
  \end{pmatrix} \in \mathcal{M}_{2 \times 2}\qty\big(\mathbb{F}_5)$, wobei
  $\mathbb{F}_5 \coloneqq \mathbb{Z}/5$

  Das charakteristische Polynom der Matrix ist
  \begin{flalign*}
    \det\qty(XI_n - C) &= \det\qty(
      \begin{pmatrix}
        X & 0 \\
        0 & X \\
      \end{pmatrix} - \begin{pmatrix}
        0 & 1 \\
        -1 & -1 \\
      \end{pmatrix}
    ) & \\
    &= \det\begin{pmatrix}
      X & 4 \\
      1 & X + 1 \\
    \end{pmatrix} \\
    &= X^2 + X + 1
  \end{flalign*}
  Dieses Polynom hat keine Nullstellen auf $\mathbb{F}_5$, daher ist die Matrix
  $C$ nicht diagonalisierbar.
\end{enumerate}

\newpage
\paragraph{Aufgabe 3} Seien $v_1 \coloneqq \begin{pmatrix}1\\1\end{pmatrix} \in
\mathbb{R}^2$ und $v_2 \coloneqq \begin{pmatrix}2\\0\end{pmatrix} \in
\mathbb{R}^2$.
Seien weiterhin $\lambda_1 \coloneqq 2$ und $\lambda_2 \coloneqq -4$.
Finden Sie eine Matrix $A$, sodass $v_i$ für $i \in \qty\big{1, 2}$ ein
Eigenvektor von $A$ mit Eigenwert $\lambda_i$ ist.

\subparagraph{Lsg.} Es ist $D = \begin{pmatrix}
  2 & 0 \\
  0 & -4 \\
\end{pmatrix}$ die Diagonalmatrix mit den Eigenwerten von $A$ als
Diagonaleneinträge.
Weiter sind $v_1, v_2$ linear unabhängig in $\mathbb{R}^2$.
Damit ist $S = \begin{pmatrix}
  1 & 2 \\
  1 & 0 \\
\end{pmatrix}$ nach Satz 7.38 der Vorlesung (\emph{``Es sind äquivalent:
  $A$ ist invertierbar und die Spalten von $A$ bilden eine Basis von $K^n$''})
invertierbar.
Aus dem Gauß-Jacobi Verfahren folgt:

\begin{flalign*}
  \qty(
  \begin{array}{cc|cc}
    1 & 2 & 1 & 0 \\
    1 & 0 & 0 & 1 \\
  \end{array})
  \overset{Z.2 + Z.1}&\leadsto
  \qty(
  \begin{array}{cc|cc}
    1 & 2 & 1 & 0 \\
    2 & 2 & 1 & 1 \\
  \end{array}) & \\
  \overset{-1 \cdot Z.1 + Z.2}&\leadsto
  \qty(
  \begin{array}{cc|cc}
    1 & 0 & 0 & 1 \\
    2 & 2 & 1 & 1 \\
  \end{array}) \\
  \overset{Z.2 - 2 \cdot Z.1}&\leadsto
  \qty(
  \begin{array}{cc|cc}
    1 & 0 & 0 & 1 \\
    0 & 2 & 1 & -1 \\
  \end{array}) \\
  \overset{\frac{1}{2} \cdot Z.2}&\leadsto
  \qty(
  \begin{array}{cc|cc}
    1 & 0 & 0 & 1 \\
    0 & 1 & \frac{1}{2} & -\frac{1}{2} \\
  \end{array})
\end{flalign*}
$S^{-1} = \frac{1}{2} \cdot \begin{pmatrix}
  0 & 2 \\
  1 & -1 \\
\end{pmatrix}$.
Schließlich ist
\[
  A = SDS^{-1} = \begin{pmatrix}
    1 & 2 \\
    1 & 0 \\
  \end{pmatrix} \cdot \begin{pmatrix}
    2 & 0 \\
    0 & -4 \\
  \end{pmatrix} \cdot \frac{1}{2} \cdot \begin{pmatrix}
    0 & 2 \\
    1 & -1 \\
  \end{pmatrix} = \begin{pmatrix}
    -4 & 6 \\
    0 & 2 \\
  \end{pmatrix}
\]
\end{document}
