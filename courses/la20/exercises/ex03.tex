\documentclass{scrreprt}

\usepackage{aligned-overset}
\usepackage{amsmath}
\usepackage{amssymb}
\usepackage{bm}
\usepackage[shortlabels]{enumitem}
\usepackage{hyperref}
\usepackage[utf8]{inputenc}
\usepackage{multicol}
\usepackage{mathtools}
\usepackage{physics}
\usepackage{tabularx}
\usepackage[table]{xcolor}
\usepackage{titling}
\usepackage{fancyhdr}
\usepackage{xfrac}
\usepackage{pgfplots}
\usepackage{tikz-3dplot}

\pgfplotsset{compat = newest}
\usetikzlibrary{intersections}
\usetikzlibrary{patterns}
\usepgfplotslibrary{fillbetween}

\author{Karsten Lehmann}
\date{SoSe 2022}
\title{Übungsblatt 03\\Lineare Algebra - Weiterführende Konzepte}

\setlength{\headheight}{26pt}
\pagestyle{fancy}
\fancyhf{}
\lhead{\thetitle}
\rhead{\theauthor}
\lfoot{\thedate}
\rfoot{Seite \thepage}

\newcommand\Grad{\text{Grad}}
\newcommand\Kern{\text{Kern}}

\begin{document}

\paragraph{Aufgabe 2} Seien $K$ ein Körper, $\lambda \in K$ und
\[
  A \coloneqq \begin{pmatrix}
    X & a & X + a \\
    X^2 & a & X^2 + a \\
    X^3 & a & X^3 + a \\
  \end{pmatrix}
  \in \mathcal{M}_{3 \times 3}\qty\big(K\qty[X])
\]
\begin{enumerate}[(i)]
\item Berechnen Sie $\det A$ mittels elementarer Zeilen- oder
  Spaltenumformungen.
\item Entscheiden Sie, ob $A$ invertierbar ist.
\end{enumerate}

\subparagraph{Lsg.}
\begin{enumerate}[(i)]
\item
  \begin{flalign*}
    \det \begin{pmatrix}
      X & a & X + a \\
      X^2 & a & X^2 + a \\
      X^3 & a & X^3 + a \\
    \end{pmatrix} \overset{S. 3 - S.2}&\leadsto \det \begin{pmatrix}
      X & a & X \\
      X^2 & a & X^2 \\
      X^3 & a & X^3 \\
    \end{pmatrix} \\
    \overset{S. 3 - S.1}&\leadsto \det \begin{pmatrix}
      X & a & 0 \\
      X^2 & a & 0 \\
      X^3 & a & 0 \\
    \end{pmatrix} \\
  \end{flalign*}
  Aus Korollar 8.23 der Vorlesung (``\emph{Falls
    $A \in \mathcal{M}_{n \times n}\qty\big(R)$ einen Null-Spalte hat, dann
    gilt $\det\qty\big(A) = 0$}'') folgt $\det\qty\big(A) = 0_K$.

\item Nach Satz 8.40 der Vorlesung ist ``\emph{$A$ ist invertierbar}''
  $\iff$ ``\emph{$\det\qty\big(A)$ ist invertierbar in $K$}''.
  Nun ist $0_K$ in $K$ nicht invertierbar.

  $\Rightarrow A$ ist nicht invertierbar.
\end{enumerate}

\paragraph{Aufgabe 3} Seien
$f \coloneqq 4X^5 + 2X^4 + X^3 + 2X^2 + 5 \in \mathbb{R}\qty[X]$ und
$g \coloneqq 2X^3 + 3X^2 + X + 2 \in \mathbb{R}\qty[X]$.
Bestimmen Sie Polynome $p, q \in \mathbb{R}\qty[X]$ mit
$f = pg + q$, wobei entweder $q = 0$ oder $\Grad\qty\big(q) < \Grad\qty\big(g)$.

\subparagraph{Lsg.} Es ist $n = \Grad\qty\big(f) = 5$ und
$m = \Grad\qty\big(g) = 3$.
Schreib man die beiden Polynome in der Form
$f = \sum_{n \geq i \in \mathbb{N}_0} a^iX^i$ und
$g = \sum_{m \geq i \in \mathbb{N}_0} b^iX^i$,
so ist $a_n = 4$ und $b_m = 2$.

Nun sei $b = \frac{a_n}{b_m} = \frac{4}{2} = 2$ und
\begin{flalign*}
  \tilde f &= f - bX^{n - m}g & \\
  &= 4X^5 + 2X^4 + X^3 + 2X^2 + 5 - 2X^2
  \qty\Big(2X^3 + 3X^2 + X + 2) \\
  &= 4X^5 + 2X^4 + X^3 + 2X^2 + 5 - \qty\Big(4X^5 + 6X^4 + 2X^3 + 4X^2) \\
  &= -4X^4 - X^3 - 2X^2 + 5
\end{flalign*}

Jetzt ist $f = 2X^2 \cdot g + \tilde f$, allerdings ist
$4 = \Grad\qty\big(\tilde f) \nless \Grad\qty\big(g) = 3$.

Schreib nun
$\tilde f = \sum_{\Grad(\tilde f) \geq i \in \mathbb{N}_0} {\tilde a}^iX^i$
und setze ${\tilde b} = \frac{{\tilde a}_{\Grad(\tilde f)}}{b_m}
= \frac{-4}{2} = -2$, außerdem setze
\begin{flalign*}
  {\tilde{\tilde f}} &= \tilde f - {\tilde b}X^{\Grad(\tilde f) - m}g & \\
  &= -4X^4 - X^3 - 2X^2 + 5 - \qty(-2X
  \qty\Big(2X^3 + 3X^2 + X + 2)) \\
  &= -4X^4 - X^3 - 2X^2 + 5 - \qty\Big(-4X^4 - 6X^3 - 2X^2 - 4X) \\
  &= 5X^3 + 4X + 5
\end{flalign*}

Jetzt ist $f = 2X^2 \cdot g + \tilde f
= 2X^2 \cdot g - 2X \cdot g + {\tilde{\tilde f}}
= \qty(2X^2 - 2X) \cdot g + {\tilde{\tilde f}}$, allerdings ist noch immer
$3 = \Grad\qty\big(\tilde{\tilde f}) \nless \Grad\qty\big(g) = 3$.

Schreib schließlich
$\tilde {\tilde f} = \sum_{\Grad(\tilde {\tilde f}) \geq i \in \mathbb{N}_0} \tilde {\tilde a}^iX^i$
und setze $\tilde {\tilde b} = \frac{\tilde {\tilde a}_{\Grad(\tilde {\tilde f})}}{b_m}
= \frac{5}{2}$, außerdem setze
\begin{flalign*}
  \tilde {\tilde{\tilde f}} &= \tilde {\tilde f} -
  \tilde \tilde {\tilde b}X^{\Grad(\tilde {\tilde f}) - m}g & \\
  &= 5X^3 + 4X + 5 - \frac{5}{2}
  \qty\Big(2X^3 + 3X^2 + X + 2) \\
  &= 5X^3 + 4X + 5 - \qty\Big(5X^3 + \frac{15}{2}X^2  + \frac{5}{2}X + 5) \\
  &= -\frac{15}{2}X^2 + \frac{3}{2}X
\end{flalign*}

Jetzt ist $f = \qty(2X^2 - 2X) \cdot g + {\tilde{\tilde f}}
= \qty(2X^2 - 2X) \cdot g + \frac{5}{2} \cdot g + \tilde {\tilde{\tilde f}}
= \qty(2X^2 - 2X + \frac{5}{2}) \cdot g + \tilde {\tilde{\tilde f}}$.

Somit lässt sich $f$ darstellen als $f = p \cdot g + q$ mit
$p = \qty(2X^2 - 2X + \frac{5}{2})$ und $q = -\frac{15}{2}X^2 + \frac{3}{2}X$
dabei ist $2 = \Grad\qty\big(q) < \Grad\qty\big(g) = 3$.

\paragraph{Aufgabe 4} Seien $K$ ein Körper und
$V \coloneqq \qty{f \in K\qty[X] \:{\big |}\: f\qty\big(1) = 0}$.
\begin{enumerate}[(i)]
\item Zeigen Sie, dass $V$ ein Unterraum von $K\qty[X]$ ist.

  \subparagraph{Lsg.} Sei $\Phi_{1_K} \colon K\qty[X] \to K,
  f \mapsto f\qty\big(1_K)$ eine Abbildung.
  Nach Lemma 9.13 der Vorlesung (``Sei $A$ eine $K$-Algebra und sei $a \in A$
  fest gewählt.
  Dann ist die Abbildung $\Phi_a \colon K\qty[X] \to A, f \mapsto f\qty\big(a)$
  ein unitaler $K$-Algebra Homomorphismus'') ist $\Phi_{1_K}$ ein
  $K$-Algebra-Homomorphismus.

  Nach Definition 7.49 der Vorlesung ist $\Phi_{1_K}$ als
  $K$-Algebra-Homomorphismus eine lineare Abbildung und nach
  Definition 6.17 der Vorlesung ist der Kern der Abbildung $\Phi_{1_K}$
  definiert als
  \[
    \Kern\qty\big(\Phi_{1_K}) = \qty{
      f \in K\qty[X] \:{\big |}\: \Phi_{1_K}(f) = 0
    }
  \]
  Somit ist $\Kern\qty\big(\Phi_{1_K}) = V$.
  Aus Lemma 6.18 der Vorlesung (``Falls $\varphi \colon V \to W$ eine
  lineare Abbildung ist, dann ist $\Kern\qty\big(\varphi)$ ein Unterraum
  von $V$'') folgt die Behauptung.

\item Für $n \in \mathbb{N}$ sei
  \[
    U_n \coloneqq \qty{
      f \in K\qty[X] \:{\big |}\: f = 0 \text{ oder } f \ne 0
      \text { und } \Grad\qty\big(f) \leq n
    }
  \]
  Bestimmen Sie $U_1 \cap V$.

  \subparagraph{Lsg.} $U_1 \cap V = \qty{0_{K\qty[X]}} \cup
  \qty{a_0X^0 + a_1X^1 \in K\qty[X] \:{\big |}\: a_0 + a_1 = 0_K}$

\item Finden Sie für jede natürliche Zahl $f_n$ ein Polynom
  $f_n \in V$ mit $\Grad\qty\big(f_n) = n$.

  \subparagraph{Lsg.} Sei $f_n = -1 \cdot X^0 + 1 \cdot X^n$

\item Zeigen Sie, dass
  ${\big\langle} f_1, \ldots, f_n {\big\rangle} = U_n \cap V$
  für jede natürliche Zahl $n$ gilt.

  \subparagraph{Lsg.} Sei $n \in \mathbb{N}, g \in U_n \cap V$ beliebig.
  Nach Definition von $U_n$ ist $\Grad\qty\big(g) \leq n$ und nach Definition
  von $V$ ist $g(1) = 0$.
  Nun hat $g$ die Form $\sum_{m \in \mathbb{N}_0}a_mX^m$ mit $a_m = 0$ für
  $m > n$.

  Da $g\qty\big(1) = 0$ folgt $\sum_{m \in \mathbb{N}_0}a_m = 0$ und
  $a_0 + \sum_{m \in \mathbb{N}}a_m = 0$.

  $\Rightarrow a_0 = -\sum_{m \in \mathbb{N}_0}a_m$.

  Somit ist $g$ darstellbar als $\sum_{m \in \mathbb{N}_{\leq n}} a_mf_n$.

  $\Rightarrow$ jedes $g \in U_n \cap V$ ist als Linearkombination von
  $f_1, \ldots, f_n$ darstellbar.

  $\Rightarrow f_1, \ldots, f_n$ spannt $U_n \cap V$ auf.

\item ${\big\langle} f_1, f_2, \ldots{\big\rangle}$ ist eine Basis von $V$.

  \subparagraph{Lsg.}
  Ähnlich zu (iv) lässt sich zeigen, dass jedes Element aus $V$ als
  Linearkombination von $f_1, f_2, \ldots$ darstellbar ist.
\end{enumerate}

\paragraph{Aufgabe 5}
\begin{enumerate}[(i)]
\item Beweisen Sie Lemma 9.27 (iii) $\Rightarrow$ (i).

  \subparagraph{Lsg.} Seien $f, g \in K\qty[X]$ zwei von Null verschiedene
  Polynome.
  Es ist zu zeigen, dass \emph{Es existieren $s, t \in K\qty[X]$, sodass
    $fs + gt = 1_{K\qty[X]}$} $\Rightarrow$ \emph{$f$ und $g$ sind teilerfremd}

  Die beiden von Null verschiedenen Polynome $f, g \in K\qty[X]$ werden
  teilerfremd genannt, falls für alle $h \in K\qty[X]$ mit $h \:{\big |}\: f$
  und $h \:{\big |}\: g$ schon $h \in K$ folgt.

  Sei nun (iii) gegeben, dass heißt es existieren $s, t \in K\qty[X]$ mit
  $fs + gt = 1_{K\qty[X]}$.
  Angenommen es existiert nun ein $h \in K\qty[X]$ mit $h \:{\big |}\: f$
  und $h \:{\big |}\: g$.
  Seien nun $a, b \in K\qty[X]$ so gewählt, dass $f = a \cdot h$ und
  $g = b \cdot h$.
  Dann ist $a \cdot h \cdot s + b \cdot h \cdot = 1_{K\qty[X]}$ und
  $h \cdot \qty\big(a \cdot s + b \cdot t) = 1_{K\qty[X]}$.

  $\Rightarrow h^{-1} = \qty\big(a \cdot s + b \cdot t)$

  $\Rightarrow h$ ist invertierbar.

  Nach Lemma 9.16 der Vorlesung (``Sei $f \in K\qty[X]$.
  Dann ist $f$ genau dann invertierbar, wenn
  $f \in K \setminus \qty\big{0_K}$ gilt.'') folgt
  $h \in \setminus \qty\big{0_K}$.

  $\Rightarrow f$ und $g$ sind teilerfremd.

\item Beweisen Sie Lemma 9.22 (ii).

  \subparagraph{Lsg.} Es ist zu beweisen, dass falls $R$ ein kommutativer Ring
  ist, gilt: ``\emph{Falls $I$ und $J$ Ideale von $R$ sind, so ist $I + J$ auch
    ein Ideal von $R$}''.

  Eine Teilmenge $I \subseteq R$ wird Ideal von $R$ genannt, falls
  \begin{enumerate}[(i)]
  \item $0_R \in I$
  \item $I + I \subseteq I$
  \item $RI \subseteq I$
  \end{enumerate}
  Dabei ist für $I, J \subseteq R$ definiert:
  \[
    IJ = \qty{ab \:{\big |}\: a \in I, b \in J},
    I + J = \qty{a + b \:{\big |}\: a \in I, b \in J}
  \]

  Da $I$ und $J$ bereits Ideale von $R$ sind, gilt $0_R \in I$ und $0_R \in J$.
  Gemäß der Definition von $I + J$ folgt auch $0_R \in I + J$.

  Weiter ist die eben definierte Addition offensichtlich kommutativ
  (, d.h. $I + J = J + I$), deswegen folgt
  \[
    I + J + I + J =
    \underset{\subseteq I}{\underbrace{I + I}} +
    \underset{\subseteq J}{\underbrace{J + J}}
  \]
  $\Rightarrow I + J + I + J \subseteq I + J$

  Schließlich ist $R\qty\big(I + J) = RI + RJ$ mit
  $RI \subseteq I$ und $RJ \subseteq J$.

  $\Rightarrow R\qty\big(I + J) \subseteq I + J$.

  $\Rightarrow I + J$ ist ein Ideal von $R$.
\end{enumerate}

\end{document}
