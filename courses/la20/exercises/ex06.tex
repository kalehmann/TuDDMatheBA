\documentclass{scrreprt}

\usepackage{aligned-overset}
\usepackage{amsmath}
\usepackage{amssymb}
\usepackage{bm}
\usepackage[shortlabels]{enumitem}
\usepackage{hyperref}
\usepackage[utf8]{inputenc}
\usepackage{multicol}
\usepackage{mathtools}
\usepackage{physics}
\usepackage{tabularx}
\usepackage[table]{xcolor}
\usepackage{titling}
\usepackage{fancyhdr}
\usepackage{xfrac}
\usepackage{pgfplots}
\usepackage{tikz-3dplot}

\pgfplotsset{compat = newest}
\usetikzlibrary{intersections}
\usetikzlibrary{patterns}
\usepgfplotslibrary{fillbetween}

\author{Karsten Lehmann}
\date{SoSe 2022}
\title{Übungsblatt 06\\Lineare Algebra - Weiterführende Konzepte}

\setlength{\headheight}{26pt}
\pagestyle{fancy}
\fancyhf{}
\lhead{\thetitle}
\rhead{\theauthor}
\lfoot{\thedate}
\rfoot{Seite \thepage}

\newcommand\Grad{\text{Grad}}
\newcommand\Kern{\text{Kern}}

\begin{document}
\paragraph{Aufgabe 2} Für jede der folgenden Matrizen $X \in \qty\big{A, B, C}$
haben Sie in der letzten Übung entschieden, ob $X$ diagonalisierbar ist.
\begin{itemize}
\item In welchen Fällen ließe sich diese Frage nur anhand des charakteristischen
  Polynoms und unter Verwendung von Satz 10.41 der Vorlesung (\emph{``Sei
    $A \in \mathcal{M}_{n \times n}\qty\big(K)$ (a) Wenn
    das charakteristische Polynom komplett in verschiedene Linearfaktoren
    zerfällt, so ist $A$ diagonalisierbar.
    (b) Wenn $A$ diagonalisierbar ist, so zerfällt das charakteristische
    Polynom von $A$ in Linearfaktoren, aber die Linearfaktoren müssen nicht
    notwendigerweise verschieden sein.''})
\item Geben Sie die algebraische Vielfachheit jedes Eigenwertes an.
  Geben Sie jeweils an, ob sich die geometrische Vielfachheit des jeweiligen
  Eigenwertes nur mit Kenntnis seiner algebraischen Vielfachheit und des
  charakteristischen Polynoms finden ließe.
\end{itemize}
\begin{enumerate}[(i)]
\item $A \coloneqq \begin{pmatrix}
    5 & 4 \\
    -3 & -2 \\
  \end{pmatrix} \in \mathcal{M}_{2 \times 2} \qty\big(\mathbb{R})$,
  $f_a = X^2 - 3X + 2 = \qty\big(X - 1)\qty\big(X - 2)$

  \subparagraph{Lsg.} Das charakteristische Polynom von $A$ zerfällt komplett
  in verschiedene Linearfaktoren, dementsprechend ist die Matrix $A$
  diagonalisierbar.

  Weiter ist $m_{\text{alg}}\qty\big(A, 1) = 1$ und
  $m_{\text{alg}}\qty\big(A, 2) = 1$.
  Da anhand des charakteristischen Polynoms zu sehen ist, dass die Matrix
  $A$ diagonalisierbar ist, folgt aus Satz 10.48 (b) der Vorlesung
  (\emph{``Falls $A$ diagonalisierbar ist, dann sind alle geometrischen
    Vielfachheiten gleich den entsprechenden algebraischen Vielfachheiten''}),
  dass $m_{\text{geom}}\qty\big(A, 1) = 1$ und
  $m_{\text{geom}}\qty\big(A, 2) = 1$.

\item $B \coloneqq \begin{pmatrix}
    0 & 1 \\
    -1 & -1 \\
  \end{pmatrix} \in \mathcal{M}_{2 \times 2} \qty\big(\mathbb{F}_3)$,
  wobei $\mathbb{F}_3 = \mathbb{Z}/3$,
  $f_a = X^2 + X + 1 = \qty\big(X - 1)^2$

  \subparagraph{Lsg.} Das charakteristische Polynom von $B$ zerfällt komplett in
  Linearfaktoren, allerdings sind diese nicht verschieden.
  Somit lässt sich mit Satz 10.41 der Vorlesung keine Aussage darüber treffen,
  ob die Matrix $B$ diagonalisierbar ist.

  Weiter ist $m_{\text{alg}}\qty\big(A, 1) = 2$.
  Da anhand des charakteristischen Polynoms nicht erkennbar ist, ob
  die Matrix diagonalisierbar ist, lässt sich nur mit Kenntnis der
  algebraischen Vielfachheit und des charakteristischen Polynoms keine
  Aussage über die geometrische Vielfachheit von dem Eigenwert $1$ treffen.

\item $C \coloneqq \begin{pmatrix}
    0 & 1 \\
    -1 & -1 \\
  \end{pmatrix} \in \mathcal{M}_{2 \times 2} \qty\big(\mathbb{F}_5)$,
  wobei $\mathbb{F}_5 = \mathbb{Z}/5$,
  $f_a = X^2 + X + 1$.

  \subparagraph{Lsg.} Das charakteristische Polynom von $C$ zerfällt in
  $\mathbb{F}_5$ nicht in Linearfaktoren.
  Dementsprechend ist $C$ nicht diagonalisierbar und es existieren keine
  Eigenwerte.
\end{enumerate}

\paragraph{Aufgabe 3} Entscheiden Sie jeweils, ob sich anhand des
charakteristischen Polynoms ermitteln lässt, ob $A$ diagonalisierbar ist.
Wenn dies nicht der Fall ist, dann geben Sie zwei Matrizen $B$ und $C$ an,
sodass $f_A = f_B = f_C$, wobei $B$ diagonalisierbar ist, während $C$ nicht
diagonalisierbar ist.

\begin{enumerate}[(i)]
\item $A \in \mathcal{M}_{3 \times 3} \qty\big(\mathbb{R})$ und
  $f_A = \qty\big(X - 1)\qty\big(X + 1)\qty\big(X + 4)$.

  \subparagraph{Lsg.} Das charakteristische Polynom von $A$ zerfällt komplett in
  verschiedene Linearfaktoren, somit ist $A$ diagonalisierbar.

\item $A \in \mathcal{M}_{2 \times 2} \qty\big(\mathbb{F}_2)$ und
  $f_A = X^2 + 1$.

  \subparagraph{Lsg.} Das charakteristische Polynom von $A$ zerfällt in
  $\mathbb{F}_3$ nicht in Linearfaktoren und hat auch keine Nullstellen
  in $\mathbb{F}_3$.
  Somit ist $A$ in $\mathbb{F}_3$ nicht diagonalisierbar.

\item $A \in \mathcal{M}_{2 \times 2} \qty\big(\mathbb{R})$ und
  $f_A = \qty\big(X - 1)^2$.

  \subparagraph{Lsg.} Das charakteristische Polynom von $A$ zerfällt komplett in
  Linearfaktoren, allerdings sind diese nicht verschieden.
  Somit lässt sich mit Satz 10.41 der Vorlesung keine Aussage darüber treffen,
  ob die Matrix $A$ diagonalisierbar ist.

  Sei nun $B = \begin{pmatrix}
    1 & 0 \\
    0 & 1 \\
  \end{pmatrix}$, dann ist $B$ offensichtlich diagonalisierbar mit $S = B$
  und $S^{-1} \cdot B \cdot S = I_2$.

  Weiter sei $C = \begin{pmatrix}
    1 & 1 \\
    0 & 1
  \end{pmatrix}$.
  Dann ist $1 \cdot I_2 - C = \begin{pmatrix}
    0 & -1 \\
    0 & 0 \\
  \end{pmatrix}$ und $V_1 = \qty{\begin{pmatrix}x\\0\end{pmatrix}
    \:{\Big |}\: x \in \mathbb{R}}$ mit $\begin{pmatrix}x\\0\end{pmatrix}$
  als Basis, allerdings ist $\qty{\begin{pmatrix}x\\0\end{pmatrix}}$ keine
  Basis von $\mathbb{R}^2$ und somit ist $C$ nicht diagonalisierbar.

\item $A \in \mathcal{M}_{4 \times 4} \qty\big(\mathbb{F}_7)$ und
  $f_A = \qty\big(X - 1)\qty\big(X - 2)\qty\big(X - 3)\qty\big(X - 5)$.

  \subparagraph{Lsg.} Das charakteristische Polynom von $A$ zerfällt komplett in
  verschiedene Linearfaktoren, somit ist $A$ diagonalisierbar.
\end{enumerate}
\end{document}
