\documentclass{scrreprt}

\usepackage{aligned-overset}
\usepackage{amsmath}
\usepackage{amssymb}
\usepackage{bm}
\usepackage[shortlabels]{enumitem}
\usepackage{hyperref}
\usepackage[utf8]{inputenc}
\usepackage{multicol}
\usepackage{mathtools}
\usepackage{physics}
\usepackage{tabularx}
\usepackage[table]{xcolor}
\usepackage{titling}
\usepackage{fancyhdr}
\usepackage{xfrac}
\usepackage{pgfplots}
\usepackage{tikz-3dplot}

\pgfplotsset{compat = newest}
\usetikzlibrary{intersections}
\usetikzlibrary{patterns}
\usepgfplotslibrary{fillbetween}

\author{Karsten Lehmann}
\date{SoSe 2022}
\title{Übungsblatt 08\\Lineare Algebra - Weiterführende Konzepte}

\setlength{\headheight}{26pt}
\pagestyle{fancy}
\fancyhf{}
\lhead{\thetitle}
\rhead{\theauthor}
\lfoot{\thedate}
\rfoot{Seite \thepage}

\newcommand\End{\text{End}}
\newcommand\id{\text{id}}
\newcommand\Mat{\text{Mat}}

\begin{document}
\paragraph{Aufgabe 2} Seien $K$ ein Körper, $V \ne \qty\big{0_V}$ ein
endlich-dimensionaler Vektorraum über $K$, $\varphi \in \End\qty\big(V)$
und $U$ ein $\varphi$-invarianter Unterraum von $V$ mit
$\qty\big{0_V} \ne U \ne V$.
Zeigen Sie: Wenn $\varphi$ diagonalisierbar ist, dann sind auch $\varphi_U$ und
$\varphi_{V/U}$ diagonalisierbar.

\subparagraph{Lsg.} Angenommen $\varphi$ ist diagonalisierbar, dann zerfällt
$m_{\varphi}$ nach Satz 11.26 der Vorlesung (\emph{``Ein Endomorphismus
  $\varphi \in \End\qty\big(V)$ ist genau dann diagonalisierbar, wenn
  $m_{\varphi}$ vollständig in paarweise verschiedene Linearfaktoren
  zerfällt''}) vollständig in paarweise verschiedene Linearfaktoren.
Weiter ist $m_{\varphi}$ per Definition 11.6 der Vorlesung normiert.
Schließlich ist $m_{\varphi}$ nach Satz 9.35 der Vorlesung in der Form
$m_{\varphi} = a \cdot f_1 \cdot f_2 \cdot \ldots \cdot f_n$ darstellbar mit
$a = 1_K$ und $f_1, \ldots, f_n \in K\qty[X]$ normiert und irreduzibel.
Linearfaktoren der Form $\qty\big(X - \lambda_i)$ sind normiert und irreduzibel.
Es folgt
$m_{\varphi} = \qty\big(X - \lambda_1)
\cdot \ldots \cdot \qty\big(X - \lambda_n)$
ist die eindeutige Darstellung von $m_{\varphi}$ als Produkt irreduzibler
Polynome mit $\lambda_1, \ldots, \lambda_n$ als Eigenwerte von $\varphi$.

Nach Korollar 11.22 der Vorlesung (\emph{``Sei $\varphi \in \End\qty\big(V)$ und
  $U$ ein $\varphi$-invarianter Unterraum von $V$.
  Dann teilt das Minimalpolynom $m_{\varphi_U}$ von $\varphi_U$ das
  Minimalpolynom $m_{\varphi}$ von $\varphi$''}) folgt $m_{\varphi_U}$ hat die
Form $\qty\big(X - \lambda_1) \cdot \ldots \cdot \qty\big(X - \lambda_m)$ mit
$m \leq n$ und $\lambda_1, \ldots, \lambda_m \in \qty\big{
  \lambda_1, \ldots, \lambda_n
}$.
Damit zerfällt auch $m_{\varphi_U}$ vollständig in paarweise verschiedene
Linearfaktoren und aus Satz 11.26 der Vorlesung folgt $m_{\varphi_U}$ ist
diagonalisierbar.

Die Abbildung $\varphi_{V \setminus U}$ ist nach Lemma 12.9 der Vorlesung ein
Endomorphismus.
Es folgt $V \setminus U$ ist $\varphi$-invariant.
Die Behauptung folgt analog dem Beweis für $U$.
\end{document}
