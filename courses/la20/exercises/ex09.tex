\documentclass{scrreprt}

\usepackage{aligned-overset}
\usepackage{amsmath}
\usepackage{amsthm}
\usepackage{amssymb}
\usepackage{bm}
\usepackage[shortlabels]{enumitem}
\usepackage{hyperref}
\usepackage[utf8]{inputenc}
\usepackage{multicol}
\usepackage{mathtools}
\usepackage{physics}
\usepackage{tabularx}
\usepackage[table]{xcolor}
\usepackage{titling}
\usepackage{fancyhdr}
\usepackage{xfrac}
\usepackage{pgfplots}
\usepackage{tikz-3dplot}

\pgfplotsset{compat = newest}
\usetikzlibrary{intersections}
\usetikzlibrary{patterns}
\usepgfplotslibrary{fillbetween}

\author{Karsten Lehmann}
\date{SoSe 2022}
\title{Übungsblatt 09\\Lineare Algebra - Weiterführende Konzepte}

\setlength{\headheight}{26pt}
\pagestyle{fancy}
\fancyhf{}
\lhead{\thetitle}
\rhead{\theauthor}
\lfoot{\thedate}
\rfoot{Seite \thepage}

\newcommand\Bild{\text{Bild}}
\newcommand\End{\text{End}}
\newcommand\Kern{\text{Kern}}
\newcommand\Mat{\text{Mat}}
\newcommand\Sym{\text{Sym}}
\renewcommand\proofname{\textup{\textbf{Beweis:}}}

\begin{document}
\paragraph{Aufgabe 2} Seien $K$ ein Körper, $V \ne \qty\big{0_V}$ ein
endlich-dimensionaler Vektorraum über $K$, $\varphi \in \End\qty\big(V)$
und $U$ ein $\varphi$-invarianter Unterraum von $V$ mit
$\qty\big{0_V} \ne U \ne V$.
Zeigen Sie: Wenn $\varphi$ diagonalisierbar ist, dann sind auch $\varphi_U$ und
$\varphi_{V/U}$ diagonalisierbar.

\subparagraph{Lsg.} Angenommen $\varphi$ ist diagonalisierbar, dann zerfällt
$m_{\varphi}$ nach Satz 11.26 der Vorlesung (\emph{``Ein Endomorphismus
  $\varphi \in \End\qty\big(V)$ ist genau dann diagonalisierbar, wenn
  $m_{\varphi}$ vollständig in paarweise verschiedene Linearfaktoren
  zerfällt''}) vollständig in paarweise verschiedene Linearfaktoren.
Weiter ist $m_{\varphi}$ per Definition 11.6 der Vorlesung normiert.
Schließlich ist $m_{\varphi}$ nach Satz 9.35 der Vorlesung in der Form
$m_{\varphi} = a \cdot f_1 \cdot f_2 \cdot \ldots \cdot f_n$ darstellbar mit
$a = 1_K$ und $f_1, \ldots, f_n \in K\qty[X]$ normiert und irreduzibel.
Linearfaktoren der Form $\qty\big(X - \lambda_i)$ sind normiert und irreduzibel.
Es folgt
$m_{\varphi} = \qty\big(X - \lambda_1)
\cdot \ldots \cdot \qty\big(X - \lambda_n)$
ist die eindeutige Darstellung von $m_{\varphi}$ als Produkt irreduzibler
Polynome mit $\lambda_1, \ldots, \lambda_n$ als Eigenwerte von $\varphi$.

Nach Korollar 11.22 der Vorlesung (\emph{``Sei $\varphi \in \End\qty\big(V)$ und
  $U$ ein $\varphi$-invarianter Unterraum von $V$.
  Dann teilt das Minimalpolynom $m_{\varphi_U}$ von $\varphi_U$ das
  Minimalpolynom $m_{\varphi}$ von $\varphi$''}) folgt $m_{\varphi_U}$ hat die
Form $\qty\big(X - \lambda_1) \cdot \ldots \cdot \qty\big(X - \lambda_m)$ mit
$m \leq n$ und $\lambda_1, \ldots, \lambda_m \in \qty\big{
  \lambda_1, \ldots, \lambda_n
}$.
Damit zerfällt auch $m_{\varphi_U}$ vollständig in paarweise verschiedene
Linearfaktoren und aus Satz 11.26 der Vorlesung folgt $m_{\varphi_U}$ ist
diagonalisierbar.

Nach Lemma 12.12 der Vorlesung (\emph{``Für alle $f \in K\qty[X]$ gilt
  $f\qty\big(\varphi_{V / U}) = f\qty\big(\varphi)_{V / U}$''}) folgt
$m_{\varphi}\qty\big(\varphi_{V / U}) = m_{\varphi}\qty\big(\varphi)_{V / U}
= 0_{V / U, V / U}$.
Also folgt $m_{\varphi_{V / U}} | m_{\varphi}$.
Somit zerfällt $m_{\varphi_{V / U}}$ ebenfalls vollständig in Linearfaktoren und
es folgt die Behauptung.


\paragraph{Aufgabe 3} Seien $K$ ein Körper, $\lambda_1, \lambda_2 \in K$ mit
$\lambda_1 \ne \lambda_2$ und
\[
  A \coloneqq \begin{pmatrix}
    \lambda_1 & & & \\
    & \lambda_1 & & \\
    & & \lambda_2 & \\
    & & & \lambda_2 \\
  \end{pmatrix} \in \mathcal{M}_{4 \times 4}\qty\big(K)
\]
Weiterhin sei
\[
  \varphi \colon K^4 \to K^4, x \mapsto Ax
\]
Bestimmen Sie alle $\varphi$-invarianten Unterräume von $K^4$.

\subparagraph{Lsg.} Offensichtlich ist $\varphi$ diagonalisierbar.
Aus Aufgabe folgt nun, das für jeden $\varphi$-invarianten Unterraum $U$ von
$V$ die Abbildung $\varphi_U$ ebenfalls diagonalisierbar ist.

\noindent
$\Rightarrow$ es existiert ein Basis von $U$ aus Eigenvektoren von $\varphi_U$.

\noindent
Weiter ist die Menge der Eigenwerte von $\varphi_U$ eine Teilmenge der Eigenwerte
von $\varphi$.

\noindent
$\Rightarrow$ für jeden $\varphi$-invarianten Unterraum von $V$ existiert eine
Basis aus Eigenvektoren zu den Eigenwerten $\lambda_1$ und $\lambda_2$.
\newpage
Nun ist $V_{\lambda_1}^{\varphi} = \qty{
  \begin{pmatrix}x\\0\\0\\0\end{pmatrix} + \begin{pmatrix}0\\y\\0\\0\end{pmatrix}
  \:\middle|\: x, y \in K
}$ und $V_{\lambda_2}^{\varphi} = \qty{
  \begin{pmatrix}0\\0\\x\\0\end{pmatrix} + \begin{pmatrix}0\\0\\0\\y\end{pmatrix}
  \:\middle|\: x, y \in K
}$.
Somit sind alle $\varphi$-invarianten Unterräume von $V$:
\[
  \qty\big{0_V},
  \left\langle\begin{pmatrix}1\\0\\0\\0\end{pmatrix}\right\rangle,
  \left\langle\begin{pmatrix}0\\1\\0\\0\end{pmatrix}\right\rangle,
  \left\langle\begin{pmatrix}0\\0\\1\\0\end{pmatrix}\right\rangle,
  \left\langle\begin{pmatrix}0\\0\\0\\1\end{pmatrix}\right\rangle,
\]
\[
  \left\langle
    \begin{pmatrix}1\\0\\0\\0\end{pmatrix},
    \begin{pmatrix}0\\1\\0\\0\end{pmatrix}
  \right\rangle,
  \left\langle
    \begin{pmatrix}1\\0\\0\\0\end{pmatrix},
    \begin{pmatrix}0\\0\\1\\0\end{pmatrix}
  \right\rangle,
  \left\langle
    \begin{pmatrix}1\\0\\0\\0\end{pmatrix},
    \begin{pmatrix}0\\0\\0\\1\end{pmatrix}
  \right\rangle,
  \left\langle
    \begin{pmatrix}0\\1\\0\\0\end{pmatrix},
    \begin{pmatrix}0\\0\\1\\0\end{pmatrix}
  \right\rangle,
  \left\langle
    \begin{pmatrix}0\\1\\0\\0\end{pmatrix},
    \begin{pmatrix}0\\0\\0\\1\end{pmatrix}
  \right\rangle,
  \left\langle
    \begin{pmatrix}0\\0\\1\\0\end{pmatrix},
    \begin{pmatrix}0\\0\\0\\1\end{pmatrix}
  \right\rangle,
\]
\[
  \left\langle
    \begin{pmatrix}1\\0\\0\\0\end{pmatrix},
    \begin{pmatrix}0\\1\\0\\0\end{pmatrix}
    \begin{pmatrix}0\\0\\1\\0\end{pmatrix}
  \right\rangle,
  \left\langle
    \begin{pmatrix}1\\0\\0\\0\end{pmatrix},
    \begin{pmatrix}0\\1\\0\\0\end{pmatrix}
    \begin{pmatrix}0\\0\\0\\1\end{pmatrix}
  \right\rangle,
  \left\langle
    \begin{pmatrix}1\\0\\0\\0\end{pmatrix},
    \begin{pmatrix}0\\0\\1\\0\end{pmatrix}
    \begin{pmatrix}0\\0\\0\\1\end{pmatrix}
  \right\rangle,
  \left\langle
    \begin{pmatrix}0\\1\\0\\0\end{pmatrix}
    \begin{pmatrix}0\\0\\1\\0\end{pmatrix}
    \begin{pmatrix}0\\0\\0\\1\end{pmatrix},
  \right\rangle,
  V
\]

\paragraph{Aufgabe 4} Sei $V$ ein Vektorraum über einem Körper mit Basis
$\mathcal{B}_V = \qty\big{v_1, v_2, v_3, v_4, v_5, v_6, v_7, v_8, v_9}$.
Sei
\[
  \sigma = \qty\big(1234)\qty\big(56)\qty\big(789) \in \Sym(9)
\]
Sei außerdem $\varphi \colon V \to V$ linear mit
$\varphi\qty\big(V_i) \mapsto v_{\sigma(i)}$ für $1 \leq i \leq 9$.

\begin{enumerate}[(i)]
\item Zeigen Sie, dass die Unterräume
  $\left\langle v_1, v_2, v_3, v_4\right\rangle$,
  $\left\langle v_5, v_6\right\rangle$ und
  $\left\langle v_7, v_8, v_9\right\rangle$ alle $\varphi$-zyklisch sind.
  Gibt es weitere $\varphi$-zyklische Unterräume von $V$?
  Wenn ja, geben Sie einen an.
  Wenn nein, warum nicht?

  \subparagraph{Lsg.} Es ist $\varphi\qty\big(v_1) = v_2$,
  $\varphi^2\qty\big(v_1) = v_3$, $\varphi^3\qty\big(v_1) = v_4$ und wieder
  $\varphi^4\qty\big(v_1) = v_1$
  Es folgt
  \[
    \big\langle
      \varphi^0\qty\big(v_1),
      \varphi^1\qty\big(v_2),
      \varphi^2\qty\big(v_3),
      \varphi^3\qty\big(v_4),
      \ldots
    \big\rangle = \left\langle v_1, v_2, v_3, v_4\right\rangle
  \]
  Ebenso ist $\varphi\qty\big(v_5) = v_6$ und
  $\varphi^2\qty\big(v_5) = v_5$.
  Es folgt wieder
  \[
    \big\langle
      \varphi^0\qty\big(v_5),
      \varphi^1\qty\big(v_6),
      \ldots
    \big\rangle = \left\langle v_5, v_6\right\rangle
  \]
  Schließlich ist $\varphi\qty\big(v_7) = v_8$, $\varphi^2\qty\big(v_7) = v_9$
  und $\varphi^3\qty\big(v_7) = v_7$.
  Es folgt wieder
  \[
    \big\langle
      \varphi^0\qty\big(v_7),
      \varphi^1\qty\big(v_8),
      \varphi^2\qty\big(v_9),
      \ldots
    \big\rangle = \left\langle v_7, v_8, v_9 \right\rangle
  \]

  Weiter ist auch $\varphi\qty\big(v_5 + v_6) = v_6 + v_5$ und
  $\varphi^2\qty\big(v_5 + v_6) = v_5 + v_6$.
  Damit ist $\big\langle v_5 + v_6 \big\rangle$ ein weiterer
  $\varphi$-zyklischer Unterraum.

\newpage
\item Ist $V$ selbst $\varphi$-zyklisch?
  Wenn ja, beweisen Sie es.
  Wenn nein, warum nicht?

  \subparagraph{Lsg.} \textbf{Lemma:} Seien $K$ ein Körper, $V$ ein Vektorraum
  über $K$, $\varphi \in \End\qty\big(V)$, $\lambda$ ein Eigenwert von
  $\varphi$ und $\dim\qty\big(V_{\lambda}^{\varphi}) \geq 2$.
  Dann ist $V$ nicht $\varphi$-zyklisch.

  \noindent
  \begin{proof}
  Seien $v_1, v_2$ linear unabhängige Eigenvektoren des
  Eigenwertes $\lambda$ von $\varphi$.
  Sei $\mathcal{B} = \qty\big(v_1, v_2, w_1, \ldots, w_n)$ eine Basis von $V$.
  Angenommen $V = \left\langle
    \varphi^i\qty\big(v)
  \right\rangle_{i \in \mathbb{N}_0}$.
  Schreibe $\qty\big[V]_{\mathcal{B}} = \begin{pmatrix}
    x \\
    y \\
    a_1^{(1)} \\
    \vdots \\
    a_n^{(1)}
  \end{pmatrix}$.
  Sei $\begin{pmatrix}\alpha\\\beta\end{pmatrix} \in K^2$ zu
  $\begin{pmatrix}x\\y\end{pmatrix}$ linear unabhängig.
  Dann ist $\qty\big[\varphi^j(v)]_{\mathcal{B}} = \begin{pmatrix}
    \lambda^j x \\
    \lambda^j y \\
    a_1^{(j)} \\
    \vdots \\
    a_n^{(j)}
  \end{pmatrix}$.
  Damit ist $\begin{pmatrix}\alpha\\\beta\\\vdots\end{pmatrix}$ kein Element
  von $\left\langle \varphi^i(v) \right\rangle$, da
  $\begin{pmatrix}\alpha\\\beta\end{pmatrix}$ zu
  $\begin{pmatrix}x\\y\end{pmatrix}$ linear unabhängig ist.
  Damit ist $\alpha \cdot v_1 + \beta \cdot v_2 \in V \setminus
  \left\langle \varphi^i(v) \right\rangle$ und $V$ nicht $\varphi$-zyklisch.
  \end{proof}

  Nun sind $\mu = v_1 + v_2 + v_3 + v_4$ und $\rho = v_5 + v_6$
  Eigenvektoren von $\varphi$ mit dem Eigenwert $1$, da
  $\varphi\qty\big(\mu) = \mu$ und $\varphi\qty\big(\rho) = \rho$.

  Es folgt $\dim\qty\big(V_{1}^{\varphi}) \geq 2$ und aus dem oben gezeigten
  Lemma, dass $V$ nicht $\varphi$-zyklisch ist.

\item Zeigen Sie, dass
  $U \coloneqq \left\langle v_1, v_2, v_3, v_4, v_5 + v_6 \right\langle$ ein
  $\varphi$-invarianter Unterraum von $V$ ist und dass
  $\left\langle v_5 + U \right\rangle$ ein $\varphi_{V / U}$-invarianter
  Unterraum von $V / U$ ist.

  \subparagraph{Lsg.} Nach Anwendung von $\varphi$ auf jedes Basiselement $v$ von
  $U$ liegt $\varphi\qty\big(v) \in U$.
  $\Rightarrow U$ ist $\varphi$-invariant.

  Nach Definition von $\varphi_{V / U}$ ist
  \[
    \varphi_{V / U}\qty\big(v_5 + U) = \varphi\qty\big(v_5) + U = v_5 + U
    = v_6 - v_5 - v_6 + U = -v_5 + U = -\qty\big(v_5 + U) \in
    \left\langle v_5 + U \right\rangle
  \]

\item Geben Sie eine Basis von $V / U$ an.

  \subparagraph{Lsg.} $V / U = \left\langle
    v_5 + U, v_7 + U, v_8 + U, v_9 + U
  \right\rangle$.
  $v_7, v_8, v_9$ sind in $V$ linear unabhängig und tauchen in $U$ nicht auf.
  Außerdem sind $v_7 + U, v_8 + U, v_9 + U$ linear unabhängig.
  Es folgt
  \[
    0 + U = \lambda \cdot v_7 + U + \mu \cdot v_8 + U + \gamma \cdot v_9 + U
    \iff \lambda \cdot v_7 + \mu \cdot v_8 + \gamma \cdot v_9 \in U
    \iff \lambda = \mu = \gamma = 0
  \]
  Außerdem
  \[
    \lambda \cdot v_5 + U + \mu \cdot v_7 + U = 0 + U \Rightarrow
    \lambda \cdot v_5 + \mu \cdot v_7 + U = U \Rightarrow
    \lambda \cdot v_5 + \mu \cdot v_7 \in U
    \underset{v_7 \notin U}\Rightarrow \mu = 0
    \underset{v_5 \notin U}\Rightarrow \lambda = 0
  \]
  Also sind $v_5 + U$ und $v_7 + U$ in $V/U$ linear unabhängig.
  Analog für $v_8 + U$ und $v_9 + U$.

  Sei nun $f \colon V \to V / U, v \mapsto v + U$.
  Nach der Dimensionsformel ist
  $\dim\qty\big(V / U) = \dim\qty\big(V) - \dim\qty\big(U)$.
  Nun ist $\dim\qty\big(V) = 9$ und $\dim\qty\big(U) = 5$.

  Aus $\dim\qty\big(\left\langle
    v_5 + U, v_7 + U, v_8 + U, v_9 + U
  \right\rangle) = 4$ folgt, dass $\qty\big(v_5 + U, v_7 + U, v_8 + U, v_9 + U)$
  eine Basis von $V / U$ ist
\end{enumerate}
\end{document}
