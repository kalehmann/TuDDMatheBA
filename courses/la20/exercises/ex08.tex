\documentclass{scrreprt}

\usepackage{aligned-overset}
\usepackage{amsmath}
\usepackage{amssymb}
\usepackage{bm}
\usepackage[shortlabels]{enumitem}
\usepackage{hyperref}
\usepackage[utf8]{inputenc}
\usepackage{multicol}
\usepackage{mathtools}
\usepackage{physics}
\usepackage{tabularx}
\usepackage[table]{xcolor}
\usepackage{titling}
\usepackage{fancyhdr}
\usepackage{xfrac}
\usepackage{pgfplots}
\usepackage{tikz-3dplot}

\pgfplotsset{compat = newest}
\usetikzlibrary{intersections}
\usetikzlibrary{patterns}
\usepgfplotslibrary{fillbetween}

\author{Karsten Lehmann}
\date{SoSe 2022}
\title{Übungsblatt 08\\Lineare Algebra - Weiterführende Konzepte}

\setlength{\headheight}{26pt}
\pagestyle{fancy}
\fancyhf{}
\lhead{\thetitle}
\rhead{\theauthor}
\lfoot{\thedate}
\rfoot{Seite \thepage}

\newcommand\End{\text{End}}
\newcommand\id{\text{id}}
\newcommand\Mat{\text{Mat}}

\begin{document}
\paragraph{Aufgabe 2} Sei
\[
  A \coloneqq \begin{pmatrix}
    3 & 2 & -2 \\
    0 & 3 & -5 \\
    0 & 5 & -7 \\
  \end{pmatrix}
\]
\begin{enumerate}[(i)]
\item Berechnen Sie $f_A$ als Produkt von Linearfaktoren.
  Folgern Sie, dass $A$ genau die Eigenwerte $3$ und $-2$ hat.

  \subparagraph{Lsg.}
  \begin{flalign*}
    \det\qty\big(X \cdot I_3 - A) &=
    \det\begin{pmatrix}
      X - 3 & -2 & 2 \\
      0 & X - 3 & 5 \\
      0 & -5 & X + 7 \\
    \end{pmatrix} \\
    &= \qty\big(X - 3) \cdot \det\begin{pmatrix}
      X - 3 & 5 \\
      -5 & X + 7 \\
    \end{pmatrix} \\
    &= \qty\big(X - 3) \cdot \qty\Big(\qty\big(X - 3)\qty\big(X + 7) + 25) \\
    &= \qty\big(X - 3) \cdot \qty\big(X^2 + 4X + 4) \\
    &= \qty\big(X - 3) \cdot \qty\big(X + 2)^2
  \end{flalign*}
  Aus Satz 10.13 der Vorlesung (\emph{``Die Eigenwerte einer Matrix
    $A \in \mathcal{M}_{n \times n}\qty\big(K)$ sind die Nullstellen des
    zugehörigen charakteristischen Polynoms $f_A$''}) folgt
  $\lambda_1 = 3$ und $\lambda_2 = -2$.

\item Finden Sie eine Basis des Eigenraums $V_{-2}^A$.
  Nutzen Sie dies, um die geometrische Vielfachheit des Eigenwerts $-2$
  anzugeben.

  \subparagraph{Lsg.} $\qty\big(-2 \cdot I_3 - A) =
  \begin{pmatrix}
    -5 & -2 & 2 \\
    0 & -5 & 5 \\
    0 & -5 & 5 \\
  \end{pmatrix}$ und
  \[
    \begin{pmatrix}
      -5 & -2 & 2 \\
      0 & -5 & 5 \\
      0 & -5 & 5 \\
    \end{pmatrix} \leadsto \begin{pmatrix}
      1 & 0 & 0 \\
      0 & 1 & -1 \\
      0 & 0 & 0 \\
    \end{pmatrix}
  \]
  Daraus folgt $V_{-2}^A = \qty{
    \begin{pmatrix}0\\1\\1\end{pmatrix} \cdot x,
    \: \middle| \:
    x \in \mathbb{R}
  }$ und $m_{\text{geom}}\qty\big(A, -2) = 1$.

\item Geben Sie die algebraischen Vielfachheiten der Eigenwerte $3$ und $-2$ an.
  \subparagraph{Lsg.} $m_{\text{alg}}\qty\big(A, 3) = 1$ und
  $m_{\text{alg}}\qty\big(A, -2) = 2$.

\newpage
\item Geben Sie die geometrische Vielfachheit des Eigenwertes $3$ an, ohne
  $V_3^A$ zu bestimmen.
  \subparagraph{Lsg.} Nach Bemerkung 10.45 der Vorlesung gilt für jeden Eigenwert
  $\lambda$ von $A$, dass $m_{\text{geom}}\qty\big(A, \lambda) \geq 1$.
  Aus Satz 10.48 (a) der Vorlesung (\emph{``Für jeden Eigenwert $\lambda$ von
    $\varphi$ gilt $m_{\text{geom}}\qty\big(\varphi, \lambda) \leq
    m_{\text{alg}}\qty\big(\varphi, \lambda)$''}) folgt direkt
  $m_{\text{geom}}\qty\big(A, 3) = 1$.

\item Begründen Sie, ob $A$ diagonalisierbar ist.

  \subparagraph{Lsg.} Nach Satz 10.48 (b) der Vorlesung (\emph{``Falls $\varphi$
    diagonalisierbar ist, dann sind alle geometrischen Vielfachheiten gleich den
    entsprechenden algebraischen Vielfachheiten''}) und
  $2 = m_{\text{alg}}\qty\big(A, -2) \ne m_{\text{geom}}\qty\big(A, -2) = 1$
  folgt, dass $A$ nicht diagonalisierbar ist.

\item Nutzen Sie (i) und (v) um das Minimalpolynom $m_A$ zu finden.
  \subparagraph{Lsg.} Aus dem Satz 11.14 der Vorlesung (\emph{``Die Eigenwerte
    der Matrix $A$ sind genau die Nullstellen des Minimalpolynoms''}) und dem
  Satz von Cayley-Hamilton folgt $m_A = \qty\big(X - 3)\qty\big(X + 2)$ oder
  $m_A = \qty\big(X - 3)\qty\big(X + 2)^2$.
  Da $A$ nicht diagonalisierbar ist folgt mit Satz 11.26 der Vorlesung (\emph{``
    Es ist $A \in \mathcal{M}_{n \times n}\qty\big(K)$ genau dann
    diagonalisierbar, wenn $m_A$ vollständig in paarweise verschiedene
    Linearfaktoren zerfällt.''}) folgt $m_A = \qty\big(X - 3)\qty\big(X + 2)^2$.

\item Nutzen Sie (i) und den Satz von Cayley-Hamilton um zu zeigen, dass $A$
  invertierbar ist und um $A^{-1}$ als Linearkombination von Potenzen von $A$ zu
  schreiben.
  Schreiben Sie außerdem $A^3$ und $A^4$ als Linearkomibnation von $I_3$, $A$ und
  $A^3$.
  \subparagraph{Lsg.} $A$ ist invertierbar, wenn
  $A^{-1} \in \mathcal{M}_{3 \times 3}\qty\big(K)$ existiert mit
  $A \cdot A^{-1} = I_3$.

  Nun ist nach (i) und dem Satz von Cayley-Hamilton
  $\qty\big(A - 3)\qty\big(A + 2)^2 = 0$ oder $A^3 + A^2 - 8A - 12I = 0$
  oder $A \cdot \frac{1}{12}\qty\big(A^2 + A - 8) = I$.

  $\Rightarrow A^{-1} = \frac{1}{12}\qty\big(A^2 + A - 8)$.

  Ebenso lässt sich das charakteristische Polynom nach $A^3$ umstellen:
  $A^3 = -A^2 + 8A + 12I$.

  Schließlich ist
  \begin{flalign*}
    A^4 &= A\qty\big(A^3) \\
    &= A \cdot \qty\big(-A^2 + 8A + 12I) \\
    &= -A^3 + 8A^2 + 12A \\
    \overset{\text{Subst. von $A^3$}}&= -\qty\big(-A^2 + 8A + 12I)
     + 8A^2 + 12A \\
    &= 9A^2 + 4A - 12I
  \end{flalign*}
\end{enumerate}
\end{document}
