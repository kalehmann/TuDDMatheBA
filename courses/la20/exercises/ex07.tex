\documentclass{scrreprt}

\usepackage{aligned-overset}
\usepackage{amsmath}
\usepackage{amssymb}
\usepackage{bm}
\usepackage[shortlabels]{enumitem}
\usepackage{hyperref}
\usepackage[utf8]{inputenc}
\usepackage{multicol}
\usepackage{mathtools}
\usepackage{physics}
\usepackage{tabularx}
\usepackage[table]{xcolor}
\usepackage{titling}
\usepackage{fancyhdr}
\usepackage{xfrac}
\usepackage{pgfplots}
\usepackage{tikz-3dplot}

\pgfplotsset{compat = newest}
\usetikzlibrary{intersections}
\usetikzlibrary{patterns}
\usepgfplotslibrary{fillbetween}

\author{Karsten Lehmann}
\date{SoSe 2022}
\title{Übungsblatt 07\\Lineare Algebra - Weiterführende Konzepte}

\setlength{\headheight}{26pt}
\pagestyle{fancy}
\fancyhf{}
\lhead{\thetitle}
\rhead{\theauthor}
\lfoot{\thedate}
\rfoot{Seite \thepage}

\newcommand\End{\text{End}}
\newcommand\id{\text{id}}
\newcommand\Mat{\text{Mat}}

\begin{document}
\paragraph{Aufgabe 2} Bestimmen Sie das Minimalpolynom der folgenden Matrizen.
\begin{enumerate}[(a)]
\item $A \coloneqq \begin{pmatrix}
    5 & 4 \\
    -3 & -2 \\
  \end{pmatrix} \in \mathcal{M}_{2 \times 2}\qty\big(\mathbb{R}),
  f_A = \qty\big(X - 1)\qty\big(X - 2)$.

  \subparagraph{Lsg.} Nach Beispiel 11.16 der Vorlesung (\emph{``
    Wenn $f_A$ komplett in paarweise verschiedene Linearfaktoren zerfällt, so
    gilt $m_A = f_A$
  ''}) folgt $m_A = \qty\big(X - 1)\qty\big(X - 2)$.

\item $B \coloneqq \begin{pmatrix}
    0 & 1 \\
    -1 & -1 \\
  \end{pmatrix} \in \mathcal{M}_{2 \times 2}\qty\big(\mathbb{F}_3),
  f_B = \qty\big(X - 1)^2$, $B$ ist nicht diagonalisierbar.

  \subparagraph{Lsg.} Es gilt
  \[
    \qty\big(B - 1 \cdot I_2)^2 =
    \qty(\begin{pmatrix}
      0 & 1 \\
      -1 & -1 \\
    \end{pmatrix} - \begin{pmatrix}
      1 & 0 \\
      0 & 1 \\
    \end{pmatrix})^2
    = \begin{pmatrix}
      -1 & 1 \\
      -1 & -2 \\
    \end{pmatrix}^2
    = 0
  \]
  Daraus folgt, dass $m_B$ ein Teiler von $\qty\big(X - 1)^2$ ist.
  Da $\qty\big(B - 1 \cdot I_2) = \begin{pmatrix}
    -1 & 1 \\
    -1 & -2 \\
  \end{pmatrix}$ folgt $m_B = \qty\big(X - 1)^2$.

\item $C \coloneqq \begin{pmatrix}
    0 & 1 \\
    -1 & -1 \\
  \end{pmatrix} \in \mathcal{M}_{2 \times 2}\qty\big(\mathbb{F}_5),
  f_C = X^2 + X + 1$, $f_C$ hat keine Nullstellen in $\mathbb{F}_5$.

  \subparagraph{Lsg.} Nach dem Satz von Cayley-Hamilton teilt
  $m_C$ das Polynom $f_C$.
  Es folgt $m_C = f_C$.
\end{enumerate}

\paragraph{Aufgabe 3} Sei $V$ ein endlich-dimensionaler Vektorraum über
$\mathbb{C}$ und $\varphi \in \End\qty\big(V)$.
Entscheiden Sie jeweils, ob $\varphi$ diagonalisierbar ist.

\begin{enumerate}[(i)]
\item $f_{\varphi} = \qty\big(X - 1)\qty\big(X + 2i)\qty\big(X - 3 - 5i)$

  \subparagraph{Lsg.} Nach Satz 10.40 (a) der Vorlesung (\emph{``Sei $\varphi$
    ein Endomorphismus von $V$.
    Wenn das charakteristische Polynom von $A$ komplett in verschiedene
    Linearfaktoren zerfällt, so ist $A$ diagonalisierbar.''}) ist
  $\varphi$ diagonalisierbar.

\item $m_{\varphi} = \qty\big(X - 3i)\qty\big(X + 2 + 4i)\qty\big(X + 1)^2$.

  \subparagraph{Lsg.} Nach Satz 11.26 der Vorlesung (\emph{``Ein Endomorphismus
    $\varphi \in \End\qty\big(V)$ ist genau dann diagonalisierbar, wenn
    $m_{\varphi}$ vollständig in paarweise verschiedene Linearfaktoren
    zerfällt''}) ist $\varphi$ nicht diagonalisierbar.

\item $f_{\varphi} = \qty\big(X - 1)\qty\big(X + 3i)$

  \subparagraph{Lsg.} Nach Satz 10.40 (a) der Vorlesung ist $\varphi$
  diagonalisierbar.
\end{enumerate}
\newpage
\paragraph{Aufgabe 4} Seien $K$ ein Körper, $V$ ein endlich-dimensionaler
Vektorraum über $K$ und $\varphi \in \End\qty\big(V)$ mit
\[
  2 \varphi^3 - 3 \varphi^2 - 3 \varphi = -2 \id_V
\]
\begin{enumerate}[(i)]
\item Bestimmen Sie die Eigenwerte von $\varphi$ für $K = \mathbb{R}$
  und $K = \mathbb{Z} / 3$.

  \subparagraph{Lsg.} Sei $K = \mathbb{R}$.
  Dann ist $2 \varphi^3 - 3 \varphi^2 - 3 \varphi + 2 \id_V = 0$.
  Sei nun
  \[
    f \in K\qty[X], f = X^3 - \frac{3}{2}X^2 - \frac{3}{2}X + 1 =
    \qty\big(X - 2)\qty\big(X + 1)\qty(X - \frac{1}{2})
  \]
  Dann ist $f\qty\big(\varphi) = 0$ und $m_{\varphi} = f_{\varphi} = f$.
  Die Eigenwerte von $\varphi$ sind $\lambda_1 = 2, \lambda_2 = -1$,
  $\lambda_3 = \frac{1}{2}$.

  Sei $K = \mathbb{Z}/3$.
  Dann ist $2 \varphi^3 + 2 \id_V = 0$.
  Sei nun
  \[
    f \in K\qty[X], f = X^3 + 1 =
    \qty\big(X + 1)^3
  \]
  Die Eigenwerte von $\varphi$ sind $\lambda_{1|2|3} = -1$.

\item Beweisen oder widerlegen Sie für $K = \mathbb{R}$ und
  $K = \mathbb{Z}/3$, dass $\varphi$ diagonalisierbar ist.

  \subparagraph{Lsg.} Sei $K = \mathbb{R}$.
  Dann zerfällt das charakteristische Polynom von $\varphi$ komplett in
  verschiedene Linearfaktoren.
  Nach Satz 10.40 der Vorlesung ist $\varphi$ diagonalisierbar.

  Sei nun $K = \mathbb{Z}/3$.
  Dann lässt sich keine Aussage über die Diagonalisierbarkeit von $\varphi$
  treffen.
  Sei zum Beispiel $\mathcal{B}$ die Standardbasis in $\qty\big(\mathbb{Z}/3)^3$
  und
  \[
    \Mat\qty\big(\varphi, \mathcal{B}) = \begin{pmatrix}
      -1 & 0 & 0 \\
      0 & -1 & 0 \\
      0 & 0 & -1 \\
    \end{pmatrix}
  \]
  Dann ist $\varphi$ diagonalisierbar und das charakteristische Polynom
  $f_{\varphi} = \qty\big(X + 1)^3$.
  Wäre hingegen
  \[
    \Mat\qty\big(\varphi, \mathcal{B}) = \begin{pmatrix}
      -1 & 1 & 0 \\
      0 & -1 & 1 \\
      0 & 0 & -1 \\
    \end{pmatrix}
  \]
  so ist das charakteristisceh Polynom ebenfalls
  $f_{\varphi} = \qty\big(X + 1)^3$, allerdings ist auch
  $V_{\lambda_{1|2|3}} = \qty{
    \begin{pmatrix}1\\0\\0\end{pmatrix} \cdot x
    \:\middle|\:
    x \in \mathbb{Z}/3
  }$ und $\varphi$ nicht diagonalisierbar.

\end{enumerate}
\end{document}
