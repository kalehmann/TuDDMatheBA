\documentclass{scrreprt}

\usepackage{aligned-overset}
\usepackage{amsmath}
\usepackage{amssymb}
\usepackage{bm}
\usepackage[shortlabels]{enumitem}
\usepackage{hyperref}
\usepackage[utf8]{inputenc}
\usepackage{multicol}
\usepackage{mathtools}
\usepackage{physics}
\usepackage{tabularx}
\usepackage[table]{xcolor}
\usepackage{titling}
\usepackage{fancyhdr}
\usepackage{xfrac}
\usepackage{pgfplots}
\usepackage{tikz-3dplot}

\pgfplotsset{compat = newest}
\usetikzlibrary{intersections}
\usetikzlibrary{patterns}
\usepgfplotslibrary{fillbetween}

\author{Karsten Lehmann}
\date{SoSe 2022}
\title{Übungsblatt 06\\Lineare Algebra - Weiterführende Konzepte}

\setlength{\headheight}{26pt}
\pagestyle{fancy}
\fancyhf{}
\lhead{\thetitle}
\rhead{\theauthor}
\lfoot{\thedate}
\rfoot{Seite \thepage}

\newcommand\Grad{\text{Grad}}
\newcommand\Kern{\text{Kern}}

\begin{document}
\paragraph{Aufgabe 2} Bestimmen Sie das Minimalpolynom der folgenden Matrizen.
\begin{enumerate}[(a)]
\item $A \coloneqq \begin{pmatrix}
    5 & 4 \\
    -3 & -2 \\
  \end{pmatrix} \in \mathcal{M}_{2 \times 2}\qty\big(\mathbb{R}),
  f_A = \qty\big(X - 1)\qty\big(X - 2)$.

  \subparagraph{Lsg.} Nach Beispiel 11.16 der Vorlesung (\emph{``
    Wenn $f_A$ komplett in paarweise verschiedene Linearfaktoren zerfällt, so
    gilt $m_A = f_A$
  ''}) folgt $m_A = \qty\big(X - 1)\qty\big(X - 2)$.

\item $B \coloneqq \begin{pmatrix}
    0 & 1 \\
    -1 & -1 \\
  \end{pmatrix} \in \mathcal{M}_{2 \times 2}\qty\big(\mathbb{F}_3),
  f_B = \qty\big(X - 1)^2$, $B$ ist nicht diagonalisierbar.

  \subparagraph{Lsg.} Es gilt
  \[
    \qty\big(B - 1 \cdot I_2)^2 =
    \qty(\begin{pmatrix}
      0 & 1 \\
      -1 & -1 \\
    \end{pmatrix} - \begin{pmatrix}
      1 & 0 \\
      0 & 1 \\
    \end{pmatrix})^2
    = \begin{pmatrix}
      -1 & 1 \\
      -1 & -2 \\
    \end{pmatrix}^2
    = 0
  \]
  Daraus folgt, dass $m_B$ ein Teiler von $\qty\big(X - 1)^2$ ist.
  Da $\qty\big(B - 1 \cdot I_2) = \begin{pmatrix}
    -1 & 1 \\
    -1 & -2 \\
  \end{pmatrix}$ folgt $m_B = \qty\big(X - 1)^2$.

\item $C \coloneqq \begin{pmatrix}
    0 & 1 \\
    -1 & -1 \\
  \end{pmatrix} \in \mathcal{M}_{2 \times 2}\qty\big(\mathbb{F}_5),
  f_C = X^2 + X + 1$, $f_C$ hat keine Nullstellen in $\mathbb{F}_5$.

  \subparagraph{Lsg.} Nach dem Satz von Cayley-Hamilton teilt
  $m_C$ das Polynom $f_C$.
  Es folgt $m_C = f_C$.
\end{enumerate}
\end{document}
