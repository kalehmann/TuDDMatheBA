\documentclass{scrreprt}

\usepackage{aligned-overset}
\usepackage{amsmath}
\usepackage{amsthm}
\usepackage{amssymb}
\usepackage{bm}
\usepackage[shortlabels]{enumitem}
\usepackage{hyperref}
\usepackage[utf8]{inputenc}
\usepackage{multicol}
\usepackage{mathtools}
\usepackage{physics}
\usepackage{tabularx}
\usepackage[table]{xcolor}
\usepackage{titling}
\usepackage{fancyhdr}
\usepackage{xfrac}
\usepackage{pgfplots}
\usepackage{tikz-3dplot}

\pgfplotsset{compat = newest}
\usetikzlibrary{intersections}
\usetikzlibrary{patterns}
\usepgfplotslibrary{fillbetween}

\author{Karsten Lehmann}
\date{SoSe 2022}
\title{Übungsblatt 10\\Lineare Algebra - Weiterführende Konzepte}

\setlength{\headheight}{26pt}
\pagestyle{fancy}
\fancyhf{}
\lhead{\thetitle}
\rhead{\theauthor}
\lfoot{\thedate}
\rfoot{Seite \thepage}

\newcommand\Bild{\text{Bild}}
\newcommand\End{\text{End}}
\newcommand\id{\text{id}}
\newcommand\Kern{\text{Kern}}
\newcommand\Mat{\text{Mat}}
\newcommand\Sym{\text{Sym}}
\renewcommand\proofname{\textup{\textbf{Beweis:}}}

\begin{document}
\paragraph{Aufgabe 2} Seien $n \geq 2$ eine natürliche Zahl, $K$ ein Körper,
$V$ ein $n$-dimensionaler Vektorraum über $K$ und
$\qty\big(v_1, v_2, \ldots, v_n)$ eine Basis von $V$.
Seien weiterhin $\sigma \in \Sym\qty\big(n)$ und
$\varphi \colon V \to V$ linear mit
\[
  \varphi\qty\big(v_i) = v_{\sigma(i)}
\]
für alle $1 \geq i \geq n$.
\begin{enumerate}[(i)]
\item Beweisen Sie, dass $V$ genau dann $\varphi$-zyklisch ist, wenn
  $\sigma$ ein $n$-Zyklus ist.
  \begin{itemize}
  \item[``$\Rightarrow$''] Angenommen $\sigma$ ist kein $n$-Zyklus, dann ist
    $\sigma = \id$ oder $\sigma$ ein Produkt paarweise elementfremder Zyklen ist.
    Das $V$ für $\sigma = \id$ nicht $\varphi$-zyklisch ist, ist offensichtlich.
    Sei nun $\rho, \mu$ zwei disjunkte Faktoren von $\sigma$ der Form
    $\rho = \qty\big(a_1\ldots a_m), \mu = \qty\big(b_1\ldots b_l) l + m \leq n$.
    Dann ist
    $\varphi\qty\big(v_{a_1} + \ldots + v_{a_m}) = v_{a_1} + \ldots + v_{a_m}$ und
    $\varphi\qty\big(v_{b_1} + \ldots + v_{b_l}) = v_{b_1} + \ldots + v_{b_l}$.
    Nach Lemma 13.5 (c) der Vorlesung (\emph{``Sei $V$ $\varphi$-zyklisch und
      $\lambda$ ein Eigenwert von $\varphi$.
      Dann hat der Eigenraum $V_{\lambda}^{\varphi}$ die Dimension 1''}) wäre
    $\dim\qty\big(V_{1}^{\varphi}) = 1$, allerdings ist $\left\langle
      v_{a_1} + \ldots + v_{a_m}, v_{b_1} + \ldots + v_{b_l}
    \right\rangle \subseteq V_{1}^{\varphi}$.
    Ein Widerspruch.
  \item[``$\Leftarrow$''] Wenn $\sigma$ ein $n$-Zyklus ist (nach Notation 8.1
    der Vorlesung $\sigma = \qty\big(a_1\ldots a_n)$, mit
    $\sigma\qty\big(a_i) = a_{i + 1}$ für $i \in 1, \ldots, n - 1$ und
    $\sigma\qty\big(a_n) = a_1$), dann ist
    $\qty\big{\varphi^i\qty\big(v_{a_1}) \:\big|\: i \in \mathbb{N}_0} =
    \qty\big{v_{a_1}, \ldots, v_{a_n}}$.
    Es folgt $\left\langle
      \varphi^i\qty\big(v_1) \:\big|\: i \in \mathbb{N}_0
    \right\rangle = V$.
  \end{itemize}

\item Nutzen Sie (i), um Aufgabe 4 (ii) von Übungsblatt erneut zu beantworten.

  \subparagraph{Lsg.} Aufgabe 4 (ii) von Übungsblatt 9 war \emph{``Sei $V$ ein
    Vektorraum über einem Körper $K$ mit Basis $\mathcal{B}_V = \qty\big(
      v_1, v_2, v_3, v_4, v_5, v_6, v_7, v_8, v_9
    )$ und $\sigma = \qty\big(1\:2\:3\:4)\qty\big(5\:6)\qty\big(7\:8\:9)$.
    Sei außerdem $\varphi \colon V \to V, \varphi\qty\big(v_i) = v_{\sigma(i)}$.
    Ist $V$ selbst $\varphi$-zyklisch?''}.
  Angenommen $V$ wäre nun $\varphi$-zyklisch, dann wäre $\sigma$ nach (i) ein
  $n$-Zyklus.
  Nun ist $\sigma$ offensichtlich kein $n$-Zyklus, ein Widerspruch.

  $\Rightarrow V$ ist nicht $\varphi$-zyklisch.
\end{enumerate}
\end{document}
