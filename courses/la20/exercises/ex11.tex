\documentclass{scrreprt}

\usepackage{aligned-overset}
\usepackage{amsmath}
\usepackage{amsthm}
\usepackage{amssymb}
\usepackage{bm}
\usepackage[shortlabels]{enumitem}
\usepackage{hyperref}
\usepackage[utf8]{inputenc}
\usepackage{multicol}
\usepackage{mathtools}
\usepackage{physics}
\usepackage{tabularx}
\usepackage[table]{xcolor}
\usepackage{titling}
\usepackage{fancyhdr}
\usepackage{xfrac}
\usepackage{pgfplots}
\usepackage{tikz-3dplot}

\pgfplotsset{compat = newest}
\usetikzlibrary{intersections}
\usetikzlibrary{patterns}
\usepgfplotslibrary{fillbetween}

\author{Karsten Lehmann}
\date{SoSe 2022}
\title{Übungsblatt 11\\Lineare Algebra - Weiterführende Konzepte}

\setlength{\headheight}{26pt}
\pagestyle{fancy}
\fancyhf{}
\lhead{\thetitle}
\rhead{\theauthor}
\lfoot{\thedate}
\rfoot{Seite \thepage}

\newcommand\Bild{\text{Bild}}
\newcommand\End{\text{End}}
\newcommand\id{\text{id}}
\newcommand\Kern{\text{Kern}}
\newcommand\Mat{\text{Mat}}
\newcommand\Rang{\text{Rang}}
\newcommand\Sym{\text{Sym}}
\renewcommand\proofname{\textup{\textbf{Beweis:}}}

\begin{document}
\paragraph{Aufgabe 2} Sei $V$ ein 5-dimensionaler $\mathbb{R}$-Vektorraum und
$\varphi \in \End(V)$ mit $f_{\varphi} = \qty\big(X^2 - X + 1)^2\qty\big(X - 2)$.
Bestimmen Sie alle möglichen allgemeinen Normalformen von $\varphi$.

\subparagraph{Lsg.} Mit den gegebenen Informationen lässt sich das Minimalpolynom
$m_{\varphi}$ nicht eindeutig bestimmen.
Nach Bemerkung 11.17 der Vorlesung kommt in einem Vektorraum über $\mathbb{R}$
jeder irreduzible Faktor aus dem charakteristischen Polynom auch in dem
Minimalpolynom vor.
Somit ist $m_{\varphi} = \qty\big(X^2 - X + 1)^2\qty\big(X - 2)$ oder
$m_{\varphi} = \qty\big(X^2 - X + 1)\qty\big(X - 2)$.
\begin{itemize}
\item ``$\qty\big(X^2 - X + 1)^2\qty\big(X - 2)$'' Nach Bemerkung 13.18 der
  Vorlesung gibt es in einer Normalform immer ein Kästchen der Größe 4 und der
  Form $A\qty\big((X^2 - X + 1)^2)$ und ein Kästchen der Größe 1 und Form
  $A\qty\big(X - 2)$.

  Ein Normalform sieht sieht somit bis auf die Vertauschung der Kästchen wie
  folgt aus:
  \[
    N \coloneqq \begin{pmatrix}
      0 & 0 & 0 & -1 & 0 \\
      1 & 0 & 0 &  2 & 0 \\
      0 & 1 & 0 & -3 & 0 \\
      0 & 0 & 1 &  2 & 0 \\
      0 & 0 & 0 &  0 & 2 \\
    \end{pmatrix}
  \]

\item ``$\qty\big(X^2 - X + 1)\qty\big(X - 2)$'' In einer Normalform kommt
  immer (mindestens) ein Block der Größe 2 und der Form
  \[
    A\qty\big(X^2 - X + 1) = \begin{pmatrix}
      0 & -1 \\
      1 &  1 \\
    \end{pmatrix}
  \]
  und (mindestens) ein Block der Größe 1 und der Form
  \[
    A\qty\big(X - 2) = \begin{pmatrix}2\end{pmatrix}
  \]

  Mögliche bis auf die Vertauschung der Blöcke eindeutige Normalformen sind somit
  \[
    N_1 \coloneqq \begin{pmatrix}
      0 & -1 &  0 &  0 & 0 \\
      1 & 1  &  0 &  0 & 0 \\
      0 & 0  &  0 & -1 & 0 \\
      0 & 0  &  1 &  1 & 0 \\
      0 & 0  &  0 &  0 & 2 \\
    \end{pmatrix}, N_2 \coloneqq \begin{pmatrix}
      0 & -1 & 0 & 0 & 0 \\
      1 &  1 & 0 & 0 & 0 \\
      0 &  0 & 2 & 0 & 0 \\
      0 &  0 & 0 & 2 & 0 \\
      0 &  0 & 0 & 0 & 2 \\
    \end{pmatrix}
  \]
  Wobei das charakteristische Polynom
  $f_{N_2} = \qty\big(X^2 - X + 1)\qty\big(X - 2)^3 \ne f_{\varphi}$.

  Somit folgt $N_1$ ist die bis auf die Vertauschung der Blöcke eindeutige
  Normalform von $\varphi$
  (für $m_{\varphi} = \qty\big(X^2 - X + 1)\qty\big(X - 2)$)
\end{itemize}

\newpage
\paragraph{Aufgabe 3} Sei $\qty\big{0_V} \ne V$ ein endlich-dimensionaler
Vektorraum über $\mathbb{R}$, sei $\mathcal{B}'$ eine Basis von $V$ und
$\varphi \in \End\qty\big(V)$.
Bestimmen Sie für die folgenden Fälle jeweils eine Jordansche Normalform von
$\varphi$.
\begin{enumerate}[(i)]
\item $\Mat\qty\big(\varphi, \mathcal{B}') = \begin{pmatrix}
    3 & 2 & -2 \\
    0 & 3 & -5 \\
    0 & 5 & -7 \\
  \end{pmatrix}$

  \subparagraph{Lsg.} Nach Aufgabe 2 auf Blatt 8 sind $\lambda_1 = 3$ und
  $\lambda_2 = -2$ die Eigenwerte von $\varphi$, außerdem ist
  $f_{\varphi} = m_{\varphi} = \qty\big(X - 3)\qty\big(X + 2)^2$ und
  $V_{-2}^A = \left\langle \begin{pmatrix}0\\1\\1\end{pmatrix} \right\rangle$.
  Es folgt $\dim\qty\big(V_{-2}^A) = 1$.
  Nach Satz 13.48 der Vorlesung ist die Zahl der Jordanblöcke für jeden
  Eigenwert von $\varphi$ gleich der Dimension des zugehörigen Eigenraums.
  Dementsprechend kommt in einer Jordannormalform ein Block $J\qty\big(3, 1)$
  und ein Block $J\qty\big(-2, 2)$ vor.
  Somit ist die Jordannormalform (bis auf Vertauschung der Blöcke)
  \[
    \begin{pmatrix}
      3 &  0 &  0 \\
      0 & -2 &  0 \\
      0 &  1 & -2 \\
    \end{pmatrix}
  \]

\item $\Mat\qty\big(\varphi, \mathcal{B}') = \begin{pmatrix}
     0 & 1 & 0 & 1 \\
     0 & 1 & 0 & 0 \\
    -1 & 1 & 1 & 1 \\
    -1 & 1 & 0 & 2 \\
  \end{pmatrix}$

  \subparagraph{Lsg.} Es ist $f_{\varphi} = \qty\big(X - 1)^4$.
  Durch Caley-Hamilton und ausprobieren erhält man
  $m_{\varphi} = \qty\big(X - 1)^2$.
  Nach Lemma 13.48 der Vorlesung entspricht die Zahl der Jordanblöcke für jeden
  Eigenwert von $\varphi$ der Dimension des zugehörigen Eigenraums.
  Außerdem ist für einen Eigenwert $\lambda$ von $\varphi$ die Dimension
  $\dim\qty\big(V_{\lambda}^{\varphi}) =
  n - \Rang\qty\big(\Mat(\varphi, \mathcal{B}') - \lambda \cdot I_n)$.
  Somit existieren für den Eigenwert $1$ genau
  \[
    4 - \Rang\begin{pmatrix}
      -1 & 1 & 0 & 1 \\
       0 & 0 & 0 & 0 \\
      -1 & 1 & 0 & 1 \\
      -1 & 1 & 0 & 1 \\
    \end{pmatrix} = 3
  \]
  Jordanblöcke.
  Somit ist die Jordannormalform (bis auf Vertauschung der Blöcke)
  \[
    \begin{pmatrix}
      1 & 0 & 0 & 0 \\
      1 & 1 & 0 & 0 \\
      0 & 0 & 1 & 0 \\
      0 & 0 & 0 & 1 \\
    \end{pmatrix}
  \]

\newpage
\item $f_{\varphi} = m_{\varphi} = \qty\big(X - 4)^2\qty\big(X - 5)^3$

  \subparagraph{Lsg.} Nach Bemerkung 13.42 der Vorlesung besitzt eine
  Jordan-Normalform ein Kästchen $J\qty\big(4, 2)$ und ein Kästchen
  $J\qty\big(5, 3)$.
  Somit ist die Jordannormalform (bis auf Vertauschung der Blöcke)
  \[
    \begin{pmatrix}
      4 & 0 & 0 & 0 & 0 \\
      1 & 4 & 0 & 0 & 0 \\
      0 & 0 & 5 & 0 & 0 \\
      0 & 0 & 1 & 5 & 0 \\
      0 & 0 & 0 & 1 & 5 \\
    \end{pmatrix}
  \]
\end{enumerate}
\end{document}
