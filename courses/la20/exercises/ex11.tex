\documentclass{scrreprt}

\usepackage{aligned-overset}
\usepackage{amsmath}
\usepackage{amsthm}
\usepackage{amssymb}
\usepackage{bm}
\usepackage[shortlabels]{enumitem}
\usepackage{hyperref}
\usepackage[utf8]{inputenc}
\usepackage{multicol}
\usepackage{mathtools}
\usepackage{physics}
\usepackage{tabularx}
\usepackage[table]{xcolor}
\usepackage{titling}
\usepackage{fancyhdr}
\usepackage{xfrac}
\usepackage{pgfplots}
\usepackage{tikz-3dplot}

\pgfplotsset{compat = newest}
\usetikzlibrary{intersections}
\usetikzlibrary{patterns}
\usepgfplotslibrary{fillbetween}

\author{Karsten Lehmann}
\date{SoSe 2022}
\title{Übungsblatt 11\\Lineare Algebra - Weiterführende Konzepte}

\setlength{\headheight}{26pt}
\pagestyle{fancy}
\fancyhf{}
\lhead{\thetitle}
\rhead{\theauthor}
\lfoot{\thedate}
\rfoot{Seite \thepage}

\newcommand\Bild{\text{Bild}}
\newcommand\End{\text{End}}
\newcommand\id{\text{id}}
\newcommand\Kern{\text{Kern}}
\newcommand\Mat{\text{Mat}}
\newcommand\Sym{\text{Sym}}
\renewcommand\proofname{\textup{\textbf{Beweis:}}}

\begin{document}
\paragraph{Aufgabe 2} Sei $V$ ein 5-dimensionaler $\mathbb{R}$-Vektorraum und
$\varphi \in \End(V)$ mit $f_{\varphi} = \qty\big(X^2 - X + 1)^2\qty\big(X - 2)$.
Bestimmen Sie alle möglichen allgemeinen Normalformen von $\varphi$.

\subparagraph{Lsg.} Mit den gegebenen Informationen lässt sich das Minimalpolynom
$m_{\varphi}$ nicht eindeutig bestimmen.
Nach Bemerkung 11.17 der Vorlesung kommt in einem Vektorraum über $\mathbb{R}$
jeder irreduzible Faktor aus dem charakteristischen Polynom auch in dem
Minimalpolynom vor.
Somit ist $m_{\varphi} = \qty\big(X^2 - X + 1)^2\qty\big(X - 2)$ oder
$m_{\varphi} = \qty\big(X^2 - X + 1)\qty\big(X - 2)$.
\begin{itemize}
\item ``$\qty\big(X^2 - X + 1)^2\qty\big(X - 2)$'' Nach Bemerkung 13.18 der
  Vorlesung gibt es in einer Normalform immer ein Kästchen der Größe 4 und der
  Form $A\qty\big((X^2 - X + 1)^2)$ und ein Kästchen der Größe 1 und Form
  $A\qty\big(X - 2)$.

  Ein Normalform sieht sieht somit bis auf die Vertauschung der Kästchen wie
  folgt aus:
  \[
    N \coloneqq \begin{pmatrix}
      0 & 0 & 0 & -1 & 0 \\
      1 & 0 & 0 &  2 & 0 \\
      0 & 1 & 0 & -3 & 0 \\
      0 & 0 & 1 &  2 & 0 \\
      0 & 0 & 0 &  0 & 2 \\
    \end{pmatrix}
  \]

\item ``$\qty\big(X^2 - X + 1)\qty\big(X - 2)$'' In einer Normalform kommt
  immer (mindestens) ein Block der Größe 2 und der Form
  \[
    A\qty\big(X^2 - X + 1) = \begin{pmatrix}
      0 & -1 \\
      1 &  1 \\
    \end{pmatrix}
  \]
  und (mindestens) ein Block der Größe 1 und der Form
  \[
    A\qty\big(X - 2) = \begin{pmatrix}2\end{pmatrix}
  \]

  Mögliche bis auf die Vertauschung der Blöcke eindeutige Normalformen sind somit
  \[
    N_1 \coloneqq \begin{pmatrix}
      0 & -1 &  0 &  0 & 0 \\
      1 & 1  &  0 &  0 & 0 \\
      0 & 0  &  0 & -1 & 0 \\
      0 & 0  &  1 &  1 & 0 \\
      0 & 0  &  0 &  0 & 2 \\
    \end{pmatrix}, N_2 \coloneqq \begin{pmatrix}
      0 & -1 & 0 & 0 & 0 \\
      1 &  1 & 0 & 0 & 0 \\
      0 &  0 & 2 & 0 & 0 \\
      0 &  0 & 0 & 2 & 0 \\
      0 &  0 & 0 & 0 & 2 \\
    \end{pmatrix}
  \]
  Wobei das charakteristische Polynom
  $f_{N_2} = \qty\big(X^2 - X + 1)\qty\big(X - 2)^3 \ne f_{\varphi}$.

  Somit folgt $N_1$ ist die bis auf die Vertauschung der Blöcke eindeutige
  Normalform von $\varphi$
  (für $m_{\varphi} = \qty\big(X^2 - X + 1)\qty\big(X - 2)$)
\end{itemize}
\end{document}
