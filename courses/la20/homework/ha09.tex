\documentclass{scrreprt}

\usepackage{aligned-overset}
\usepackage{amsmath}
\usepackage{amssymb}
\usepackage{bm}
\usepackage[shortlabels]{enumitem}
\usepackage{hyperref}
\usepackage[utf8]{inputenc}
\usepackage{multicol}
\usepackage{mathtools}
\usepackage{physics}
\usepackage{tabularx}
\usepackage{titling}
\usepackage{fancyhdr}
\usepackage{xfrac}
\usepackage[dvipsnames]{xcolor}
\usepackage{pgfplots}

\pgfplotsset{compat = newest}
\usetikzlibrary{intersections}
\usetikzlibrary{patterns}
\usepgfplotslibrary{fillbetween}

\newcommand{\End}{\text{End}}
\newcommand{\Grad}{\text{Grad}}
\newcommand{\Mat}{\text{Mat}}
\newcommand{\Spur}{\text{Spur}}

\author{Karsten Lehmann\\Mat. Nr 4935758}
\date{SoSe 2022}
\title{Hausaufgaben Blatt 09\\Lineare Algebra - Weiterführende Konzepte}

\setlength{\headheight}{26pt}
\pagestyle{fancy}
\fancyhf{}
\lhead{\thetitle}
\rhead{\theauthor}
\lfoot{\thedate}
\rfoot{Seite \thepage}

\begin{document}
\paragraph{Aufgabe 6} Sei $\mathbb{F}_2 = \mathbb{Z} / 2$.
Seien außerdem
\[
  A \coloneqq \begin{pmatrix}
    0 & 1 \\
    1 & 0 \\
  \end{pmatrix} \in \mathcal{M}_{2 \times 2}\qty\big(\mathbb{F}_2)
  \text{ und }
  B \coloneqq \begin{pmatrix}
    1 & 1 \\
    0 & 1 \\
  \end{pmatrix} \in \mathcal{M}_{2 \times 2}\qty\big(\mathbb{F}_2)
\]
Weiterhin seien $V \coloneqq \qty\big(\mathbb{F}_2)^2$ und
$\varphi \colon V \to V, x \mapsto Ax \text{ und }
  \psi \colon V \to V, x \mapsto Bx$.
\begin{enumerate}[(i)]
\item Bestimmen Sie alle $\varphi$-zyklischen Unterräume von $V$.
  Ist $V$ selbst $\varphi$-zyklisch?

  \subparagraph{Lsg.} Der Vektorraum $V$ umfasst die 4 Elemente
  $
    V \coloneqq \qty{
      \begin{pmatrix}0\\0\end{pmatrix},
      \begin{pmatrix}1\\0\end{pmatrix},
      \begin{pmatrix}0\\1\end{pmatrix},
      \begin{pmatrix}1\\1\end{pmatrix}
    }
  $
  und die Unterräume
  \[
    U_1 = \left\langle \begin{pmatrix}0\\0\end{pmatrix} \right\rangle =
    \qty{\begin{pmatrix}0\\0\end{pmatrix}},
    U_2 = \left\langle \begin{pmatrix}1\\0\end{pmatrix} \right\rangle =
    \qty{\begin{pmatrix}0\\0\end{pmatrix}, \begin{pmatrix}1\\0\end{pmatrix}},
    U_3 = \left\langle \begin{pmatrix}0\\1\end{pmatrix} \right\rangle =
    \qty{\begin{pmatrix}0\\0\end{pmatrix}, \begin{pmatrix}0\\1\end{pmatrix}},
  \]
  \[
    U_4 = \left\langle \begin{pmatrix}1\\1\end{pmatrix} \right\rangle =
    \qty{\begin{pmatrix}0\\0\end{pmatrix}, \begin{pmatrix}1\\1\end{pmatrix}},
    U_5 = \left\langle
      \begin{pmatrix}1\\0\end{pmatrix},
      \begin{pmatrix}0\\1\end{pmatrix}
    \right\rangle = \qty{
      \begin{pmatrix}0\\0\end{pmatrix},
      \begin{pmatrix}1\\0\end{pmatrix},
      \begin{pmatrix}0\\1\end{pmatrix},
      \begin{pmatrix}1\\1\end{pmatrix}
    } = V
  \]
  Nun ist
  \begin{multicols}{2}
  \begin{itemize}
  \item $\psi\qty(
      \begin{pmatrix}0\\0\end{pmatrix}
    ) = \begin{pmatrix}0\\0\end{pmatrix}$

  \item $\psi\qty(
      \begin{pmatrix}1\\0\end{pmatrix}
    ) = \begin{pmatrix}0\\1\end{pmatrix}$

  \item $\psi\qty(
      \begin{pmatrix}0\\1\end{pmatrix}
    ) = \begin{pmatrix}1\\0\end{pmatrix}$

  \item$\psi\qty(
      \begin{pmatrix}1\\1\end{pmatrix}
    ) = \begin{pmatrix}1\\1\end{pmatrix}$
  \end{itemize}
  \end{multicols}
  Folglich sind $U_1$ mit $\left\langle
    \varphi^i\qty(\begin{pmatrix}0\\0\end{pmatrix})
  \right\rangle_{i \in \mathbb{N}_0} = U_1$,
  $U_4$ mit $\left\langle
    \varphi^i\qty(\begin{pmatrix}1\\1\end{pmatrix})
  \right\rangle_{i \in \mathbb{N}_0} = U_3$
  und $V$ mit $\left\langle
    \varphi^i\qty(\begin{pmatrix}1\\0\end{pmatrix})
  \right\rangle_{i \in \mathbb{N}_0} = V$ die $\varphi$-zyklischen
  Unterräume von $V$.

\item Bestimmen Sie alle $\psi$-zyklischen Unterräume von $V$.
  Ist $V$ selbst $\psi$-zyklisch?

  \subparagraph{Lsg.} Es ist
  \begin{multicols}{2}
  \begin{itemize}
  \item $\psi\qty(
      \begin{pmatrix}0\\0\end{pmatrix}
    ) = \begin{pmatrix}0\\0\end{pmatrix}$

  \item $\psi\qty(
      \begin{pmatrix}1\\0\end{pmatrix}
    ) = \begin{pmatrix}1\\0\end{pmatrix}$

  \item $\psi\qty(
      \begin{pmatrix}0\\1\end{pmatrix}
    ) = \begin{pmatrix}1\\1\end{pmatrix}$

  \item$\psi\qty(
      \begin{pmatrix}1\\1\end{pmatrix}
    ) = \begin{pmatrix}0\\1\end{pmatrix}$
  \end{itemize}
  \end{multicols}
  Folglich sind $U_1$ mit $\left\langle
    \varphi^i\qty(\begin{pmatrix}0\\0\end{pmatrix})
  \right\rangle_{i \in \mathbb{N}_0} = U_1$,
  $U_2$ mit $\left\langle
    \varphi^i\qty(\begin{pmatrix}1\\0\end{pmatrix})
  \right\rangle_{i \in \mathbb{N}_0} = U_2$,
  und $V$ mit $\left\langle
    \varphi^i\qty(\begin{pmatrix}0\\1\end{pmatrix})
  \right\rangle_{i \in \mathbb{N}_0} = V$ die $\psi$-zyklischen
  Unterräume von $V$.

\item Geben Sie ein $\alpha \in \End\qty\big(V)$ an, sodass $V$ nicht
  $\alpha$-zyklisch ist.

  \subparagraph{Lsg.} Sei $\alpha = 0_{V, V}$, dann ist für $v \in V$ immer
  $\left\langle \alpha^i\qty\big(v) \right\rangle_{i \in \mathbb{N}_0} = 0_V$.
\end{enumerate}
\end{document}
