\documentclass{scrreprt}

\usepackage{aligned-overset}
\usepackage{amsmath}
\usepackage{amssymb}
\usepackage{bm}
\usepackage[shortlabels]{enumitem}
\usepackage{hyperref}
\usepackage[utf8]{inputenc}
\usepackage{multicol}
\usepackage{mathtools}
\usepackage{physics}
\usepackage{tabularx}
\usepackage{titling}
\usepackage{fancyhdr}
\usepackage{xfrac}
\usepackage[dvipsnames]{xcolor}
\usepackage{pgfplots}

\pgfplotsset{compat = newest}
\usetikzlibrary{intersections}
\usetikzlibrary{patterns}
\usepgfplotslibrary{fillbetween}

\newcommand{\End}{\text{End}}
\newcommand{\Grad}{\text{Grad}}
\newcommand{\Mat}{\text{Mat}}
\newcommand{\Spur}{\text{Spur}}

\author{Karsten Lehmann\\Mat. Nr 4935758}
\date{SoSe 2022}
\title{Hausaufgaben Blatt 09\\Lineare Algebra - Weiterführende Konzepte}

\setlength{\headheight}{26pt}
\pagestyle{fancy}
\fancyhf{}
\lhead{\thetitle}
\rhead{\theauthor}
\lfoot{\thedate}
\rfoot{Seite \thepage}

\begin{document}
\paragraph{Aufgabe 6} Sei $\mathbb{F}_2 = \mathbb{Z} / 2$.
Seien außerdem
\[
  A \coloneqq \begin{pmatrix}
    0 & 1 \\
    1 & 0 \\
  \end{pmatrix} \in \mathcal{M}_{2 \times 2}\qty\big(\mathbb{F}_2)
  \text{ und }
  B \coloneqq \begin{pmatrix}
    1 & 1 \\
    0 & 1 \\
  \end{pmatrix} \in \mathcal{M}_{2 \times 2}\qty\big(\mathbb{F}_2)
\]
Weiterhin seien $V \coloneqq \qty\big(\mathbb{F}_2)^2$ und
$\varphi \colon V \to V, x \mapsto Ax \text{ und }
  \psi \colon V \to V, x \mapsto Bx$.
\begin{enumerate}[(i)]
\item Bestimmen Sie alle $\varphi$-zyklischen Unterräume von $V$.
  Ist $V$ selbst $\varphi$-zyklisch?

  \subparagraph{Lsg.} Der Vektorraum $V$ umfasst die 4 Elemente
  $
    V \coloneqq \qty{
      \begin{pmatrix}0\\0\end{pmatrix},
      \begin{pmatrix}1\\0\end{pmatrix},
      \begin{pmatrix}0\\1\end{pmatrix},
      \begin{pmatrix}1\\1\end{pmatrix}
    }
  $
  und die Unterräume
  \[
    U_1 = \left\langle \begin{pmatrix}0\\0\end{pmatrix} \right\rangle =
    \qty{\begin{pmatrix}0\\0\end{pmatrix}},
    U_2 = \left\langle \begin{pmatrix}1\\0\end{pmatrix} \right\rangle =
    \qty{\begin{pmatrix}0\\0\end{pmatrix}, \begin{pmatrix}1\\0\end{pmatrix}},
    U_3 = \left\langle \begin{pmatrix}0\\1\end{pmatrix} \right\rangle =
    \qty{\begin{pmatrix}0\\0\end{pmatrix}, \begin{pmatrix}0\\1\end{pmatrix}},
  \]
  \[
    U_4 = \left\langle \begin{pmatrix}1\\1\end{pmatrix} \right\rangle =
    \qty{\begin{pmatrix}0\\0\end{pmatrix}, \begin{pmatrix}1\\1\end{pmatrix}},
    U_5 = \left\langle
      \begin{pmatrix}1\\0\end{pmatrix},
      \begin{pmatrix}0\\1\end{pmatrix}
    \right\rangle = \qty{
      \begin{pmatrix}0\\0\end{pmatrix},
      \begin{pmatrix}1\\0\end{pmatrix},
      \begin{pmatrix}0\\1\end{pmatrix},
      \begin{pmatrix}1\\1\end{pmatrix}
    } = V
  \]
  Nun ist
  \begin{multicols}{2}
  \begin{itemize}
  \item $\psi\qty(
      \begin{pmatrix}0\\0\end{pmatrix}
    ) = \begin{pmatrix}0\\0\end{pmatrix}$

  \item $\psi\qty(
      \begin{pmatrix}1\\0\end{pmatrix}
    ) = \begin{pmatrix}0\\1\end{pmatrix}$

  \item $\psi\qty(
      \begin{pmatrix}0\\1\end{pmatrix}
    ) = \begin{pmatrix}1\\0\end{pmatrix}$

  \item$\psi\qty(
      \begin{pmatrix}1\\1\end{pmatrix}
    ) = \begin{pmatrix}1\\1\end{pmatrix}$
  \end{itemize}
  \end{multicols}
  Folglich sind $U_1$ mit $\left\langle
    \varphi^i\qty(\begin{pmatrix}0\\0\end{pmatrix})
  \right\rangle_{i \in \mathbb{N}_0} = U_1$,
  $U_4$ mit $\left\langle
    \varphi^i\qty(\begin{pmatrix}1\\1\end{pmatrix})
  \right\rangle_{i \in \mathbb{N}_0} = U_3$
  und $V$ mit $\left\langle
    \varphi^i\qty(\begin{pmatrix}1\\0\end{pmatrix})
  \right\rangle_{i \in \mathbb{N}_0} = V$ die $\varphi$-zyklischen
  Unterräume von $V$.

\item Bestimmen Sie alle $\psi$-zyklischen Unterräume von $V$.
  Ist $V$ selbst $\psi$-zyklisch?

  \subparagraph{Lsg.} Es ist
  \begin{multicols}{2}
  \begin{itemize}
  \item $\psi\qty(
      \begin{pmatrix}0\\0\end{pmatrix}
    ) = \begin{pmatrix}0\\0\end{pmatrix}$

  \item $\psi\qty(
      \begin{pmatrix}1\\0\end{pmatrix}
    ) = \begin{pmatrix}1\\0\end{pmatrix}$

  \item $\psi\qty(
      \begin{pmatrix}0\\1\end{pmatrix}
    ) = \begin{pmatrix}1\\1\end{pmatrix}$

  \item$\psi\qty(
      \begin{pmatrix}1\\1\end{pmatrix}
    ) = \begin{pmatrix}0\\1\end{pmatrix}$
  \end{itemize}
  \end{multicols}
  Folglich sind $U_1$ mit $\left\langle
    \varphi^i\qty(\begin{pmatrix}0\\0\end{pmatrix})
  \right\rangle_{i \in \mathbb{N}_0} = U_1$,
  $U_2$ mit $\left\langle
    \varphi^i\qty(\begin{pmatrix}1\\0\end{pmatrix})
  \right\rangle_{i \in \mathbb{N}_0} = U_2$,
  und $V$ mit $\left\langle
    \varphi^i\qty(\begin{pmatrix}0\\1\end{pmatrix})
  \right\rangle_{i \in \mathbb{N}_0} = V$ die $\psi$-zyklischen
  Unterräume von $V$.

\item Geben Sie ein $\alpha \in \End\qty\big(V)$ an, sodass $V$ nicht
  $\alpha$-zyklisch ist.

  \subparagraph{Lsg.} Sei $\alpha = 0_{V, V}$, dann ist für $v \in V$ immer
  $\left\langle \alpha^i\qty\big(v) \right\rangle_{i \in \mathbb{N}_0} = 0_V$.
\end{enumerate}
\newpage
\paragraph{Aufgabe 7} Seien $K$ ein Körper, $V$ ein endlich-dimensionaler
Vektorraum über $K$ und $\varphi \in \End\qty\big(V)$.
Angenommen, es existiert ein $0_V \ne v \in V$ mit $V = \left\langle
  \varphi^i\qty\big(v) \:{\big |}\: i \in \mathbb{N}_0
\right\rangle$.
Schreibe
\[
  m_{\varphi} = X^r + a_{r - i}X^{r - 1} + \ldots + a_0 \in K\qty[X]
\]
mit $r \in \mathbb{N}_0$ und $a_0, \ldots, a_{r - 1} \in K$.
\begin{enumerate}[(a)]
\item Zeigen Sie, dass $r \geq 1$.

  \subparagraph{Lsg.} Aus der Annahme, dass ein Element $v \in V$ mit
  $v \ne 0_V$ existiert folgt $V \ne \qty\big{0_V}$.
  Aus Bemerkung 11.10 der Vorlesung (\emph{``Ist $V \ne \qty\big{0_V}$, so ist
    immer $\Grad\qty\big(m_{\varphi}) \geq 1$''}) folgt die Behauptung.

\item Zeigen Sie, dass $\mathcal{B} = \qty\big(
    v, \varphi(v), \ldots, \varphi^{r - 1}(v)
  )$ eine Basis von $V$ ist.
  Gehen Sie dazu in den folgenden Schritten vor:
  \begin{enumerate}[(i)]
  \item Sei $g \in K\qty[X]$ mit $\qty\big(g(\varphi))(v) = 0_V$.
    Zeigen Sie, dass $g\qty\big(\varphi) = 0_{V, V}$.

    \subparagraph{Lsg.} Es ist
    \[
      \qty(\varphi^i \circ g\qty\big(\varphi))(v) =
      \varphi^i\qty(\qty(g\qty\big(\varphi))(v)) = \varphi^i\qty\big(0_V) = 0_V
    \]
    Im Beweis zu Lemma 11.19 (c,d) der Vorlesung wurde
    bereits gezeigt, dass
    $f\qty\big(\varphi) \circ \varphi^i = \varphi^i \circ f\qty\big(\varphi)$
    für $f \in K\qty[X]$.
    Somit folgt $ \qty(g\qty\big(\varphi) \circ \varphi^i)(v) = 0_V$,
    $i \in \mathbb{N}_0$.
    Nach Lemma 6.22 der Vorlesung reicht es eine Lineare Abbildung auf einer
    Basis zu definieren, $\left\langle
      \varphi^i\qty\big(v) \:{\big |}\: i \in \mathbb{N}_0
    \right\rangle$ spannt $V$ auf, somit schließt
    $\qty\big{\varphi^i(v) \:\big |\: i \in \mathbb{N}_0}$ eine Basis von $V$ ein
    und es folgt die Behauptung.

  \item Zeigen Sie, dass $v, \varphi(v), \ldots, \varphi^{r - 1}(v)$ linear
    unabhängig ist.
    Nehmen Sie hierzu an, dass $k_0, \ldots, k_{r - 1}$ Elemente von $K$ sind,
    sodass $\sum_{i = 0}^{r - 1} k_i\varphi^i(v) = 0_V$.
    Wenden Sie (i) auf $g = \sum_{i = 0}^{r - 1} k_iX^i$ an, um zu folgern, dass
    $g = 0_{K\qty[X]}$.

    \subparagraph{Lsg.} Es ist $g \in K\qty[X]$ und
    $g\qty\big(\varphi) = \sum_{i = 0}^{r - 1} k_i\varphi^i$.
    Per Annahme ist außerdem $\qty\big(g(\varphi))(v) = 0_V$.
    Aus (i) folgt $g\qty\big(\varphi) = 0_{V, V;}$.
    Da $V$ nicht leer ist, folgt $\varphi \ne 0_{V, V}$ und
    $g = 0_{K\qty[X]}$
    Somit gilt
    $\sum_{i = 0}^{r - 1} k_i\varphi^i(v) = 0_V \Rightarrow
    k_0 = \ldots = k_{r - 1} = 0$.

    $\Rightarrow$ $v, \varphi(v), \ldots, \varphi^{r - 1}(v)$ sind linear
    unabhängig.

  \item Zeigen Sie, dass
    $\left\langle v, \varphi(v), \ldots, \varphi^{r - 1}(v)\right\rangle = V$.
    Nehmen Sie dazu an, dass
    $\left\langle v, \varphi(v), \ldots, \varphi^{r - 1}(v)\right\rangle \ne V$.
    Wegen $V = \left\langle
      \varphi^i\qty\big(v) \:{\big |}\: i \in \mathbb{N}_0
    \right\rangle$ existiert ein $m \in \mathbb{N}_0$ mit
    $\varphi^m(v) \notin
    \left\langle v, \varphi(v), \ldots, \varphi^{r - 1}(v)\right\rangle$ und wir
    wählen ein kleinstmögliches solches $m$.
    Dann gilt $m \geq r \geq 1$, $\varphi^{m - 1}(v) \in \left\langle
      v, \varphi(v), \ldots, \varphi^{r - 1}(v)
    \right\rangle$.
    Zeigen Sie, dass $\varphi^r(v) \in \left\langle
      v, \varphi(v), \ldots, \varphi^{r - 1}(v)
    \right\rangle$ und nutzen Sie dies, um einen Widerspruch zu
    $\varphi^m(v) \notin \left\langle
      v, \varphi(v), \ldots, \varphi^{r - 1}(v)
    \right\rangle$ zu erhalten.

    \subparagraph{Lsg.}
  \end{enumerate}

\newpage
\item Geben Sie (mit Begründung) die Matrix $\Mat\qty\big(\varphi, \mathcal{B})$
  an.

  \subparagraph{Lsg.} Es ist $m_{\varphi}\qty\big(\varphi) = 0_{V, V}$.
  Somit ist $m_{\varphi}\qty\big(\varphi)(v) = 0$.
  \[
    0 = \varphi^r(v) + a_{r - 1}\varphi^{r - 1}(v) + \ldots + a_0 \cdot v
  \]
  $\Rightarrow \varphi\qty\big(\varphi^{r - 1}(v)) = -a_{r - 1}\varphi^{r - 1}(v) - \ldots - a_0 \cdot v$.
  Wenn man $\varphi$ nun auf die anderen Basiselemente anwendet erhält man
  jeweils das nächste Basis-Element.
  $\Rightarrow \Mat\qty\big(\varphi, \mathcal{B}) = \begin{pmatrix}
    0 & 0 & \ldots & 0 & -a_0 \\
    1 & 0 & \ldots & 0 & -a_1 \\
    \vdots & \vdots & \vdots & \vdots & \vdots \\
    0 & 0 & \ldots & 0 & -a_{r - 2} \\
    0 & 0 & \ldots & 1 & -a_{r - 1}
  \end{pmatrix}$.
\end{enumerate}
\end{document}
