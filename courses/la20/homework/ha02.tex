\documentclass{scrreprt}

\usepackage{aligned-overset}
\usepackage{amsmath}
\usepackage{amssymb}
\usepackage{bm}
\usepackage[shortlabels]{enumitem}
\usepackage{hyperref}
\usepackage[utf8]{inputenc}
\usepackage{multicol}
\usepackage{mathtools}
\usepackage{physics}
\usepackage{tabularx}
\usepackage{titling}
\usepackage{fancyhdr}
\usepackage{xfrac}
\usepackage[dvipsnames]{xcolor}
\usepackage{pgfplots}

\pgfplotsset{compat = newest}
\usetikzlibrary{intersections}
\usetikzlibrary{patterns}
\usepgfplotslibrary{fillbetween}

\newcommand{\sgn}{\text{sgn}}

\author{Karsten Lehmann\\Mat. Nr 4935758}
\date{SoSe 2022}
\title{Hausaufgaben Blatt 01\\Lineare Algebra - Weiterführende Konzepte}

\setlength{\headheight}{26pt}
\pagestyle{fancy}
\fancyhf{}
\lhead{\thetitle}
\rhead{\theauthor}
\lfoot{\thedate}
\rfoot{Seite \thepage}

\begin{document}
\paragraph{Aufgabe 7} Weisen Sie Eigenschaft (iii) aus dem Beweis von Satz
9.11 nach.

\subparagraph{Lsg.} In der Vorlesung wurde die Multiplikation zweier Elemente
$f = \qty\big(a_0, a_1, a_2, \ldots)$ und $g = \qty\big(b_0, b_1, b_2, \ldots)$
aus $K\qty[X]$ durch $f \cdot g = \qty\big(p_0, p_1, p_2, \ldots)$ mit
\[
  p_k = \sum_{\substack{i, k \in \mathbb{N}_0 \\ i + j = k}} a_i \cdot b_j
\]
für alle $k \in \mathbb{N}_0$ definiert.

Satz 9.11 der Vorlesung besagt nun ``\emph{$K\qty[X]$ zusammen mit der
  Vektorraum-Addition und der eben definierten Multiplikation bildet
  einen kommunitativen Ring mit
  $1_{K\qty[X]} = \qty\big(1_K, 0_K, 0_K, \ldots) = 1_KX^0$.
  Tatsächlich ist $K\qty[X]$ sogar eine $K$-Algebra.}''

Nach Punkt (iii) des Beweises aus der Vorlesung gilt für
$f = \qty\big(a_0, a_1, a_2, \ldots)$,  $g = \qty\big(b_0, b_1, b_2, \ldots)$
und $h = \qty\big(c_0, c_1, c_2, \ldots)$ aus $K\qty[X]$
\[
  f\qty\big(g + h) = f \cdot g + f \cdot h
\]

Sei $f \cdot g = r$ mit $r = \qty\big(r_0, r_1, \ldots)$ und
$f \cdot h = s$ mit $s = \qty\big(s_0, s_1, \ldots)$
Es ist $g + h = \qty\big(b_0 + c_0, b_1 + c_1, \ldots)$.
Dann ist $f \cdot \qty\big(g + h) = \qty\big(p_0, p_1, \ldots)$ mit
\begin{flalign*}
  p_k &= \sum_{\substack{i, k \in \mathbb{N}_0 \\ i + j = k}} a_i \cdot (b_j + c_j) \\
  \overset{\text{Distributivität in $K$}}&=
  \sum_{\substack{i, k \in \mathbb{N}_0 \\ i + j = k}} a_i \cdot b_j + a_i \cdot c_j \\
  &= \underset{r_k}{
    \underbrace{\sum_{\substack{i, k \in \mathbb{N}_0 \\ i + j = k}} a_i \cdot b_j}
  } + \underset{s_k}{
    \underbrace{\sum_{\substack{i, k \in \mathbb{N}_0 \\ i + j = k}} a_i \cdot c_j}
  }
\end{flalign*}
Somit ist $f \cdot \qty\big(g + h) = \qty\big(p_0, p_1, \ldots) =
\qty\big(r_0 + s_0, r_1 + s_1, \ldots) = \qty\big(r_0, r_1, \ldots) +
\qty\big(s_0, s_1, \ldots) = r + s = f \cdot g + f \cdot h$.
\end{document}