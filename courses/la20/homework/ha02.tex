\documentclass{scrreprt}

\usepackage{aligned-overset}
\usepackage{amsmath}
\usepackage{amssymb}
\usepackage{bm}
\usepackage[shortlabels]{enumitem}
\usepackage{hyperref}
\usepackage[utf8]{inputenc}
\usepackage{multicol}
\usepackage{mathtools}
\usepackage{physics}
\usepackage{tabularx}
\usepackage{titling}
\usepackage{fancyhdr}
\usepackage{xfrac}
\usepackage[dvipsnames]{xcolor}
\usepackage{pgfplots}

\pgfplotsset{compat = newest}
\usetikzlibrary{intersections}
\usetikzlibrary{patterns}
\usepgfplotslibrary{fillbetween}

\newcommand{\Kern}{\text{Kern}}

\author{Karsten Lehmann\\Mat. Nr 4935758}
\date{SoSe 2022}
\title{Hausaufgaben Blatt 02\\Lineare Algebra - Weiterführende Konzepte}

\setlength{\headheight}{26pt}
\pagestyle{fancy}
\fancyhf{}
\lhead{\thetitle}
\rhead{\theauthor}
\lfoot{\thedate}
\rfoot{Seite \thepage}

\begin{document}
\paragraph{Aufgabe 6} Seien $G$ und $H$ Gruppen und sei $\varphi \colon G \to H$
ein Gruppenhomomorphismus.
Wir definieren
\[
  \Kern\qty\big(\varphi) \coloneqq
  \qty{ g \in G \:{\big |}\: \varphi\qty\big(g) = 1_H}
\]
\begin{enumerate}[(i)]
\item Zeigen Sie, dass $\Kern\qty\big(\varphi)$ eine Untergruppe von $G$ ist.

  \subparagraph{Lsg.} Eine Abbildung $\sigma \colon G \to H$ heißt
  Gruppenhomomorphismus, falls für alle $g \in G, h \in H$ gilt
  $\sigma\qty\big(gh) = \sigma\qty\big(g)\sigma\qty\big(h)$.

  Nun ist gilt für ein beliebiges $g \in G$, dass
  $\varphi\qty\big(g) = \varphi\qty\big(1_G \cdot g)
  = \varphi\qty\big(1_G) \cdot \varphi\qty\big(g)$.

  $\Rightarrow \varphi\qty\big(1_G) = 1_H$, somit ist $\Kern\qty\big(\varphi)$
  nicht leer.

  Seien nun $e, g \in \Kern\qty\big(\varphi)$ beliebig.
  Dann ist $\varphi\qty\big(e) = 1_H = \varphi\qty\big(1_G)
  = \varphi\qty\big(e \cdot e^{-1})
  = \varphi\qty\big(e) \cdot \varphi\qty\big(e^{-1})
  = 1_H \cdot \varphi\qty\big(e^{-1}) = \varphi\qty\big(e^{-1})$.

  $\Rightarrow e^{-1} \in \Kern\qty\big(\varphi)$.

  Weiter ist $\varphi\qty\big(g \cdot e^{-1})
  = \varphi\qty\big(g) \cdot \varphi\qty\big(e^{-1}) = 1_H \cdot 1_H = 1_H$.

  $\Rightarrow g \cdot e^{-1} \in \Kern\qty\big(\varphi)$

  Nach Lemma 2.9 der Vorlesung (``\emph{Sei $\qty\big(G, \cdot)$ eine Gruppe
    und $\emptyset \ne U \subseteq G$.
    Dann ist $U$ eine Untergruppe genau dann wenn $g \cdot h^{-1} \in U$
    für alle $g, h \in U$}'') folgt die Behauptung.

\item Sei $K$ ein Körper und $n \in \mathbb{N}$.
  Nach Aufgabe 5 auf Übungsblatt 1 ist
  \[
    GL_n\qty\big(K) \coloneqq \qty{
      A \in \mathcal{M}_{n \times n}\qty\big(K)
      \:{\big |}\:
      \det\qty\big(A) \ne 0
    }
  \]
  zusammen mit der üblichen Multiplikation von Matrizen eine Gruppe.
  Zeigen Sie unter Verwendung von $(i)$ und Aufgabe 5 auf Übungsblatt 1, dass
  \[
    SL_n\qty\big(K) \coloneqq \qty{
      A \in GL_n\qty\big(K)
      \:{\big |}\:
      \det\qty\big(A) = 1
    }
  \]
  eine Untergruppe von $GL_n\qty\big(K)$ ist.

  \subparagraph{Lsg.} Sei $\phi \colon  GL_n(K) \to K \setminus \qty\big{0},
  A \mapsto \det\qty\big(A)$ eine Abbildung.
  In Aufgabe 5 (ii) auf dem 1. Übungsblatt wurde bereits gezeigt, dass
  $\phi$ ein Gruppenhomomorphismus ist.

  Offensichtlich ist $SL_n\qty\big(K) = \qty{
      A \in GL_n\qty\big(K)
      \:{\big |}\:
      \phi\qty\big(A) = 1_K
    } = \Kern\qty\big(\phi)$.
    Aus (i) folgt nun die Behauptung.
\end{enumerate}

\newpage
\paragraph{Aufgabe 7} Weisen Sie Eigenschaft (iii) aus dem Beweis von Satz
9.11 nach.

\subparagraph{Lsg.} In der Vorlesung wurde die Multiplikation zweier Elemente
$f = \qty\big(a_0, a_1, a_2, \ldots)$ und $g = \qty\big(b_0, b_1, b_2, \ldots)$
aus $K\qty[X]$ durch $f \cdot g = \qty\big(p_0, p_1, p_2, \ldots)$ mit
\[
  p_k = \sum_{\substack{i, k \in \mathbb{N}_0 \\ i + j = k}} a_i \cdot b_j
\]
für alle $k \in \mathbb{N}_0$ definiert.

Satz 9.11 der Vorlesung besagt nun ``\emph{$K\qty[X]$ zusammen mit der
  Vektorraum-Addition und der eben definierten Multiplikation bildet
  einen kommunitativen Ring mit
  $1_{K\qty[X]} = \qty\big(1_K, 0_K, 0_K, \ldots) = 1_KX^0$.
  Tatsächlich ist $K\qty[X]$ sogar eine $K$-Algebra.}''

Nach Punkt (iii) des Beweises aus der Vorlesung gilt für
$f = \qty\big(a_0, a_1, a_2, \ldots)$,  $g = \qty\big(b_0, b_1, b_2, \ldots)$
und $h = \qty\big(c_0, c_1, c_2, \ldots)$ aus $K\qty[X]$
\[
  f\qty\big(g + h) = f \cdot g + f \cdot h
\]

Sei $f \cdot g = r$ mit $r = \qty\big(r_0, r_1, \ldots)$ und
$f \cdot h = s$ mit $s = \qty\big(s_0, s_1, \ldots)$
Es ist $g + h = \qty\big(b_0 + c_0, b_1 + c_1, \ldots)$.
Dann ist $f \cdot \qty\big(g + h) = \qty\big(p_0, p_1, \ldots)$ mit
\begin{flalign*}
  p_k &= \sum_{\substack{i, k \in \mathbb{N}_0 \\ i + j = k}} a_i \cdot (b_j + c_j) \\
  \overset{\text{Distributivität in $K$}}&=
  \sum_{\substack{i, k \in \mathbb{N}_0 \\ i + j = k}} a_i \cdot b_j + a_i \cdot c_j \\
  &= \underset{r_k}{
    \underbrace{\sum_{\substack{i, k \in \mathbb{N}_0 \\ i + j = k}} a_i \cdot b_j}
  } + \underset{s_k}{
    \underbrace{\sum_{\substack{i, k \in \mathbb{N}_0 \\ i + j = k}} a_i \cdot c_j}
  }
\end{flalign*}
Somit ist $f \cdot \qty\big(g + h) = \qty\big(p_0, p_1, \ldots) =
\qty\big(r_0 + s_0, r_1 + s_1, \ldots) = \qty\big(r_0, r_1, \ldots) +
\qty\big(s_0, s_1, \ldots) = r + s = f \cdot g + f \cdot h$.

\newpage
\paragraph{Aufgabe 8} Beweisen Sie Lemma 9.13:
Seien $K$ ein Körper und $A$ eine $K$-Algebra.
Sei $a \in A$ fest gewählt.
Dann ist die Abbildung
\[
  \Phi_a \colon K\qty[X] \to A, f \mapsto f(a)
\]
ein unitärer $K$-Algebra-Homomorphismus.

\subparagraph{Lsg.} $\Phi_a$ ist ein $K$-Algebra-Homomorphismus, falls für
$g, h \in K\qty[X]$ gilt, dass
$\Phi\qty\big(g \cdot h) = \Phi\qty\big(g)\Phi\qty\big(h)$.
Nun ist
\begin{flalign*}
  \Phi_a\qty\big(g) \cdot \Phi_a\qty\big(h)
  &= \qty(\sum_{m \in \mathbb{N}_0} g_ma^m)
    \cdot \qty(\sum_{n \in \mathbb{N}_0} h_na^n) &\\
  &= \sum_{m \in \mathbb{N}_0}
    \qty(g_ma^m \cdot \sum_{n \in \mathbb{N}_0} h_na^n) \\
  &= \sum_{m \in \mathbb{N}_0}
    \qty(\sum_{n \in \mathbb{N}_0} g_mh_na^{m + n}) \\
  &= \sum_{k \in \mathbb{N}_0} \qty(\sum_{\substack{m, n \in \mathbb{N}_0 \\ m + n = k}}
  g_mh_na^k) \\
  &= \Phi_a\qty\big(g \cdot h)
\end{flalign*}
Weiter wird $\Phi_a$ unitär genannt, falls $\Phi_a\qty(1_{K\qty[X]}) = 1_A$.
Nun ist $1_{K\qty[X]} = 1_A \cdot X^0 + 0 \cdot X^1 + 0 \cdot X^2 + \ldots$

$\Rightarrow \Phi_a\qty(1_{K\qty[X]}) = 1_A \cdot a^0 + 0 \cdot a^1 + \ldots
= 1_A$

$\Rightarrow \Phi_a$ ist ein unitärer $K$-Algebra-Homomorphismus.
\end{document}