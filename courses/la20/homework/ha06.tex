\documentclass{scrreprt}

\usepackage{aligned-overset}
\usepackage{amsmath}
\usepackage{amssymb}
\usepackage{bm}
\usepackage[shortlabels]{enumitem}
\usepackage{hyperref}
\usepackage[utf8]{inputenc}
\usepackage{multicol}
\usepackage{mathtools}
\usepackage{physics}
\usepackage{tabularx}
\usepackage{titling}
\usepackage{fancyhdr}
\usepackage{xfrac}
\usepackage[dvipsnames]{xcolor}
\usepackage{pgfplots}

\pgfplotsset{compat = newest}
\usetikzlibrary{intersections}
\usetikzlibrary{patterns}
\usepgfplotslibrary{fillbetween}

\newcommand{\Grad}{\text{Grad}}
\newcommand{\Spur}{\text{Spur}}

\author{Karsten Lehmann\\Mat. Nr 4935758}
\date{SoSe 2022}
\title{Hausaufgaben Blatt 06\\Lineare Algebra - Weiterführende Konzepte}

\setlength{\headheight}{26pt}
\pagestyle{fancy}
\fancyhf{}
\lhead{\thetitle}
\rhead{\theauthor}
\lfoot{\thedate}
\rfoot{Seite \thepage}

\begin{document}
\paragraph{Aufgabe 5} Begründen Sie jeweils an, ob sich nur anhand des
charakteristischen Polynoms und unter Verwendung der Sätze 10.40 und 10.41
der Vorlesung entscheiden ließe, ob die jeweilige Matrix bzw. lineare
Abbildung diagonalisierbar ist.
Falls dies nicht möglich ist, so geben Sie eine zweite Matrix bzw. eine zweite
lineare Abbildung an, die dasselbe charakteristische Polynom hat, sich aber
bezüglich Diagonalisierbarkeit anders verhält als die gegebene Matrix bzw. die
gegebene Abbildung.

\begin{enumerate}[(i)]
\item $B \coloneqq \begin{pmatrix}
    0 & 1 \\
    -1 & -1 \\
  \end{pmatrix} \in \mathcal{M}_{2 \times 2}\qty\big(F)_7$, wobei
  $\mathbb{F}_7 = \mathbb{Z}/7$,
  $f_B = X^2 + X + 1 = \qty\big(X - 2)\qty\big(X - 4)$.

  \subparagraph{Lsg.} Das charakteristische Polynom von $B$ zerfällt komplett in
  verschiedene Linearfaktoren und somit ist $B$ nach Satz 10.41 (a) der Vorlesung
  (\emph{``Sei $A \in \mathcal{M}_{n \times n}\qty\big(K)$.
    Wenn das charakteristische Polynom von $A$ komplett in verschiedene
    Linearfaktoren zerfällt, so ist $A$ diagonalisierbar''}) diagonalisierbar.

\item $C \coloneqq \begin{pmatrix}
    0 & 1 \\
    1 & 0 \\
  \end{pmatrix} \in \mathcal{M}_{2 \times 2}\qty\big(F)_2$, wobei
  $\mathbb{F}_2 = \mathbb{Z}/2$,
  $f_C = X^2 + 1 = \qty\big(X - 2)\qty\big(X - 4)$.

  \subparagraph{Lsg.} Es ist $X^2 + 1 = \qty\big(X - 1)\qty\big(X - 1)$ in
  $\mathbb{F}_2$.
  Somit zerfällt das charakteristische Polynom der Matrix $C$ zwar komplett
  in Linearfaktoren, jedoch sind diese nicht verschieden.
  Mit Satz 10.41 der Vorlesung lässt sich keine Aussage darüber treffen, ob
  die Matrix $C$ diagonalisierbar ist.

  Nach Anwendung den Gauß-Verfahrens auf $1 \cdot I_n - C$ folgt
  \[
    \begin{pmatrix}
      1 & -1 \\
      -1 & 1 \\
    \end{pmatrix}
    \leadsto
    \begin{pmatrix}
      1 & -1 \\
    \end{pmatrix}    
  \]
  Also ist $V_1 = \qty{
    \begin{pmatrix}0\\0\end{pmatrix},
    \begin{pmatrix}1\\1\end{pmatrix}
  }$ und $m_{\text{geom}}\qty\big(C, 1) = 1 \ne
  m_{\text{alg}}\qty\big(C, 1) = 2$.

  $\Rightarrow$ Die Matrix $C$ ist in $\mathbb{F}_2$ nicht diagonalisierbar.

  Eine diagonalisierbare Matrix mit dem selben charakteristischen Polynom
  wäre $C' = I_2$ mit $S = S^{-1} = I_2$, $D = I_2$ und $S^{-!}C'S = D$.

\item $\varphi \colon \mathbb{R}^2 \to \mathbb{R}^2,
  \begin{pmatrix}x\\y\end{pmatrix} \mapsto \begin{pmatrix}x\\x+2y\end{pmatrix}$,
  $f_{\varphi} = \qty\big(X - 1)\qty\big(X - 2)$

  \subparagraph{Lsg.} Das charakteristische Polynom von $\varphi$ zerfällt komplett in
  verschiedene Linearfaktoren und somit ist $\varphi$ nach Satz 10.40 (a) der Vorlesung
  (\emph{``Sei $\varphi$ ein Endomorphismus von $V$.
    Wenn das charakteristische Polynom von $\varphi$ komplett in verschiedene
    Linearfaktoren zerfällt, so ist $\varphi$ diagonalisierbar''}) diagonalisierbar.
\end{enumerate}
\end{document}
