\documentclass{scrreprt}

\usepackage{aligned-overset}
\usepackage{amsmath}
\usepackage{amssymb}
\usepackage{bm}
\usepackage[shortlabels]{enumitem}
\usepackage{hyperref}
\usepackage[utf8]{inputenc}
\usepackage{multicol}
\usepackage{mathtools}
\usepackage{physics}
\usepackage{tabularx}
\usepackage{titling}
\usepackage{fancyhdr}
\usepackage{xfrac}
\usepackage[dvipsnames]{xcolor}
\usepackage{pgfplots}

\pgfplotsset{compat = newest}
\usetikzlibrary{intersections}
\usetikzlibrary{patterns}
\usepgfplotslibrary{fillbetween}

\newcommand{\End}{\text{End}}
\newcommand{\Grad}{\text{Grad}}
\newcommand{\Mat}{\text{Mat}}
\newcommand{\Spur}{\text{Spur}}

\author{Karsten Lehmann\\Mat. Nr 4935758}
\date{SoSe 2022}
\title{Hausaufgaben Blatt 06\\Lineare Algebra - Weiterführende Konzepte}

\setlength{\headheight}{26pt}
\pagestyle{fancy}
\fancyhf{}
\lhead{\thetitle}
\rhead{\theauthor}
\lfoot{\thedate}
\rfoot{Seite \thepage}

\begin{document}
\paragraph{Aufgabe 5} Begründen Sie jeweils an, ob sich nur anhand des
charakteristischen Polynoms und unter Verwendung der Sätze 10.40 und 10.41
der Vorlesung entscheiden ließe, ob die jeweilige Matrix bzw. lineare
Abbildung diagonalisierbar ist.
Falls dies nicht möglich ist, so geben Sie eine zweite Matrix bzw. eine zweite
lineare Abbildung an, die dasselbe charakteristische Polynom hat, sich aber
bezüglich Diagonalisierbarkeit anders verhält als die gegebene Matrix bzw. die
gegebene Abbildung.

\begin{enumerate}[(i)]
\item $B \coloneqq \begin{pmatrix}
    0 & 1 \\
    -1 & -1 \\
  \end{pmatrix} \in \mathcal{M}_{2 \times 2}\qty\big(F)_7$, wobei
  $\mathbb{F}_7 = \mathbb{Z}/7$,
  $f_B = X^2 + X + 1 = \qty\big(X - 2)\qty\big(X - 4)$.

  \subparagraph{Lsg.} Das charakteristische Polynom von $B$ zerfällt komplett in
  verschiedene Linearfaktoren und somit ist $B$ nach Satz 10.41 (a) der Vorlesung
  (\emph{``Sei $A \in \mathcal{M}_{n \times n}\qty\big(K)$.
    Wenn das charakteristische Polynom von $A$ komplett in verschiedene
    Linearfaktoren zerfällt, so ist $A$ diagonalisierbar''}) diagonalisierbar.

\item $C \coloneqq \begin{pmatrix}
    0 & 1 \\
    1 & 0 \\
  \end{pmatrix} \in \mathcal{M}_{2 \times 2}\qty\big(F)_2$, wobei
  $\mathbb{F}_2 = \mathbb{Z}/2$,
  $f_C = X^2 + 1 = \qty\big(X - 2)\qty\big(X - 4)$.

  \subparagraph{Lsg.} Es ist $X^2 + 1 = \qty\big(X - 1)\qty\big(X - 1)$ in
  $\mathbb{F}_2$.
  Somit zerfällt das charakteristische Polynom der Matrix $C$ zwar komplett
  in Linearfaktoren, jedoch sind diese nicht verschieden.
  Mit Satz 10.41 der Vorlesung lässt sich keine Aussage darüber treffen, ob
  die Matrix $C$ diagonalisierbar ist.

  Nach Anwendung den Gauß-Verfahrens auf $1 \cdot I_n - C$ folgt
  \[
    \begin{pmatrix}
      1 & -1 \\
      -1 & 1 \\
    \end{pmatrix}
    \leadsto
    \begin{pmatrix}
      1 & -1 \\
    \end{pmatrix}    
  \]
  Also ist $V_1 = \qty{
    \begin{pmatrix}0\\0\end{pmatrix},
    \begin{pmatrix}1\\1\end{pmatrix}
  }$ und $m_{\text{geom}}\qty\big(C, 1) = 1 \ne
  m_{\text{alg}}\qty\big(C, 1) = 2$.

  $\Rightarrow$ Die Matrix $C$ ist in $\mathbb{F}_2$ nicht diagonalisierbar.

  Eine diagonalisierbare Matrix mit dem selben charakteristischen Polynom
  wäre $C' = I_2$ mit $S = S^{-1} = I_2$, $D = I_2$ und $S^{-!}C'S = D$.

\item $\varphi \colon \mathbb{R}^2 \to \mathbb{R}^2,
  \begin{pmatrix}x\\y\end{pmatrix} \mapsto \begin{pmatrix}x\\x+2y\end{pmatrix}$,
  $f_{\varphi} = \qty\big(X - 1)\qty\big(X - 2)$

  \subparagraph{Lsg.} Das charakteristische Polynom von $\varphi$ zerfällt komplett in
  verschiedene Linearfaktoren und somit ist $\varphi$ nach Satz 10.40 (a) der Vorlesung
  (\emph{``Sei $\varphi$ ein Endomorphismus von $V$.
    Wenn das charakteristische Polynom von $\varphi$ komplett in verschiedene
    Linearfaktoren zerfällt, so ist $\varphi$ diagonalisierbar''}) diagonalisierbar.
\end{enumerate}

\newpage
\paragraph{Aufgabe 6} Seien $K$ ein Körper, $n \in \mathbb{N}$, $V$ ein
$n$-dimensionaler Vektorraum über $K$ und $\varphi \in \End\qty\big(V)$.
Zeigen Sie, dass $\varphi$ genau dann diagonalisierbar ist, wenn
$f_{\varphi}$ in Linearfaktoren zerfällt und die geometrische Vielfachheit eines
jeden Eigenwertes von $\varphi$ mit seiner algebraischen Vielfachheit
übereinstimmmt.

\subparagraph{Lsg.}
\begin{itemize}
\item[``$\Rightarrow$''] Siehe Satz 10.48 (b) der Vorlesung.
\item[``$\Leftarrow$''] Sei $l \in \mathbb{N}, l \leq n$ und
  $\lambda_1, \ldots, \lambda_l$ die Eigenwerte von $\varphi$.
  Angenommen für $i \in \qty\big{1, \ldots, l}$ sei
  $m_{\text{alg}}\qty\big(\varphi, \lambda_i)
  = m_{\text{geom}}\qty\big(\varphi, \lambda_i)$.

  Da das charakteristische Polynom von $\varphi$ komplett in Linearfaktoren
  zerfällt, ist $f_{\varphi}$ in der Form
  $\qty\big(X - \lambda_1)^{m_1} \cdot \qty\big(X - \lambda_2)^{m_2} \cdot \ldots
  \cdot \qty\big(X - \lambda_l)^{m_l}$ darstellbar, wobei
  $m_i = m_{\text{alg}}\qty\big(\varphi, \lambda_i)$.
  Es folgt $\sum_{i = 0}^l m_i = \Grad\qty\big(f_{\varphi}) = \dim\qty\big(V)$.

  Per Definition der geometrischen Vielfachheit gilt für
  $i \in \qty\big{1, \ldots, l}$, dass
  $m_i = \dim\qty\big(V_{\lambda_i}^{\varphi})$.

  $\Rightarrow \sum_{i = 1}^l \dim\qty(V_{\lambda_i}^{\varphi}) =
  \dim\qty\big(V)$

  Aus Satz 10.35 der Vorlesung (\emph{``Sei $\varphi \in |End\qty\big(V)$ und
    $\lambda_1, \ldots, \lambda_r$ die paarweise verschiedenen Eigenwerte von
    $\varphi$.
    Dann sind äquivalent: $\varphi$ ist diagonalisierbar und
    $\dim\qty\big(V) = \sum_{i = 1}^r \dim\qty(V_{\lambda_i}^{\varphi})$''})
  folgt die Behauptung.
\end{itemize}

\paragraph{Aufgabe 7} Seien $p$ eine Primzahl,
$\mathbb{F}_p \coloneqq \mathbb{Z}/p$, $V \coloneqq \mathbb{F}_p^3$ und
\[
  \varphi \colon V \to V,
  \begin{pmatrix}x\\y\\z\end{pmatrix} \mapsto
  \begin{pmatrix}x\\-3z\\y\end{pmatrix}
\]
\begin{enumerate}[(i)]
\item Zeigen Sie, dass $\varphi$ linear ist.

  \subparagraph{Lsg.} Seien $u, v \in V, \lambda \in \mathbb{F}_p$
  beliebig.
  Dann ist
  \begin{flalign*}
    \varphi\qty\big(u + \lambda \cdot v) &= \begin{pmatrix}
      u_1 + \lambda \cdot v_1 \\
      -3 \cdot \qty\big(u_3 + \lambda \cdot v_3) \\
      u_2 + \lambda \cdot v_2
    \end{pmatrix} \\
    \overset{
      \substack{
        \text{Distributivität} \\
        \text{in }\mathbb{F}_p
      }
    }&= \begin{pmatrix}
      u_1 + \lambda \cdot v_1 \\
      -3 \cdot u_3  + \qty\big(- 3 \cdot \lambda \cdot v_3) \\
      u_2 + \lambda \cdot v_2
    \end{pmatrix} \\
    &= \begin{pmatrix}
      u_1 \\
      -3 \cdot u_3 \\
      u_2
    \end{pmatrix} + \lambda \cdot \begin{pmatrix}
      v_1 \\
      -3 \cdot v_3 \\
      \cdot v_2
    \end{pmatrix} \\
    &= \varphi\qty\big(u) + \lambda \cdot \varphi\qty\big(v)
  \end{flalign*}
  Aus Lemma 6.2 der Vorlesung folgt die Behauptung.

\item Bestimmen Sie $\text{Mat}\qty\big(\varphi, \mathcal{E})$ für die
  Standardbasis $\mathcal{E} = \qty\big(e_1, e_2, e_3)$.

  \subparagraph{Lsg.} Es ist
  \[
    \varphi\qty\big(e_1) = \begin{pmatrix}1\\0\\0\end{pmatrix},
    \varphi\qty\big(e_2) = \begin{pmatrix}0\\0\\1\end{pmatrix}
    \varphi\qty\big(e_3) = \begin{pmatrix}0\\-3\\0\end{pmatrix}
  \]
  Somit ist
  \[
    \text{Mat}\qty\big(\varphi, \mathcal{E}) = \begin{pmatrix}
      1 & 0 & 0 \\
      0 & 0 & -3 \\
      0 & 1 & 0 \\
    \end{pmatrix}
  \]

\item Bestimmen Sie $f_{\varphi}$.

  \subparagraph{Lsg.} Es ist $f_{\varphi} =
  \det\qty\big(XI_3 - \text{Mat}\qty\big(\varphi, \mathcal{E}))
  = \det\begin{pmatrix}
    X - 1 & 0 & 0 \\
    0 & X & 3 \\
    0 & -1 & X \\
  \end{pmatrix}$.

  Mit Entwicklung der Determinante nach der ersten Spalte folgt
  \begin{flalign*}
    f_{\varphi} &= \qty\big(X - 1) \cdot \det \begin{pmatrix}
      X & 3 \\
      -1 & X \\
    \end{pmatrix} \\
    &= \qty\big(X - 1) \cdot \qty\big(X^2 + 3)
  \end{flalign*}

\item Bestimmen Sie für $p \in \qty\big{2, 3, 5, 7}$ die Eigenwerte und
  Eigenräume von $\varphi$.

  \subparagraph{Lsg.} Sei $p = 2$, dann ist
  \[
    f_{\varphi} = \qty\big(X - 1)^3
  \]
  und $\lambda_{1|2|3} = 1$.
  Weiter ist $V_{\lambda_{1|2|3}}^{\varphi} = \qty{
    \begin{pmatrix}0\\1\\1\end{pmatrix} \cdot x +
    \begin{pmatrix}1\\0\\0\end{pmatrix} \cdot y
    \:\middle |\:
    x, y \in \mathbb{F}_2
  }$.

  Sei $p = 3$, dann ist
  \[
    f_{\varphi} = X^2 \cdot \qty\big(X - 1)
  \]
  und $\lambda_{1|2} = 0, \lambda_3 = 1$.
  Weiter ist $V_{\lambda_{1|2}}^{\varphi} = \qty{
    \begin{pmatrix}0\\0\\1\end{pmatrix} \cdot x
    \:\middle |\:
    x \in \mathbb{F}_3
  }$ und $V_{\lambda_3}^{\varphi} = \qty{
    \begin{pmatrix}1\\0\\0\end{pmatrix} \cdot x
    \:\middle |\:
    x \in \mathbb{F}_3
  }$.

  Sei $p = 5$, dann ist
  \[
    f_{\varphi} = \qty\big(X - 1) \cdot \qty\big(X^2 + 3)
  \]
  und $\lambda_1 = 1$.
  Weiter ist $V_{\lambda_1}^{\varphi} = \qty{
      \begin{pmatrix}1\\0\\0\end{pmatrix} \cdot x
      \:\middle |\:
      x \in \mathbb{F}_5
  }$.

  Sei schließlich $p = 7$, dann ist
  \[
    f_{\varphi} = \qty\big(X - 1) \cdot \qty\big(X - 2) \cdot \qty\big(X - 5)
  \]
  und $\lambda_1 = 1, \lambda_2 = 2, \lambda_3 = 5$.
  Weiter ist $V_{\lambda_1} = \qty{
    \begin{pmatrix}1\\0\\0\end{pmatrix} \cdot x
    \:\middle |\:
    x \in \mathbb{F}_5
  }$.

  Seien $A = 1 \cdot I_3 - \begin{pmatrix}
    1 & 0 & 0 \\
    0 & 0 & -3 \\
    0 & 1 & 0 \\
  \end{pmatrix} = \begin{pmatrix}
    0 & 0 & 0 \\
    0 & 1 & 3 \\
    0 & -1 & 1 \\
  \end{pmatrix}$,

  $B = 2 \cdot I_3 - \begin{pmatrix}
    1 & 0 & 0 \\
    0 & 0 & -3 \\
    0 & 1 & 0 \\
  \end{pmatrix} = \begin{pmatrix}
    1 & 0 & 0 \\
    0 & 2 & 3 \\
    0 & -1 & 2 \\
  \end{pmatrix}$ und

  $C = 5 \cdot I_3 - \begin{pmatrix}
    1 & 0 & 0 \\
    0 & 0 & -3 \\
    0 & 1 & 0 \\
  \end{pmatrix} = \begin{pmatrix}
    4 & 0 & 0 \\
    0 & 5 & 3 \\
    0 & -1 & 5 \\
  \end{pmatrix}$.

  Es folgen $V_{\lambda_1}^{\varphi} = \qty{
    \begin{pmatrix}1\\0\\0\end{pmatrix} \cdot x
    \:\middle |\: x \in \mathbb{F}_7
  }$, $V_{\lambda_2}^{\varphi} = \qty{
    \begin{pmatrix}0\\4\\2\end{pmatrix} \cdot x
    \:\middle |\: x \in \mathbb{F}_7
  }$ und

  $V_{\lambda_3}^{\varphi} = \qty{
    \begin{pmatrix}0\\2\\6\end{pmatrix} \cdot x
    \:\middle |\: x \in \mathbb{F}_7
  }$.

\item Beweisen oder widerlegen Sie für $p \in \qty\big{2, 3, 5, 7}$, dass
  $\varphi$ diagonalisierbar ist und bestimmen Sie gegebenenfalls eine Basis
  $\mathcal{B}$ von $V$, sodass $\Mat\qty\big(\mathcal{B}, \varphi)$ eine
  Diagonalmatrix ist.

  \subparagraph{Lsg.} Sei $p = 2$, dann ist nach (iv)
  \[
    m_{\text{geom}}\qty\big(\varphi, 1) =
    \dim\qty\big(V_{\lambda_{1|2|3}}^{\varphi}) = 2 \ne
    3 = m_{\text{alg}}\qty\big(\varphi, 1)
  \]
  $\Rightarrow \varphi$ ist in $\mathbb{F}_2$ nicht diagonalisierbar.

  Sei $p = 3$, dann ist nach (iv)
  \[
    m_{\text{geom}}\qty\big(\varphi, 0) =
    \dim\qty\big(V_{\lambda_{1|2}}^{\varphi}) = 1 \ne
    2 = m_{\text{alg}}\qty\big(\varphi, 0)
  \]
  $\Rightarrow \varphi$ ist in $\mathbb{F}_3$ nicht diagonalisierbar.

  Sei $p = 5$, dann ist nach (iv) $\lambda_1 = 1$ der einzige Eigenwert von
  $\varphi$ und
  \[
    m_{\text{geom}}\qty\big(\varphi, 1) =
    \dim\qty\big(V_{\lambda_{1}}^{\varphi}) = 1 \ne
    3 = \dim\qty\big(V)
  \]
  $\Rightarrow \varphi$ ist in $\mathbb{F}_5$ nicht diagonalisierbar.

  Sei $p = 7$, dann sind nach (iv) $\lambda_1 = 1$, $\lambda_2 = 2$ und
  $\lambda_3 = 5$ die Eigenwerte von $\varphi$.
  Da
  \[
    \sum_{i = 1}^3
    m_{\text{geom}}\qty\big(\varphi, \lambda_i) =
    m_{\text{alg}}\qty\big(\varphi, \lambda_i)
  \]
  ist $\varphi$ diagonalisierbar.
  Sei weiter $\mathcal{B} = \qty(
    \begin{pmatrix}1\\0\\0\end{pmatrix},
    \begin{pmatrix}0\\4\\2\end{pmatrix},
    \begin{pmatrix}0\\2\\6\end{pmatrix}
  )$, dann ist
  \[
    \Mat\qty\big(\varphi, \mathcal{B}) = \begin{pmatrix}
      1 & 0 & 0 \\
      0 & 2 & 0 \\
      0 & 0 & 5 \\
    \end{pmatrix}
  \]
\end{enumerate}
\end{document}
