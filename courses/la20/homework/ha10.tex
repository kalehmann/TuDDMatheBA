\documentclass{scrreprt}

\usepackage{aligned-overset}
\usepackage{amsmath}
\usepackage{amssymb}
\usepackage{bm}
\usepackage[shortlabels]{enumitem}
\usepackage{hyperref}
\usepackage[utf8]{inputenc}
\usepackage{multicol}
\usepackage{mathtools}
\usepackage{physics}
\usepackage{tabularx}
\usepackage{titling}
\usepackage{fancyhdr}
\usepackage{xfrac}
\usepackage[dvipsnames]{xcolor}
\usepackage{pgfplots}

\pgfplotsset{compat = newest}
\usetikzlibrary{intersections}
\usetikzlibrary{patterns}
\usepgfplotslibrary{fillbetween}

\newcommand{\Bild}{\text{Bild}}
\newcommand{\Dim}{\text{Dim}}
\newcommand{\End}{\text{End}}
\newcommand{\id}{\text{id}}
\newcommand{\Grad}{\text{Grad}}
\newcommand{\Mat}{\text{Mat}}
\newcommand{\Rang}{\text{Rang}}
\newcommand{\Spur}{\text{Spur}}

\author{Karsten Lehmann\\Mat. Nr 4935758}
\date{SoSe 2022}
\title{Hausaufgaben Blatt 10\\Lineare Algebra - Weiterführende Konzepte}

\setlength{\headheight}{26pt}
\pagestyle{fancy}
\fancyhf{}
\lhead{\thetitle}
\rhead{\theauthor}
\lfoot{\thedate}
\rfoot{Seite \thepage}

\begin{document}
\paragraph{Aufgabe 5}
\begin{enumerate}[(i)]
\item Beweisen Sie Lemma 13.4 (a)

  \subparagraph{Lsg.} Lemma 13.4 (a) der Vorlesung besagt (\emph{``Sei $U$ ein
    $\varphi$-zyklischer Unterraum von $V$.
    Dann gilt: Ist $W$ ein $\varphi$-invarianter Unterraum von $U$, so ist $U/W$
    ein $\varphi_{U/W}$-zyklischer Unterraum''}).
  Nach Lemma 12.9 der Vorlesung ist $\varphi_{U/W}$ ein Endomorphismus auf $U / W$.
  Seien nun $u \in U$ so gewählt, dass $\left\langle
    \varphi^i\qty\big(u) \:\big|\: i \in \mathbb{N}_0
  \right\rangle = U$ und $\rho \colon U \to U / W$ der natürliche Homomorphismus.
  Dann ist nach Lemma 6.11 (c) der Vorlesung $\left\langle
    \rho\qty\big(\varphi^i\qty\big(u)) \:\big|\: i \in \mathbb{N}_0
  \right\rangle = U / W = \left\langle
    \varphi^i\qty\big(u) + W \:\big|\: i \in \mathbb{N}_0
  \right\rangle = \left\langle
    \varphi_{U / W}^i\qty\big(u + W) \:\big|\: i \in \mathbb{N}_0
  \right\rangle$.

\item Beweisen Sie Lemma 13.4 (b)

  \subparagraph{Lsg.}  Lemma 13.4 (a) der Vorlesung besagt (\emph{``Sei $U$ ein
    $\varphi$-zyklischer Unterraum von $V$.
    Dann gilt: Ist $\psi \in \End\qty\big(V)$ mit
    $\varphi \circ \psi = \psi \circ \varphi$, so ist $\psi\qty\big(U)$ wieder
    $\varphi$-zyklisch''}).
  Sei nun $u \in U$ so gewählt, dass $\left\langle
    \varphi^i\qty\big(u) \:\big|\: i \in \mathbb{N}_0
  \right\rangle = U$.
  Nun ist
  \begin{flalign*}
    \psi\qty\big(U) &= \left\langle
      \psi\qty\big(\varphi^i(u)) \:\big|\: i \in \mathbb{N}_0
    \right\rangle \\
    &= \left\langle
      \qty\big(\psi \circ \varphi^i(u)) \:\big|\: i \in \mathbb{N}_0
    \right\rangle \\
    &= \left\langle
      \qty\big(\varphi^i \circ \psi(u)) \:\big|\: i \in \mathbb{N}_0
    \right\rangle \\
    &= \left\langle
      \varphi^i\qty\big(\psi(u)) \:\big|\: i \in \mathbb{N}_0
    \right\rangle
  \end{flalign*}
\end{enumerate}

\paragraph{Aufgabe 6} Finden Sie ein Beispiel für einen endlich-dimensionalen
Vektorraum $\qty\big{0_V}$ über einem Körper $K$ und einen Endomorphismus
$\varphi$ von $V$, sodass $V$ zwar $\varphi$-zyklisch, aber nicht
$\varphi$-unzerlegbar ist.

\subparagraph{Lsg.} Sei $K = \mathbb{Z}/3$ und $V = \qty\big(\mathbb{Z}/3)^2$.
Sei weiter $\varphi \colon V \to V$ so gewählt, dass $v \mapsto A \cdot v$ mit
\[
  A = \begin{pmatrix}
    1 & 0 \\
    0 & -1 \\
  \end{pmatrix}
\]
Dann ist $V$ ein $\varphi$-zyklischer Raum mit
\[
  \left\langle
    \begin{pmatrix}1\\1\end{pmatrix},
    \varphi\qty(\begin{pmatrix}1\\1\end{pmatrix})
  \right\rangle = V
\]
Die nach Lemma 11.19 (b) $\varphi$-invarianten Eigenräume von $\varphi$ sind
\[
  V_1^{\varphi} = \left\langle
    \begin{pmatrix}1\\0\end{pmatrix}
  \right\rangle \text{ und } V_{-1}^{\varphi} = \left\langle
    \begin{pmatrix}0\\1\end{pmatrix}
  \right\rangle
\]
Schließlich ist $V$ nicht $\varphi$-unzerlegbar, da
$V = V_1^{\varphi} \oplus V_{-1}^{\varphi}$.

\newpage
\paragraph{Aufgabe 7} Sei $\qty\big{0_V} \ne V$ ein endlich-dimensionaler
Vektorraum über $\mathbb{R}$, sei $\mathcal{B}'$ eine Basis von $V$ und
$\varphi \in \End\qty\big(V)$.
Bestimmen Sie für die folgenden Fälle jeweils eine allgemeine Normalform von
$\varphi$.
\begin{enumerate}[(i)]
\item $\Mat\qty\big(\varphi, \mathcal{B}') = \begin{pmatrix}
    1 & 1 \\
    0 & 1 \\
  \end{pmatrix}$

  \subparagraph{Lsg.} Es ist $f_{\varphi} = \qty\big(X - 1)^2$.
  Nach Caley-Hamilton ist $f_{\varphi} = m_{\varphi} = 1 -2X + X^2$.
  Nach Bemerkung 13.18 der Vorlesung ist in einer allgemeinen Normalform
  immer ein Kästchen der Form $A(m_{\varphi})$.
  Es folgt $N \coloneqq \begin{pmatrix}
    0 & -1 \\
    1 & 2 \\
  \end{pmatrix}$ ist die allgemeine Normalform von $\varphi$.

\item $m_{\varphi} = \qty\big(X - 1)^2 \qty\big(X - 2)^2\qty\big(X - 3)^2$ und
  $f_{\varphi} = \qty\big(X - 1)^2 \qty\big(X - 2)^2\qty\big(X - 3)^3$.

  \subparagraph{Lsg.} Wegen $\Grad\qty\big(f_{\varphi}) = 7$ ist die Matrix in
  der allgemeinen Normalform eine $7 \times 7$ Matrix.
  Nach Bemerkung 13.18 gibt es in einer allgemeinen Normalform immer Kästchen der
  Form
  \[
    A\qty\big((X - 1)^2) = \begin{pmatrix}
      0 & -1 \\
      1 & 2 \\
    \end{pmatrix}, A\qty\big((X - 2)^2) = \begin{pmatrix}
      0 & -4 \\
      1 & 4 \\
    \end{pmatrix}, A\qty\big((X - 3)^2) = \begin{pmatrix}
      0 & -9 \\
      1 & 6 \\
    \end{pmatrix},
  \]
  Damit bleibt noch Platz für ein Kästchen der Größe 1 in der Form
  \[
    A\qty\big(X - 1) = 1, A\qty\big(X - 2) = 2 \text{ oder }
    A\qty\big(X - 3) = 3
  \]
  Somit sieht eine allgemeine Normalform von $\varphi$ wie folgt aus
  \[
    N \coloneqq \begin{pmatrix}
      0 & -1 & 0 &  0 & 0 &  0 & 0 \\
      1 &  2 & 0 &  0 & 0 &  0 & 0 \\
      0 &  0 & 0 & -4 & 0 &  0 & 0 \\
      0 &  0 & 1 &  4 & 0 &  0 & 0 \\
      0 &  0 & 0 &  0 & 0 & -9 & 0 \\
      0 &  0 & 0 &  0 & 1 &  6 & 0 \\
      0 &  0 & 0 &  0 & 0 &  0 & a \\
    \end{pmatrix}
  \]
  wobei $a \in \qty\big{1, 2, 3}$.
  Nach Definition 13.20 der Vorlesung ist $N$ ähnlich zu $\varphi$,
  aus Lemma 10.15 der Vorlesung folgt $f_N = f_{\varphi}$.
  Nun ist
  $f_N = \qty\big(X - 1)^2 \qty\big(X - 2)^2\qty\big(X - 3)^2\qty\big(X - a)$,
  es folgt $a = 3$.
  Damit ist
  \[
    N \coloneqq \begin{pmatrix}
      0 & -1 & 0 &  0 & 0 &  0 & 0 \\
      1 &  2 & 0 &  0 & 0 &  0 & 0 \\
      0 &  0 & 0 & -4 & 0 &  0 & 0 \\
      0 &  0 & 1 &  4 & 0 &  0 & 0 \\
      0 &  0 & 0 &  0 & 0 & -9 & 0 \\
      0 &  0 & 0 &  0 & 1 &  6 & 0 \\
      0 &  0 & 0 &  0 & 0 &  0 & 3 \\
    \end{pmatrix}
  \]
  die (bis auf Vertauschung der Blöcke) eindeutige Normalform von $\varphi$.

\newpage
\item $\Mat\qty\big(\varphi, \mathcal{B}') = \begin{pmatrix}
    3 & 0 & 0 & -1 \\
    1 & 2 & 0 & -1 \\
    0 & 0 & 2 &  0 \\
    1 & 0 & 0 &  1 \\
  \end{pmatrix}$

  \subparagraph{Lsg.} Es ist
  \begin{flalign*}
    f_{\varphi} &= \qty\big(X - 2)\det\begin{pmatrix}
      X - 3 & 0 & 1 \\
      -1 & X - 2 & 1 \\
      -1 & 0 &  X - 1 \\
    \end{pmatrix} &\\
    &= \qty\big(X - 2)^2\det\begin{pmatrix}
      X - 3 & 1 \\
      -1 & X - 1 \\
    \end{pmatrix} \\
    &= \qty\big(X - 2)^2\qty\big(X^2 - 4X + 4) \\
    &= \qty\big(X - 2)^4
  \end{flalign*}
  Setze nun $A \coloneqq \Mat\qty\big(\varphi, \mathcal{B}') - 2I_4
  = \begin{pmatrix}
    1 & 0 & 0 & -1 \\
    1 & 0 & 0 & -1 \\
    0 & 0 & 0 &  0 \\
    1 & 0 & 0 & -1 \\
  \end{pmatrix}$.
  Dann ist $A \ne 0$, aber $A \cdot A = 0$.
  Es folgt $m_{\varphi} = \qty\big(X - 2)^2$.
  Nun hat die allgemeine Normalform von $\varphi$ mindestens einen Block der Form
  $A\qty\big((X - 2)^2)$ und eventuell zwei Blöcke der Form $A(X - 2)$.
  Lässt man die Reihenfolge der Blöcke außer Acht, stehen die folgenden beiden
  Kandidaten für die allgemeine Normalform im Raum
  \[
    N_1 \coloneqq \begin{pmatrix}
      0 & -4 & 0 &  0 \\
      1 &  4 & 0 &  0 \\
      0 &  0 & 0 & -4 \\
      0 &  0 & 1 &  4 \\
    \end{pmatrix} \text { und } N_2 \coloneqq \begin{pmatrix}
      0 & -4 & 0 & 0 \\
      1 &  4 & 0 & 0 \\
      0 &  0 & 2 & 0 \\
      0 &  0 & 0 & 2 \\
    \end{pmatrix}
  \]
  Nun ist $\Dim\qty\big(\Bild\qty\big(\varphi - 2\id)) =
  \Rang\qty\big(\Mat\qty\big(\varphi, \mathcal{B}') - 2I_4) = 1$.
  Weiter sind $\Rang\qty\big(N_1 - 2I_4) = 2$ und
  $\Rang\qty\big(N_2 - 2I_4) = 4$.

  Es folgt $N_2$ ist die (bis auf Vertauschung der Blöcke) eindeutige
  Normalform von $\varphi$.
\end{enumerate}
\end{document}
