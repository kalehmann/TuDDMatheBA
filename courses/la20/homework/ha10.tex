\documentclass{scrreprt}

\usepackage{aligned-overset}
\usepackage{amsmath}
\usepackage{amssymb}
\usepackage{bm}
\usepackage[shortlabels]{enumitem}
\usepackage{hyperref}
\usepackage[utf8]{inputenc}
\usepackage{multicol}
\usepackage{mathtools}
\usepackage{physics}
\usepackage{tabularx}
\usepackage{titling}
\usepackage{fancyhdr}
\usepackage{xfrac}
\usepackage[dvipsnames]{xcolor}
\usepackage{pgfplots}

\pgfplotsset{compat = newest}
\usetikzlibrary{intersections}
\usetikzlibrary{patterns}
\usepgfplotslibrary{fillbetween}

\newcommand{\Bild }{\text{Bild}}
\newcommand{\End}{\text{End}}
\newcommand{\Grad}{\text{Grad}}
\newcommand{\Mat}{\text{Mat}}
\newcommand{\Spur}{\text{Spur}}

\author{Karsten Lehmann\\Mat. Nr 4935758}
\date{SoSe 2022}
\title{Hausaufgaben Blatt 10\\Lineare Algebra - Weiterführende Konzepte}

\setlength{\headheight}{26pt}
\pagestyle{fancy}
\fancyhf{}
\lhead{\thetitle}
\rhead{\theauthor}
\lfoot{\thedate}
\rfoot{Seite \thepage}

\begin{document}
\paragraph{Aufgabe 5}
\begin{enumerate}[(i)]
\item Beweisen Sie Lemma 13.4 (a)

  \subparagraph{Lsg.} Lemma 13.4 (a) der Vorlesung besagt (\emph{``Sei $U$ ein
    $\varphi$-zyklischer Unterraum von $V$.
    Dann gilt: Ist $W$ ein $\varphi$-invarianter Unterraum von $U$, so ist $U/W$
    ein $\varphi_{U/W}$-zyklischer Unterraum''}).
  Nach Lemma 12.9 der Vorlesung ist $\varphi_{U/W}$ ein Endomorphismus auf $U / W$.
  Seien nun $u \in U$ so gewählt, dass $\left\langle
    \varphi^i\qty\big(u) \:\big|\: i \in \mathbb{N}_0
  \right\rangle = U$ und $\rho \colon U \to U / W$ der natürliche Homomorphismus.
  Dann ist nach Lemma 6.11 (c) der Vorlesung $\left\langle
    \rho\qty\big(\varphi^i\qty\big(u)) \:\big|\: i \in \mathbb{N}_0
  \right\rangle = U / W = \left\langle
    \varphi^i\qty\big(u) + W \:\big|\: i \in \mathbb{N}_0
  \right\rangle = \left\langle
    \varphi_{U / W}^i\qty\big(u + W) \:\big|\: i \in \mathbb{N}_0
  \right\rangle$.

\item Beweisen Sie Lemma 13.4 (b)

  \subparagraph{Lsg.}  Lemma 13.4 (a) der Vorlesung besagt (\emph{``Sei $U$ ein
    $\varphi$-zyklischer Unterraum von $V$.
    Dann gilt: Ist $\psi \in \End\qty\big(V)$ mit
    $\varphi \circ \psi = \psi \circ \varphi$, so ist $\psi\qty\big(U)$ wieder
    $\varphi$-zyklisch''}).
  Sei nun $u \in U$ so gewählt, dass $\left\langle
    \varphi^i\qty\big(u) \:\big|\: i \in \mathbb{N}_0
  \right\rangle = U$.
  Nun ist
  \begin{flalign*}
    \psi\qty\big(U) &= \left\langle
      \psi\qty\big(\varphi^i(u)) \:\big|\: i \in \mathbb{N}_0
    \right\rangle \\
    &= \left\langle
      \qty\big(\psi \circ \varphi^i(u)) \:\big|\: i \in \mathbb{N}_0
    \right\rangle \\
    &= \left\langle
      \qty\big(\varphi^i \circ \psi(u)) \:\big|\: i \in \mathbb{N}_0
    \right\rangle \\
    &= \left\langle
      \varphi^i\qty\big(\psi(u)) \:\big|\: i \in \mathbb{N}_0
    \right\rangle
  \end{flalign*}
\end{enumerate}

\paragraph{Aufgabe 6} Finden Sie ein Beispiel für einen endlich-dimensionalen
Vektorraum $\qty\big{0_V}$ über einem Körper $K$ und einen Endomorphismus
$\varphi$ von $V$, sodass $V$ zwar $\varphi$-zyklisch, aber nicht
$\varphi$-unzerlegbar ist.

\subparagraph{Lsg.} Sei $K = \mathbb{Z}/3$ und $V = \qty\big(\mathbb{Z}/3)^2$.
Sei weiter $\varphi \colon V \to V$ so gewählt, dass $v \mapsto A \cdot v$ mit
\[
  A = \begin{pmatrix}
    1 & 0 \\
    0 & -1 \\
  \end{pmatrix}
\]
Dann ist $V$ ein $\varphi$-zyklischer Raum mit
\[
  \left\langle
    \begin{pmatrix}1\\1\end{pmatrix},
    \varphi\qty(\begin{pmatrix}1\\1\end{pmatrix})
  \right\rangle = V
\]
Die nach Lemma 11.19 (b) $\varphi$-invarianten Eigenräume von $\varphi$ sind
\[
  V_1^{\varphi} = \left\langle
    \begin{pmatrix}1\\0\end{pmatrix}
  \right\rangle \text{ und } V_{-1}^{\varphi} = \left\langle
    \begin{pmatrix}0\\1\end{pmatrix}
  \right\rangle
\]
Schließlich ist $V$ nicht $\varphi$-unzerlegbar, da
$V = V_1^{\varphi} \oplus V_{-1}^{\varphi}$.
\end{document}
