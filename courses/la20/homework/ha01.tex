\documentclass{scrreprt}

\usepackage{aligned-overset}
\usepackage{amsmath}
\usepackage{amssymb}
\usepackage{bm}
\usepackage[shortlabels]{enumitem}
\usepackage{hyperref}
\usepackage[utf8]{inputenc}
\usepackage{multicol}
\usepackage{mathtools}
\usepackage{physics}
\usepackage{tabularx}
\usepackage{titling}
\usepackage{fancyhdr}
\usepackage{xfrac}
\usepackage[dvipsnames]{xcolor}
\usepackage{pgfplots}

\pgfplotsset{compat = newest}
\usetikzlibrary{intersections}
\usetikzlibrary{patterns}
\usepgfplotslibrary{fillbetween}

\author{Karsten Lehmann\\Mat. Nr 4935758}
\date{SoSe 2022}
\title{Hausaufgaben Blatt 01\\Lineare Algebra - Weiterführende Konzepte}

\setlength{\headheight}{26pt}
\pagestyle{fancy}
\fancyhf{}
\lhead{\thetitle}
\rhead{\theauthor}
\lfoot{\thedate}
\rfoot{Seite \thepage}

\begin{document}
\paragraph{Aufgabe 6} Beweisen Sie Lemma 7.50: Sei $K$ ein Körper.
Seien $A$ und $B$ beides $K$-Algebren und $\varphi \colon A \to B$ ein
surjektiver $K$-Algebrahomomorphismus.
Dann ist $\varphi$ unitär.
Insbesondere ist jeder Isomorphismus von $K$-Algebren schon unitär.

\subparagraph{Lsg.} $\varphi \colon A \to B$ ist ein $K$-Algebrahomomorphismus
wenn $\varphi$ eine lineare Abbildung zwischen $A$ und $B$ ist, sowie für alle
$a, b \in A$ gilt $\varphi\qty\big(ab) = \varphi\qty\big(a)\varphi\qty\big(b)$.
Der $K$-Algebrahomomorphismus $\varphi$ heißt unitär, wenn
$\varphi\qty\big(1_A) = 1_B$ gilt.
Weiter heißt der $K$-Algebrahomomorphismus $\varphi$ surjektiv, wenn für jedes
Element $b \in B$ ein Element $a \in A$ existiert mit
$\varphi\qty\big(a) = b$ existiert. \\

Sei nun $b \in B$ beliebig gewählt.
Da $\varphi$ surjektiv ist, existiert ein $a \in A$ mit
$\varphi\qty\big(a) = b$.
Nun ist auch $\varphi\qty\big(1_A)\varphi\qty\big(a) = \varphi\qty\big(1_A)b$.
Aus der Definition des $K$-Algebrahomomorphismus folgt
$\varphi\qty\big(1_A)\varphi\qty\big(a) = \varphi\qty\big(1_A \cdot a)
= \varphi\qty\big(a) = b$.
Also ist $\varphi\qty\big(1_A) \cdot b = b$ für jedes beliebige $b \in B$.

$\Rightarrow \varphi\qty\big(1_A) = 1_B$.

Ein $K$-Algebrahomomorphismus $f \colon A \to B$ heißt Isomorphismus
von $K$-Algebren, wenn $f$ bijektiv - also injektiv und surjektiv ist.
Wie eben gezeigt folgt aus der Surjektivität eines $K$-Algebrahomomorphismus
$f$ bereits, dass er unitär ist.

$\Rightarrow$ jeder Isomorphismus von $K$-Algebren ist unitär.

\end{document}