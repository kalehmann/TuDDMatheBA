\documentclass{scrreprt}

\usepackage{aligned-overset}
\usepackage{amsmath}
\usepackage{amssymb}
\usepackage{bm}
\usepackage[shortlabels]{enumitem}
\usepackage{hyperref}
\usepackage[utf8]{inputenc}
\usepackage{multicol}
\usepackage{mathtools}
\usepackage{physics}
\usepackage{tabularx}
\usepackage{titling}
\usepackage{fancyhdr}
\usepackage{xfrac}
\usepackage[dvipsnames]{xcolor}
\usepackage{pgfplots}

\pgfplotsset{compat = newest}
\usetikzlibrary{intersections}
\usetikzlibrary{patterns}
\usepgfplotslibrary{fillbetween}

\newcommand{\sgn}{\text{sgn}}

\author{Karsten Lehmann\\Mat. Nr 4935758}
\date{SoSe 2022}
\title{Hausaufgaben Blatt 01\\Lineare Algebra - Weiterführende Konzepte}

\setlength{\headheight}{26pt}
\pagestyle{fancy}
\fancyhf{}
\lhead{\thetitle}
\rhead{\theauthor}
\lfoot{\thedate}
\rfoot{Seite \thepage}

\begin{document}
\paragraph{Aufgabe 6} Beweisen Sie Lemma 7.50: Sei $K$ ein Körper.
Seien $A$ und $B$ beides $K$-Algebren und $\varphi \colon A \to B$ ein
surjektiver $K$-Algebrahomomorphismus.
Dann ist $\varphi$ unitär.
Insbesondere ist jeder Isomorphismus von $K$-Algebren schon unitär.

\subparagraph{Lsg.} $\varphi \colon A \to B$ ist ein $K$-Algebrahomomorphismus
wenn $\varphi$ eine lineare Abbildung zwischen $A$ und $B$ ist, sowie für alle
$a, b \in A$ gilt $\varphi\qty\big(ab) = \varphi\qty\big(a)\varphi\qty\big(b)$.
Der $K$-Algebrahomomorphismus $\varphi$ heißt unitär, wenn
$\varphi\qty\big(1_A) = 1_B$ gilt.
Weiter heißt der $K$-Algebrahomomorphismus $\varphi$ surjektiv, wenn für jedes
Element $b \in B$ ein Element $a \in A$ existiert mit
$\varphi\qty\big(a) = b$ existiert. \\

Sei nun $b \in B$ beliebig gewählt.
Da $\varphi$ surjektiv ist, existiert ein $a \in A$ mit
$\varphi\qty\big(a) = b$.
Nun ist auch $\varphi\qty\big(1_A)\varphi\qty\big(a) = \varphi\qty\big(1_A)b$.
Aus der Definition des $K$-Algebrahomomorphismus folgt
$\varphi\qty\big(1_A)\varphi\qty\big(a) = \varphi\qty\big(1_A \cdot a)
= \varphi\qty\big(a) = b$.
Also ist $\varphi\qty\big(1_A) \cdot b = b$ für jedes beliebige $b \in B$.

$\Rightarrow \varphi\qty\big(1_A) = 1_B$.

Ein $K$-Algebrahomomorphismus $f \colon A \to B$ heißt Isomorphismus
von $K$-Algebren, wenn $f$ bijektiv - also injektiv und surjektiv ist.
Wie eben gezeigt folgt aus der Surjektivität eines $K$-Algebrahomomorphismus
$f$ bereits, dass er unitär ist.

$\Rightarrow$ jeder Isomorphismus von $K$-Algebren ist unitär.

\paragraph{Aufgabe 7} In dieser Aufgabe beweisen Sie Lemma 8.34.
Seien $R$ ein kommutativer Ring und $k, l \in \mathbb{N}$.
\begin{enumerate}[(i)]
\item Seien $A \in \mathcal{M}_{k \times k}\qty\big(R)$,
  $C \in \mathcal{M}_{l \times l}\qty\big(R)$ und
  $B \in \mathcal{M}_{l \times k}\qty\big(R)$.
  Sei außerdem $n \coloneqq k + l$.
  Nehmen wir an, eine Matrix $D \in \mathcal{M}_{n \times n}\qty\big(R)$
  ist in Blockform gegeben als
  \[
    D = \begin{pmatrix}
      A & \textbf{0} \\
      B & C \\
    \end{pmatrix}
  \]
  Zeigen Sie, dass $\det\qty\big(D) = \det\qty\big(A)\det\qty\big(C)$ gilt.

  \subparagraph{Lsg.} Sei $d_{i, j}$ der $i$-te Eintrag aus der $j$-ten Spalte
  von $D$.
  Dann ist $d_{i, j} = 0$, falls $k < i \leq n$ und $1 \leq j \leq k$.
  Generell ist
  \begin{flalign*}
    \det\qty\big(D) &= \sum_{\sigma \in S_n} \sgn\qty\big(\sigma)
    d_{\sigma(1), 1} \ldots d_{\sigma(n), n} & \\
    &= \sum_{\sigma \in S_n} \sgn\qty\big(\sigma)
    d_{\sigma(1), 1} \ldots d_{\sigma(k), k}d_{\sigma(k + 1), k + 1} \ldots d_{\sigma(n), n}
  \end{flalign*}
  Nun interessieren nur Elemente dieser Summe, welche ungleich $0$ sind.
  Damit ein Produkt aus dieser Summe ungleich $0$ sein kann, muss
  $\sigma\qty\big(j) \leq k$ gelten für $j \leq k$.
  (anderenfalls wäre $d_{\sigma(j), j} = 0$).
  Da für $a, b \in 1 .. n$ und $\sigma \in S_n$ gilt $\sigma(a) \ne \sigma(b)$
  folgt auch $\sigma\qty\big(j) > k$ gelten für $j > k$.
  Somit ergibt sich
  \begin{flalign*}
    \det\qty\big(D) &= \sum_{\rho \in S_k, \tau \in \text{Sym}(\qty{k + 1, n})}
    \sgn\qty\big(\rho) \cdot \sgn\qty\big(\tau)
    d_{\rho(1), 1} \ldots d_{\rho(k), k}
    d_{\tau(k + 1), k + 1} \ldots d_{\tau(n), n} \\
    &= \qty(
      \sum_{\rho \in S_k} \sgn\qty\big(\rho) d_{\rho(1), 1} \ldots d_{\rho(k), k}
    ) \cdot \qty(
      \sum_{\tau \in \text{Sym}(\qty{k + 1, n})}
      \sgn\qty\big(\tau) d_{\tau(k + 1), k + 1} \ldots d_{\tau(n), n}
    ) \\
    &= \det\qty\big(A) \cdot \det\qty\big(C)
  \end{flalign*}

\item Seien $A \in \mathcal{M}_{k \times k}\qty\big(R)$,
  $C \in \mathcal{M}_{l \times l}\qty\big(R)$ und
  $B \in \mathcal{M}_{k \times l}\qty\big(R)$.
  Sei außerdem $n \coloneqq k + l$.
  Nehmen wir an, eine Matrix $D \in \mathcal{M}_{n \times n}\qty\big(R)$
  ist in Blockform gegeben als
  \[
    D = \begin{pmatrix}
      A & B \\
      \textbf{0} & C \\
    \end{pmatrix}
  \]
  Zeigen Sie, dass $\det\qty\big(D) = \det\qty\big(A)\det\qty\big(C)$ gilt.

  \subparagraph{Lsg.} Nach Satz 8.26 der Vorlesung gilt für eine $n \times n$
  Matrix $A$, dass $\det\qty\big(A) = \det\qty\big(A^t)$.
  Nun ist $D^t = \begin{pmatrix}A^t & 0 \\ B^t & C^t\end{pmatrix}$ und
  nach (i) $\det\qty\big(D^t) = \det\qty\big(A^t)\det\qty\big(C^t)$.

  Aus erneuter Anwendung von Satz 8.26 folgt die Behauptung.
\end{enumerate}
\end{document}