\documentclass{scrreprt}

\usepackage{aligned-overset}
\usepackage{amsmath}
\usepackage{amssymb}
\usepackage{bm}
\usepackage[shortlabels]{enumitem}
\usepackage{hyperref}
\usepackage[utf8]{inputenc}
\usepackage{multicol}
\usepackage{mathtools}
\usepackage{physics}
\usepackage{tabularx}
\usepackage{titling}
\usepackage{fancyhdr}
\usepackage{xfrac}
\usepackage[dvipsnames]{xcolor}
\usepackage{pgfplots}

\pgfplotsset{compat = newest}
\usetikzlibrary{intersections}
\usetikzlibrary{patterns}
\usepgfplotslibrary{fillbetween}

\newcommand{\Grad}{\text{Grad}}
\newcommand{\Spur}{\text{Spur}}

\author{Karsten Lehmann\\Mat. Nr 4935758}
\date{SoSe 2022}
\title{Hausaufgaben Blatt 05\\Lineare Algebra - Weiterführende Konzepte}

\setlength{\headheight}{26pt}
\pagestyle{fancy}
\fancyhf{}
\lhead{\thetitle}
\rhead{\theauthor}
\lfoot{\thedate}
\rfoot{Seite \thepage}

\begin{document}
\paragraph{Aufgabe 5}
\begin{enumerate}[(i)]
\item Sei
  \[
    B \coloneqq \begin{pmatrix}
      0 & 1 \\
      -1 & -1 \\
    \end{pmatrix} \in \mathcal{M}_{n \times n}\qty\big(\mathbb{F}_7)
  \]
  wobei $\mathbb{F}_7 = \mathbb{Z}/7$.
  Entscheiden Sie, ob $B$ diagonalisierbar ist.
  Falls dies der Fall ist, dann finden Sie eine Basis von
  $\qty\big(\mathbb{F}_7)^2$, die aus Eigenvektoren von $B$
  besteht und eine invertierbare Matrix $S$, sodass $S^{-1}BS$ eine
  Diagonalmatrix ist.

  \subparagraph{Lsg.} Das charakteristische Polynom der Matrix ist
  \begin{flalign*}
    \det\qty(XI_n - B) &= \det\qty(
      \begin{pmatrix}
        X & 0 \\
        0 & X \\
      \end{pmatrix} - \begin{pmatrix}
        0 & 1 \\
        -1 & -1 \\
      \end{pmatrix}
    ) & \\
    &= \det\begin{pmatrix}
      X & -1 \\
      1 & X + 1 \\
    \end{pmatrix} \\
    &= X^2 + X + 1 = \qty\big(X - 4)\qty\big(X - 2)
  \end{flalign*}
  Nach Satz 10.41 (a) der Vorlesung (\emph{``Wenn das charakteristische
    Polynom von $A$ komplett in verschiedene Linearfaktoren zerfällt, so ist
    $A$ diagonalisierbar''}), ist die Matrix $B$ diagonalisierbar.

  Außerdem sind nach Satz 10.13 der Vorlesung (\emph{``Die Eigenwerte einer
    Matrix $A \in \mathcal{M}_{n \times n}\qty\big(K)$ sind die Nullstellen
    des zugehörigen charakteristischen Polynoms.''}) $\lambda_1 = 4$ und
  $\lambda_2 = 2$ die Eigenwerte von $A$.

  Seien nun $A_{\lambda_1} = 4 \cdot I_n - B = \begin{pmatrix}
    4 & -1 \\
    1 & 5 \\
  \end{pmatrix}$ und
  $A_{\lambda_2} = 2 \cdot I_n - B = \begin{pmatrix}
    2 & -1 \\
    1 & 3 \\
  \end{pmatrix}$.
  Nach Anwendung des Gauß-Verfahrens bleiben übrig:
  \[
    \begin{pmatrix}
      4 & -1 \\
      1 & 5 \\
    \end{pmatrix}
    \overset{Z.2 = 3 \cdot Z.2 + Z.1}\leadsto
    \begin{pmatrix}
      4 & -1 \\
      0 & 0 \\
    \end{pmatrix}
  \]
  und
  \[
    \begin{pmatrix}
      2 & -1 \\
      1 & 3 \\
    \end{pmatrix}
    \overset{Z.1 = Z1 + 5 \cdot Z.2}\leadsto
    \begin{pmatrix}
      0 & 0 \\
      1 & 3 \\
    \end{pmatrix}
  \]
  Nun ist $V_{\lambda_1} = \qty{
    \begin{pmatrix}x\\y\end{pmatrix} \in \mathbb{F}_7^2
    \:{\Big |}\:
    4 \cdot x - y = 0
  } = \qty{
    \begin{pmatrix}1\\4\end{pmatrix} \cdot x
    \:{\Big |}\:
    x \in \mathbb{F}_7
  }$ und

  $V_{\lambda_2} = \qty{
    \begin{pmatrix}x\\y\end{pmatrix} \in \mathbb{F}_7^2
    \:{\Big |}\:
    x + 3 \cdot y = 0
  } = \qty{
    \begin{pmatrix}4\\1\end{pmatrix} \cdot x
    \:{\Big |}\:
    x \in \mathbb{F}_7
  }$.
   Jetzt bilden $\qty{
    \begin{pmatrix}1\\4\end{pmatrix},
    \begin{pmatrix}4\\1\end{pmatrix}
  }$ ein Basis in $\mathbb{F}_7^2$ und
  $S = \begin{pmatrix}
    1 & 4 \\
    4 & 1 \\
  \end{pmatrix}$ ist nach Satz 7.38 der Vorlesung (\emph{``Es sind äquivalent:
    $A$ ist invertierbar und die Spalten von $A$ bilden eine Basis von $K^n$''})
  invertierbar.

  Aus dem Gauß-Jacobi Verfahren folgt:
  \begin{flalign*}
    \qty(
    \begin{array}{cc|cc}
      1 & 4 & 1 & 0 \\
      4 & 1 & 0 & 1 \\
    \end{array})
    \overset{Z.2 = Z.2 + 3 \cdot Z.1}&\leadsto
    \qty(
    \begin{array}{cc|cc}
      1 & 4 & 1 & 0 \\
      0 & 6 & 3 & 1 \\
    \end{array}) & \\
    \overset{Z.1 = Z.1 + 4 \cdot Z.2}&\leadsto
    \qty(
    \begin{array}{cc|cc}
      1 & 0 & 6 & 4 \\
      0 & 6 & 3 & 1 \\
    \end{array}) \\
    \overset{Z.2 = 6 \cdot Z.2}&\leadsto
    \qty(
    \begin{array}{cc|cc}
      1 & 0 & 6 & 4 \\
      0 & 1 & 4 & 6 \\
    \end{array})
  \end{flalign*}
  $S^{-1} = \begin{pmatrix}
    6 & 4 \\
    4 & 6 \\
  \end{pmatrix}$.
  Schließlich ist
  \begin{flalign*}
    S^{-1} \cdot B \cdot S &= \begin{pmatrix}
      6 & 4 \\
      4 & 6 \\
    \end{pmatrix} \cdot \begin{pmatrix}
      0 & 1 \\
      -1 & -1 \\
    \end{pmatrix} \cdot \begin{pmatrix}
      1 & 4 \\
      4 & 1 \\
    \end{pmatrix} \\
    &= \begin{pmatrix}
      6 \cdot 0 + 4 \cdot -1 & 6 \cdot 1 + 4 \cdot -1 \\
      4 \cdot 0 + 6 \cdot -1 & 4 \cdot 1 + 6 \cdot -1 \\
    \end{pmatrix} \cdot \begin{pmatrix}
      1 & 4 \\
      4 & 1 \\
    \end{pmatrix} \\
    &= \begin{pmatrix}
      -4 & 2 \\
      -6 & -2 \\
    \end{pmatrix} \cdot \begin{pmatrix}
      1 & 4 \\
      4 & 1 \\
    \end{pmatrix} \\
    &= \begin{pmatrix}
      -4 \cdot 1 + 2 \cdot 4 & -4 \cdot 4 + 2 \cdot 1 \\
      -6 \cdot 1 + -2 \cdot 4 & -6 \cdot 4 + -2 \cdot 1 \\
    \end{pmatrix} \\
    &= \begin{pmatrix}
      4 & 0 \\
      0 & 2 \\
    \end{pmatrix}
  \end{flalign*}
  eine Diagonalmatrix mit den Eigenwerten von $B$ als Diagonaleneinträgen.

\item Sei
  \[
    C \coloneqq \begin{pmatrix}
      0 & 1 \\
      1 & 0 \\
    \end{pmatrix} \in \mathcal{M}_{n \times n}\qty\big(\mathbb{F}_2)
  \]
  wobei $\mathbb{F}_2 = \mathbb{Z}/2$.
  Entscheiden Sie, ob $C$ diagonalisierbar ist.
  Falls dies der Fall ist, dann finden Sie eine Basis von
  $\qty\big(\mathbb{F}_2)^2$, die aus Eigenvektoren von $C$
  besteht und eine invertierbare Matrix $S$, sodass $S^{-1}CS$ eine
  Diagonalmatrix ist.

  \subparagraph{Lsg.} Das charakteristische Polynom der Matrix ist
  \begin{flalign*}
    \det\qty(XI_n - C) &= \det\qty(
      \begin{pmatrix}
        X & 0 \\
        0 & X \\
      \end{pmatrix} - \begin{pmatrix}
        0 & 1 \\
        1 & 0 \\
      \end{pmatrix}
    ) & \\
    &= \det\begin{pmatrix}
      X & -1 \\
      -1 & X \\
    \end{pmatrix} \\
    &= X^2 + 1 = \qty\big(X - 1)\qty\big(X - 1)
  \end{flalign*}
  Nach Satz 10.13 der Vorlesung ist nun $\lambda = 1$ der einzige
  Eigenwert der Matrix $C$.

  Sei nun $A_{\lambda} = 1 \cdot I_n - C = \begin{pmatrix}
    1 & -1 \\
    -1 & 1 \\
  \end{pmatrix}$.
  Offensichtlich ist der Eigenraum $V_{\lambda} = \qty{
    \begin{pmatrix}0\\0\end{pmatrix}, \begin{pmatrix}1\\1\end{pmatrix}}$ und
  $\begin{pmatrix}1\\1\end{pmatrix}$ der einzige Eigenvektor von $C$.
  Nun ist $\qty{\begin{pmatrix}1\\1\end{pmatrix}}$ keine Basis von
  $\mathbb{F}_2^2$ und damit ist $C$ nicht diagonalisierbar.

\item Sei
  \[
    \varphi \colon \mathbb{R}^2 \to \mathbb{R}^2,
    \begin{pmatrix}x\\y\end{pmatrix} \mapsto
    \begin{pmatrix}x\\x + 2y\end{pmatrix}
  \]
  Entscheiden Sie, ob $\varphi$ diagonalisierbar ist.
  Falls dies der Fall ist, dann finden Sie eine Basis $\mathcal{B}$ von
  $\mathbb{R}^2$, sodass $\text{Mat}\qty\big(\varphi, \mathcal{B})$ eine
  Diagonalmatrix ist.

  \subparagraph{Lsg.} Sei $B$ die Standardbasis von $\mathbb{R}^2$.
  Dann ist
  $\text{Mat}\qty\big(\varphi, B) = \begin{pmatrix}1&0\\1&2\end{pmatrix}$.
  Das charakteristische Polynom ist
  \begin{flalign*}
    \det\qty(
    \begin{pmatrix}
      X & 0 \\
      0 & X \\
    \end{pmatrix} - \begin{pmatrix}
      1 & 0 \\
      1 & 2 \\
    \end{pmatrix}
    ) &= \det \begin{pmatrix}
      X - 1 & 0 \\
      -1 & X - 2 \\
    \end{pmatrix} \\
    &= \qty\big(X - 1)\qty\big(X - 2)
  \end{flalign*}
  Nach Satz 10.41 (a) der Vorlesung ist die Matrix diagonalisierbar und
  nach Satz 10.13 der Vorlesung sind $\lambda_1 = 1$ und
  $\lambda_2 = 2$ die Eigenwerte der Matrix.
  Seien nun $A_{\lambda_1}
  = 1 \cdot I_n - \text{Mat}\qty\big(\varphi, B) = \begin{pmatrix}
    0 & 0 \\
    -1 & 1 \\
  \end{pmatrix}$ und
  $A_{\lambda_2}
  = 2 \cdot I_n - \text{Mat}\qty\big(\varphi, B) = \begin{pmatrix}
    1 & 0 \\
    -1 & 0 \\
  \end{pmatrix}$.
  Damit sind
  $V_{\lambda_1} = \qty{
    \begin{pmatrix}
      1 \\
      -1 \\
    \end{pmatrix} \cdot x
    \:{\Big |}\:
    x \in \mathbb{R}
  }$ mit $\begin{pmatrix}1\\-1\end{pmatrix}$ als Basis und
  $V_{\lambda_2} = \qty{
    \begin{pmatrix}
      0 \\
      1 \\
    \end{pmatrix} \cdot x
    \:{\Big |}\:
    x \in \mathbb{R}
  }$ mit $\begin{pmatrix}0\\1\end{pmatrix}$ als Basis..
  Sei nun $\mathcal{B} = \qty{
    \begin{pmatrix}1\\-1\end{pmatrix}, \begin{pmatrix}0\\1\end{pmatrix}
  }$, dann ist $\mathcal{B}$ eine Basis von $\mathbb{R}^2$ und nach Bemerkung
  10.34 der Vorlesung ist
  \[
    \text{Mat}\qty\big(\varphi, \mathcal{B}) = \begin{pmatrix}
      1 & 0 \\
      0 & 2 \\
    \end{pmatrix}.
  \]
\end{enumerate}

\paragraph{Aufgabe 6} Seien $K$ ein Körper, $n \in \mathbb{N}$ und
$v_1, \ldots, v_n$ eine Basis von $K^n$.
Seien außerdem $\lambda_1, \ldots, \lambda_n \in K$.
Bemerkung 10.39 (ii) der Vorlesung gibt Ihnen ein Rezept, um eine Matrix
$A \in \mathcal{M}_{n \times n}\qty\big(K)$ so zu konstruieren, dass $v_i$
ein Eigenvektor von $A$ mit Eigenwert $\lambda_i$ ist für jedes
$i \in \qty\big{1, \ldots, n}$.
Ist die Matrix $A$ eindeutig?

\noindent
Genauer: Wenn $A, A' \in \mathcal{M}_{n \times n}\qty\big(K)$ Matrizen sind,
sodass $v_i$ für jedes $i \in \qty\big(1, \ldots, n)$ ein Eigenvektor von
$A$ und $A'$ mit Eigenwert $\lambda_i$ ist, gilt dann $A = A'$?

\noindent
Geben Sie entweder einen Beweis an oder ein Beispiel, in dem $A \ne A'$ gilt.

\subparagraph{Lsg.} Seien $A, A' \in \mathcal{M}_{n \times n}\qty\big(K)$ zwei
Matrizen, sodass $A \cdot v_i = A' \cdot v_i = \lambda_i \cdot v_i$ für
$i \in \qty\big{1, \ldots, n}$.
Per Voraussetzung sind $v_1, \ldots, v_n$ eine Basis von $K^n$, es folgt aus
Korollar 10.38 der Vorlesung (\emph{``Es sind äquivalent: $A$ ist
  diagonalisierbar und $K^n$ besitzt eine Basis aus Eigenvektoren von $A$''})
sind $A$ und $A'$ diagonalisierbar.
Sei nun $D$ die Matrix, deren Diagonaleneinträge die Eigenwerte von $A$ sind,
also
\[
  D = \begin{pmatrix}
    \lambda_1 & 0 & \ldots & 0 \\
    0 & \lambda_2 & \ldots & 0 \\
    \vdots & \vdots & \ddots & \vdots \\
    0 & 0 & \ldots & \lambda_n \\
  \end{pmatrix}
\]
und $S$ die Matrix deren Spaltenvektoren die Eigenvektoren von $A$
sind.
Nach Satz 10.37 der Vorlesung (\emph{``Wenn $A$ und $S$ beides $n \times n$
  Matrizen mit Einträgen in $K$ sind und $S$ invertierbar ist, dann ist
  $S^{-1} \cdot A \cdot S = D$ genau dann diagonal wenn die Spalten von $S$
  Eigenwerte von $A$ sind, deren zugehörigen Eigenwerte die Diagonaleneinträge
  von $D$ sind.''})
ist dann $S^{-1} \cdot A \cdot S = D$.
Da $A'$ per Annahme die selben Eigenvektoren und Eigenwerte besitzt gilt
außerdem $S^{-1} \cdot A' \cdot S = D$.
Somit folgt $S^{-1} \cdot A \cdot S = D = S^{-1} \cdot A' \cdot S$ und $A = A'$.

\noindent
$\Rightarrow$ die Matrix $A$ ist eindeutig.
\end{document}