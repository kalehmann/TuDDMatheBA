\documentclass{scrreprt}

\usepackage{aligned-overset}
\usepackage{amsmath}
\usepackage{amssymb}
\usepackage{bm}
\usepackage[shortlabels]{enumitem}
\usepackage{hyperref}
\usepackage[utf8]{inputenc}
\usepackage{multicol}
\usepackage{mathtools}
\usepackage{physics}
\usepackage{tabularx}
\usepackage{titling}
\usepackage{fancyhdr}
\usepackage{xfrac}
\usepackage[dvipsnames]{xcolor}
\usepackage{pgfplots}

\pgfplotsset{compat = newest}
\usetikzlibrary{intersections}
\usetikzlibrary{patterns}
\usepgfplotslibrary{fillbetween}

\newcommand{\sgn}{\text{sgn}}

\author{Karsten Lehmann\\Mat. Nr 4935758}
\date{SoSe 2022}
\title{Hausaufgaben Blatt 02\\Lineare Algebra - Weiterführende Konzepte}

\setlength{\headheight}{26pt}
\pagestyle{fancy}
\fancyhf{}
\lhead{\thetitle}
\rhead{\theauthor}
\lfoot{\thedate}
\rfoot{Seite \thepage}

\begin{document}
\paragraph{Aufgabe 6}
\begin{enumerate}[(i)]
\item Sei
  \[
    A \coloneqq \begin{pmatrix}
      0 & 0 & 3X & 2 \\
      X + 2 & X - 5 & 5X & X \\
      0 & X & 2 & 1 \\
      0 & 0 & 1 & 0 \\
    \end{pmatrix} \in
    \mathcal{M}_{4 \times 4}\qty\big(\mathbb{R}\qty[X])
  \]
  Bestimmen Sie $\det\qty\big(A)$.
  Ist $A$ invertierbar?
  Ist $\det\qty\big(A)$ irreduzibel?

  \subparagraph{Lsg.} Aus den Bemerkungen 8.25 und 8.29 der Vorlesung geht
  hervor, dass
  \begin{flalign*}
    \det \begin{pmatrix}
      0 & 0 & 3X & 2 \\
      X + 2 & X - 5 & 5X & X \\
      0 & X & 2 & 1 \\
      0 & 0 & 1 & 0 \\
    \end{pmatrix} &= -\det \begin{pmatrix}
      X + 2 & X - 5 & 5X & X \\
      0 & 0 & 3X & 2 \\
      0 & X & 2 & 1 \\
      0 & 0 & 1 & 0 \\
    \end{pmatrix} &\\
    &= \det \begin{pmatrix}
      X + 2 & X - 5 & 5X & X \\
      0 & X & 2 & 1 \\
      0 & 0 & 3X & 2 \\
      0 & 0 & 1 & 0 \\
    \end{pmatrix} \\
    &= -\det \begin{pmatrix}
      X + 2 & X - 5 & X & 5X \\
      0 & X & 1 & 2 \\
      0 & 0 & 2 & 3X \\
      0 & 0 & 0 & 1 \\
    \end{pmatrix}
  \end{flalign*}
  Aus Lemma 8.24 der Vorlesung (\emph{``Falls eine Matrix
    $A \in \mathcal{M}_{n \times n}\qty\big(R)$ eine obere Dreiecksmatrix ist,
    dann gilt $\det A = a_{11}a_{22}\ldots a_{nn}$''}) folgt nun
  \[
    \det A = -\qty\big(X + 2) \cdot X \cdot 2 \cdot 1 = -2X^2 - 4X
  \]
  Da die Determinante $-2X^2 - 4X$ als Polynom mit Grad 2 nicht invertierbar ist,
  ist auch $A$ nicht invertierbar.

  Nach Korollar 9.38 der Vorlesung (\emph{``Sei $f \in K\qty[X]$.
    Falls $f$ Grad 2 oder 3 hat, so ist $f$ genau dann nicht irreduzibel,
    wenn $f$ eine Nullstelle in $K$ besitzt''}) ist $\det\qty\big(A)$
  nicht irreduzibel mit $\qty\Big(-2X^2 - 4X)(0) = 0$.

\item Beweisen Sie Lemma 10.12: Seien $K$ ein Körper, $n \in \mathbb{N}$ und
  $B = \qty\big(b_{ij}) \in \mathcal{M}_{n \times n}\qty\big(K\qty[X])$.
  Für $\lambda \in K$ sei
  $B\qty\big(\lambda) \in \mathcal{M}_{n \times n}\qty\big(K)$ die Matrix, deren
  $\qty\big(i, j)$-Eintrag $b_{ij}\qty\big(\lambda)$ ist.
  Für $f = \det\qty\big(B) \in K\qty[X]$ gilt dann
  \[
    f\qty\big(\lambda) = \det\qty\big(B\qty(\lambda))
  \]

  \subparagraph{Lsg.} Es ist
  $f\qty\big(\lambda) = \qty\Big(\det\qty\big(B))\qty\big(\lambda)$ und nach
  der Definition der Determinante ist
  $\qty\Big(\det\qty\big(B))\qty\big(\lambda) =
  \qty\Big(
    \sum_{\sigma \in S_n}
    \sgn\qty\big(\sigma)
    b_{\sigma(1)1} \ldots b_{\sigma(n)n}
  )\big(\lambda)$.

  Nun ist die Abbildung
  $\Phi_{\lambda} \colon K\qty[X] \to K, a \mapsto a\qty\big(\lambda)$
  nach Lemma 9.13 der Vorlesung (\emph{``Sei $A$ eine $K$-Algebra und sei
    $a \in A$ fest gewählt.
    Dann ist die Abbildung
    $\Phi_a \colon K\qty[X] \to A, f \mapsto f\qty\big(a)$
    ein unitaler $K$-Algebra Homomorphismus.''})
  ein unitaler $K$-Algebra Homomorphismus.

  Aus der Definition eines  $K$-Algebra Homomorphismus und der linearen
  Abbildung folgt
  \[
    \qty\Big(
      \sum_{\sigma \in S_n}
      \sgn\qty\big(\sigma)
      b_{\sigma(1)1} \ldots b_{\sigma(n)n}
    )\big(\lambda) =
    \sum_{\sigma \in S_n}
    \sgn\qty\big(\sigma)
    b_{\sigma(1)1}\qty\big(\lambda) \ldots b_{\sigma(n)n}\qty\big(\lambda)
    = \det\qty\big(B(\lambda))
  \]

\item Nutzen Sie (i) und (ii), um die Determinante der Matrix
  \[
    B \coloneqq \begin{pmatrix}
      0 & 0 & 4 & 2 \\
      \frac{10}{3} & -\frac{11}{3} & \frac{20}{3} & \frac{4}{3} \\
      0 & \frac{4}{3} & 2 & 1 \\
      0 & 0 & 1 & 0 \\
    \end{pmatrix} \mathcal{M}_{4 \times 4}\qty\big(\mathbb{R})
  \]
  zu finden.

  \subparagraph{Lsg.} Offensichtlich ist $B = A\qty(\frac{4}{3})$.
  Aus (ii) folgt nun
  $\det\qty\big(B) = \qty\Big(\det\qty\big(A))\qty(\frac{4}{3})$.

  Mit dem Ergebnis aus (i):
  \[
    \det\qty\big(B) = -2\qty(\frac{4}{3})^2 - 4\frac{4}{3}
    = -\frac{80}{9}
  \]
\end{enumerate}
\end{document}