\documentclass{scrreprt}

\usepackage{aligned-overset}
\usepackage{amsmath}
\usepackage{amssymb}
\usepackage{bm}
\usepackage[shortlabels]{enumitem}
\usepackage{hyperref}
\usepackage[utf8]{inputenc}
\usepackage{multicol}
\usepackage{mathtools}
\usepackage{physics}
\usepackage{tabularx}
\usepackage{titling}
\usepackage{fancyhdr}
\usepackage{xfrac}
\usepackage[dvipsnames]{xcolor}
\usepackage{pgfplots}

\pgfplotsset{compat = newest}
\usetikzlibrary{intersections}
\usetikzlibrary{patterns}
\usepgfplotslibrary{fillbetween}

\newcommand{\Bild}{\text{Bild}}
\newcommand{\Dim}{\text{Dim}}
\newcommand{\End}{\text{End}}
\newcommand{\id}{\text{id}}
\newcommand{\Grad}{\text{Grad}}
\newcommand{\Mat}{\text{Mat}}
\newcommand{\Rank}{\text{Rank}}
\newcommand{\Spur}{\text{Spur}}

\author{Karsten Lehmann\\Mat. Nr 4935758}
\date{SoSe 2022}
\title{Hausaufgaben Blatt 11\\Lineare Algebra - Weiterführende Konzepte}

\setlength{\headheight}{26pt}
\pagestyle{fancy}
\fancyhf{}
\lhead{\thetitle}
\rhead{\theauthor}
\lfoot{\thedate}
\rfoot{Seite \thepage}

\begin{document}
\paragraph{Aufgabe 5} Sei $\qty\big{0_V} \ne V$ ein endlich-dimensionaler
Vektorraum über $\mathbb{R}$, sei $\mathcal{B}'$ eine Basis von $V$ und
$\varphi \in \End\qty\big(V)$.
Bestimmen Sie für die folgenden Fälle jeweilseine Jordansche Normalform von
$\varphi$.
\begin{enumerate}[(i)]
\item $\Mat\qty\big(\varphi, \mathcal{B}') = \begin{pmatrix}
    1 & 1 \\
    0 & 1 \\
  \end{pmatrix}$

  \subparagraph{Lsg.} Es ist $f_{\varphi} = \det\qty\big(
    X \cdot I_2 - \Mat\qty\big(\varphi, \mathcal{B}')
  ) = \qty\big(X - 1)^2$.
  Durch Caley-Hamilton und Probieren folgt $m_{\varphi} = f_{\varphi}$.
  Per Bemerkung 13.42 der Vorlesung besitzt eine Jordansche Normalform einen
  Block der Form $J\qty\big(1, 2)$.
  Somit ist die Jordansche Normalform von $\varphi$
  \[
    \begin{pmatrix}
      1 & 0 \\
      1 & 1 \\
    \end{pmatrix}
  \]

\item $m_{\varphi} = \qty\big(X - 1)^2 \qty\big(X - 2)^2\qty\big(X - 3)^2$ und
  $f_{\varphi} = \qty\big(X - 1)^2 \qty\big(X - 2)^2\qty\big(X - 3)^3$.

  \subparagraph{Lsg.} Da $\Grad\qty\big(f_{\varphi}) = 7$, ist die Jordansche
  Normalform von $\varphi$ ein $7 \times 7$ Matrix.
  Nach Bemerkung 13.42 der Vorlesung besitzt eine Jordansche Normalform
  mindestens einen Block der Form $J\qty\big(1, 2)$, mindestens einen Block der
  Form $J\qty\big(2, 2)$ und mindestens einen Block der Form $J\qty\big(3, 2)$.
  Der verbleibende Block ist entweder $J\qty\big(1, 1)$, $J\qty\big(2, 1)$ oder
  $J\qty\big(3, 1)$.
  Mögliche Normalformen sind somit (bis auf Vertauschung der Blöcke)
  \[
    N_a \coloneqq \begin{pmatrix}
      1 & 0 & 0 & 0 & 0 & 0 & 0 \\
      1 & 1 & 0 & 0 & 0 & 0 & 0 \\
      0 & 0 & 2 & 0 & 0 & 0 & 0 \\
      0 & 0 & 1 & 2 & 0 & 0 & 0 \\
      0 & 0 & 0 & 0 & 3 & 0 & 0 \\
      0 & 0 & 0 & 0 & 1 & 3 & 0 \\
      0 & 0 & 0 & 0 & 0 & 0 & a \\
    \end{pmatrix}, a \in \qty\big{1, 2, 3}
  \]
  Per Definition 13.40 der Vorlesung ist die Jordansche Normalform von $\varphi$
  ähnlich zu $\Mat\qty\big(\varphi, \mathcal{B}')$ und nach Lemma 10.15
  der Vorlesung haben ähnliche Matrizen das gleiche charakteristische Polynom.
  Somit ist die Jordansche Normalform von $\varphi$
  \[
    N_3 \coloneqq \begin{pmatrix}
      1 & 0 & 0 & 0 & 0 & 0 & 0 \\
      1 & 1 & 0 & 0 & 0 & 0 & 0 \\
      0 & 0 & 2 & 0 & 0 & 0 & 0 \\
      0 & 0 & 1 & 2 & 0 & 0 & 0 \\
      0 & 0 & 0 & 0 & 3 & 0 & 0 \\
      0 & 0 & 0 & 0 & 1 & 3 & 0 \\
      0 & 0 & 0 & 0 & 0 & 0 & 3 \\
    \end{pmatrix}
  \]

\newpage
\item $\Mat\qty\big(\varphi, \mathcal{B}') = \begin{pmatrix}
    3 & 0 & 0 & -1 \\
    1 & 2 & 0 & -1 \\
    0 & 0 & 2 &  0 \\
    1 & 0 & 0 &  1 \\
  \end{pmatrix}$

  \subparagraph{Lsg.} Wie in Aufgabe 7 (iii) von Übungsblatt 10 bereits
  gezeigt ist $f_{\varphi} = \qty\big(X - 2)^4$.
  Durch Caley-Hamilton und Probieren folgt $m_{\varphi} = \qty\big(X - 2)^2$.
  Nach Bemerkung 13.42 der Vorlesung besitzt eine Jordansche Normalform
  mindestens einen Block der Form $J\qty\big(2, 2)$.
  Nach dem Beweis von Lemma 13.48 der Vorlesung ist
  \[
    \dim\qty\big(V_2^{\varphi}) =
    4 - \Rank\qty\big(\Mat\qty\big(\varphi, \mathcal{B}') - 2 \cdot I_4)
    = 4 - \Rank\begin{pmatrix}
      1 & 0 & 0 & -1 \\
      1 & 0 & 0 & -1 \\
      0 & 0 & 0 &  0 \\
      1 & 0 & 0 & -1 \\
    \end{pmatrix} = 3
  \]
  Aus Lemma 13.48 der Vorlesung folgt, dass eine Jordansche Normalform
  von $\varphi$ 3 Blöcke für den Eigenwert 2 enthält, somit ist die
  (bis auf Vertauschung der Blöcke) eindeutige Jordansche Normalform von
  $\varphi$
  \[
    \begin{pmatrix}
      2 & 0 & 0 & 0 \\
      1 & 2 & 0 & 0 \\
      0 & 0 & 2 & 0 \\
      0 & 0 & 0 & 2 \\
    \end{pmatrix}
  \]
\end{enumerate}

\end{document}
