\documentclass{scrreprt}

\usepackage{aligned-overset}
\usepackage{amsmath}
\usepackage{amssymb}
\usepackage{bm}
\usepackage[shortlabels]{enumitem}
\usepackage{hyperref}
\usepackage[utf8]{inputenc}
\usepackage{multicol}
\usepackage{mathtools}
\usepackage{physics}
\usepackage{tabularx}
\usepackage{titling}
\usepackage{fancyhdr}
\usepackage{xfrac}
\usepackage[dvipsnames]{xcolor}
\usepackage{pgfplots}

\pgfplotsset{compat = newest}
\usetikzlibrary{intersections}
\usetikzlibrary{patterns}
\usepgfplotslibrary{fillbetween}

\newcommand{\End}{\text{End}}
\newcommand{\Grad}{\text{Grad}}
\newcommand{\Mat}{\text{Mat}}
\newcommand{\Spur}{\text{Spur}}

\author{Karsten Lehmann\\Mat. Nr 4935758}
\date{SoSe 2022}
\title{Hausaufgaben Blatt 08\\Lineare Algebra - Weiterführende Konzepte}

\setlength{\headheight}{26pt}
\pagestyle{fancy}
\fancyhf{}
\lhead{\thetitle}
\rhead{\theauthor}
\lfoot{\thedate}
\rfoot{Seite \thepage}

\begin{document}
\paragraph{Aufgabe 6} Seien $K$ ein Körper, $\lambda \in K, n \in \mathbb{N}$ und
$A \in \mathcal{M}_{n \times n}\qty\big(K)$.
Beweisen Sie die folgenden Aussagen:
\begin{enumerate}[(i)]
\item Wenn $\lambda$ der einzige Eigenwert von $A$ ist und außerdem $A$
  diagonalisierbar ist, dann gilt $A = \lambda \cdot I_n$.

  \subparagraph{Lsg.} Da $A$ diagonalisierbar ist, zerfällt $m_A$ vollständig in
  verschiedene Linearfaktoren.
  Außerdem sind die Eigenwerte von $A$ gleich den Nullstellen von $m_A$.
  Es folgt $m_A = \qty\big(X - \lambda)$.

  Nach Cayley-Hamilton gilt schließlich $A - \lambda \cdot I_n = 0$, es folgt
  $A = \lambda \cdot I_n$.

\item Wenn $f_A$ irreduzibel ist, dann gilt $f_A = m_A$.

  \subparagraph{Lsg.} Nach Cayley-Hamilton gilt $m_A \mid f_A$,
  dass heißt es existiert ein $g \in K\qty[X]$ mit
  $f_A = g \cdot m_A$.
  Da $f_A$ irreduzibel ist, folgt $g \in K \setminus \qty\big{0_K}$ oder
  $m_A \in K \setminus \qty\big{0_K}$.
  Da $f_A$ und $m_A$ per Definition normiert sind gilt somit $g = 1_K$ oder
  $m_A = 1_K$.

  Nun ist $\qty\big(1_K)\qty\big(A) = 1_K \ne 0$, ein Widerspruch zu
  Cayley-Hamilton.
  Es folgt $m_A \ne 1_K$ und $g = 1_K$, sowie $f_A = 1_K \cdot m_A = m_A$.
\end{enumerate}

\paragraph{Aufgabe 7} Seien $n \in \mathbb{N}$ und
$A \in \mathcal{M}_{n \times n}\qty\big(\mathbb{R})$ mit
\[
  A^3 - 3A^2 -A = -3I_n
\]
Bestimmen Sie die möglichen Eigenwerte von $A$.

\subparagraph{Lsg.} Es ist $A^3 - 3A^2 - A + 3I_n = 0$.

$\Rightarrow \qty\Big(
  \qty\big(X + 1)\qty\big(X - 1)\qty\big(X - 3)
)\qty\big(A) = 0$

$\Rightarrow m_A$ ist ein Teiler von
$\qty\big(X + 1)\qty\big(X - 1)\qty\big(X - 3)$.

$\Rightarrow$ alle Nullstellen von
$\qty\big(X + 1)\qty\big(X - 1)\qty\big(X - 3)$ sind mögliche Eigenwerte.

Mögliche Eigenwerte von $A$ sind nun $-1$, $1$ und $3$

\newpage
\paragraph{Aufgabe 8} Sei
\[
  A \coloneqq \begin{pmatrix}
    2 & 0 & 0 \\
    5 & -1 & -1 \\
    2 & 1 & -1 \\
  \end{pmatrix} \in \mathcal{M}_{3 \times 3}\qty\big(\mathbb{F}_7)
\]
wobei $\mathbb{F}_7 = \mathbb{Z}/7$.
\begin{enumerate}[(i)]
\item Berechnen Sie $f_A$ und schreiben Sie $f_A$ als Produkt zweier
  irreduzibler Polynome.

  \subparagraph{Lsg.} Es ist
  \begin{flalign*}
    \det\qty\big(X \cdot I_n - A) &= \det\begin{pmatrix}
      X - 2 & 0 & 0 \\
      -5 & X + 1 & 1 \\
      -2 & -1 & X + 1 \\
    \end{pmatrix} \\
    &= \qty\big(X - 2) \cdot \det\begin{pmatrix}
      X + 1 & 1 \\
      -1 & X + 1 \\
    \end{pmatrix} \\
    &= \qty\big(X - 2) \cdot \qty\big(X + 1)^2 + 1 \\
    &= \qty\big(X - 2) \cdot \qty\big(X^2 + 2X + 2)
  \end{flalign*}
  Nun hat $X^2 + 2X + 2$ keine Nullstelle in $\mathbb{F}_7$.
  Dabei ist $X - 2$ nach Beispiel 9.29 der Vorlesung irreduzibel
  und $X^2 + 2X + 2$ nach Korollar 9.38 der Vorlesung irreduzibel.

\item Begründen Sie anhand ihrer Lösung zu (i), dass $2$ der einzige
  Eigenwert von $A$ ist.

  \subparagraph{Lsg.} Nach Satz 10.13 der Vorlesung sind die Eigenwerte einer
  Matrix $A \in \mathcal{M}_{n \times n}\qty\big(K)$ genau die Nullstellen des
  charakteristischen Polynoms $f_A$.
  Wie in (i) bereits gesehen ist $2$ die einzige Nullstelle von $f_A$.

\item Begründen Sie, ob $A$ diagonalisierbar ist.

  \subparagraph{Lsg.} Nach Satz 10.40 (b)der Vorlesung zerfällt das
  charakteristische Polynom einer diagonalisierbaren Matrix komplett in
  Linearfaktoren.
  $f_A$ zerfällt nicht komplett in Linearfaktoren.

  $\Rightarrow A$ ist nicht diagonalisierbar.

\item Finden Sie die algebraische und die geometrische Vielfachheit des
  Eigenwerts $2$.

  \subparagraph{Lsg.} $m_{\text{alg}}\qty\big(A, 2) = 1$.
  Weiter ist $\qty\big(2 \cdot I_n - A) = \begin{pmatrix}
    0 & 0 & 0 \\
    -5 & 3 & 1 \\
    -2 & -1 & 3 \\
  \end{pmatrix}$ und
  \[
    \begin{pmatrix}
      0 & 0 & 0 \\
      -5 & 3 & 1 \\
      -2 & -1 & 3 \\
    \end{pmatrix} \leadsto \begin{pmatrix}
      2 & 3 & 1 \\
      5 & 6 & 3 \\
      0 & 0 & 0 \\
    \end{pmatrix} \overset{
      \substack{
        Z.2 = 6 \cdot Z.2 - Z.1 \\
        Z.1 = 4 \cdot Z.1 !!
      }
    }\leadsto \begin{pmatrix}
      1 & 5 & 4 \\
      0 & 5 & 3 \\
      0 & 0 & 0 \\
    \end{pmatrix} \overset{
      \substack{
        Z.1 = Z.1 - Z.2 \\
        Z.2 = 3 \cdot Z.1
      }
    }\leadsto \begin{pmatrix}
      1 & 0 & 1 \\
      0 & 1 & 2 \\
      0 & 0 & 0 \\
    \end{pmatrix}
  \]

  Es folgt $V_2^A = \qty{
    \begin{pmatrix}x\\y\\z\end{pmatrix}
    \:\middle|\:
    x,y,z \in \mathbb{F}_7, x + z = 0 \land y + 2z = 0
  } = \qty{
    \begin{pmatrix}-1\\5\\1\end{pmatrix} \cdot x
    \:\middle|\:
    x \in \mathbb{F}_7
  }$ und $\dim\qty\big(V_2^A) = 1$.

  $\Rightarrow m_{\text{geom}}\qty\big(2, A) = 1$.

\item Finden Sie das Minimalpolynom $m_A$ von $A$.
  Verwenden Sie in ihrer Argumentation Satz 9.35.

  \subparagraph{Lsg.} Das Minimalpolynom ist das eindeutige normierte Polynom
  von minimalem Grad, sodass $m_A\qty\big(A) = 0$.
  Weiter teilt $m_A$ nach Cayley-Hamilton $f_A$.

  Nach Satz 9.35 der Vorlesung (\emph{``Jedes von Null verschiedene Polynom
    $f \in K\qty[X]$ kann als Produkt $f = a \cdot f_1 \cdot \ldots \cdot f_n$
    geschrieben werden für ein $a \in K$, ein $n \in \mathbb{N}_0$ und normiert
    irreduzible Polynome $f_1, \ldots, f_n \in K\qty[X]$.
    Dabei sind $a, n$ und $f_1, \ldots, f_n$ bis auf die Reihenfolge eindeutig.
    ''}) ist die Zerlegung von $f_A$ in
  $f_A = 1 \qty\big(X - 2) \cdot \qty\big(X^2 + 2X + 2)$ eindeutig, d.h. $f_A$
  hat genau zwei Teiler, $\qty\big(X - 2)$ und $\qty\big(X^2 + 2X + 2)$.
  Da $\qty\big(X - 2)\qty\big(A) \ne 0$ und
  $\qty\big(X^2 + 2X + 2)\qty\big(A) \ne 0$ folgt $m_A = f_A$.
\end{enumerate}
\end{document}
