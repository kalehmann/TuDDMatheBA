\documentclass{scrreprt}

\usepackage{aligned-overset}
\usepackage{amsmath}
\usepackage{amssymb}
\usepackage{bm}
\usepackage[shortlabels]{enumitem}
\usepackage{hyperref}
\usepackage[utf8]{inputenc}
\usepackage{multicol}
\usepackage{mathtools}
\usepackage{physics}
\usepackage{tabularx}
\usepackage{titling}
\usepackage{fancyhdr}
\usepackage{xfrac}
\usepackage[dvipsnames]{xcolor}
\usepackage{pgfplots}

\pgfplotsset{compat = newest}
\usetikzlibrary{intersections}
\usetikzlibrary{patterns}
\usepgfplotslibrary{fillbetween}

\newcommand{\End}{\text{End}}
\newcommand{\Grad}{\text{Grad}}
\newcommand{\Mat}{\text{Mat}}
\newcommand{\Spur}{\text{Spur}}

\author{Karsten Lehmann\\Mat. Nr 4935758}
\date{SoSe 2022}
\title{Hausaufgaben Blatt 08\\Lineare Algebra - Weiterführende Konzepte}

\setlength{\headheight}{26pt}
\pagestyle{fancy}
\fancyhf{}
\lhead{\thetitle}
\rhead{\theauthor}
\lfoot{\thedate}
\rfoot{Seite \thepage}

\begin{document}
\paragraph{Aufgabe 6} Seien $K$ ein Körper, $\lambda \in K, n \in \mathbb{N}$ und
$A \in \mathcal{M}_{n \times n}\qty\big(K)$.
Beweisen Sie die folgenden Aussagen:
\begin{enumerate}[(i)]
\item Wenn $\lambda$ der einzige Eigenwert von $A$ ist und außerdem $A$
  diagonalisierbar ist, dann gilt $A = \lambda \cdot I_n$.

  \subparagraph{Lsg.} Da $A$ diagonalisierbar ist, zerfällt $m_A$ vollständig in
  verschiedene Linearfaktoren.
  Außerdem sind die Eigenwerte von $A$ gleich den Nullstellen von $m_A$.
  Es folgt $m_A = \qty\big(X - \lambda)$.

  Nach Cayley-Hamilton gilt schließlich $A - \lambda \cdot I_n = 0$, es folgt
  $A = \lambda \cdot I_n$.

\item Wenn $f_A$ irreduzibel ist, dann gilt $f_A = m_A$.

  \subparagraph{Lsg.} Nach Cayley-Hamilton gilt $m_A \mid f_A$,
  dass heißt es existiert ein $g \in K\qty[X]$ mit
  $f_A = g \cdot m_A$.
  Da $f_A$ irreduzibel ist, folgt $g \in K \setminus \qty\big{0_K}$ oder
  $m_A \in K \setminus \qty\big{0_K}$.
  Da $f_A$ und $m_A$ per Definition normiert sind gilt somit $g = 1_K$ oder
  $m_A = 1_K$.

  Nun ist $\qty\big(1_K)\qty\big(A) = 1_K \ne 0$, ein Widerspruch zu
  Cayley-Hamilton.
  Es folgt $m_A \ne 1_K$ und $g = 1_K$, sowie $f_A = 1_K \cdot m_A = m_A$.
\end{enumerate}

\paragraph{Aufgabe 7} Seien $n \in \mathbb{N}$ und
$A \in \mathcal{M}_{n \times n}\qty\big(\mathbb{R})$ mit
\[
  A^3 - 3A^2 -A = -3I_n
\]
Bestimmen Sie die möglichen Eigenwerte von $A$.

\subparagraph{Lsg.} Es ist $A^3 - 3A^2 - A + 3I_n = 0$.

$\Rightarrow \qty\Big(
  \qty\big(X + 1)\qty\big(X - 1)\qty\big(X - 3)
)\qty\big(A) = 0$

$\Rightarrow m_A$ ist ein Teiler von
$\qty\big(X + 1)\qty\big(X - 1)\qty\big(X - 3)$.

$\Rightarrow$ alle Nullstellen von
$\qty\big(X + 1)\qty\big(X - 1)\qty\big(X - 3)$ sind mögliche Eigenwerte.

Mögliche Eigenwerte von $A$ sind nun $-1$, $1$ und $3$
\end{document}
