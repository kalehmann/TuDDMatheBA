\documentclass{scrreprt}

\usepackage{aligned-overset}
\usepackage{amsmath}
\usepackage{amssymb}
\usepackage{bm}
\usepackage[shortlabels]{enumitem}
\usepackage{hyperref}
\usepackage[utf8]{inputenc}
\usepackage{multicol}
\usepackage{mathtools}
\usepackage{physics}
\usepackage{tabularx}
\usepackage{titling}
\usepackage{fancyhdr}
\usepackage{xfrac}
\usepackage[dvipsnames]{xcolor}
\usepackage{pgfplots}

\pgfplotsset{compat = newest}
\usetikzlibrary{intersections}
\usetikzlibrary{patterns}
\usepgfplotslibrary{fillbetween}

\newcommand{\End}{\text{End}}
\newcommand{\Grad}{\text{Grad}}
\newcommand{\Mat}{\text{Mat}}
\newcommand{\Spur}{\text{Spur}}

\author{Karsten Lehmann\\Mat. Nr 4935758}
\date{SoSe 2022}
\title{Hausaufgaben Blatt 07\\Lineare Algebra - Weiterführende Konzepte}

\setlength{\headheight}{26pt}
\pagestyle{fancy}
\fancyhf{}
\lhead{\thetitle}
\rhead{\theauthor}
\lfoot{\thedate}
\rfoot{Seite \thepage}

\begin{document}
\paragraph{Aufgabe 6}
\begin{enumerate}[(a)]
\item Bestimmen Sie die Minimalpolynome der folgenden Matrizen
  bzw. linearen Abbildungen.
  \begin{enumerate}[(i)]
  \item $B \coloneqq \begin{pmatrix}
      0 & 1 \\
      -1 & -1 \\
    \end{pmatrix} \in \mathcal{M}_{2 \times 2}\qty\big(\mathbb{F}_7)$,
    wobei $\mathbb{F}_7 = \mathbb{Z}/7$,
    $f_B = \qty\big(X - 2)\qty\big(X - 4)$.

    \subparagraph{Lsg.} Aus dem charakteristischen Polynom der Matrix $B$ lassen
    sich die Eigenwerte $\lambda_1 = 2$ und $\lambda_2 = 4$ ablesen.
    Nach Satz 11.14 der Vorlesung (\emph{``Ebenso sind die Eigenwerte der Matrix
      $A$ genau die Nullstellen des Minimalpolynoms''}) und dem Satz von
    Cayley-Hamilton (\emph{``$m_A$ teilt $f_A$''}) folgt
    $m_B = \qty\big(X - 2)\qty\big(X - 4)$.

  \item $C \coloneqq \begin{pmatrix}
      0 & 1 \\
      1 & 0 \\
    \end{pmatrix} \in \mathcal{M}_{2 \times 2}\qty\big(\mathbb{F}_2)$,
    wobei $\mathbb{F}_2 = \mathbb{Z}/2$,
    $f_C = X^2 + 1$, $C$ ist nicht diagonalisierbar.

    \subparagraph{Lsg.} Es ist $f_C = X^2 + 1 = \qty\big(X + 1)\qty\big(X + 1)$.
    Aus Satz 11.14 der Vorlesung und dem Satz von Cayley-Hamilton folgt
    $m_C = X + 1$ oder $m_c = \qty\big(X + 1)^2$.

    Durch Einsetzen von $C$ in die beiden Kandidaten für $m_c$ erhält man
    \[
      \begin{pmatrix}
        0 & 1 \\
        1 & 0 \\
      \end{pmatrix} + \begin{pmatrix}
        1 & 0 \\
        0 & 1 \\
      \end{pmatrix} = \begin{pmatrix}
        1 & 1 \\
        1 & 1 \\
      \end{pmatrix}
    \]
    und
    \[
      \qty(\begin{pmatrix}
        0 & 1 \\
        1 & 0 \\
      \end{pmatrix} + \begin{pmatrix}
        1 & 0 \\
        0 & 1 \\
      \end{pmatrix})^2 = \begin{pmatrix}
        0 & 0 \\
        0 & 0 \\
      \end{pmatrix}
    \]
    $\Rightarrow m_C = \qty\big(X + 1)^2$

  \item $\varphi \colon \mathbb{R}^2 \to \mathbb{R}^2,
    \begin{pmatrix}x\\y\end{pmatrix} \mapsto
    \begin{pmatrix}x\\x + 2y\end{pmatrix}$,
    $f_{\varphi} = \qty\big(X - 1)\qty\big(X - 2)$

    \subparagraph{Lsg.} Nach Satz 11.14 der Vorlesung und dem Satz von
    Cayley-Hamilton ist $m_{\varphi} = \qty\big(X - 1)\qty\big(X - 2)$.
  \end{enumerate}

\item Sei $V$ ein endlich-dimensionaler Vektorraum über $\mathcal{C}$ und
  $\varphi \in \End\qty\big(V)$.
  Entscheiden Sie jeweils, ob $\varphi$ diagonalisierbar ist.
  \begin{enumerate}[(i)]
  \item $m_{\varphi} = \qty\big(X + 2)\qty\big(X + 3i + 7)\qty\big(X - 5i)^2$
    \subparagraph{Lsg.} Das Minimalpolynom $m_{\varphi}$ zerfällt vollständig in
    Linearfaktoren, allerdings sind diese nicht paarweise verschieden -
    $\qty\big(X - 5i)$ kommt zwei mal vor.
    Nach Satz 11.26 der Vorlesung (\emph{``Ein Endomorphismus
      $\varphi \in \End\qty\big(V)$ ist genau dann diagonalisierbar, wenn
      $m_{\varphi}$ vollständig in paarweise verschiedene Linearfaktoren
      zerfällt.''}) ist $\varphi$ somit nicht diagonalisierbar.

\newpage
  \item $m_{\varphi} = \qty\big(X - 2)\qty\big(X + 3i + 1)\qty\big(X + 7i)$
    \subparagraph{Lsg.} Das Minimalpolynom zerfällt vollständig in paarweise
    verschiedene Linearfaktoren, nach Satz 11.26 der Vorlesung ist
    $\varphi$ diagonalisierbar.
  \end{enumerate}
\end{enumerate}
\end{document}
