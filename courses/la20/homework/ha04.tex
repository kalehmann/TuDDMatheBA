\documentclass{scrreprt}

\usepackage{aligned-overset}
\usepackage{amsmath}
\usepackage{amssymb}
\usepackage{bm}
\usepackage[shortlabels]{enumitem}
\usepackage{hyperref}
\usepackage[utf8]{inputenc}
\usepackage{multicol}
\usepackage{mathtools}
\usepackage{physics}
\usepackage{tabularx}
\usepackage{titling}
\usepackage{fancyhdr}
\usepackage{xfrac}
\usepackage[dvipsnames]{xcolor}
\usepackage{pgfplots}

\pgfplotsset{compat = newest}
\usetikzlibrary{intersections}
\usetikzlibrary{patterns}
\usepgfplotslibrary{fillbetween}

\newcommand{\Spur}{\text{Spur}}

\author{Karsten Lehmann\\Mat. Nr 4935758}
\date{SoSe 2022}
\title{Hausaufgaben Blatt 04\\Lineare Algebra - Weiterführende Konzepte}

\setlength{\headheight}{26pt}
\pagestyle{fancy}
\fancyhf{}
\lhead{\thetitle}
\rhead{\theauthor}
\lfoot{\thedate}
\rfoot{Seite \thepage}

\begin{document}
\paragraph{Aufgabe 5} Seien $K$ ein Körper, $\lambda \in K$, $n \in \mathbb{N}$
und $A, B \in \mathcal{M}_{n \times n}\qty\big(K)$.
Beweisen Sie die richtigen Aussagen und widerlegen Sie die falschen Aussagen mit
einem Gegenbeispiel.

\begin{enumerate}[(i)]
\item Wenn $\lambda$ ein Eigenwert von $A$ ist und $A$ invertierbar ist, dann ist
  $\lambda \ne 0$ und es ist $\lambda^{-1}$ ein Eigenwert von $A^{-1}$.

  \subparagraph{Lsg.} Sei
  $A = I_1 = \begin{pmatrix}1\end{pmatrix} \in \mathcal{M}_{1 \times 1}\qty\big(K)$.
  Dann ist $A$ invertierbar mit $A^{-1} = \begin{pmatrix}1\end{pmatrix}$.
  Weiter ist $1_K$ ein Eigenwert von $A$, da $I_1 \cdot v = v = 1_K \cdot v$
  für $v \in K^1$.
  Nun ist $-1_K$ aber kein Eigenwert von $A^{-1}$, da $I_1 \cdot v = -1_K \cdot v$
  keine Lösung außer $v = 0_K$ besitzt.

  $\Rightarrow$ die Aussage ist falsch.

\item Ist $A^2 = I_n$ und ist $\lambda$ ein Eigenwert von $A$, so ist
  $\lambda = 1$ oder $\lambda = -1$.

  \subparagraph{Lsg.} Sei $\lambda$ ein Eigenwert von $A$.
  Dann existiert ein $v \in K^n$ mit $A \cdot v = \lambda \cdot v$.
  Multipliziert man beide Seiten der Gleichung erneut mit $A$, so
  erhält man $I_n \cdot v = \lambda^2 \cdot v$.

  $\Rightarrow \sqrt{\lambda} = 1, \lambda = 1 \lor \lambda = -1$,
  die Aussage ist wahr.

\item $A$ und $A^t$ haben dieselben Eigenwerte.

  \subparagraph{Lsg.} Es $A^t + B^t = \qty\big(A + B)^t$.
  Weiter ist $I_n^t = I_n$ und auch $\qty\big(X \cdot I_n)^t = X \cdot I_n$.

  $\Rightarrow X \cdot I_n - A^t = \qty\big(X \cdot I_n - A)^t$

  Nun ist $\det\qty\big(X \cdot I_n - A) = \det\qty\big(X \cdot I_n - A)$

  $\Rightarrow A$ und $A^t$ haben das selbe charakteristische Polynom und
  somit auch die selben Eigenwerte.

\item Ist $\lambda$ ein Eigenwert von $AB$, so ist $\lambda$ ein Eigenwert von
  $A$ oder $B$.

  \subparagraph{Lsg.} Sei $A = B = \begin{pmatrix}2 & 0 \\0 & 2\end{pmatrix}$.
  Dann ist $AB = \begin{pmatrix}4 & 0\\0 & 4\end{pmatrix}$.
  Da
  \[
    \begin{pmatrix}
      2 & 0 \\
      0 & 2 \\
    \end{pmatrix} \cdot \begin{pmatrix}
      1 \\
      1 \\
    \end{pmatrix} = 4 \cdot \begin{pmatrix}
      1 \\
      1 \\
    \end{pmatrix}
  \]
  ist $4$ ein Eigenwert von $AB$.
  Nun besitzt aber
  \[
    \begin{pmatrix}
      4 & 0 \\
      0 & 4 \\
    \end{pmatrix} \cdot v = 4 \cdot v
  \]
  keine Lösung außer $v = 0_{K^2}$ und somit ist $4$ kein Eigenwert von $A$ oder
  $B$.

  $\Rightarrow$ die Aussage ist falsch.
\end{enumerate}
\newpage

\paragraph{Aufgabe 6}
\begin{enumerate}[(i)]
\item Sei $A \in \mathcal{M}_{2 \times 2}\qty\big(\mathbb{R})$, sodass $A$
  zwei verschiedene Eigenwerte $\lambda_1$ und $\lambda_2$ besitzt.
  Sei außerdem angenommen, dass $\Spur\qty\big(A) = 7$ und
  $\det\qty\big(A) = 10$.
  Bestimmen Sie $\lambda_1$ und $\lambda_2$.
  Geben Sie außerdem ein konkretes Beispiel für eine Matrix mit den genannten
  Eigenschaften an.

  \subparagraph{Lsg.} Wie in Beispiel 10.11 der Vorlesung bereits dargelegt, ist
  das charakteristische Polynom von $A$ als $2 \times 2$ Matrix
  \begin{flalign*}
    f_A &= \det\qty\big(XI_n - A) = X^2 - \Spur\qty\big(A)X + \det(A) & \\
    &= X^2 - 7X + 10 \\
    &= (X - 2)(X - 5)
  \end{flalign*}
  $\Rightarrow \lambda_1 = 2$, $\lambda_2 = 5$

  Ein konkretes Beispiel für eine solche Matrix ist
  \[
    A = \begin{pmatrix}
      6 & 2 \\
      -2 & 1 \\
    \end{pmatrix}
  \]
  mit
  \[
    \begin{pmatrix}
      6 & 2 \\
      -2 & 1 \\
    \end{pmatrix} \cdot \begin{pmatrix}
      -2 \\
      1 \\
    \end{pmatrix} = \begin{pmatrix}
      -10 \\
      5 \\
    \end{pmatrix} = 5 \cdot \begin{pmatrix}
      -2 \\
      1 \\
    \end{pmatrix}
  \]
  und
  \[
    \begin{pmatrix}
      6 & 2 \\
      -2 & 1 \\
    \end{pmatrix} \cdot \begin{pmatrix}
      -1 \\
      2 \\
    \end{pmatrix} = \begin{pmatrix}
      -2 \\
      4 \\
    \end{pmatrix} = 2 \cdot \begin{pmatrix}
      -1 \\
      2 \\
    \end{pmatrix}
  \]

\item Bestimmen Sie das charakteristische Polynom und die Eigenwerte der
  Matrix
  \[
    B \coloneqq \begin{pmatrix}
      0 & 1 \\
      -1 & -1 \\
    \end{pmatrix} \in \mathcal{M}_{2 \times 2}\qty\big(\mathbb{F}_7)
  \]
  wobei $\mathbb{F}_7 = \mathbb{Z}/7$.
  Bestimmen Sie für jeden Eigenwert eine Basis des zugehörigen Eigenraums.

  \subparagraph{Lsg.} Es ist
  \begin{flalign*}
    f_B &= \det\qty\big(XI_n - B) &\\
    &= \det(
    \begin{pmatrix}
      X & 0 \\
      0 & X \\
    \end{pmatrix} - \begin{pmatrix}
      0 & 1 \\
      -1 & -1 \\
    \end{pmatrix}
    ) \\
    &= \det(\begin{pmatrix}
      X & -1 \\
      1 & X + 1 \\
    \end{pmatrix}) \\
    &= X^2 + X + 1 \\
    &= \qty\big(X - 2)\qty\big(X - 4)
  \end{flalign*}
  mit den Eigenwerten $\lambda_1 = 2_{\mathbb{F}_7}$ und $\lambda_2 = 4_{\mathbb{F}_7}$
  \newpage
  Nun sei
  $B = \lambda_1 \cdot I_2 - A = \begin{pmatrix}2 & -1\\1 & 3\end{pmatrix}$
  und
  $C = \lambda_2 \cdot I_2 - A = \begin{pmatrix}4 & -1\\1 & 5\end{pmatrix}$
  Durch Anwendung des Gauß-Verfahrens erhält man
  \begin{flalign*}
    \begin{pmatrix}
      2 & -1 \\
      1 & 3 \\
    \end{pmatrix} \overset{Z.1 + 5 \cdot Z.2}\leadsto
    \begin{pmatrix}
      0 & 0  \\
      1 & 3 \\
    \end{pmatrix}
  \end{flalign*}
  $\Rightarrow E(\lambda_1) = \qty{
    \begin{pmatrix}x \\ y\end{pmatrix} \in \mathbb{F}_7^2
    \:{\Big |}\: x + 3 \cdot y = 0_{\mathbb{F}_7}
  } = \qty{
    \begin{pmatrix}1 \\ 2\end{pmatrix} \cdot v
    \:{\Big |}\: v \in \mathbb{F}_7
  }$ und $\begin{pmatrix}5 \\ 6\end{pmatrix}$ ist eine Basis des
  Eigenraums.
  \begin{flalign*}
    \begin{pmatrix}
      4 & -1 \\
      1 & 5 \\
    \end{pmatrix} &\overset{Z.1 + 3 \cdot Z.2}\leadsto
    \begin{pmatrix}
      5 & 0 \\
      1 & 3 \\
    \end{pmatrix} \\
    &\overset{2 \cdot Z.2 + Z.1}\leadsto
    \begin{pmatrix}
      5 & 0 \\
      0 & 6 \\
    \end{pmatrix}
  \end{flalign*}
  $\Rightarrow E(\lambda_2) = \qty{
    \begin{pmatrix}5 \\ 6\end{pmatrix} \cdot v
    \:{\Big |}\: v \in \mathbb{F}_7
  }$ und $\begin{pmatrix}5 \\ 6\end{pmatrix}$ ist eine Basis des
  Eigenraums.

\item Sei
  \[
    \varphi \colon \mathbb{R}^2 \to \mathbb{R}^2,
    \begin{pmatrix}
      x \\
      y \\
    \end{pmatrix} \mapsto \begin{pmatrix}
      x \\
      x + 2y \\
    \end{pmatrix}
  \]
  Bestimmen Sie das charakteristische Polynom und die Eigenwerte von $\varphi$.
  Geben Sie für jeden Eigenwert eine Basis des zugehörigen Eigenraums an.

  \subparagraph{Lsg.} Sei $B$ die Standardbasis von $\mathbb{R}^2$.
  Dann ist
  $\text{Mat}\qty\big(\varphi, B) = \begin{pmatrix}1&0\\1&2\end{pmatrix}$.
  Das charakteristische Polynom ist
  \begin{flalign*}
    \det\qty(
    \begin{pmatrix}
      X & 0 \\
      0 & X \\
    \end{pmatrix} - \begin{pmatrix}
      1 & 0 \\
      1 & 2 \\
    \end{pmatrix}
    ) &= \det \begin{pmatrix}
      X - 1 & 0 \\
      -1 & X - 2 \\
    \end{pmatrix} \\
    &= \qty\big(X - 1)\qty\big(X - 2)
  \end{flalign*}
  mit den Eigenwerten $\lambda_1 = 1$ und $\lambda_2 = 2$.
  Nach dem Verfahren aus der vorherigen Aufgabe erhält man
  $E\qty\big(\lambda_1) = \qty{
    \begin{pmatrix}
      1 \\
      -1 \\
    \end{pmatrix} \cdot x
    \:{\Big |}\:
    x \in \mathbb{R}
  }$ mit $\begin{pmatrix}1\\-1\end{pmatrix}$ als Basis und
  $E\qty\big(\lambda_2) = \qty{
    \begin{pmatrix}
      0 \\
      1 \\
    \end{pmatrix} \cdot x
    \:{\Big |}\:
    x \in \mathbb{R}
  }$ mit $\begin{pmatrix}0\\1\end{pmatrix}$ als Basis.
\end{enumerate}
\end{document}