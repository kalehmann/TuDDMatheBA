\documentclass{scrreprt}

\usepackage{aligned-overset}
\usepackage{amsmath}
\usepackage{amssymb}
\usepackage{bm}
\usepackage[shortlabels]{enumitem}
\usepackage{hyperref}
\usepackage[utf8]{inputenc}
\usepackage{multicol}
\usepackage{mathtools}
\usepackage{physics}
\usepackage{tabularx}
\usepackage{titling}
\usepackage{fancyhdr}
\usepackage{xfrac}
\usepackage[dvipsnames]{xcolor}
\usepackage{pgfplots}

\pgfplotsset{compat = newest}
\usetikzlibrary{intersections}
\usetikzlibrary{patterns}
\usepgfplotslibrary{fillbetween}

\newcommand{\sgn}{\text{sgn}}

\author{Karsten Lehmann\\Mat. Nr 4935758}
\date{SoSe 2022}
\title{Hausaufgaben Blatt 04\\Lineare Algebra - Weiterführende Konzepte}

\setlength{\headheight}{26pt}
\pagestyle{fancy}
\fancyhf{}
\lhead{\thetitle}
\rhead{\theauthor}
\lfoot{\thedate}
\rfoot{Seite \thepage}

\begin{document}
\paragraph{Aufgabe 5} Seien $K$ ein Körper, $\lambda \in K$, $n \in \mathbb{N}$
und $A, B \in \mathcal{M}_{n \times n}\qty\big(K)$.
Beweisen Sie die richtigen Aussagen und widerlegen Sie die falschen Aussagen mit
einem Gegenbeispiel.

\begin{enumerate}[(i)]
\item Wenn $\lambda$ ein Eigenwert von $A$ ist und $A$ invertierbar ist, dann ist
  $\lambda \ne 0$ und es ist $\lambda^{-1}$ ein Eigenwert von $A^{-1}$.

  \subparagraph{Lsg.} Sei
  $A = I_1 = \begin{pmatrix}1\end{pmatrix} \in \mathcal{M}_{1 \times 1}\qty\big(K)$.
  Dann ist $A$ invertierbar mit $A^{-1} = \begin{pmatrix}1\end{pmatrix}$.
  Weiter ist $1_K$ ein Eigenwert von $A$, da $I_1 \cdot v = v = 1_K \cdot v$
  für $v \in K^1$.
  Nun ist $-1_K$ aber kein Eigenwert von $A^{-1}$, da $I_1 \cdot v = -1_K \cdot v$
  keine Lösung außer $v = 0_K$ besitzt.

  $\Rightarrow$ die Aussage ist falsch.

\item Ist $A^2 = I_n$ und ist $\lambda$ ein Eigenwert von $A$, so ist
  $\lambda = 1$ oder $\lambda = -1$.

  \subparagraph{Lsg.} Sei $\lambda$ ein Eigenwert von $A$.
  Dann existiert ein $v \in K^n$ mit $A \cdot v = \lambda \cdot v$.
  Multipliziert man beide Seiten der Gleichung erneut mit $A$, so
  erhält man $I_n \cdot v = \lambda^2 \cdot v$.

  $\Rightarrow \sqrt{\lambda} = 1, \lambda = 1 \lor \lambda = -1$,
  die Aussage ist wahr.

\item $A$ und $A^t$ haben dieselben Eigenwerte.

  \subparagraph{Lsg.} Es $A^t + B^t = \qty\big(A + B)^t$.
  Weiter ist $I_n^t = I_n$ und auch $\qty\big(X \cdot I_n)^t = X \cdot I_n$.

  $\Rightarrow X \cdot I_n - A^t = \qty\big(X \cdot I_n - A)^t$

  Nun ist $\det\qty\big(X \cdot I_n - A) = \det\qty\big(X \cdot I_n - A)$

  $\Rightarrow A$ und $A^t$ haben das selbe charakteristische Polynom und
  somit auch die selben Eigenwerte.

\item Ist $\lambda$ ein Eigenwert von $AB$, so ist $\lambda$ ein Eigenwert von
  $A$ oder $B$.

  \subparagraph{Lsg.} Sei $A = B = \begin{pmatrix}2 & 0 \\0 & 2\end{pmatrix}$.
  Dann ist $AB = \begin{pmatrix}4 & 0\\0 & 4\end{pmatrix}$.
  Da
  \[
    \begin{pmatrix}
      2 & 0 \\
      0 & 2 \\
    \end{pmatrix} \cdot \begin{pmatrix}
      1 \\
      1 \\
    \end{pmatrix} = 4 \cdot \begin{pmatrix}
      1 \\
      1 \\
    \end{pmatrix}
  \]
  ist $4$ ein Eigenwert von $AB$.
  Nun besitzt aber
  \[
    \begin{pmatrix}
      4 & 0 \\
      0 & 4 \\
    \end{pmatrix} \cdot v = 4 \cdot v
  \]
  keine Lösung außer $v = 0_{K^2}$ und somit ist $4$ kein Eigenwert von $A$ oder
  $B$.

  $\Rightarrow$ die Aussage ist falsch.
\end{enumerate}
\end{document}