\documentclass{scrreprt}

\usepackage{aligned-overset}
\usepackage{amsmath}
\usepackage{amssymb}
\usepackage{bm}
\usepackage[shortlabels]{enumitem}
\usepackage{hyperref}
\usepackage[utf8]{inputenc}
\usepackage{multicol}
\usepackage{mathtools}
\usepackage{physics}
\usepackage{stmaryrd}
\usepackage{tabularx}
\usepackage{titling}
\usepackage{fancyhdr}
\usepackage{xfrac}
\usepackage{pgfplots}

\pgfplotsset{compat = newest}
\usetikzlibrary{intersections}
\usetikzlibrary{patterns}
\usetikzlibrary{positioning}
\usetikzlibrary{shapes.misc}
\usepgfplotslibrary{fillbetween}

\author{Karsten Lehmann}
\date{WiSe 2021/2022}
\title{Übungsblatt 03\\Algorithmen und Datenstrukturen}

\setlength{\headheight}{26pt}
\pagestyle{fancy}
\fancyhf{}
\lhead{\thetitle}
\rhead{\theauthor}
\lfoot{\thedate}
\rfoot{Seite \thepage}

\begin{document}
\paragraph{Aufgabe 1}
\begin{enumerate}[(a)]
\item Sei $\Sigma = \qty\big{a, b, c}$.
  Geben Sie die Mengen $V$ und $R$ einer EBNF-Definition
  $\varepsilon = \qty\big(V, \Sigma, S, R)$ an, so dass
  $W\qty\big(\varepsilon, S) = \qty\big{a^kb^lc^{2k}c^m \,{\big |}\,
    k \geq 1, l \geq m \geq 0}$ gilt.

  \subparagraph{Lsg.} $W\qty\big(\varepsilon, S) = \qty\big{a^kb^lc^{2k}c^m
    \,{\big |}\, k \geq 1, l \geq m \geq 0}$ ist äquivalent zu \\
  $W\qty\big(\varepsilon, S) = \qty\big{a^kb^lc^mc^{2k} \,{\big |}\,
    k \geq 1, l \geq m \geq 0}$.
  Somit ist
  \[
    R = \qty{
      S \Coloneqq a\hat{\Big(} S\,\hat{\Big|}\, \hat{\big\{} b \hat{[}c\hat{]}\hat{\big\}}\hat{\Big)}cc
    }, V = \qty\big{ S }
  \]

  Alternativ als Syntaxdiagramm \\
  \begin{tikzpicture}
    \node at (0, 0) {S};
    \node[circle, draw] (a) at (1, -1) {a};
    \node[draw, rectangle, right = 1.5 of a] (S) {S};
    \node[circle, draw, right = 2.5 of S] (c1) {c};
    \node[circle, draw, right = 0.5 of c1] (c2) {c};
    \node[below = 1.5 of S, circle, draw] (b) {b};
    \node[circle, draw, right = 0.5 of b] (c3) {c};
    \draw (0,-1) -- (a) -- (S) -- (c1) -- (c2) -- ++(1, 0);
    \draw (1.5,-1) .. controls ++(0.5,0) and ++(-0.5,0) .. ++(0.5,-1)
      -- ++(3, 0) .. controls ++(0.5,0) and ++(-0.5,0) .. ++(0.5,1);
    \draw (2.5, -2) .. controls ++(-0.3,0) and ++(0,0.3) .. ++(-0.3,-0.3)
      -- ++(0,-0.55) .. controls ++(0,-0.3) and ++(-0.3,0) .. ++(0.3,-0.3)
      -- (b) -- (c3) -- ++(0.5,0)
      .. controls ++(0.3,0) and ++(0,0.3) .. ++(0.3,0.3) -- ++(0,0.55)
      .. controls ++(0,-0.3) and ++(0.3,0) .. ++(-0.3,0.27);
    \draw (3.6, -3.12) .. controls ++(0.5,0) and ++(-0.5,0) .. ++(0.5,0.7)
      -- ++(0.25,0) .. controls ++(0.5,0) and ++(-0.5,0) .. ++(0.5,-0.7);
    \end{tikzpicture}

\item Sei $\Sigma' = \qty\big{a, b}$ und
  $\varepsilon' = \qty\big(V', \Sigma', X, R')$ eine EBNF-Definition mit
  $V' = \qty\big{X, Y}$ sowie
  \[
    R' = \qty\big{
      X \Coloneqq \hat{(}aXa \,\hat{|}\, Y \hat{)},
      Y \Coloneqq \hat{[}bY\hat{]}
    }
  \]
  \begin{enumerate}[(1)]
  \item Geben Sie die für die Fixpunktsemantik von $\varepsilon'$ zu iterierende
    Funktion $f$ an.
  \item Dokumentieren Sie vier Iterationsschritte der Fixpunktsemantik von
    $\varepsilon'$.
  \end{enumerate}

  \subparagraph{Lsg.} Es bezeichnet \\
  \begin{tabular}{cl}
    $\Sigma$
      & die Menge an Terminalsymbolen / ein Alphabet \\
    $\Sigma^*$
      & die Menge aller Wörter (endlichen Folgen von Symbolen aus $\Sigma$) \\
    $V$
      & die Menge der syntaktischen Variablen der EBNF-Definition \\
    $R$
      & die Menge der EBNF-Regeln \\
    $T\qty\big(\Sigma, V)$
      & die Menge an EBNF-Termen über $V$ und $\Sigma$ \\
    $\mathcal{P}\qty\big(\Sigma^*)$
      & die Potenzmenge (Menge aller Teilmengen) von $\Sigma^*$ \\
    $\varepsilon$
      & das leere Wort / die Kombination von Terminalsymbolen mit der Länge 0.
  \end{tabular}
  \begin{enumerate}[(1)]
  \item Gesucht wird eine Funktion, die zwei Funktionen
    \[
      \llbracket\cdot\rrbracket \colon
      T\qty\big(\Sigma, V) \to \qty\Big(
        \qty\big(V \to \mathcal{P}\qty\big(\Sigma^*)) \to
        \mathcal{P}\qty\big(\Sigma^*)
      )
    \]
    - also Funktionen, die einen EBNF-Term auf eine Funktion, die eine
    Abbildung einer syntaktischen Variable auf eine Teilmenge von
    $\mathcal{P}\qty\big(\Sigma^*)$ nach einer Teilmenge von
    $\mathcal{P}\qty\big(\Sigma^*)$ abbildet - zusammenfasst.

    Sei nun $p \colon V \to \mathcal{P}\qty\big(\Sigma^*)$ eine beliebige
    Funktion.
    \begin{flalign*}
      \llbracket \hat{(}aXa \,\hat{|}\, Y \hat{)} \rrbracket\qty\big(p) &=
      \llbracket aXa \rrbracket\qty\big(p) \cup
      \llbracket Y \rrbracket\qty\big(p) \\
      &= \qty\big{a} \cdot p\qty\big(X) \cdot \qty\big{a} \cup p\qty\big(Y) \\
      \llbracket \hat{[}bY\hat{]} \rrbracket\qty\big(p)
      &= \llbracket bY \rrbracket\qty\big(p) \cup \qty\big{\varepsilon} \\
      &= \qty\big{b} \cdot p\qty\big(Y) \cup \qty\big{\varepsilon}
    \end{flalign*}

    Nun ist
    \[
      f\qty\big(p) = \begin{pmatrix}
        \llbracket \hat{(}aXa \,\hat{|}\, Y \hat{)} \rrbracket\qty\big(p) \\
        \llbracket bY \rrbracket\qty\big(p)
      \end{pmatrix} = \begin{pmatrix}
        \qty\big{a} \cdot p\qty\big(X) \cdot \qty\big{a} \cup p\qty\big(Y) \\
        \qty\big{b} \cdot p\qty\big(Y) \cup \qty\big{\varepsilon}
      \end{pmatrix}
    \]

  \item
    \begin{flalign*}
      \begin{pmatrix} \emptyset \\ \emptyset \end{pmatrix}
      \overset{f}&\longrightarrow
      \begin{pmatrix}
        \qty\big{a} \cdot \emptyset \cdot \qty\big{a} \cup \emptyset \\
        \qty\big{b} \cdot \emptyset \cup \qty\big{\varepsilon}
      \end{pmatrix} = \begin{pmatrix}
        \emptyset \\
        \qty\big{\varepsilon}
      \end{pmatrix} \\
      \begin{pmatrix}
        \emptyset \\
        \qty\big{\varepsilon}
      \end{pmatrix}
      \overset{f}&\longrightarrow
      \begin{pmatrix}
        \qty\big{a} \cdot \emptyset \cdot \qty\big{a} \cup
        \qty\big{\varepsilon} \\
        \qty\big{b} \cdot \qty\big{\varepsilon} \cup \qty\big{\varepsilon}
      \end{pmatrix} = \begin{pmatrix}
        \qty\big{\varepsilon} \\
        \qty\big{b, \varepsilon}
      \end{pmatrix} \\
      \begin{pmatrix}
        \qty\big{\varepsilon} \\
        \qty\big{b, \varepsilon}
      \end{pmatrix}
      \overset{f}&\longrightarrow
      \begin{pmatrix}
        \qty\big{a} \cdot \qty\big{\varepsilon} \cdot \qty\big{a} \cup
        \qty\big{b, \varepsilon} \\
        \qty\big{b} \cdot \qty\big{b, \varepsilon} \cup \qty\big{\varepsilon}
      \end{pmatrix} = \begin{pmatrix}
        \qty\big{aa, b, \varepsilon} \\
        \qty\big{bb, b, \varepsilon}
      \end{pmatrix} \\
      \begin{pmatrix}
        \qty\big{aa, b, \varepsilon} \\
        \qty\big{bb, b, \varepsilon}
      \end{pmatrix}
      \overset{f}&\longrightarrow
      \begin{pmatrix}
        \qty\big{a} \cdot \qty\big{aa, b, \varepsilon} \cdot \qty\big{a} \cup
        \qty\big{bb, b, \varepsilon} \\
        \qty\big{b} \cdot \qty\big{bb, b, \varepsilon} \cup \qty\big{\varepsilon}
      \end{pmatrix} \\
      &= \begin{pmatrix}
        \qty\big{aaaa, aba, aa, bb, b, \varepsilon} \\
        \qty\big{bbb, bb, b, \varepsilon}
      \end{pmatrix} \\
      \begin{pmatrix}
        \qty\big{aaaa, aba, aa, bb, b, \varepsilon} \\
        \qty\big{bbb, bb, b, \varepsilon}
      \end{pmatrix}
      \overset{f}&\longrightarrow
      \begin{pmatrix}
        \qty\big{a} \cdot \qty\big{aaaa, aba, aa, bb, b, \varepsilon}
        \cdot \qty\big{a} \cup \qty\big{bbb, bb, b, \varepsilon} \\
        \qty\big{b} \cdot \qty\big{bbb, bb, b, \varepsilon}
        \cup \qty\big{\varepsilon}
      \end{pmatrix} \\
      &= \begin{pmatrix}
        \qty\big{aaaaaa, aabaa, aaaa, abba, aba, aa, bbb, bb, b, \varepsilon} \\
        \qty\big{bbbb, bbb, bb, b, \varepsilon}
      \end{pmatrix}
    \end{flalign*}
  \end{enumerate}

\item Geben Sie die Sprache $W\qty\big(\varepsilon', X)$ an, die durch die
  EBNF-Definition $\varepsilon'$ beschrieben wird.

  \subparagraph{Lsg.} $W\qty\big(\varepsilon', X) = \qty\big{a^mb^na^m
    \,{\big |}\, m,n \geq 0}$
\end{enumerate}

\newpage
\paragraph{Aufgabe 2}
Sei $\varepsilon = \qty\big(V, \Sigma, S, R)$ eine EBNF-Definition mit
$V = \qty\big{S}, \Sigma = \qty\big{a, b}$ und
$R = \qty\big{
  S \Coloneqq \hat{\big (} aSa \,\hat{\big |}\, \hat{[}b\hat{]}\hat{\big )}
}$.
Weiterhin sei $p \colon V \to \mathcal{P}\qty\big(\Sigma^*)$, sodass
\[
  p\qty\big(S) = \qty\big{
    a^nwa^n \,{\big |}\, n \geq 0, w \in \qty\big{\varepsilon, b}
  }
\]
Zeigen Sie, dass ${\big \llbracket}
\hat{\big (} aSa \,\hat{\big |}\, \hat{[}b\hat{]}\hat{\big )}
{\big \rrbracket}\qty\big(p) = p\qty\big(S)$ gilt.

\subparagraph{Lsg.} Es ist
\begin{flalign*}
  {\big \llbracket}
    \hat{\big (} aSa \,\hat{\big |}\, \hat{[}b\hat{]}\hat{\big )}
  {\big \rrbracket}\qty\big(p) &=
  \llbracket aSa \rrbracket\qty\big(p) \cup
  \llbracket \hat{[}b\hat{]} \rrbracket\qty\big(p) \\
  &= \qty\big{a} \cdot p\qty\big(S) \cdot \qty\big{a} \cup
  \qty\big{b} \cup \qty\big{\varepsilon} \\
  &= \qty\big{a} \cdot \qty\big{
    a^nwa^n \,{\big |}\, n \geq 0, w \in \qty\big{\varepsilon, b}
  } \cdot \qty\big{a} \cup
  \qty\big{b, \varepsilon} \\
  &= \qty\big{
    a^{n+1}wa^{n+1} \,{\big |}\, n \geq 0, w \in \qty\big{\varepsilon, b}
  } \cup \qty\big{b, \varepsilon} \\
  &= \qty\big{
    a^nwa^n \,{\big |}\, n \geq 1, w \in \qty\big{\varepsilon, b}
  } \cup \qty\big{
    a^0wa^0 \,{\big |}\, w \in \qty\big{\varepsilon, b}
  } \\
  &= \qty\big{
    a^nwa^n \,{\big |}\, n \geq 0, w \in \qty\big{\varepsilon, b}
  } = p\qty\big(S) \\
\end{flalign*}
\end{document}