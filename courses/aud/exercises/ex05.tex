\documentclass{scrreprt}

\usepackage{aligned-overset}
\usepackage{amsmath}
\usepackage{amssymb}
\usepackage{bm}
\usepackage[shortlabels]{enumitem}
\usepackage{hyperref}
\usepackage[utf8]{inputenc}
\usepackage{listings}
\usepackage{multicol}
\usepackage{mathtools}
\usepackage{physics}
\usepackage{stmaryrd}
\usepackage{tabularx}
\usepackage{titling}
\usepackage{fancyhdr}
\usepackage{xfrac}
\usepackage{pgfplots}

\pgfplotsset{compat = newest}
\usetikzlibrary{intersections}
\usetikzlibrary{patterns}
\usetikzlibrary{positioning}
\usetikzlibrary{shapes.misc}
\usepgfplotslibrary{fillbetween}

\author{Karsten Lehmann}
\date{WiSe 2021/2022}
\title{Übungsblatt 05\\Algorithmen und Datenstrukturen}

\setlength{\headheight}{26pt}
\pagestyle{fancy}
\fancyhf{}
\lhead{\thetitle}
\rhead{\theauthor}
\lfoot{\thedate}
\rfoot{Seite \thepage}

\begin{document}
\paragraph{Aufgabe 2}
\begin{enumerate}[(a)]
\item Das folgende C-Programm ist gegeben:
  \begin{lstlisting}[language=C, showstringspaces=false]
#include<stdio.h>

void swoop(int a, int b)
{
        /* label 1 */
        a = b;
        b = a;
        /* label 2 */
}

int main()
{
        int x = 3, y = 6;
        /* label 3 */
        swoop(x, y); /*$1*/
        /* label 4 */
        printf("x = %d, y = %d", x, y);
        return 0;
}
  \end{lstlisting}
  Geben Sie ein Speicherbelegungsprotokoll für dieses Programm an.

  \subparagraph{Lsg.}\,\:

  \begin{tabular}{|c|c|c|c|c|c|c|c|c|c|c|c|c|}
    \hline
    HP & RM & 1 & 2 & 3 & 4 & 5 & 6 & 7 & 8 & 9 & 10 & 11 \\
    \hline
    label3 & - & $\substack{x \\ 3}$ & $\substack{y \\ 6}$ & & & & & & & & & \\
    \hline
    label1 & \$1 & & & $\substack{a \\ 3}$ & $\substack{b \\ 6}$ & & & & & & & \\
    \hline
    label2 & \$1 & & & $\substack{a \\ 6}$ & $\substack{b \\ 6}$ & & & & & & & \\
    \hline
    label4 & - & $\substack{x \\ 3}$ & $\substack{y \\ 6}$ & & & & & & & & & \\
    \hline
  \end{tabular}

\newpage
\item Schreiben Sie eine C-Funktion \textbf{swap} mit zwei Parametern, welche die
  Werte der aktuellen Parameter vertauscht.

  \subparagraph{Lsg.}
  \begin{lstlisting}[language=C, showstringspaces=false]
void swap(int *a, int *b)
{
        int tmp = *a;
        *a = *b;
        *b = tmp;
}
  \end{lstlisting}
\end{enumerate}

\newpage
\paragraph{Aufgabe 3}
Gegeben sei das folgende C-Programm:
\begin{lstlisting}[language=C, numbers=left, showstringspaces=false]
#include<stdio.h>
int a;

void g(int a, int *b);

void f(int *i, int j) {
        /* label1 */
        if (*i + j < a) {
                *i = *i + 1;
                f(i, j); /*$1*/
        }
        /* label2 */
}

void g(int a, int *b) {
        int i = 2;
        /*label3*/
        while (a != 1) {
                f(&i, a); /*$2*/
                a = a / 2;
                *b = *b + 1;
                /*label4*/
        }
}

int main() {
        int x = 0;
        scanf("%i", &a);
        /*label5*/
        g(a, &x); /*$3*/
        /*label6*/
        return 0;
}
\end{lstlisting}
\newpage
\begin{enumerate}[a)]
\item Geben Sie den Gültigkeitsbereich jedes Objektes an.
  Nutzen Sie dazu die Zeilennummern.

  \subparagraph{Lsg.} \;\\
  \begin{tabular}{c|c}
    \hline
    a & Z. 2 - 14, 25 - 33 \\
    g & Z. 4 - 33 \\
    f & Z. 6 - 33 \\
    i (f) & Z. 6 - 13 \\
    j (f) & Z. 6 - 13 \\
    a (g) & Z. 15 - 24 \\
    b (g) & Z. 15 - 24 \\
    i (g) & Z. 16 - 24 \\
    main & Z. 26 - 33 \\
    x (main) & Z. 27 - 33 \\
  \end{tabular}

\item Führen Sie das folgende Speicherbelegungsprotokoll weiter bis das Programm
  endet.
  Dokumentieren Sie jeweils die aktuelle Situation beim Passieren der Marken
  \emph{/*label1*/} bis \emph{/*label6*/}.
  Geben Sie jeweils den Rücksprungmarkenkeller und die \textbf{sichtbaren}
  Variablen mit ihrer Wertbelegung an.
  Die Inhalte von Speicherzellen nicht sichtbarer Variablen müssen Sie nur bei
  Änderung eintragen.
  Die bereits festgelegten Rücksprungmarken sind \emph{/*\$1*/} bis
  \emph{/*\$3*/}.

  \subparagraph{Lsg.} \:\\
  \begin{tabular}{|c|c|c|c|c|c|c|c|c|c|c|c|c|}
    \hline
    HP & RM & 1 & 2 & 3 & 4 & 5 & 6 & 7 & 8 & 9 & 10 & 11 \\
    \hline
    label5 & - & $\substack{a \\ 7}$ & $\substack{x \\ 0}$ & & & & & & & & & \\
    \hline
    label3 & \$3 & & & $\substack{a \\ 7}$ & $\substack{b \\ 2}$ & $\substack{i \\ 2}$ & & & & & & \\
    \hline
    label1 & \$2\$3 & $\substack{a \\ 7}$ & & & & & $\substack{i \\ 5}$ & $\substack{j \\ 7}$ & & & & \\
    \hline
    label2 & \$2\$3 & $\substack{a \\ 7}$ & & & & & $\substack{i \\ 5}$ & $\substack{j \\ 7}$ & & & & \\
    \hline
    label4 & \$3 & & $\substack{x (main) \\ 1}$ & $\substack{a \\ 3}$ & $\substack{b \\ 2}$ & $\substack{i \\ 2}$ & & & & & & \\
    \hline
    label1 & \$2\$3 & $\substack{a \\ 7}$ & & & & & $\substack{i \\ 5}$ & $\substack{j \\ 3}$ & & & & \\
    \hline
    label1 & \$1\$2\$3 & $\substack{a \\ 7}$ & & & & $\substack{i (g) \\ 3}$ & & & $\substack{i \\ 5}$ & $\substack{j \\ 3}$ & & \\
    \hline
    label1 & \$1\$1\$2\$3 & $\substack{a \\ 7}$ & & & & $\substack{i (g) \\ 4}$ & & & & & $\substack{i \\ 5}$ & $\substack{j \\ 3}$ \\
    \hline
    label2 & \$1\$1\$2\$3& $\substack{a \\ 7}$ & & & & & & & & & $\substack{i \\ 5}$ & $\substack{j \\ 3}$ \\
    \hline
    label2 & \$1\$2\$3 & $\substack{a \\ 7}$ & & & & & & & $\substack{i \\ 5}$ & $\substack{j \\ 3}$ & & \\
    \hline
    label2 & \$2\$3 & $\substack{a \\ 7}$ & & & & & $\substack{i \\ 5}$ & $\substack{j \\ 3}$ & & & & \\
    \hline
    label4 & \$3 & & $\substack{x (main) \\ 2}$ & $\substack{a \\ 1}$ & $\substack{b \\ 2}$ & $\substack{i \\ 4}$ & & & & & & \\
    \hline
    label6 & - & $\substack{a \\ 7}$ & $\substack{x \\ 2}$ & & & & & & & & & \\
    \hline
  \end{tabular}
\end{enumerate}
\end{document}