\documentclass{scrreprt}

\usepackage{aligned-overset}
\usepackage{amsmath}
\usepackage{amssymb}
\usepackage{bm}
\usepackage[shortlabels]{enumitem}
\usepackage{hyperref}
\usepackage[utf8]{inputenc}
\usepackage{multicol}
\usepackage{mathtools}
\usepackage{physics}
\usepackage{tabularx}
\usepackage{titling}
\usepackage{fancyhdr}
\usepackage{xfrac}
\usepackage{pgfplots}

\pgfplotsset{compat = newest}
\usetikzlibrary{intersections}
\usetikzlibrary{patterns}
\usepgfplotslibrary{fillbetween}

\author{Karsten Lehmann}
\date{WiSe 2021/2022}
\title{Übungsblatt 01\\Algorithmen und Datenstrukturen}

\setlength{\headheight}{26pt}
\pagestyle{fancy}
\fancyhf{}
\lhead{\thetitle}
\rhead{\theauthor}
\lfoot{\thedate}
\rfoot{Seite \thepage}

\begin{document}
\paragraph{Definitionen}
\,\\
\begin{tabular}{l l}
  \textbf{Syntax} & umfasst Regeln zur Kombination von Zeichen. \\
  \textbf{Semantik} & ist die Theorie der Bedeutung von Zeichen. \\
  \textbf{Objektsprache} & ist Sprache, die Gegenstand einer Metasprache ist. \\
  \textbf{Metasprache} & dient zur Beschreibung der Syntax einer Objektsprache. \\
  \textbf{Alphabet} & nichtleere, endliche Menge von Symbolen. \\
  \textbf{Wort} & Eine endliche Folgen von Symbolen aus einem Alphabet. \\
  \textbf{Konkatenation} & Operation der Zeichenverkettung. \\
  \textbf{formale Sprache} & die Menge von Wörtern über einem Alphabet. \\
  \textbf{Komplexprodukt} & Operation der Verknüpfung zweier formaler Sprachen. \\
  \textbf{Stern $L^*$ einer formalen Sprache $L$} & die Menge von Wörter der formalen Sprache $L$. \\
\end{tabular}

\end{document}