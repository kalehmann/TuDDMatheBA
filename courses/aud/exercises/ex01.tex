\documentclass{scrreprt}

\usepackage{aligned-overset}
\usepackage{amsmath}
\usepackage{amssymb}
\usepackage{bm}
\usepackage[shortlabels]{enumitem}
\usepackage{hyperref}
\usepackage[utf8]{inputenc}
\usepackage{multicol}
\usepackage{mathtools}
\usepackage{physics}
\usepackage{tabularx}
\usepackage{titling}
\usepackage{fancyhdr}
\usepackage{xfrac}
\usepackage{pgfplots}

\pgfplotsset{compat = newest}
\usetikzlibrary{intersections}
\usetikzlibrary{patterns}
\usepgfplotslibrary{fillbetween}

\author{Karsten Lehmann}
\date{WiSe 2021/2022}
\title{Übungsblatt 01\\Algorithmen und Datenstrukturen}

\setlength{\headheight}{26pt}
\pagestyle{fancy}
\fancyhf{}
\lhead{\thetitle}
\rhead{\theauthor}
\lfoot{\thedate}
\rfoot{Seite \thepage}

\begin{document}
\paragraph{Aufgabe 1} Finden Sie die Definitionen der folgenden Begriffe:

\begin{tabular}{l l}
  \textbf{Syntax} & umfasst Regeln zur Kombination von Zeichen. \\
  \textbf{Semantik} & ist die Theorie der Bedeutung von Zeichen. \\
  \textbf{Objektsprache} & ist Sprache, die Gegenstand einer Metasprache ist. \\
  \textbf{Metasprache} & dient zur Beschreibung der Syntax einer Objektsprache. \\
  \textbf{Alphabet} & nichtleere, endliche Menge von Symbolen. \\
  \textbf{Wort} & Eine endliche Folgen von Symbolen aus einem Alphabet. \\
  \textbf{Konkatenation} & Operation der Zeichenverkettung. \\
  \textbf{formale Sprache} & die Menge von Wörtern über einem Alphabet. \\
  \textbf{Komplexprodukt} & Operation der Verknüpfung zweier formaler Sprachen. \\
  \textbf{Stern $L^*$ einer formalen Sprache $L$} & die Menge von Wörter der formalen Sprache $L$. \\
\end{tabular}

\paragraph{Aufgabe 2} Sei $\Sigma = \qty\big{1, 2, a, b}$.
Geben Sie einige Wörter über $\Sigma$ an.
Setzen Sie diese Wörter in Beziehung zu $\Sigma^*$ und geben Sie einige Elemente
von $\mathcal{P}\qty\big(\Sigma^*)$ an.

\subparagraph{Lsg.} Es sind zum Beispiel \emph{aa12}, \emph{ab12} und \emph{1a}
Wörter über $\Sigma$.
Für diese Wörter gilt \emph{aa12} $\in \Sigma^*$, \emph{ab12} $\in \Sigma^*$ und
\emph{1a} $\in \Sigma^*$.
Elemente aus $\mathcal{P}\qty\big(\Sigma^*)$ sind zum Beispiel
$\qty\big{\emph{1a}, \emph{1122}}$ oder $\qty\big{\epsilon}$.

\paragraph{Aufgabe 3} Gegeben seien die Sprachen $L_1 = \qty\big{a}$,
$L_2 = \qty\big{b}$, $L_3 = \qty\big{a, ba}$.
Ermitteln Sie das Ergebnis der folgenden Ausdrücke''
\begin{enumerate}[(a)]
\item $L_1 \cdot L_2 \cdot L_3 = \qty\big{aba, abba}$
\item $L_1^* = \qty\big{\epsilon, a, aa, \ldots}$
\item $L_3^* = \qty\big{\epsilon, a, ba, aa, aba, baa, baba, \ldots}$
\item $L_2^* \cdot L_1 = \qty\big{a, ba, bba, \ldots}$
\item $\mathcal{P}\qty\big(L_1^*)
  = \qty\big{\emptyset, \qty{\epsilon}, \qty{a}, \qty{\epsilon, a}, \qty{aa},
      \qty{\epsilon, a, aa}, \ldots, L_1^*}$
\end{enumerate}
\end{document}