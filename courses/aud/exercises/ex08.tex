\documentclass{scrreprt}

\usepackage{aligned-overset}
\usepackage{amsmath}
\usepackage{amssymb}
\usepackage{bm}
\usepackage[shortlabels]{enumitem}
\usepackage{hyperref}
\usepackage[utf8]{inputenc}
\usepackage{listings}
\usepackage{multicol}
\usepackage{mathtools}
\usepackage{physics}
\usepackage{stmaryrd}
\usepackage{tabularx}
\usepackage{titling}
\usepackage{fancyhdr}
\usepackage{xfrac}
\usepackage{pgfplots}

\pgfplotsset{compat = newest}
\usetikzlibrary{intersections}
\usetikzlibrary{patterns}
\usetikzlibrary{positioning}
\usetikzlibrary{shapes.misc}
\usepgfplotslibrary{fillbetween}

\author{Karsten Lehmann}
\date{WiSe 2021/2022}
\title{Übungsblatt 08\\Algorithmen und Datenstrukturen}

\setlength{\headheight}{26pt}
\pagestyle{fancy}
\fancyhf{}
\lhead{\thetitle}
\rhead{\theauthor}
\lfoot{\thedate}
\rfoot{Seite \thepage}

\begin{document}
\paragraph{Aufgabe 1}
\begin{enumerate}[a)]
\item Geben Sie zu dem Pattern \textbf{aabaaacaab} die mit Hilfe des
  KMP-Algorithmus (Knuth-Morris-Pratt) berechnete Verschiebetabelle an.

  \subparagraph{Lsg.}
  \begin{tabular}{c|cccccccccc}
    Position & 0 & 1 & 2 & 3 & 4 & 5 & 6 & 7 & 8 & 9 \\
    \hline
    Pattern & a & a & b & a & a & a & c & a & a & b \\
    \hline
    Tabelle & -1 & -1 & 1 & -1 & -1 & 2 & 2 & -1 & -1 & 1
  \end{tabular}

\item Mit Hilfe des KMP-Algorithmus ist die folgende Verschiebetabelle berechnet
  worden.
  Vervollständigen Sie das aus den Symbolen a, b und c bestehende Pattern.

  \subparagraph{Lsg.}
  \begin{tabular}{c|cccccc}
    Position & 0 & 1 & 2 & 3 & 4 & 5 \\
    \hline
    Pattern & \textbf{c} & \textbf{b} & c & c & b & \textbf{a}  \\
    \hline
    Tabelle & -1 & 0 & -1 & 1 & 0 & 2
  \end{tabular}
\end{enumerate}

\paragraph{Aufgabe 2}
Gegeben seien die Wörter $w$ = espen und $v$ = beispiele.
\begin{enumerate}[a)]
\item Berechnen Sie die Levenshtein-Distanz $d\qty\big(v, w)$.
  Geben Sie dazu die Berechnungsmatrix an.
  Tragen Sie alle Zelleneinträge zusammen mit den dazugehörigen Pfeilen ein.

  \subparagraph{Lsg.} \:\\
  \begin{tabular}{c|ccccccccccccccccccc}
    $d\qty\big(w, v)$
      & & & b & & e & & i & & s & & p & & i & & e & & l & & e \\
    \hline
      & 0 & \color{blue} $\rightarrow$ & 1 & $\rightarrow$ & 2 & $\rightarrow$ & 3 & $\rightarrow$ & 4 & $\rightarrow$ & 5 & $\rightarrow$ & 6 & $\rightarrow$ & 7 & $\rightarrow$ & 8 & $\rightarrow$ & 9 \\
      & $\downarrow$ & $\searrow$ & & \color{blue} $\searrow$ & & & & & & & & & & $\searrow$ & & & & $\searrow$ \\
    e & 1 & & 1 & & 1 & \color{blue} $\rightarrow$ & 2 & $\rightarrow$ & 3 & $\rightarrow$ & 4 & $\rightarrow$ & 5 & $\rightarrow$ & 6 & $\rightarrow$ & 7 & $\rightarrow$ & 8 \\
      & $\downarrow$ & $\searrow$ & $\downarrow$ & $\searrow$ & & $\searrow$ & & \color{blue} $\searrow$ \\
    s & 2 & & 2 & & 2 &  & 2 & & 2 & $\rightarrow$ & 3 & $\rightarrow$ & 4 & $\rightarrow$ & 5 & $\rightarrow$ & 6 & $\rightarrow$ & 7 \\
      & $\downarrow$ & $\searrow$ & $\downarrow$ & & $\downarrow$ & $\searrow$ & & $\searrow$ & $\downarrow$ & \color{blue} $\searrow$ \\
    p & 3 & & 3 & & 3 & & 3 & $\rightarrow$ & 3 & & 2 & \color{blue} $\rightarrow$ & 3 & $\rightarrow$ & 4 & $\rightarrow$ & 5 & $\rightarrow$ & 6 \\
      & $\downarrow$ & $\searrow$ & $\downarrow$ & $\searrow$ & & $\searrow$ & $\downarrow$ & $\searrow$ & $\downarrow$ & & $\downarrow$ & $\searrow$ & & \color{blue} $\searrow$ & & & & $\searrow$ \\
    e & 4 & & 4 & & 3 & & 4 & & 4 & & 3 & & 3 & & 3 &\color{red} $\rightarrow$ & \colorbox{green!30}{4} & $\rightarrow$ & 5 \\
      & $\downarrow$ & $\searrow$ & $\downarrow$ & & $\downarrow$ & $\searrow$ & & $\searrow$ & $\downarrow$ & & $\downarrow$ & $\searrow$ & $\downarrow$ & $\searrow$ & $\downarrow$ & \color{red} $\searrow$ & & \color{red} $\searrow$ \\
    n & 5 & & 5 & & 4 & & 4 & $\rightarrow$ & 5 & & 4 & & 4 & & 4 & & 4 & \color{red} $\rightarrow$ & \colorbox{yellow}{5} \\
  \end{tabular}

  Die Levenshtein-Distanz beträgt \colorbox{yellow}{5}.

\item Geben Sie die Levenshtein-Distanz $d\qty\big(\text{espe, beispiel})$ an.
  Beachten Sie, dass espe und beispiel Präfixe von espen, bzw. beispiele sind.

  \subparagraph{Lsg.} Die Levenshtein-Distanz beträgt \colorbox{green!30}{4}.

\newpage
\item Geben Sie zwei Alignments zwischen \emph{espen} und \emph{beispiele} an,
  die zu den minimalen Kosten führen.
  Dabei sollen die Alignments die jeweils angewendeten Editieroperation enthalten.

  \subparagraph{Lsg.} \:\\
  \begin{tabular}{ccccccccc}
    b & e & i & s & p & i & e & l & e\\
    $\mid$ & $\mid$ & $\mid$ & $\mid$ & $\mid$ & $\mid$ & $\mid$ & $\mid$ & $\mid$ \\
    * & e & * & s & p & * & e & n & *\\
    d &   & d &   &   & d &   & s & d
  \end{tabular}

  \begin{tabular}{ccccccccc}
    b & e & i & s & p & i & e & l & e\\
    $\mid$ & $\mid$ & $\mid$ & $\mid$ & $\mid$ & $\mid$ & $\mid$ & $\mid$ & $\mid$ \\
    * & e & * & s & p & * & e & * & n\\
    d &   & d &   &   & d &   & d & s
  \end{tabular}

\item Wie viele Alignments enthält die in Aufgabe a) angegebene Berechnungsmatrix?

  \subparagraph{Lsg.} Es gibt 2 Alignments.
\end{enumerate}

\end{document}