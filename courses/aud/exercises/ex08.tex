\documentclass{scrreprt}

\usepackage{aligned-overset}
\usepackage{amsmath}
\usepackage{amssymb}
\usepackage{bm}
\usepackage[shortlabels]{enumitem}
\usepackage{hyperref}
\usepackage[utf8]{inputenc}
\usepackage{listings}
\usepackage{multicol}
\usepackage{mathtools}
\usepackage{physics}
\usepackage{stmaryrd}
\usepackage{tabularx}
\usepackage{titling}
\usepackage{fancyhdr}
\usepackage{xfrac}
\usepackage{pgfplots}

\pgfplotsset{compat = newest}
\usetikzlibrary{intersections}
\usetikzlibrary{patterns}
\usetikzlibrary{positioning}
\usetikzlibrary{shapes.misc}
\usepgfplotslibrary{fillbetween}

\author{Karsten Lehmann}
\date{WiSe 2021/2022}
\title{Übungsblatt 08\\Algorithmen und Datenstrukturen}

\setlength{\headheight}{26pt}
\pagestyle{fancy}
\fancyhf{}
\lhead{\thetitle}
\rhead{\theauthor}
\lfoot{\thedate}
\rfoot{Seite \thepage}

\begin{document}
\paragraph{Aufgabe 1}
\begin{enumerate}[a)]
\item Geben Sie zu dem Pattern \textbf{aabaaacaab} die mit Hilfe des
  KMP-Algorithmus (Knuth-Morris-Pratt) berechnete Verschiebetabelle an.

  \subparagraph{Lsg.}
  \begin{tabular}{c|cccccccccc}
    Position & 0 & 1 & 2 & 3 & 4 & 5 & 6 & 7 & 8 & 9 \\
    \hline
    Pattern & a & a & b & a & a & a & c & a & a & b \\
    \hline
    Tabelle & -1 & -1 & 1 & -1 & -1 & 2 & 2 & -1 & -1 & 1
  \end{tabular}

\item Mit Hilfe des KMP-Algorithmus ist die folgende Verschiebetabelle berechnet
  worden.
  Vervollständigen Sie das aus den Symbolen a, b und c bestehende Pattern.

  \subparagraph{Lsg.}
  \begin{tabular}{c|cccccc}
    Position & 0 & 1 & 2 & 3 & 4 & 5 \\
    \hline
    Pattern & \textbf{c} & \textbf{b} & c & c & b & \textbf{a}  \\
    \hline
    Tabelle & -1 & 0 & -1 & 1 & 0 & 2
  \end{tabular}
\end{enumerate}

\end{document}