\documentclass{article}
\usepackage{aligned-overset}
\usepackage{amsmath}
\usepackage{amssymb}
\usepackage{bm}
\usepackage[shortlabels]{enumitem}
\usepackage{hyperref}
\usepackage[utf8]{inputenc}
\usepackage{mathtools}
\usepackage{physics}
\usepackage{titling}
\usepackage{tikz}
\usetikzlibrary{calc}
\usepackage{fancyhdr}
\usepackage{xfrac}

\newcommand{\tikzmark}[1]{\tikz[overlay,remember picture]{ \node (#1) {};}}

\author{Karsten Lehmann}
\date{WiSe 2020}
\title{Hausaufgabe 06 Lineare Algebra}

\pagestyle{fancy}
\fancyhf{}
\lhead{\thetitle}
\rhead{\theauthor}
\lfoot{\thedate}
\rfoot{Seite \thepage}

\begin{document}
\section*{Übung 23}

\begin{enumerate}[(i)]
\item
  Beweisen Sie, dass $B = \left\{ \begin{pmatrix}3\\2\end{pmatrix}_,\begin{pmatrix}2\\1\end{pmatrix}\right\}$
  sowie $B^{'} = \left\{ \begin{pmatrix}2\\-2\end{pmatrix}_,\begin{pmatrix}-1\\6\end{pmatrix}\right\}$
  Basen in $\mathbb{R}^2$ sind und finden Sie die Basiswechselmatrix $M^{B^{'}}_B$.
  Finden Sie außerdem die Koordinaten des Vektors $v \in \mathbb{R}^2$ bezüglich $B^{'}$, wenn seine
  Koordinaten bezüglich $B$ gleich $(-3, 2)$ sind.

  $B$ ist Basis von $\mathbb{R}^2$ genau dann, wenn gilt:
  \begin{enumerate}[1)]
  \item $B$ ist linear unabhängig
  \item $span B = \mathbb{R}^2$
  \end{enumerate}
  \begin{enumerate}[label={Zu \arabic*):}]
  \item
    die beiden Vektoren aus $B$ sind linear Abhängig genau dann, wenn ein $\lambda \in \mathbb{R}$
    existiert mit
    \[
      \begin{pmatrix}3\\2\end{pmatrix} = \lambda \cdot \begin{pmatrix}2\\1\end{pmatrix}
    \]
    ein solches $\lambda$ existiert offensichtlich nicht.
    Für $B^{'}$ existiert ein solches $\lambda$ mit
    \[
      \begin{pmatrix}2\\-2\end{pmatrix} = \lambda \cdot \begin{pmatrix}-1\\6\end{pmatrix}
    \]
    ebenfalls nicht.

    Somit sind $B$ und $B^{'}$ linear unabhängig.
  \item
    $\dim \mathbb{R}^2 = 2$ und nach Korollar 4.34 der Vorlesung bildet dann jede linear unabhängige Teilmenge
    aus 2 Vektoren eine Basis in $\mathbb{R}^2$. $B$ und $B^{'}$ sind linear unabhängig und umfassen 2 Vektoren,
    somit ist
    \[
      \text{span} B = \text{span} B^{'} = \mathbb{R}^2
    \]
  \end{enumerate}
  \begin{align*}
    \left(
    \begin{array}{cc|cc}
      3 & 2 & 2  & -1 \\
      2 & 1 & -2 & 6  \\
    \end{array}
    \right)
    &=
    \left(
    \begin{array}{cc|cc}
      3 & 2 & 2 & -1 \\
      5 & 3 & 0 & 5  \\
    \end{array}
    \right) \\
    &=
    \left(
    \begin{array}{cc|cc}
      15 & 10 & 10 & -5 \\
      5  & 3  & 0  & 5  \\
    \end{array}
    \right) \\
    &=
    \left(
    \begin{array}{cc|cc}
      20 & 13 & 10 & 0 \\
      5  & 3  & 0  & 5 \\
    \end{array}
    \right) \\
    &=
    \left(
    \begin{array}{cc|cc}
      2 & \sfrac{13}{10} & 1 & 0 \\
      1 & \sfrac{3}{5}   & 0 & 1 \\
    \end{array}
    \right) \\
  \end{align*}
  \[
    M^{B^{'}}_B = \begin{pmatrix}2&\sfrac{13}{10}\\1&\sfrac{3}{5}\end{pmatrix}
  \]

  \[
    \begin{pmatrix}2&\sfrac{13}{10}\\1&\sfrac{3}{5}\end{pmatrix} \cdot \begin{pmatrix}-3\\2\end{pmatrix}
    =
    \begin{pmatrix}
      -3 \cdot 2 + \sfrac{13}{10} \cdot 2 \\
      -3 \cdot 1 + \sfrac{3}{5} \cdot 2 \\
    \end{pmatrix}
    =
    \begin{pmatrix}
      -\sfrac{17}{5} \\
      -\sfrac{9}{5}  \\
    \end{pmatrix}
  \]
\item
  Beweise Sie, dass die Legendre-Polynome
  \[
    1, t, \frac{1}{2}(3t^2 - 1), \frac{1}{2}(5t^3 - 3t)
  \]
  eine Basis in $\mathbb{R}[t]_3$ bilden. Finden Sie die Basiswechselmatrizen von der Standardbasis
  \[
    1,t,t^2,t^3
  \]
  zu diese Basis und zurück.

  \begin{align*}
    \left(
      \begin{array}{cccc}
        1             & 0             & 0            & 0            \\
        0             & 1             & 0            & 0            \\
        -\sfrac{1}{2} & 0             & \sfrac{3}{2} & 0            \\
        0             & -\sfrac{3}{2} & 0            & \sfrac{5}{2} \\
      \end{array}
    \right)
    &=
    \left(
      \begin{array}{cccc}
        1             & 0 & 0            & 0            \\
        0             & 1 & 0            & 0            \\
        -\sfrac{1}{2} & 0 & \sfrac{3}{2} & 0            \\
        0             & 0 & 0            & \sfrac{5}{2} \\
      \end{array}
    \right) \\
    &=
    \left(
      \begin{array}{cccc}
        1 & 0 & 0            & 0            \\
        0 & 1 & 0            & 0            \\
        0 & 0 & \sfrac{3}{2} & 0            \\
        0 & 0 & 0            & \sfrac{5}{2} \\
      \end{array}
    \right) \\
  \end{align*}
  Somit sind die Polynome nicht linear abhängig. Nach 4.30 der Vorlesung ist $\dim \mathbb{R}[t]_3 = 4$, somit
  bilden die 4 linear unabhängigen Polynome nach Korollar 4.34 der Vorlesung eine Basis in $\mathbb{R}[t]_3$.

  Die ersten 4 Legendre-Polynome werden im folgenden mit $L$ bezeichnet und die Standardbasis mit $B$

  \begin{itemize}[itemindent=4em]
  \item[Von $B$ zu $L$ :]
    \[
      M^{L}_{B} = \left(
        \begin{array}{cccc}
          1             & 0             & 0            & 0            \\
          0             & 1             & 0            & 0            \\
          -\sfrac{1}{2} & 0             & \sfrac{3}{2} & 0            \\
          0             & -\sfrac{3}{2} & 0            & \sfrac{5}{2} \\
        \end{array}
      \right) 
    \]
  \item[Von $L$ zu $B$ :]
    \begin{align*}
      M^{B}_{L} =
      \left(
        \begin{array}{cccc | cccc}
          1             & 0             & 0            & 0            & 1 & 0 & 0 & 0 \\
          0             & 1             & 0            & 0            & 0 & 1 & 0 & 0 \\
          -\sfrac{1}{2} & 0             & \sfrac{3}{2} & 0            & 0 & 0 & 1 & 0 \\
          0             & -\sfrac{3}{2} & 0            & \sfrac{5}{2} & 0 & 0 & 0 & 1 \\
        \end{array}
      \right)
      &=
      \left(
        \begin{array}{cccc | cccc}
          1 & 0             & 0            & 0            & 1            & 0 & 0 & 0 \\
          0 & 1             & 0            & 0            & 0            & 1 & 0 & 0 \\
          0 & 0             & \sfrac{3}{2} & 0            & \sfrac{1}{2} & 0 & 1 & 0 \\
          0 & -\sfrac{3}{2} & 0            & \sfrac{5}{2} & 0            & 0 & 0 & 1 \\
        \end{array}
      \right) \\
      &=
      \left(
        \begin{array}{cccc | cccc}
          1 & 0 & 0            & 0            & 1            & 0            & 0 & 0 \\
          0 & 1 & 0            & 0            & 0            & 1            & 0 & 0 \\
          0 & 0 & \sfrac{3}{2} & 0            & \sfrac{1}{2} & 0            & 1 & 0 \\
          0 & 0 & 0            & \sfrac{5}{2} & 0            & \sfrac{3}{2} & 0 & 1 \\
        \end{array}
      \right) \\
      &=
      \left(
        \begin{array}{cccc | cccc}
          1 & 0 & 0 & 0 & 1            & 0            & 0            & 0 \\
          0 & 1 & 0 & 0 & 0            & 1            & 0            & 0 \\
          0 & 0 & 1 & 0 & \sfrac{1}{3} & 0            & \sfrac{2}{3} & 0 \\
          0 & 0 & 0 & 1 & 0            & \sfrac{3}{5} & 0            & \sfrac{2}{5} \\
        \end{array}
      \right) \\
      M^{B}_{L} &=
      \left(
        \begin{array}{cccc}
          1            & 0            & 0            & 0 \\
          0            & 1            & 0            & 0 \\
          \sfrac{1}{3} & 0            & \sfrac{2}{3} & 0 \\
          0            & \sfrac{3}{5} & 0            & \sfrac{2}{5} \\
        \end{array}
      \right)
    \end{align*}
  \end{itemize}
\end{enumerate}

\end{document}