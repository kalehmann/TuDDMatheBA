\documentclass{scrreprt}

\usepackage{aligned-overset}
\usepackage{amsmath}
\usepackage{amssymb}
\usepackage{bm}
\usepackage[shortlabels]{enumitem}
\usepackage{hyperref}
\usepackage[utf8]{inputenc}
\usepackage{multicol}
\usepackage{mathtools}
\usepackage{physics}
\usepackage{tabularx}
\usepackage{titling}
\usepackage{fancyhdr}
\usepackage{xfrac}
\usepackage{pgfplots}

\pgfplotsset{compat = newest}
\usetikzlibrary{intersections}
\usetikzlibrary{patterns}
\usepgfplotslibrary{fillbetween}

\author{Karsten Lehmann\\Mat. Nr 4935758}
\date{WiSe 2021/2022}
\title{Hausaufgaben Blatt 02\\Lineare Algebra - Grundlegende Konzepte}

\setlength{\headheight}{26pt}
\pagestyle{fancy}
\fancyhf{}
\lhead{\thetitle}
\rhead{\theauthor}
\lfoot{\thedate}
\rfoot{Seite \thepage}

\begin{document}
\paragraph{Aufgabe 6} Untersuchen Sie jeweils, ob $\sim$ ein Äquivalenzrelation
auf $M$ ist.
Falls dies der Fall ist, geben Sie die Äquivalenzklassen von $\sim$ auf $M$ an.
\begin{enumerate}[(i)]
\item $M$ sei die Menge aller deutschen Städte; für $x, y \in M$ gilt
  $x \sim y$ genau dann, wenn $x$ und $y$ im selben Bundesland liegen.

\item $M$ sei die Menge aller deutschen Städte; für $x, y \in M$ gilt
  $x \sim y$ genau dann, wenn $x$ oder $y$ in Sachsen liegt.

\item $M = \mathbb{Z}$, für $x, y \in \mathbb{Z}$ gilt $x \sim y$, wenn
  $xy > 0$.

\item $M = \mathbb{Z} \setminus \qty{0}$; für
  $x, y \in \mathbb{Z} \setminus \qty{0}$ gilt $x \sim y$ genau dann, wenn
  $xy > 0$.
\end{enumerate}

\subparagraph{Lsg.}
\begin{enumerate}[(i)]
\item Seien $x, y, z \in M$ drei beliebige deutsche Städte.
  Dann gilt $x \sim x$, da die Stadt $x$ im selben Bundesland wie sie selbst
  liegt.
  Weiterhin gilt $x \sim y \Rightarrow y \sim x$.
  Schließlich folgt aus $x \sim y \land y \sim z$ auch $x \sim z$.

  $\Rightarrow$ ``$\sim$'' ist eine Äquivalenzrelation über $M$.

  $\sim$ hat auf $M$ 16 Äquivalenzklassen - jeweils die Menge der Städte für
  jedes Bundesland.

\item Sei $x \in M$ die Stadt Berlin.
  Dann gilt $x \sim x$ nicht, da Berlin nicht in Sachsen liegt.
  Somit ist $\sim$ über $M$ nicht reflexiv.

  $\Rightarrow$ ``$\sim$'' ist keine Äquivalenzrelation über $M$

\item Sei $x = 0 \in \mathbb{Z}$.
  Dann ist $x \cdot x = 0 \Rightarrow \neg \qty\big(x \sim x)$.
  Somit ist $\sim$ über $M$ nicht reflexiv.

  $\Rightarrow$ ``$\sim$'' ist keine Äquivalenzrelation über $M$

\item Seien $x, y, z \in \mathbb{Z} \setminus \qty{0}$ beliebig.
  Da $x \ne 0$ folgt $x \cdot x > 0 \Rightarrow x \sim x \Rightarrow \sim$
  ist auf $M$ reflexiv.
  Weiterhin folgt aus $x \sim y$ aufgrund der Kommutativität der Multiplikation
  der ganzen Zahlen $y \sim x$.

  Angenommen $x \sim y$ und $y \sim z$.
  Dann ist entweder $x, y, z < 0$ oder $x, y, z > 0$.
  $\Rightarrow x \sim z$.

  $\Rightarrow$ ``$\sim$'' ist eine Äquivalenzrelation über $M$.

  $\sim$ hat auf $M$ zwei Äquivalenzklassen:
  \begin{itemize}
  \item $\qty\big{x \in \mathbb{Z} \: {\big |} \: x > 0}$
  \item $\qty\big{x \in \mathbb{Z} \: {\big |} \: x < 0}$
  \end{itemize}
\end{enumerate}

\end{document}