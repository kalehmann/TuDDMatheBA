\documentclass{scrreprt}

\usepackage{aligned-overset}
\usepackage{amsmath}
\usepackage{amssymb}
\usepackage{bm}
\usepackage[shortlabels]{enumitem}
\usepackage{hyperref}
\usepackage[utf8]{inputenc}
\usepackage{multicol}
\usepackage{mathtools}
\usepackage{physics}
\usepackage{tabularx}
\usepackage{titling}
\usepackage{fancyhdr}
\usepackage{xfrac}
\usepackage{pgfplots}

\pgfplotsset{compat = newest}
\usetikzlibrary{intersections}
\usetikzlibrary{patterns}
\usepgfplotslibrary{fillbetween}

\author{Karsten Lehmann\\Mat. Nr 4935758}
\date{WiSe 2021/2022}
\title{Hausaufgaben Blatt 02\\Lineare Algebra - Grundlegende Konzepte}

\setlength{\headheight}{26pt}
\pagestyle{fancy}
\fancyhf{}
\lhead{\thetitle}
\rhead{\theauthor}
\lfoot{\thedate}
\rfoot{Seite \thepage}

\begin{document}
\paragraph{Aufgabe 6} Untersuchen Sie jeweils, ob ``$\sim$'' eine
Äquivalenzrelation auf $M$ ist.
Falls dies der Fall ist, geben Sie die Äquivalenzklassen von ``$\sim$'' auf
$M$ an.
\begin{enumerate}[(i)]
\item $M$ sei die Menge aller deutschen Städte; für $x, y \in M$ gilt
  $x \sim y$ genau dann, wenn $x$ und $y$ im selben Bundesland liegen.

\item $M$ sei die Menge aller deutschen Städte; für $x, y \in M$ gilt
  $x \sim y$ genau dann, wenn $x$ oder $y$ in Sachsen liegt.

\item $M = \mathbb{Z}$, für $x, y \in \mathbb{Z}$ gilt $x \sim y$, wenn
  $xy > 0$.

\item $M = \mathbb{Z} \setminus \qty{0}$; für
  $x, y \in \mathbb{Z} \setminus \qty{0}$ gilt $x \sim y$ genau dann, wenn
  $xy > 0$.
\end{enumerate}

\subparagraph{Lsg.}
\begin{enumerate}[(i)]
\item Seien $x, y, z \in M$ drei beliebige deutsche Städte.
  Dann gilt $x \sim x$, da die Stadt $x$ im selben Bundesland wie sie selbst
  liegt.
  Weiterhin gilt $x \sim y \Rightarrow y \sim x$, da die Reihenfolge der
  Städte beim Vergleich ihres Bundeslandes keine Rolle spielt.
  Schließlich folgt aus $x \sim y \land y \sim z$ auch $x \sim z$.

  $\Rightarrow$ ``$\sim$'' ist eine Äquivalenzrelation über $M$.

  ``$\sim$'' hat auf $M$ 16 Äquivalenzklassen - jeweils die Menge der Städte für
  jedes Bundesland.

\item Sei $x \in M$ die Stadt Berlin.
  Dann gilt $x \sim x$ nicht, da Berlin nicht in Sachsen liegt.
  Somit ist $\sim$ über $M$ nicht reflexiv.

  $\Rightarrow$ ``$\sim$'' ist keine Äquivalenzrelation über $M$

\item Sei $x = 0 \in \mathbb{Z}$.
  Dann ist $x \cdot x = 0 \Rightarrow \neg \qty\big(x \sim x)$.
  Somit ist ``$\sim$'' über $M$ nicht reflexiv.

  $\Rightarrow$ ``$\sim$'' ist keine Äquivalenzrelation über $M$

\item Seien $x, y, z \in \mathbb{Z} \setminus \qty{0}$ beliebig.
  Da $x \ne 0$ folgt $x \cdot x > 0 \Rightarrow x \sim x \Rightarrow$
  ``$\sim$'' ist auf $M$ reflexiv.
  Weiterhin folgt aus $x \sim y$ aufgrund der Kommutativität der Multiplikation
  der ganzen Zahlen $y \sim x$.

  Angenommen $x \sim y$ und $y \sim z$.
  Dann ist entweder $x, y, z < 0$ oder $x, y, z > 0$.
  $\Rightarrow x \sim z$.

  $\Rightarrow$ ``$\sim$'' ist eine Äquivalenzrelation über $M$.

  $\sim$ hat auf $M$ zwei Äquivalenzklassen:
  \begin{itemize}
  \item $\qty\big{x \in \mathbb{Z} \: {\big |} \: x > 0}$ (die natürlichen
    Zahlen)

  \item $\qty\big{x \in \mathbb{Z} \: {\big |} \: x < 0}$ (die ganzen Zahlen
    kleiner als $0$)
  \end{itemize}
\end{enumerate}

\newpage
\paragraph{Aufgabe 7} Beweisen Sie Lemma 1.8 aus der Vorlesung:
Sei $\mathcal{P}$ eine Partition auf $M$.
Sei
\[
  R \coloneqq \qty\Big{\qty\big(x, y) \: {\Big |} \:
    \text{Es gibt ein } N \in \mathcal{P} \text{, so dass } x, y \in N}
\]
Dann ist $R$ ein Äquivalenzrelation auf $M$ und die Menge der Äquivalenzklassen
von $R$ ist $\mathcal{P}$.

\subparagraph{Lsg.} $R$ heißt Äquivalenzrelation, wenn
\begin{enumerate}[(1)]
\item $\forall x \in M \colon \qty\big(x, x) \in R$ (\emph{Reflexivität})
\item $\forall x, y \in M \colon \qty\big(x, y) \in R
  \Rightarrow \qty\big(y, x) \in R$ (\emph{Transitivität})

\item $\forall x, y, z \in M \colon \qty\big(x, y), \qty\big(y, z) \in R
  \Rightarrow \qty\big(x, z) \in R$
\end{enumerate}

\begin{enumerate}[label={Zu (\arabic*)}]
\item Da $\mathcal{P}$ eine Partition ist gilt $\bigcup P = M$.
  Also existiert für jedes $x \in M$ eine Menge $N \in \mathcal{P}$
  mit $x \in N$.

  $\Rightarrow \qty(x, x) \in R$

  $\Rightarrow R$ ist \emph{reflexiv}.

\item Sei $x, y \in M$ mit $\qty\big(x, y) \in R$.
  Nach der Definition von $R$ existiert somit eine Menge $N \in \mathcal{P}$
  mit $x, y \in N$.
  Da hierbei die Reihenfolge keine Rolle spielt gilt auch $y, x \in N$ und
  $\qty\big(y, x) \in R$.

  $\Rightarrow R$ ist \emph{symmetrisch}.

\item Seien $x, y, z \in M$ mit $\qty\big(x, y), \qty\big(y, z) \in R$.

  $\Rightarrow$ es existiert $N_1, N_2 \in \mathcal{P}$ mit $x, y \in N_1$
  und $y, z \in N_2$.

  Angenommen $N_1 \ne N_2$.
  Da $\mathcal{P}$ eine Partition ist, gilt $N_1 = N_2 \lor N_1 \cap N_2 = \emptyset$.
  Nun ist aber $y \in N_1$ und $y \in N_2$ - ein Widerspruch.

  $\Rightarrow N_1 = N_2 \Rightarrow x, z \in N_1 \Rightarrow (x, z) \in R$.

  $\Rightarrow R$ ist \emph{transitiv}.
\end{enumerate}

$\Rightarrow R$ ist eine Äquivalenzrelation auf $M$. \\

\noindent
Sei nun $N \in \mathcal{P}$ beliebig.
Für zwei Elemente $x, y \in N$ gilt jeweils $x \sim y$.
Sei $z \in M$ mit $z \notin N$, dann gilt $\neg (x \sim z)$.
Somit ist $N$ eine Äquivalenzklasse in $M$ unter $R$.

$\Rightarrow \mathcal{P} \subset \text{``Menge der Äquivalenzklassen von $M$''}$
\\

\noindent
Sei nun $x \in M$ beliebig.
Dann existiert $N \in \mathcal{P}$ mit $x \in N$.
Weiter ist $\qty[x] = \qty\big{y \in M {\big |} x, y \in N} = N$.

$\Rightarrow \text{``Menge der Äquivalenzklassen von $M$''} \subset \mathcal{P}$

$\Rightarrow \mathcal{P}$ ist die Menge der Äquivalenzklassen von $M$.

\end{document}