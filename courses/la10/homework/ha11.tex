\documentclass{scrreprt}

\usepackage{aligned-overset}
\usepackage{amsmath}
\usepackage{amssymb}
\usepackage{bm}
\usepackage[shortlabels]{enumitem}
\usepackage{hyperref}
\usepackage[utf8]{inputenc}
\usepackage{multicol}
\usepackage{mathtools}
\usepackage{physics}
\usepackage{tabularx}
\usepackage{titling}
\usepackage{fancyhdr}
\usepackage{xfrac}
\usepackage[dvipsnames]{xcolor}
\usepackage{pgfplots}

\pgfplotsset{compat = newest}
\usetikzlibrary{intersections}
\usetikzlibrary{patterns}
\usepgfplotslibrary{fillbetween}

\author{Karsten Lehmann\\Mat. Nr 4935758}
\date{WiSe 2021/2022}
\title{Hausaufgaben Blatt 11\\Lineare Algebra - Grundlegende Konzepte}

\setlength{\headheight}{26pt}
\pagestyle{fancy}
\fancyhf{}
\lhead{\thetitle}
\rhead{\theauthor}
\lfoot{\thedate}
\rfoot{Seite \thepage}

\newcommand\Bild{\text{Bild}}
\newcommand\Kern{\text{Kern}}
\newcommand\hull[1]{\left\langle #1 \right\rangle}

\begin{document}
\paragraph{Aufgabe 6} Sei
\[
  f \colon \mathbb{R}^4 \to \mathbb{R}^2,
  \begin{pmatrix} x \\ y \\ z \\ t \end{pmatrix} \mapsto
  \begin{pmatrix} 3x - 2y + z \\ 2y - t \end{pmatrix}
\]
Es lässt sich leicht nachprüfen, dass $f$ linear ist.
Sie dürfen dies ohne Beweis verwenden.
Sei außerdem
\[
  W \coloneqq \qty{
    \begin{pmatrix} x \\ y \\ z \\ t \end{pmatrix} \in \mathbb{R}^4
    \,\middle|\, 2x - y - t = 0
  }
\]
\begin{enumerate}[(a)]
\item Berechnen Sie $\Kern\qty\big(f)$, $\Bild\qty\big(g)$ sowie die Dimensionen
  von $\Kern\qty\big(f)$ und $\Bild\qty\big(f)$.
  \subparagraph{Lsg.} Sei $\varphi \colon V \to W$ eine lineare Abbildung.
  Dann ist $\Kern\qty\big(\varphi) =
  \qty\big{v \in V \,{\big |}\, \varphi\qty\big(v) = 0_W}$.

  $f\qty(\begin{pmatrix} x \\ y \\ z \\ t\end{pmatrix}) =
  \begin{pmatrix} 0 \\ 0 \end{pmatrix}$, wenn $2y = t$ und
  $3x - 2y + z = 0$.
  Substituiert man nun $2y$ mit $t$, dann erhält man
  \[
    \Kern\qty\big(f) = \qty{
      \begin{pmatrix}
        x \\
        y \\
        z \\
        \frac{y}{2}
      \end{pmatrix} \,\middle|\,
      x, y, z \in \mathbb{R}, 3x - 2y + z = 0
    } \text{ mit } \dim\qty\big(\Kern(f)) = 3
  \]
  Weiter ist $\Bild\qty\big(\varphi)$ definiert als
  $\qty\big{\varphi\qty\big(v) \,{\big|}\, v \in V}$.
  Nun ist $\qty{
    f\qty(\begin{pmatrix} 0 \\ 0 \\ 1 \\ 0\end{pmatrix}),
    f\qty(\begin{pmatrix} 0 \\ 0 \\ 0 \\ -1\end{pmatrix})
  } = \qty{
    \begin{pmatrix} 1 \\ 0 \end{pmatrix},
    \begin{pmatrix} 0 \\ 1 \end{pmatrix}
  } = \varepsilon_2$ (die Standardbasis von $\mathbb{R}^2$).
  Da das Bild einer linearen Abbildung nach Lemma 6.18 der Vorlesung
  ein Unterraum ist, folgt $\Bild\qty\big(f) = \hull{\varepsilon_2} = \mathbb{R}^2$.
  Da die Basis 2 Elemente hat, folgt weiter $\dim\qty\big(\Bild(f)) = 2$.

\item Finden Sie eine lineare Abbildung $g \colon \mathbb{R}^4 \to \mathbb{R}$,
  so dass $\Kern\qty\big(g) = W$.
  Gibt es eine Abbildung $g$ mit dieser Eigenschaft, so dass $g$ nicht
  surjektiv ist?
  \subparagraph{Lsg.} Sei
  \[
    g\qty(\begin{pmatrix} x \\ y \\ z \\ t \end{pmatrix}) = 2x - y - t
  \]

  Da $\Bild\qty\big(g)$ ein Unterraum von $\mathbb{R}$ ist,
  gilt für alle $w \in \Bild\qty\big(g), \lambda \in \mathbb{R}$, dass
  $\lambda \cdot w \in \Bild\qty\big(g)$.
  Angenommen $g$ wäre nicht surjektiv, dann existiert ein Element
  $x \in \mathbb{R}$, so dass $x \notin \Bild\qty\big(g)$ und
  $\lambda \cdot w \ne x$ für alle
  $\lambda \in \mathbb{R}, w \in \Bild\qty\big(g)$.

  $\Rightarrow \Bild\qty\big(g) = \qty\big{0}$

  $\Rightarrow g(v) = 0$, somit wäre auch
  $g\qty(\begin{pmatrix} 1 \\ 0 \\ 0 \\ 0\end{pmatrix}) = 0$, aber
  $\begin{pmatrix} 1 \\ 0 \\ 0 \\ 0\end{pmatrix} \notin W$, ein Widerspruch zu
  $\Kern\qty\big(g) = W$.

  $\Rightarrow$ \underline{Es gibt keine lineare Abbildung $g$ mit
    $\Kern\qty\big(g) = W$, so dass $g$ nicht surjektiv ist.}
\end{enumerate}
\end{document}
