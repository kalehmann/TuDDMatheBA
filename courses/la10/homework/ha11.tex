\documentclass{scrreprt}

\usepackage{aligned-overset}
\usepackage{amsmath}
\usepackage{amssymb}
\usepackage{bm}
\usepackage[shortlabels]{enumitem}
\usepackage{hyperref}
\usepackage[utf8]{inputenc}
\usepackage{multicol}
\usepackage{mathtools}
\usepackage{physics}
\usepackage{tabularx}
\usepackage{titling}
\usepackage{fancyhdr}
\usepackage{xfrac}
\usepackage[dvipsnames]{xcolor}
\usepackage{pgfplots}

\pgfplotsset{compat = newest}
\usetikzlibrary{intersections}
\usetikzlibrary{patterns}
\usepgfplotslibrary{fillbetween}

\author{Karsten Lehmann\\Mat. Nr 4935758}
\date{WiSe 2021/2022}
\title{Hausaufgaben Blatt 11\\Lineare Algebra - Grundlegende Konzepte}

\setlength{\headheight}{26pt}
\pagestyle{fancy}
\fancyhf{}
\lhead{\thetitle}
\rhead{\theauthor}
\lfoot{\thedate}
\rfoot{Seite \thepage}

\newcommand\Bild{\text{Bild}}
\newcommand\Kern{\text{Kern}}
\newcommand\Mat{\text{Mat}}
\newcommand\hull[1]{\left\langle #1 \right\rangle}

\begin{document}
\paragraph{Aufgabe 6} Sei
\[
  f \colon \mathbb{R}^4 \to \mathbb{R}^2,
  \begin{pmatrix} x \\ y \\ z \\ t \end{pmatrix} \mapsto
  \begin{pmatrix} 3x - 2y + z \\ 2y - t \end{pmatrix}
\]
Es lässt sich leicht nachprüfen, dass $f$ linear ist.
Sie dürfen dies ohne Beweis verwenden.
Sei außerdem
\[
  W \coloneqq \qty{
    \begin{pmatrix} x \\ y \\ z \\ t \end{pmatrix} \in \mathbb{R}^4
    \,\middle|\, 2x - y - t = 0
  }
\]
\begin{enumerate}[(a)]
\item Berechnen Sie $\Kern\qty\big(f)$, $\Bild\qty\big(g)$ sowie die Dimensionen
  von $\Kern\qty\big(f)$ und $\Bild\qty\big(f)$.
  \subparagraph{Lsg.} Sei $\varphi \colon V \to W$ eine lineare Abbildung.
  Dann ist $\Kern\qty\big(\varphi) =
  \qty\big{v \in V \,{\big |}\, \varphi\qty\big(v) = 0_W}$.

  $f\qty(\begin{pmatrix} x \\ y \\ z \\ t\end{pmatrix}) =
  \begin{pmatrix} 0 \\ 0 \end{pmatrix}$, wenn $2y = t$ und
  $3x - 2y + z = 0$.
  Substituiert man nun $2y$ mit $t$, dann erhält man
  \[
    \Kern\qty\big(f) = \qty{
      \begin{pmatrix}
        x \\
        y \\
        z \\
        \frac{y}{2}
      \end{pmatrix} \,\middle|\,
      x, y, z \in \mathbb{R}, 3x - 2y + z = 0
    } \text{ mit } \dim\qty\big(\Kern(f)) = 3
  \]
  Weiter ist $\Bild\qty\big(\varphi)$ definiert als
  $\qty\big{\varphi\qty\big(v) \,{\big|}\, v \in V}$.
  Nun ist $\qty{
    f\qty(\begin{pmatrix} 0 \\ 0 \\ 1 \\ 0\end{pmatrix}),
    f\qty(\begin{pmatrix} 0 \\ 0 \\ 0 \\ -1\end{pmatrix})
  } = \qty{
    \begin{pmatrix} 1 \\ 0 \end{pmatrix},
    \begin{pmatrix} 0 \\ 1 \end{pmatrix}
  } = \varepsilon_2$ (die Standardbasis von $\mathbb{R}^2$).
  Da das Bild einer linearen Abbildung nach Lemma 6.18 der Vorlesung
  ein Unterraum ist, folgt $\Bild\qty\big(f) = \hull{\varepsilon_2} = \mathbb{R}^2$.
  Da die Basis 2 Elemente hat, folgt weiter $\dim\qty\big(\Bild(f)) = 2$.

\item Finden Sie eine lineare Abbildung $g \colon \mathbb{R}^4 \to \mathbb{R}$,
  so dass $\Kern\qty\big(g) = W$.
  Gibt es eine Abbildung $g$ mit dieser Eigenschaft, so dass $g$ nicht
  surjektiv ist?
  \subparagraph{Lsg.} Sei
  \[
    g\qty(\begin{pmatrix} x \\ y \\ z \\ t \end{pmatrix}) = 2x - y - t
  \]

  Da $\Bild\qty\big(g)$ ein Unterraum von $\mathbb{R}$ ist,
  gilt für alle $w \in \Bild\qty\big(g), \lambda \in \mathbb{R}$, dass
  $\lambda \cdot w \in \Bild\qty\big(g)$.
  Angenommen $g$ wäre nicht surjektiv, dann existiert ein Element
  $x \in \mathbb{R}$, so dass $x \notin \Bild\qty\big(g)$ und
  $\lambda \cdot w \ne x$ für alle
  $\lambda \in \mathbb{R}, w \in \Bild\qty\big(g)$.

  $\Rightarrow \Bild\qty\big(g) = \qty\big{0}$

  $\Rightarrow g(v) = 0$, somit wäre auch
  $g\qty(\begin{pmatrix} 1 \\ 0 \\ 0 \\ 0\end{pmatrix}) = 0$, aber
  $\begin{pmatrix} 1 \\ 0 \\ 0 \\ 0\end{pmatrix} \notin W$, ein Widerspruch zu
  $\Kern\qty\big(g) = W$.

  $\Rightarrow$ \underline{Es gibt keine lineare Abbildung $g$ mit
    $\Kern\qty\big(g) = W$, so dass $g$ nicht surjektiv ist.}

\item Sei $\mathcal{E}$ die Standardbasis von $\mathbb{R}$,
  $\mathcal{E}_2$ die Standardbasis von $\mathbb{R}^2$ und $\mathcal{E}_4$ die
  Standardbasis von $\mathbb{R}^4$.
  Berechnen Sie $\Mat\qty\big(f, \mathcal{E}_4, \mathcal{E}_2)$ sowie
  $\Mat\qty\big(g, \mathcal{E}_4, \mathcal{E})$.

  \subparagraph{Lsg.} $\Mat\qty\big(f, \mathcal{E}_4, \mathcal{E}_2)$ ist die
  Matrix mit
  \[
    \Mat\qty\big(f, \mathcal{E}_4, \mathcal{E}_2) \cdot
    \begin{pmatrix} x \\ y \\ z \\ t \end{pmatrix} =
    \begin{pmatrix} 3x - 2y + z \\ 2y - t \end{pmatrix}
  \]
  Somit ist $\Mat\qty\big(f, \mathcal{E}_4, \mathcal{E}_2) =
  \begin{pmatrix}
    3 & -2 & 1 & 0 \\
    0 & 2 & 0 & -1
  \end{pmatrix}$
  und
  $\Mat\qty\big(g, \mathcal{E}_4, \mathcal{E}) =
  \begin{pmatrix}
    2 & -1 & 0 & -1
  \end{pmatrix}$.
\end{enumerate}

\paragraph{Aufgabe 7} Gegeben ist die Abbildung
\[
  \varphi \colon \mathbb{R}^3 \to \mathbb{R}^4,
  \begin{pmatrix} x \\ y \\ z \end{pmatrix} \mapsto
  \begin{pmatrix} x - z \\ y - z \\ 0 \\ x - y \end{pmatrix}
\]
Es lässt sich leicht zeigen, dass $\varphi$ eine lineare Abbildung ist.
Die dürfen Sie ohne Beweis verwenden.
Nutzen Sie Lemma 6.21 um zu überprüfen, ob $\varphi$ sogar ein Monomorphismus ist.

\subparagraph{Lsg.} Lemma 6.21 der Vorlesung besagt \emph{``Seien $U$ und $V$
  Vektorräume über einem Körper $K$ und sei $\varphi \colon U \to V$ eine lineare
  Abbildung.
  Es ist $\varphi$ genau dann ein Monomorphismus, wenn
  $\Kern\qty\big(\varphi) = \qty\big{0_U}$ gilt.}

Nun ist $\varphi\qty(
  \begin{pmatrix} 1 \\ 1 \\ 1 \end{pmatrix}
) = \begin{pmatrix} 1 - 1 \\ 1 - 1 \\ 0 \\ 1 - 1 \end{pmatrix}
= \begin{pmatrix} 0 \\ 0 \\ 0 \\ 0 \end{pmatrix}$, also ist
$\begin{pmatrix} 1 \\ 1 \\ 1 \end{pmatrix} \in \Kern\qty\big(\varphi)$.

$\Rightarrow$ \underline{$\varphi$ ist kein Monomorphismus.}

\newpage
\paragraph{Aufgabe 8} Sei $K$ ein Körper und $n \in \mathbb{N}$.
Nach Beispiel 3.3 ist $\mathcal{M}_{n \times n}\qty\big(K)$ ein
Vektorraum über $K$.
Sei
\[
  T \colon \mathcal{M}_{n \times n} \qty\big(K) \to
  \mathcal{M}_{n \times n} \qty\big(K), A \mapsto A^t - A
\]
\begin{enumerate}[(i)]
\item Zeigen Sie, dass $T$ linear ist.
  \subparagraph{Lsg.} Die Abbildung $T$ heißt linear, falls
  \begin{enumerate}[(1)]
  \item Für alle $A, B \in \mathcal{M}_{n \times n} \qty\big(K)$ gilt
    $T\qty\big(A + B) = T\qty\big(A) + T\qty\big(B)$

    Nun ist nach Beispiel 6.10 der Vorlesung $\qty\big(A + B)^t = A^t + B^t$.
    Also ist $T\qty\big(A + B) = \qty\big(A + B)^t - \qty\big(A + B)
    = A^t + B^t - A - B = \qty\big(A^t - A) + \qty\big(B^t - B)
    = T\qty\big(A) + T\qty\big(B)$

  \item Für alle $\lambda \in K$ und $A \in \mathcal{M}_{n \times n} \qty\big(K)$
    gilt $\lambda T\qty\big(A) = T\qty\big(\lambda A)$.

    Nach Beispiel 6.10 der Vorlesung ist $\qty\big(\lambda A)^t = \lambda A^t$.
    Also ist \\
    $\lambda T\qty\big(A) = \lambda \cdot \qty\big(A^t - A)
    = \qty\big(\lambda A^t) - \qty\big(\lambda A)
    = \qty\big(\lambda A)^t - \qty\big(\lambda A) = T\qty\big(\lambda A)$.
  \end{enumerate}

  $\Rightarrow$ \underline{$T$ ist linear.}
\end{enumerate}
\end{document}
