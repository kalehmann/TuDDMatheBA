\documentclass{scrreprt}

\usepackage{aligned-overset}
\usepackage{amsmath}
\usepackage{amssymb}
\usepackage{bm}
\usepackage[shortlabels]{enumitem}
\usepackage{hyperref}
\usepackage[utf8]{inputenc}
\usepackage{multicol}
\usepackage{mathtools}
\usepackage{physics}
\usepackage{tabularx}
\usepackage{titling}
\usepackage{fancyhdr}
\usepackage{xfrac}
\usepackage[dvipsnames]{xcolor}
\usepackage{pgfplots}

\pgfplotsset{compat = newest}
\usetikzlibrary{intersections}
\usetikzlibrary{patterns}
\usepgfplotslibrary{fillbetween}

\author{Karsten Lehmann\\Mat. Nr 4935758}
\date{WiSe 2021/2022}
\title{Hausaufgaben Blatt 09\\Lineare Algebra - Grundlegende Konzepte}

\setlength{\headheight}{26pt}
\pagestyle{fancy}
\fancyhf{}
\lhead{\thetitle}
\rhead{\theauthor}
\lfoot{\thedate}
\rfoot{Seite \thepage}

\newcommand\Abb{\text{Abb}}
\newcommand\hull[1]{\left\langle #1 \right\rangle}

\begin{document}
\paragraph{Aufgabe 6} Lösen Sie das lineare Gleichungssystem indem Sie die
folgenden Schritte ausführen:
\begin{itemize}
\item Bilden Sie die erweiterte Koeffizientenmatrix des Gleichungssystems und
  bringen Sie diese durch elementare Zeilenumformungen in reduzierte
  Stufenform.

\item Bestimmen Sie die Lösungsmenge dess Gleichungssystems.
\end{itemize}
Seien $L_{\mathbb{Q}}$ die Lösungsmenge über $\mathbb{Q}$ und
$L_{\mathbb{R}}$ die Lösungsmenge über $\mathbb{R}$.
Was ist die Beziehung zwischen $L_{\mathbb{Q}}$ und $L_{\mathbb{R}}$?
Geben Sie auch $L_{\mathbb{Z}/19}$ - also die Lösungsmenge über
$\mathbb{Z}/19$ - an.
\begin{flalign*}
  x_1 - 2 x_2 + 4 x_3 + 5 x_4 &= 0 \\
  x_1 + 4 x_2 + 2 x_4 &= 0 \\
  x_1 + 5 x_2 + 7 x_3 + x_4 &= 1 \\
  x_1 + 6 x_2 + 3 x_3 + 13 x_4 &= 4
\end{flalign*}

\subparagraph{Lsg.}
\begin{flalign*}
  \qty(\begin{array}{cccc|c}
    1 & -2 & 4 &  5 & 0 \\
    1 &  4 & 0 &  2 & 0 \\
    1 &  5 & 7 &  1 & 1 \\
    1 &  6 & 3 & 13 & 4
  \end{array})
  \overset{\substack{
      R_2 - R_3 \\
      R_3 - R_1 \\
      R_4 - R_3
  }}&\longrightarrow
  \qty(\begin{array}{cccc|c}
    1 & -2 &  4 &  5 &  0 \\
    0 & -1 & -7 &  1 & -1 \\
    0 &  7 &  3 & -4 &  1 \\
    0 &  1 & -4 & 12 &  3
  \end{array}) \\
  \overset{\substack{
      R_3 + 7\cdot R_2 \\
      R_4 + R_2
  }}&\longrightarrow
  \qty(\begin{array}{cccc|c}
    1 & -2 &   4 &  5 &  0 \\
    0 & -1 &  -7 &  1 & -1 \\
    0 &  0 & -46 &  3 & -6 \\
    0 &  0 & -11 & 13 &  2
  \end{array}) \\
  \overset{\substack{
      \qty\big(46 \cdot R_4 - 11 \cdot R_3) / 565
  }}&\longrightarrow
  \qty(\begin{array}{cccc|c}
    1 & -2 &   4 & 5 &   0 \\
    0 & -1 &  -7 & 1 &  -1 \\
    0 &  0 & -46 & 3 &  -6 \\
    0 &  0 &   0 & 1 & \frac{158}{565}
  \end{array}) \\
  \overset{\substack{
      R_1 - 5 \cdot R_4 \\
      -1 \cdot R_2 + R_4 \\
      \qty(-1 \cdot R_3 + 3 \cdot R_4) / 46
  }}&\longrightarrow
  \qty(\begin{array}{cccc|c}
    1 & -2 & 4 & 0 & -\frac{790}{565} \\
    0 &  1 & 7 & 0 &  \frac{723}{565} \\
    0 &  0 & 1 & 0 &   \frac{84}{565} \\
    0 &  0 & 0 & 1 &  \frac{158}{565}
  \end{array}) \\
  \overset{\substack{
      R_1 - 4 \cdot R_3 \\
       R_2 - 7 \cdot R_4
  }}&\longrightarrow
  \qty(\begin{array}{cccc|c}
    1 & -2 & 0 & 0 & -\frac{1126}{565} \\
    0 &  1 & 0 & 0 &  \frac{135}{565} \\
    0 &  0 & 1 & 0 &   \frac{84}{565} \\
    0 &  0 & 0 & 1 &  \frac{158}{565}
  \end{array}) \\
  \overset{\substack{
       R_1 + 2 \cdot R_2
  }}&\longrightarrow
  \qty(\begin{array}{cccc|c}
    1 & 0 & 0 & 0 & -\frac{856}{565} \\
    0 & 1 & 0 & 0 &  \frac{135}{565} \\
    0 & 0 & 1 & 0 &   \frac{84}{565} \\
    0 & 0 & 0 & 1 &  \frac{158}{565}
  \end{array}) \\
\end{flalign*}
Somit ist
\[
  L = \qty{\begin{pmatrix}
    -\frac{856}{565} \\
    \frac{135}{565} \\
    \frac{84}{565} \\
    \frac{158}{565}
    \end{pmatrix}}
\]
Weiter ist in dem Fall $L_{\mathbb{Q}} = L_{\mathbb{R}}$.
Generell gilt $L_{\mathbb{Q}} \subseteq L_{\mathbb{R}}$, so kann
$L_{\mathbb{R}}$ irrationale Elemente enthalten, die nicht in
$L_{\mathbb{Q}}$ vorkommen.
\newpage
Es ist $-2_{\mathbb{R}} = 17_{\mathbb{Z}/19}$, da
$2_{\mathbb{Z}/19} + 17_{\mathbb{Z}/19} = 0_{\mathbb{Z}/19}$.
\begin{flalign*}
  \qty(\begin{array}{cccc|c}
    1 & 17 & 4 &  5 & 0 \\
    1 &  4 & 0 &  2 & 0 \\
    1 &  5 & 7 &  1 & 1 \\
    1 &  6 & 3 & 13 & 4
  \end{array})
  \overset{
    \substack{
      R_2 + 18 \cdot R_1 \\
      R_3 + 18 \cdot R_1 \\
      R_4 + 18 \cdot R_1
    }
  }&\longrightarrow
  \qty(\begin{array}{cccc|c}
    1 & 17 &  4 &  5 & 0 \\
    0 &  6 & 15 & 16 & 0 \\
    0 &  7 &  3 & 15 & 1 \\
    0 &  8 & 18 &  8 & 4
  \end{array}) \\
  \overset{
    \substack{
      R_3 + 2 \cdot R_2 \\
      R_4 + 5 \cdot R_1
    }
  }&\longrightarrow
  \qty(\begin{array}{cccc|c}
    1 & 17 &  4 &  5 & 0 \\
    0 &  6 & 15 & 16 & 0 \\
    0 &  0 & 14 &  9 & 1 \\
    0 &  0 & 17 & 12 & 4
  \end{array}) \\
  \overset{
    \substack{
      2 \cdot R_4 + 3 \cdot R_3 \\
    }
  }&\longrightarrow
  \qty(\begin{array}{cccc|c}
    1 & 17 &  4 &  5 & 0 \\
    0 &  6 & 15 & 16 & 0 \\
    0 &  0 & 14 &  9 & 1 \\
    0 &  0 &  0 & 13 & 11
  \end{array}) \\
  \overset{
    \substack{
       5 \cdot R_1 + R_4 \\
      R_2 + 9 \cdot R_4 \\
      7 \cdot R_3 + R_4 \\
    }
  }&\longrightarrow
  \qty(\begin{array}{cccc|c}
    5 & 9 &  1 &  0 & 11 \\
    0 & 6 & 15 &  0 &  4 \\
    0 & 0 &  3 &  0 & 18 \\
    0 & 0 &  0 & 13 & 11
  \end{array}) \\
  \overset{
    \substack{
      R_1 + 6 \cdot R_3 \\
      3 \cdot R_2 + 4 \cdot R_3 \\
    }
  }&\longrightarrow
  \qty(\begin{array}{cccc|c}
    5 &  9 & 0 &  0 &  5 \\
    0 & 18 & 0 &  0 &  8 \\
    0 &  0 & 3 &  0 & 18 \\
    0 &  0 & 0 & 13 & 11
  \end{array}) \\
  \overset{
    \substack{
      R_1 + 9 \cdot R_2 \\
    }
  }&\longrightarrow
  \qty(\begin{array}{cccc|c}
    5 &  0 & 0 &  0 &  1 \\
    0 & 18 & 0 &  0 &  8 \\
    0 &  0 & 3 &  0 & 18 \\
    0 &  0 & 0 & 13 & 11
  \end{array}) \\
  \overset{
    \substack{
      4 \cdot R_1 \\
      18 \cdot R_2 \\
      13 \cdot R_3 \\
      3 \cdot R_4 \\
    }
  }&\longrightarrow
  \qty(\begin{array}{cccc|c}
    1 & 0 & 0 & 0 &  4 \\
    0 & 1 & 0 & 0 & 11 \\
    0 & 0 & 1 & 0 &  6 \\
    0 & 0 & 0 & 1 & 14
  \end{array})
\end{flalign*}
Somit ist
\[
  L_{\mathbb{Z}/19} + \qty{\begin{pmatrix}
      4 \\
      11 \\
      6 \\
      14
    \end{pmatrix}}
\]
Probe:
\begin{flalign*}
  1 \cdot 4 + 17 \cdot 11 + 4 \cdot 6 + 5 \cdot 14
  &= 4 + 16 + 5 + 13 &= 0 && && \\
  1 \cdot 4 + 4 \cdot 11 + 2 \cdot 14
  &= 4 + 6 + 9  &= 0\\
  1 \cdot 4 + 5 \cdot 11 + 7 \cdot 6 + 1 \cdot 14
  &= 4 + 17 + 4 + 14 &= 1 \\
  1 \cdot 4 + 6 \cdot 11 + 3 \cdot 6 + 13 \cdot 14
  &= 4 + 9 + 18 + 11 &= 4
\end{flalign*}
\end{document}
