\documentclass{scrreprt}

\usepackage{aligned-overset}
\usepackage{amsmath}
\usepackage{amssymb}
\usepackage{bm}
\usepackage[shortlabels]{enumitem}
\usepackage{hyperref}
\usepackage[utf8]{inputenc}
\usepackage{multicol}
\usepackage{mathtools}
\usepackage{physics}
\usepackage{tabularx}
\usepackage{titling}
\usepackage{fancyhdr}
\usepackage{xfrac}
\usepackage{pgfplots}

\pgfplotsset{compat = newest}
\usetikzlibrary{intersections}
\usetikzlibrary{patterns}
\usepgfplotslibrary{fillbetween}

\author{Karsten Lehmann\\Mat. Nr 4935758}
\date{WiSe 2021/2022}
\title{Hausaufgaben Blatt 03\\Lineare Algebra - Grundlegende Konzepte}

\setlength{\headheight}{26pt}
\pagestyle{fancy}
\fancyhf{}
\lhead{\thetitle}
\rhead{\theauthor}
\lfoot{\thedate}
\rfoot{Seite \thepage}

\begin{document}
\paragraph{Aufgabe 6} Beweisen Sie durch vollständige Induktion, dass die
Gleichung
\[
  \sum_{i = 1}^{n}i^3 = \frac{1}{4}n^2 \qty\big(n + 1)^2
\]
für jede natürliche Zahl $n$ erfüllt ist.
\subparagraph{Lsg.}
\underline{Behauptung:}
\begin{flalign*}
  P(n) \colon \sum_{i = 1}^{n}i^3 &= \frac{1}{4}n^2 \qty\big(n + 1)^2 & \\
\end{flalign*}

\underline{Induktionsanfang:}
\begin{flalign*}
  P(0) \colon \sum_{i = 1}^{0}i^3 &= \frac{1}{4}0^2 \qty\big(0 + 1)^2 & \\
  0 &= 0 \\
  P(1) \colon \sum_{i = 1}^{1}i^3 &= \frac{1}{4}1^2 \qty\big(1 + 1)^2 \\
  1^3 &= \frac{1}{4}^2 \cdot 2^2 \\
  1 &= \frac{4}{4} = 1
\end{flalign*}
$P(0)$ und $P(1)$ sind also wahr.

\underline{Induktionsschritt:}
Sei nun $P(n)$ auch für ein beliebiges $n \in \mathbb{N}$ wahr.
\begin{flalign*}
  P(n + 1) \colon \sum_{i = 1}^{n + 1}i^3 &= \frac{1}{4}(n + 1)^2 \qty\big((n + 1) + 1)^2 & \\
  (n + 1)^3 + \sum_{i = 1}^{n}i^3 &= \frac{1}{4}(n + 1)^2 \qty\big(n + 2)^2 \\
  &= \frac{1}{4} \qty\big(n^2 + 2n + 1) \qty\big(n^2 + 4n + 4) \\
  &= \frac{1}{4} \qty\big(n^4 + 6n^3 + 13n^2 + 12n + 4) \\
  &= \frac{1}{4} \qty\big(n^4 + 2n^3 + n^2 + 4n^3 + 12n^2 + 12n + 4) \\
  &= \frac{1}{4}  n^2 \qty\big(n^2 + 2n + 1) + (n^3 + 3n^2 + 3n + 1) \\
  (n + 1)^3 + \sum_{i = 1}^{n}i^3 &= \frac{1}{4} n^2 \qty\big(n + 1)^2 + (n + 1)^3
\end{flalign*}

$\Rightarrow$ aus der vollständigen Induktion folgt die Behauptung.
\end{document}