\documentclass{scrreprt}

\usepackage{aligned-overset}
\usepackage{amsmath}
\usepackage{amssymb}
\usepackage{bm}
\usepackage[shortlabels]{enumitem}
\usepackage{hyperref}
\usepackage[utf8]{inputenc}
\usepackage{multicol}
\usepackage{mathtools}
\usepackage{physics}
\usepackage{tabularx}
\usepackage{titling}
\usepackage{fancyhdr}
\usepackage{xfrac}
\usepackage{pgfplots}

\pgfplotsset{compat = newest}
\usetikzlibrary{intersections}
\usetikzlibrary{patterns}
\usepgfplotslibrary{fillbetween}

\author{Karsten Lehmann\\Mat. Nr 4935758}
\date{WiSe 2021/2022}
\title{Hausaufgaben Blatt 04\\Lineare Algebra - Grundlegende Konzepte}

\setlength{\headheight}{26pt}
\pagestyle{fancy}
\fancyhf{}
\lhead{\thetitle}
\rhead{\theauthor}
\lfoot{\thedate}
\rfoot{Seite \thepage}

\begin{document}
\paragraph{Aufgabe 6} Beweisen Sie durch vollständige Induktion, dass die
Gleichung
\[
  \sum_{i = 1}^{n}i^3 = \frac{1}{4}n^2 \qty\big(n + 1)^2
\]
für jede natürliche Zahl $n$ erfüllt ist.
\subparagraph{Lsg.}
\underline{Behauptung:}
\begin{flalign*}
  P(n) \colon \sum_{i = 1}^{n}i^3 &= \frac{1}{4}n^2 \qty\big(n + 1)^2 & \\
\end{flalign*}

\underline{Induktionsanfang:}
\begin{flalign*}
  P(0) \colon \sum_{i = 1}^{0}i^3 &= \frac{1}{4}0^2 \qty\big(0 + 1)^2 & \\
  0 &= 0 \\
  P(1) \colon \sum_{i = 1}^{1}i^3 &= \frac{1}{4}1^2 \qty\big(1 + 1)^2 \\
  1^3 &= \frac{1}{4}^2 \cdot 2^2 \\
  1 &= \frac{4}{4} = 1
\end{flalign*}
$P(0)$ und $P(1)$ sind also wahr.

\underline{Induktionsschritt:}
Sei nun $P(n)$ auch für ein beliebiges $n \in \mathbb{N}$ wahr.
\begin{flalign*}
  P(n + 1) \colon \sum_{i = 1}^{n + 1}i^3 &= \frac{1}{4}(n + 1)^2 \qty\big((n + 1) + 1)^2 & \\
  (n + 1)^3 + \sum_{i = 1}^{n}i^3 &= \frac{1}{4}(n + 1)^2 \qty\big(n + 2)^2 \\
  &= \frac{1}{4} \qty\big(n^2 + 2n + 1) \qty\big(n^2 + 4n + 4) \\
  &= \frac{1}{4} \qty\big(n^4 + 6n^3 + 13n^2 + 12n + 4) \\
  &= \frac{1}{4} \qty\big(n^4 + 2n^3 + n^2 + 4n^3 + 12n^2 + 12n + 4) \\
  &= \frac{1}{4}  n^2 \qty\big(n^2 + 2n + 1) + (n^3 + 3n^2 + 3n + 1) \\
  (n + 1)^3 + \sum_{i = 1}^{n}i^3 &= \frac{1}{4} n^2 \qty\big(n + 1)^2 + (n + 1)^3
\end{flalign*}

$\Rightarrow$ aus der vollständigen Induktion folgt die Behauptung.

\newpage
\paragraph{Aufgabe 7} Sei $M$ eine nichtleere endliche Menge.
Sei $P_g(M)$ die Menge aller Teilmengen von $M$, deren Mächtigkeit gerade ist.
Sei $P_u(M)$ die Menge aller Teilmengen von $M$, deren Mächtigkeit ungerade ist.
Zeigen Sie, dass $\abs\big{P_g(M)} = \abs\big{P_u(M)}$ gilt.

\subparagraph{Lsg.} Da nur die Mächtigkeit der Menge betrachtet wird, ist es
in der folgenden Betrachtung nicht relevant, welche Art von Elementen die Menge
$M$ enthält.
Der Einfachheit halber werden natürliche Zahlen $1, \ldots, n$ gewählt um die
Menge $M$ mit $\abs{M} = n$ darzustellen. \\

\underline{Behauptung:}
\[
  A(n) = \abs\big{P_g(\qty{1, \ldots, n})} = \abs\big{P_u(\qty{1, \ldots, n})}
\]

\underline{Induktionsanfang:} Die Aussage $A(0)$ entfällt, da $M$ nichtleer ist.
\begin{flalign*}
  A(1) &\colon \abs\big{P_g(\qty{x_1})} = \abs\big{\qty{\emptyset}}
  = \abs\big{\qty{\qty{x_1}}} = \abs\big{P_u(\qty{x_1})} & \\
  A(2) &\colon \abs\big{P_g(\qty{x_1, x_2})}
  = \abs\big{\qty{\emptyset, \qty{x_1, x_2}}}
  = \abs{\qty{\qty{x_1}, \qty{x_2}}}
  = \abs\big{P_u(\qty{x_1, x_2})}
\end{flalign*}
\underline{Induktionsschritt}: Sei nun $A(n)$ für ein beliebiges
$n \in \mathbb{N}$ wahr.
Die Menge aller Teilmengen von $1, \ldots, n + 1$ deren Mächtigkeit gerade ist
setzt sich zusammen aus
\begin{itemize}
\item der Menge aller Teilmengen von $1, \ldots, n$ deren Mächtigkeit gerade ist -
  $P_g\qty(\qty{1, \ldots, n})$.
\item der Menge aller Teilmengen von $1, \ldots, n$ deren Mächtigkeit ungerade
  ist, jeweils vereinigt mit dem Element $n + 1$ -
  $\qty{M \cup \qty{n + 1} {\big |} M \in P_u\qty(\qty{1, \ldots, n})}$.
\end{itemize}
Also gilt:
\[
  P_g(\qty{1, \ldots, n + 1}) = P_g(\qty{1, \ldots, n}) \cup
  \qty{M \cup \qty{n + 1} {\big |} M \in P_u({1, \ldots, n})}
\]
Analog gilt für die Menge aller Teilmengen von $1, \ldots, n + 1$ deren
Mächtigkeit ungerade ist:
\[
  P_u(\qty{1, \ldots, n + 1}) = P_u(\qty{1, \ldots, n}) \cup
  \qty{M \cup \qty{n + 1} {\big |} M \in P_g({1, \ldots, n})}
\]
Dabei gilt offensichtlich
\begin{equation}
  \tag{1}
  \abs{P_g(\qty{1, \ldots, n})} =
  \abs{\qty{M \cup \qty{n + 1} {\big |} M \in P_u({1, \ldots, n})}}
\end{equation}
und
\begin{equation}
  \tag{2}
  \abs{P_u(\qty{1, \ldots, n})} =
  \abs{\qty{M \cup \qty{n + 1} {\big |} M \in P_g({1, \ldots, n})}}
\end{equation}


\begin{flalign*}
  A(n + 1) \colon \abs\big{P_g(\qty{1, \ldots, n + 1})}
  &= \abs\big{P_u(\qty{1, \ldots, n + 1})} &\\
  \substack{
      \abs\big{P_g(\qty{1, \ldots, n})} + \\
      \abs{\qty{M \cup \qty{ n + 1} {\big |} M \in P_u\qty(\qty{1, \ldots, n})}}
  } &= \substack{
      \abs\big{P_u(\qty{1, \ldots, n})} + \\
      \abs{\qty{M \cup \qty{ n + 1} {\big |} M \in P_g\qty(\qty{1, \ldots, n})}}
  } && {\Big |} \text{ Siehe (1) und (2)} \\
  2 \cdot \abs{P_g(\qty{1, \ldots, n})} &= 2 \cdot \abs{P_u(\qty{1, \ldots, n})}
\end{flalign*}
$\Rightarrow$ nach der vollständigen Induktion folgt die Behauptung.

\end{document}
