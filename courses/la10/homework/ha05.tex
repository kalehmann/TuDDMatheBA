\documentclass{scrreprt}

\usepackage{aligned-overset}
\usepackage{amsmath}
\usepackage{amssymb}
\usepackage{bm}
\usepackage[shortlabels]{enumitem}
\usepackage{hyperref}
\usepackage[utf8]{inputenc}
\usepackage{multicol}
\usepackage{mathtools}
\usepackage{physics}
\usepackage{tabularx}
\usepackage{titling}
\usepackage{fancyhdr}
\usepackage{xfrac}
\usepackage{pgfplots}

\pgfplotsset{compat = newest}
\usetikzlibrary{intersections}
\usetikzlibrary{patterns}
\usepgfplotslibrary{fillbetween}

\author{Karsten Lehmann\\Mat. Nr 4935758}
\date{WiSe 2021/2022}
\title{Hausaufgaben Blatt 05\\Lineare Algebra - Grundlegende Konzepte}

\setlength{\headheight}{26pt}
\pagestyle{fancy}
\fancyhf{}
\lhead{\thetitle}
\rhead{\theauthor}
\lfoot{\thedate}
\rfoot{Seite \thepage}

\begin{document}
\paragraph{Aufgabe 6} Prüfen Sie, ob es sich um einen Körper handelt:
\[
  \qty\big(\mathbb{Z}, \oplus, \odot)
  \text{ mit }
  a \oplus b \coloneqq a + b - 1
  \text{ und }
  a \odot b \coloneqq a + b - ab
\]
\subparagraph{Lsg.}
\underline{Ist $\qty\big(\mathbb{Z}, \oplus)$ eine Gruppe?}
\begin{enumerate}[(i)]
\item Für $a, b, c \in \mathbb{Z}$ gilt
  \begin{flalign*}
    a \oplus \qty\big(b \oplus c)
    \overset{\text{Def. } \oplus}&= a \oplus \qty\big(b + c - 1) \\
    \overset{\text{Def. } \oplus}&= a + \qty\big(b + c - 1) - 1 \\
    \overset{\text{Kommutativität von $\mathbb{Z}$}}&=
    a + \qty\big(b - 1 + c) - 1 \\
    \overset{\text{Assoziativität von $\mathbb{Z}$}}&=
    \qty\big(a + b - 1) + c - 1 \\
    \overset{\text{Def. } \oplus}&= \qty\big(a \oplus b) + c - 1 \\
    \overset{\text{Def. } \oplus}&= \qty\big(a \oplus b) \oplus c
  \end{flalign*}
  $\Rightarrow \qty\big(\mathbb{Z}, \oplus)$ ist \emph{assoziativ}.

\item Sei $a \in \mathbb{Z}$ beliebig.
  Dann ist
  \begin{flalign*}
    a \oplus 1 \overset{\text{Def.} \oplus}&= a + 1 - 1 = a + 0 \\
    &= a \\
    &= a + 0 = a + 1 - 1 \\
    \overset{\text{Kommutativität von $\mathbb{Z}$}}&= 1 + a - 1 \\
    \overset{\text{Def.} \oplus}&= 1 + a
  \end{flalign*}
  $\Rightarrow 1$ ist \emph{neutrales Element} in
  $\qty\big(\mathbb{Z}, \oplus)$.

\item Sei $a \in \mathbb{Z}$ beliebig.
  Dann gilt für $2 - a$
  \begin{flalign*}
    a \oplus \qty\big(2 - a)
    \overset{\text{Def.} \oplus}&= a + \qty\big(2 - a) - 1 \\
    \overset{\text{Kommutativität in $\mathbb{Z}$}}&=
      a + \qty\big(-a + 2) - 1 \\
    \overset{\text{Assoziativität in $\mathbb{Z}$}}&=
      \qty\big(a - a) + \qty\big(2 - 1) \\
    &= 0 + 1 = 1
  \end{flalign*}
  $\Rightarrow$ für jedes Element $a \in \mathbb{Z}$ existiert mit
  $\qty\big(2 - a)$ ein \emph{inverses Element} bezüglich $\oplus$.
\end{enumerate}
$\Rightarrow \qty\big(\mathbb{Z}, \oplus)$ ist eine \textbf{Gruppe}. \\
Seien $a, b \in \mathbb{Z}$ beliebig.
Dann ist
\begin{flalign*}
  a \oplus b \overset{\text{Def.} \oplus}&= a + b + 1 \\
  \overset{\text{Kommutativität in $\mathbb{Z}$}}&= b + a + 1 \\
  \overset{\text{Def. } \oplus}&= b \oplus a
\end{flalign*}
$\Rightarrow \qty\big(\mathbb{Z}, \oplus)$ ist eine \textbf{abelsche Gruppe}.

\newpage
\underline{Ist $\qty\big(\mathbb{Z}, \oplus, \odot)$ ein Ring?}
\begin{enumerate}[(i)]
\item $\qty\big(\mathbb{Z}, \oplus)$ ist eine \emph{abelsche Gruppe}.
\item Für $a, b, c \in \mathbb{Z}$ gilt
  \begin{flalign*}
    a \odot \qty\big(b \odot c) \overset{\text{Def. } \odot}&=
      a \odot \qty\big(b + c - bc) \\
    \overset{\text{Def. } \odot}&=
      a + \qty\big(b + c - bc) - a \cdot \qty\big(b + c - bc) \\
    \overset{\text{Distributivität von $\mathbb{Z}$}}&=
      a + \qty\big(b + c - bc) - ab - ac + abc \\
    \overset{
      \substack{
        \text{Assoziativität/} \\
        \text{Kommutativität}
      } \text{ von } \mathbb{Z}
    }&= \qty\big(a + b - ab) + c - ac - bc + abc \\
    \overset{\text{Distributivität von } \mathbb{Z}}&=
      \qty\big(a + b - ab) + c - \qty\big(a + b - ab) \cdot c \\
    \overset{\text{Def. } \odot}&= \qty\big(a + b - ab) \odot c \\
    \overset{\text{Def. } \odot}&= \qty\big(a \odot b) \odot c
  \end{flalign*}
  $\Rightarrow \qty\big(\mathbb{Z}, \oplus, \odot)$ ist \emph{assoziativ}
  bezüglich der Multiplikation.

\item Für jedes $a \in \mathbb{Z}$ gilt
  \begin{flalign*}
    a \odot 0 \overset{\text{Def. } \oplus}&= a + 0 - a \cdot 0
    = a + 0 - 0 \\
    &= a \\
    &= 0 + a - 0
    = 0 + a - 0 \cdot a \\
    \overset{\text{Def. } \oplus}&= a \odot 0
  \end{flalign*}
  $\Rightarrow 0$ ist \emph{neutrales Element} bezüglich $\odot$ in
  $\mathbb{Z}$.

\item Für $a, b, c \in \mathbb{Z}$ gilt
  \begin{flalign*}
    a \odot \qty\big(b \oplus c) \overset{\text{Def. } \oplus}&=
      a \odot \qty\big(b + c - 1) \\
    \overset{\text{Def. } \odot}&=
      a + \qty\big(b + c - 1) - a \cdot \qty\big(b + c - 1) \\
    \overset{\text{Distributivität in } \mathbb{Z}}&=
      a + \qty\big(b + c - 1) - ab - ac + a \\
    \overset{
      \substack{
        \text{Assoziativität/} \\
        \text{Kommutativität}
      } \text{ von } \mathbb{Z}
    }&= \qty\big(a + b - ab) + \qty\big(a + c - ac) - 1 \\
    \overset{\text{Def. } \oplus}&=
      \qty\big(a + b - ab) \oplus \qty\big(a + c - ac) \\
    \overset{\text{Def. } \odot}&= \qty\big(a \odot b) \oplus \qty\big(a \odot c) \\
  \end{flalign*}
  und
  \begin{flalign*}
    \qty\big(b \oplus c) \odot a \overset{\text{Def. } \oplus}&=
      \qty\big(b + c - 1) \odot a \\
    \overset{\text{Def. } \odot}&=
      \qty\big(b + c - 1) + a - \qty\big(b + c - 1) \cdot a \\
    \overset{\text{Distributivität in } \mathbb{Z}}&=
      \qty\big(b + c - 1) + a - ba - ca + a \\
    \overset{
      \substack{
        \text{Assoziativität/} \\
        \text{Kommutativität}
      } \text{ von } \mathbb{Z}
    }&= \qty\big(b + a - ba) + \qty\big(c + a - ca) - 1 \\
    \overset{\text{Def. } \oplus}&=
      \qty\big(b + a - ba) \oplus \qty\big(c + a - ca) \\
    \overset{\text{Def. } \odot}&= \qty\big(b \odot a) \oplus \qty\big(c \odot a)
  \end{flalign*}
  $\Rightarrow \qty\big(\mathbb{Z}, \oplus, \odot)$ ist \emph{distributiv}.
\end{enumerate}
$\Rightarrow \qty\big(\mathbb{Z}, \oplus, \odot)$ ist ein \textbf{Ring}. \\

Für $a, b \in \mathbb{Z}$ gilt
\begin{flalign*}
  a \cdot b \overset{\text{Def. } \odot}&= a + b - ab \\
  \overset{\text{Kommutativität in } \mathbb{Z}}&= b + a - ba \\
  \overset{\text{Def. } \odot}&= b \odot a
\end{flalign*}
$\Rightarrow \qty\big(\mathbb{Z}, \oplus, \odot)$ ist ein
\textbf{kommutativer Ring}. \\

\underline{Ist $\qty\big(\mathbb{Z}, \oplus, \odot)$ ein Körper?}
\begin{enumerate}[(i)]
\item $\qty\big(\mathbb{Z}, \oplus, \odot)$ ist ein \emph{kommutativer Ring}.
\item Sei $a = 3$.
  Dann gilt
  \[
    a \odot b = 1_{\qty\big(\mathbb{Z}, \oplus, \odot)} = 0
  \]
  für $b = \frac{3}{2} \notin \mathbb{Z}$.

  $\Rightarrow$ es existiert nicht für jedes Element außer
  $0_{\qty\big(\mathbb{Z}, \oplus, \odot)}$ ein inverses Element
  der Multiplikation.
\end{enumerate}
$\Rightarrow \qty\big(\mathbb{Z}, \oplus, \odot)$ ist \textbf{kein Körper}.
\end{document}
