\documentclass{scrreprt}

\usepackage{aligned-overset}
\usepackage{amsmath}
\usepackage{amssymb}
\usepackage{bm}
\usepackage[shortlabels]{enumitem}
\usepackage{hyperref}
\usepackage[utf8]{inputenc}
\usepackage{multicol}
\usepackage{mathtools}
\usepackage{physics}
\usepackage{tabularx}
\usepackage{titling}
\usepackage{fancyhdr}
\usepackage{xfrac}
\usepackage{pgfplots}

\pgfplotsset{compat = newest}
\usetikzlibrary{intersections}
\usetikzlibrary{patterns}
\usepgfplotslibrary{fillbetween}

\author{Karsten Lehmann\\Mat. Nr 4935758}
\date{WiSe 2021/2022}
\title{Hausaufgaben Blatt 02\\Lineare Algebra - Grundlegende Konzepte}

\setlength{\headheight}{26pt}
\pagestyle{fancy}
\fancyhf{}
\lhead{\thetitle}
\rhead{\theauthor}
\lfoot{\thedate}
\rfoot{Seite \thepage}

\begin{document}
\paragraph{Aufgabe 8} Seien $M, N, L$ Mengen und
$f \colon M \to N, g \colon N \to L$ Abbildungen.
\begin{enumerate}[(a)]
\item Angenommen $g \circ f$ ist injektiv
  \begin{enumerate}[label={(a\arabic*)}]
  \item Zeigen Sie, dass $f$ injektiv ist

    \subparagraph{Lsg.} Angenommen $f$ wäre nicht injektiv,
    dann existieren zwei Elemente $m_1 \ne m_2 \in M$ mit
    $f\qty(m_1) = f\qty(m_2) = n \in N$.
    Somit ist
    \[
      (g \circ f)\qty(m_1) = g\qty\big(f(m_1)) = g(n) = g\qty(f(m_2))
      = (g \circ f)(m_2)
    \]

    Das ist ein Widerspruch zu der Annahme, dass $g \circ f$ injektiv ist.

    $\Rightarrow f$ muss auch injektiv sein.
  \item Zeigen Sie, dass $g$ injektiv ist, wenn $f$ surjektiv ist.

    \subparagraph{Lsg.} Angenommen $g$ wäre nicht injektiv,
    dann existieren $n_1 \ne n_2 \in N$ mit $g\qty(n_1) = g\qty(n_2) = l \in L$.
    Da $f$ surjektiv ist, existieren $m_1, m_2 \in M$ mit $f\qty(m_1) = n_1$
    und $f\qty(m_2) = n_2$.

    Somit ist
    \[
      (g \circ f)(m_1) = g(f(m_1)) = g(n_1) = l = g(n_2) = g(f(m_2))
    \]

    Das ist ein Widerspruch zu der Annahme, dass $g \circ f$ injektiv ist.

    $\Rightarrow g$ muss auch injektiv sein.

  \item Geben Sie ein Beispiel an, in dem $g$ nicht injektiv ist.

    \subparagraph{Lsg.} Seien $f \colon \mathbb{N} \to \mathbb{Z}, n \mapsto n$
    und $g \colon \mathbb{Z} \to \mathbb{N}, z \mapsto \abs{z}$.

    Dann ist $(g \circ f)(n) = \abs{n}$.
    Da $n \in \mathbb{N}$ gilt $n \geq 0$ und $\abs{n} = n$.
    Somit ist $(g \circ f)(n) = n$.

    Damit ist $g \circ f$ injektiv, denn für $n_1 \ne n_2 \in \mathbb{N}$ gilt
    $(g \circ f)(n_1) = n_1 \ne n_2 = (g \circ f)(n_2)$.

    Allerdings ist $g$ nicht injektiv, denn für $-1, 1 \in \mathbb{Z}$ gilt
    $g(-1) = g(1) = 1$.
  \end{enumerate}

\setcounter{enumi}{2}
\item Geben Sie ein Beispiel an, in dem $g \circ f$ eine Bijektion ist, aber
  weder $f$ noch $g$ bijektiv ist.

  \subparagraph{Lsg.} Seien $f \colon \mathbb{N}_{> 0} \to \mathbb{N}_{> 0},
  n \mapsto n + 1$ und $g \colon \mathbb{N}_{n \geq 0} \to \mathbb{N}_{\geq 0},
  n \mapsto \abs{n - 1}$.

  Dann ist
  \[
    (g \circ f)(n) = g(f(n)) = g(n + 1) = \abs{(n + 1) - 1} = \abs{n}
    \overset{n \in \mathbb{N}_{> 0}}= n
  \]
  eine Bijektion.
  Allerdings ist $f$ selbst nicht surjektiv, da für $1 \in \mathbb{N}_{> 0}$ kein
  $n \in \mathbb{N}$ existiert mit $f(n) = 1$.

  Weiterhin ist $g$ nicht injektiv, denn für $0, 2 \in \mathbb{N}_{\geq 0}$
  gilt $g(0) = 1 = g(2)$.
\end{enumerate}

\newpage
\paragraph{Aufgabe 9} Zeigen Sie die folgenden Aussagen.
\begin{enumerate}[(a)]
\item $\abs{\mathbb{Z}} = \abs{\mathbb{N}}$
\item $\abs{\mathbb{N} \times \mathbb{N}} = \abs{\mathbb{Z} \times \mathbb{N}}$
\end{enumerate}

\subparagraph{Lsg.} Zwei Mengen heißen gleichmächtig, falls eine bijektive
Abbildung zwischen den Mengen existiert.

\begin{enumerate}[(a)]
\item Sei $f \colon \mathbb{Z} \to \mathbb{N}$ eine Abbildung mit
  $f(x) = \begin{cases}
    2 \cdot (-x) - 1 & \text{, falls } x < 0 \\
    2x & \text{, falls } x \geq 0
  \end{cases}$.

  Dann ist zum Beispiel $f(-2) = 3, f(-1) = 1, f(0) = 0, f(1) = 2, f(3) = 4$.
  Die Abbildung $f$ ist injektiv, da für $z_1 \ne z_2 \in \mathbb{Z}$ gilt
  $f(z_1) \ne f(z_2)$.
  Außerdem ist $f$ surjektiv, denn für jedes $n \in \mathbb{N}$ findet sich ein
  $z \in \mathbb{Z}$ mit $f(z) = n$.

\item Nach dem Satz von Schröder Bernstein sind die Mengen gleichmächtig,
  wenn injektive Abbildungen
  $\mathbb{N} \times \mathbb{N} \to \mathbb{Z} \times \mathbb{N}$ und
  $\mathbb{Z} \times \mathbb{N} \to \mathbb{N} \times \mathbb{N}$ existieren.
  Die Abbildung
  \[
    \mathbb{N} \times \mathbb{N} \to \mathbb{Z} \times \mathbb{N},
    (m, n) \mapsto (n, n)
  \]
  ist offensichtlich injektiv.
  Sei $f$ die bijektive Abbildung aus Aufgabe (a), dann ist auch die Abbildung
  \[
    \mathbb{Z} \times \mathbb{N} \to \mathbb{N} \times \mathbb{N},
    (z, n) \mapsto (f(z), n)
  \]
  injektiv.

  $\Rightarrow$ nach Schröder-Bernstein ist
  $\abs{\mathbb{N} \times \mathbb{N}} = \abs{\mathbb{Z} \times \mathbb{N}}$.
\end{enumerate}

\end{document}