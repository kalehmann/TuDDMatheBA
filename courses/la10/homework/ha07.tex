\documentclass{scrreprt}

\usepackage{aligned-overset}
\usepackage{amsmath}
\usepackage{amssymb}
\usepackage{bm}
\usepackage[shortlabels]{enumitem}
\usepackage{hyperref}
\usepackage[utf8]{inputenc}
\usepackage{multicol}
\usepackage{mathtools}
\usepackage{physics}
\usepackage{tabularx}
\usepackage{titling}
\usepackage{fancyhdr}
\usepackage{xfrac}
\usepackage[dvipsnames]{xcolor}
\usepackage{pgfplots}

\pgfplotsset{compat = newest}
\usetikzlibrary{intersections}
\usetikzlibrary{patterns}
\usepgfplotslibrary{fillbetween}

\author{Karsten Lehmann\\Mat. Nr 4935758}
\date{WiSe 2021/2022}
\title{Hausaufgaben Blatt 07\\Lineare Algebra - Grundlegende Konzepte}

\setlength{\headheight}{26pt}
\pagestyle{fancy}
\fancyhf{}
\lhead{\thetitle}
\rhead{\theauthor}
\lfoot{\thedate}
\rfoot{Seite \thepage}

\newcommand\Char{\text{char}}
\newcommand\ggT{\text{ggT}}

\begin{document}
\paragraph{Aufgabe 6}
Sei $G$ eine Gruppe und seien $A$ und $B$ Untergruppen von $G$.
Zeigen Sie:
\begin{enumerate}[(a)]
\item $A \cap B$ ist eine Untergruppe von $G$

  \subparagraph{Lsg.} Sei $\qty\big(G, *)$ eine Gruppe, dann heißt eine
  Teilmenge $A$ von $G$ heißt Untergruppe von $G$, falls
  \begin{enumerate}[(1)]
  \item $1_G \in A$
  \item Für $a, b \in A$ gilt $a * b \in A$
  \item Für $a \in A$ gilt $a^{-1} \in A$
  \end{enumerate}

  \emph{Zu (1):} Da $A$ und $B$ Untergruppen von $G$ sind, folgt $1_G \in A$ und
  $1_G \in B$, schließlich $1_G \in A \cap B$.

  \emph{Zu (2):} Seien nun $a, b \in A \cap B$ beliebig gewählt.
  Dann gilt $a, b \in A$ und $a, b \in B$.
  Da $A, B$ Untergruppen von $G$ sind, folgt $a + b \in A$ und $a + b \in B$.
  Es folgt $a + b \in A \cap B$.

  \emph{Zu (3):} Sei $a \in A \cap B$ beliebig.
  Es gilt $a \in A$ und $a \in B$.
  Da $A, B$ Untergruppen von $G$ sind, folgt $a^{-1} \in A$ und $a^{-1} \in B$.
  Es folgt $a^{-1} \in A \cap B$.

\item $A \cup B$ ist eine Untergruppe von $G$ genau dann, wenn
  $A \subseteq B$ oder $B \subseteq A$ ist.

  \subparagraph{Lsg.}
  \begin{itemize}
  \item[``$\Rightarrow$''] Sei $A \cup B$ eine Untergruppe von $G$.
    Angenommen $A \not\subset B$ und $B \not\subset A$, dann existieren
    $a \in A$ und $b \in B$ mit $a \notin B$ und $b \notin A$.
    Wären nun $a * b \in A$, dann wäre auch $\qty\big(a * b) * a^{-1} = b \in A$
    - Widerspruch.
    Analog dazu wäre für $a * b \in B$ auch $\qty\big(a * b) * b^{-1} = a \in B$
    - ebenfalls ein Widerspruch.

    $\Rightarrow a * b \not \in A \cup B$

    $\Rightarrow$ Widerspruch zu $A \cup B$ ist Untergruppe.

    $\Rightarrow A \subset B \lor B \subset A$

  \item[``$\Leftarrow$''] Sei $A \subseteq B$, dann ist $A \cup B = B$,
    da $B$ nach Voraussetzung eine Untergruppe von $G$ ist folgt die
    Behauptung. \\
    Sei $B \subseteq A$, dann ist $A \cup B = A$,
    da $A$ nach Voraussetzung eine Untergruppe von $G$ ist folgt die
    Behauptung.
  \end{itemize}

\newpage
\item Zeigen Sie, dass (b) auch für Unterräume von Vektorräumen gilt.
  Zeigen Sie also folgendes: Sind $V$ ein Vektorraum über einem Körper $K$
  und sind $U$ und $W$ Unterräume von $V$, so ist $U \cup W$ genau dann ein
  Unterraum von $V$, wenn $U \subseteq W$ oder $W \subseteq U$ gilt.

  \subparagraph{Lsg.} Seien $K$ ein Körper und $V$ ein Vektorraum über diesen
  Körper.
  Dann heißt eine Teilmenge $W \subseteq K$ Unterraum von $V$, falls
  \begin{enumerate}[(1)]
  \item $0_V \in W$
  \item Für $v, w \in W$ gilt $v + w \in W$
  \item Für $\lambda \in K, w \in W$ gilt $\lambda \cdot w \in W$
  \end{enumerate}

  \begin{itemize}
  \item[``$\Rightarrow$''] Sei $U \cup W$ ein Unterraum von $V$.
    Angenommen $U \not\subset W$ und $W \not\subset U$, dann existieren
    $u \in U, w \in W$ mit $u \notin W, w \notin U$.
    Weil $U \cup W$ ein Unterraum ist, gilt $w + u \in W$ oder $w + u \in U$.

    \begin{minipage}{0.4\textwidth}
      Sei $w + u \in W$, da $W$ ein Unterraum ist, ist auch
      $-\qty\big(1_K) \cdot w \in W$.

      $\Rightarrow w + u + -\qty\big(1_K) \cdot w = u \in W$

      $\Rightarrow$ Widerspruch zu $u \notin W$
    \end{minipage}
    \hfill
    \vrule
    \hfill
    \begin{minipage}{0.4\textwidth}
      $w + u \in U$, da $U$ ein Unterraum ist, ist auch
      $-\qty\big(1_K) \cdot u \in U$.

      $\Rightarrow w + u + -\qty\big(1_K) \cdot u = w \in W$

      $\Rightarrow$ Widerspruch zu $w \notin U$
    \end{minipage}

    $\Rightarrow w + u \notin U \cup W$

    $\Rightarrow$ Widerspruch zur Vorraussetzung $U \cup W$ ist Unterraum.

  \item[``$\Leftarrow$''] Sei $U \subseteq W$, dann ist $U \cup W = W$,
    da $W$ nach Voraussetzung ein Unterraum von $V$ ist folgt die Behauptung. \\
    Sei $W \subseteq U$, dann ist $U \cup W = U$,
    da $U$ nach Voraussetzung ein Unterraum von $V$ ist folgt die Behauptung.
  \end{itemize}
\end{enumerate}

\end{document}
