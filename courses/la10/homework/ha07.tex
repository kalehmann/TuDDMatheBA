\documentclass{scrreprt}

\usepackage{aligned-overset}
\usepackage{amsmath}
\usepackage{amssymb}
\usepackage{bm}
\usepackage[shortlabels]{enumitem}
\usepackage{hyperref}
\usepackage[utf8]{inputenc}
\usepackage{multicol}
\usepackage{mathtools}
\usepackage{physics}
\usepackage{tabularx}
\usepackage{titling}
\usepackage{fancyhdr}
\usepackage{xfrac}
\usepackage[dvipsnames]{xcolor}
\usepackage{pgfplots}

\pgfplotsset{compat = newest}
\usetikzlibrary{intersections}
\usetikzlibrary{patterns}
\usepgfplotslibrary{fillbetween}

\author{Karsten Lehmann\\Mat. Nr 4935758}
\date{WiSe 2021/2022}
\title{Hausaufgaben Blatt 07\\Lineare Algebra - Grundlegende Konzepte}

\setlength{\headheight}{26pt}
\pagestyle{fancy}
\fancyhf{}
\lhead{\thetitle}
\rhead{\theauthor}
\lfoot{\thedate}
\rfoot{Seite \thepage}

\newcommand\Char{\text{char}}
\newcommand\ggT{\text{ggT}}

\begin{document}
\paragraph{Aufgabe 6}
Sei $G$ eine Gruppe und seien $A$ und $B$ Untergruppen von $G$.
Zeigen Sie:
\begin{enumerate}[(a)]
\item $A \cap B$ ist eine Untergruppe von $G$

  \subparagraph{Lsg.} Eine Teilmenge $A$ von $G$ heißt Untergruppe von $G$,
  falls
  \begin{enumerate}[(1)]
  \item $1_G \in A$
  \item Für $a, b \in A$ gilt $a * b \in A$
  \item Für $a \in A$ gilt $a^{-1} \in A$
  \end{enumerate}

  \emph{Zu (1):} Da $A$ und $B$ Untergruppen von $G$ sind, folgt $1_G \in A$ und
  $1_G \in B$, schließlich $1_G \in A \cap B$.

  \emph{Zu (2):} Seien nun $a, b \in A \cap B$ beliebig gewählt.
  Dann gilt $a, b \in A$ und $a, b \in B$.
  Da $A, B$ Untergruppen von $G$ sind, folgt $a + b \in A$ und $a + b \in B$.
  Es folgt $a + b \in A \cap B$.

  \emph{Zu (3):} Sei $a \in A \cap B$ beliebig.
  Es gilt $a \in A$ und $a \in B$.
  Da $A, B$ Untergruppen von $G$ sind, folgt $a^{-1} \in A$ und $a^{-1} \in B$.
  Es folgt $a^{-1} \in A \cap B$.

\item $A \cup B$ ist eine Untergruppe von $G$ genau dann, wenn
  $A \subseteq B$ oder $B \subseteq A$ ist.

\item Zeigen Sie, dass (b) auch für Unterräume von Vektorräumen gilt.
  Zeigen Sie also folgendes: Sind $V$ ein Vektorraum über einem Körper $K$
  und sind $U$ und $W$ Unterräume von $V$, so ist $U \cup W$ genau dann ein
  Unterraum von $V$, wenn $U \subseteq W$ oder $W \subseteq U$ gilt.
\end{enumerate}

\end{document}
