\documentclass{article}
\usepackage{aligned-overset}
\usepackage{amsmath}
\usepackage{amssymb}
\usepackage{bm}
\usepackage[shortlabels]{enumitem}
\usepackage{hyperref}
\usepackage[utf8]{inputenc}
\usepackage{mathtools}
\usepackage{physics}
\usepackage{titling}
\usepackage{tikz}
\usetikzlibrary{calc}
\usepackage{fancyhdr}
\usepackage{xfrac}

\newcommand{\tikzmark}[1]{\tikz[overlay,remember picture]{ \node (#1) {};}}

\author{Karsten Lehmann}
\date{WiSe 2020}
\title{Hausaufgabe 06 Lineare Algebra}

\pagestyle{fancy}
\fancyhf{}
\lhead{\thetitle}
\rhead{\theauthor}
\lfoot{\thedate}
\rfoot{Seite \thepage}

\begin{document}
\section*{Übung 28}

Eine lineare Abbildung $f \colon V \to W$ zwischen zwei $\mathbb{R}$-Vektorräumen ist durch ihre Matrix
Matrix $a = M^{\mathfrak{B}}_{\mathfrak{C}}(f)$ bezüglich der Basen $\mathfrak{B}$ und $\mathfrak{C}$
gegeben:

\[
  A = \begin{pmatrix}
    1  & -3 \\
    -2 & 6  \\
  \end{pmatrix},
  \begin{pmatrix}
    6 & 8 & 9 \\
    0 & 1 & 6 \\ 
  \end{pmatrix},
  \begin{pmatrix}
    -2 & 1  &  3 & -2 \\
     8 & 4  & 12 & -8 \\
     4 & -2 & -6 &  4 \\
  \end{pmatrix},
  \begin{pmatrix}
    0 & 0 &  0 & -1 \\
    0 & 0 & -1 &  0 \\
    0 & 1 &  0 &  9 \\
    1 & 0 &  0 &  9 \\
  \end{pmatrix}
\]

Finden Sie:

\begin{enumerate}[(i)]
\item Dimension und Basen von Kern und Bild der Abbildung
\item den Rang der Abbildung
\item Basen $\mathfrak{B}'$ und $\mathfrak{C}'$ mit
  \[
    M^{\mathfrak{B}'}_{\mathfrak{C}'}(f) = \begin{pmatrix}
      1_r & 0 \\
      0   & 0 \\
    \end{pmatrix}
  \]
\item Bestimmen Sie außerdem jeweils, ob die Abbildung injektiv, surjektiv, bijektiv ist und
  brechnen Sie im letzteren Fall die Matrix $M^{\mathfrak{B}'}_{\mathfrak{C}'}(f^{-1})$ der inversen Abbildung.
\end{enumerate}

\begin{enumerate}[1)]
\item
  Sei $A = \begin{pmatrix}
    1  & -3 \\
    -2 & 6  \\
  \end{pmatrix}$. Mit dem Gauß-Jordan-Verfahren über $A$ erhält man
  $\begin{pmatrix}
    1 & -3 \\
    0 &  0 \\
  \end{pmatrix}$.

  \[
    \begin{pmatrix}0\\0\end{pmatrix} = \begin{pmatrix}1&-3\\0&0\end{pmatrix} \cdot \begin{pmatrix}x_1\\x_2\end{pmatrix} =
    \begin{pmatrix}x_1 - 3\cdot x_2\\0\end{pmatrix}
  \]

  Somit ist $x_1 = 3 \cdot x_2$ und $\left\{ x = \begin{pmatrix}3\\1\end{pmatrix}\right\}$ eine Basis von $\ker A.$.
  Weiterhin ist $\{ \Phi_\mathfrak{B}(x)\}$  eine Basis von $\ker f$ und $\dim \ker f = 1$.
  
  $\begin{pmatrix}-3\\6\end{pmatrix}$ ist von $\begin{pmatrix}1\\-2\end{pmatrix}$ linear Abhängig.
  Die Menge $\{ \Phi_\mathfrak{C}(\begin{pmatrix}1\\-2\end{pmatrix})\}$ ist eine Basis von $\Im f$ mit $\dim \Im f = 1$ und
  der Rang von $f = 1$ 

\item
  Sei $A = \begin{pmatrix}
    6 & 8 & 9 \\
    0 & 1 & 6 \\
  \end{pmatrix}$. Mit dem Gauß-Jordan-Verfahren über $A$ erhält man
  $\begin{pmatrix}
    1 & 0 & -\frac{13}{2} \\
    0 & 1 & 6             \\
  \end{pmatrix}$.

  \[
    \begin{pmatrix}0\\0\end{pmatrix} = \begin{pmatrix}1&0&-\frac{13}{2}\\0&1&6\end{pmatrix} \cdot \begin{pmatrix}x_1\\x_2\\x_3\end{pmatrix} =
    \begin{pmatrix}x_1 - \frac{13}{2}\cdot x_3\\x_2 + 6x_3\end{pmatrix}
  \]

  Somit ist $x_1 = 3 \cdot x_2$ und $\left\{ x = \begin{pmatrix}3\\1\end{pmatrix}\right\}$ eine Basis von $\ker A.$.
  Weiterhin ist $\{ \Phi_\mathfrak{B}(x)\}$  eine Basis von $\ker f$ und $\dim \ker f = 1$.
  
  $\begin{pmatrix}-3\\6\end{pmatrix}$ ist von $\begin{pmatrix}1\\-2\end{pmatrix}$ linear Abhängig.
  Die Menge $\{ \Phi_\mathfrak{C}(\begin{pmatrix}1\\-2\end{pmatrix})\}$ ist eine Basis von $\Im f$ mit $\dim \Im f = 1$ und
  der Rang von $f = 1$ 
\end{enumerate}
  
\end{document}