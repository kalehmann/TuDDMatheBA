\documentclass{scrreprt}

\usepackage{aligned-overset}
\usepackage{amsmath}
\usepackage{amssymb}
\usepackage{bm}
\usepackage[shortlabels]{enumitem}
\usepackage{hyperref}
\usepackage[utf8]{inputenc}
\usepackage{multicol}
\usepackage{mathtools}
\usepackage{physics}
\usepackage{tabularx}
\usepackage{titling}
\usepackage{fancyhdr}
\usepackage{xfrac}
\usepackage{pgfplots}

\pgfplotsset{compat = newest}
\usetikzlibrary{intersections}
\usetikzlibrary{patterns}
\usepgfplotslibrary{fillbetween}

\author{Karsten Lehmann\\Mat. Nr 4935758}
\date{WiSe 2021/2022}
\title{Hausaufgaben Blatt 06\\Lineare Algebra - Grundlegende Konzepte}

\setlength{\headheight}{26pt}
\pagestyle{fancy}
\fancyhf{}
\lhead{\thetitle}
\rhead{\theauthor}
\lfoot{\thedate}
\rfoot{Seite \thepage}

\newcommand\ggT{\text{ggT}}

\begin{document}
\paragraph{Aufgabe 7}
\begin{enumerate}[(a)]
\item Seien $m$ und $n$ natürliche Zahlen mit $m > n$.
  Seien außerdem $k$ und $r$ natürliche Zahlen mit $m = kn + r$ und
  $r \in \qty\big{0, 1, \ldots, n - 1}$.
  Zeigen Sie, dass $\ggT\qty\big(M, n) = \ggT\qty\big(n, r)$

  \subparagraph{Lsg.} Man sagt $a | b$ oder $a$ teilt $b$, falls ein $z$
  in $\mathbb{Z}$ existiert mit $z \cdot a = b$.
  Weiter ist $\ggT\qty\big(n, r)$ das größte $l \in \mathbb{N}$ mit $l$ teilt $n$ und
  $l$ teilt $r$.

  $\Rightarrow \exists \: z_1, z_2 \in \mathbb{Z} \: \colon \:
  n = z_1 \cdot \ggT\qty\big(n, r)$ und $r = z_2 \cdot \ggT\qty\big(n, r)$.

  Nach der Aufgabenstellung sind $m, r \geq 0$ und $\ggT\qty\big(n, r)$ ist nach
  Definition größer als $0$.

  $\Rightarrow z_1, z_2 \in \mathbb{N}$
  \begin{flalign*}
    \Rightarrow m &= k \cdot z_1 \cdot \ggT\qty\big(n, r) +
    z_2 \cdot \ggT\qty\big(n, r) & \\
    &= \ggT\qty\big(n, r) \cdot \underset{\in \mathbb{N}}{\underbrace{\qty\big(z_1 \cdot k + z_2)}}
  \end{flalign*}
  $\Rightarrow \ggT\qty\big(n, r)$ teilt $m$.

  Sei nun $l \in \mathbb{N}$ mit $\ggT\qty\big(n, r) < l < n$ und
  $l$ teilt $m$, sowie $l$ teilt $n$.
  Dann existieren $y_1, y_2 \in \mathbb{Z}$ mit $m = y_1 \cdot l$ und
  $n = y_2 \cdot l$.
  \begin{flalign*}
    \Rightarrow& &y_1 \cdot l &= k \cdot y_2 \cdot l + r && && && && && \\
    &&y_1 \cdot l -  k \cdot y_2 \cdot l &= r \\
    &&\underset{\in \mathbb{Z}}{\underbrace{\qty\big(y_1 - k \cdot y_2)}} \cdot l
    &= r
  \end{flalign*}
  $\Rightarrow l$ teilt $r$

  $\Rightarrow$ Widerspruch ($l$ wäre größer als der \emph{größte} gemeinsame
  Teiler von $r$ und $n$)

  $\Rightarrow$ es gibt keinen größeren Teiler von $m$ und $n$ als
  $\ggT\qty\big(n, r)$.

  $\Rightarrow$ \underline{$\ggT\qty\big(m, n) = \ggT\qty\big(n, r)$}
\end{enumerate}
\end{document}
