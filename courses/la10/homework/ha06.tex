\documentclass{scrreprt}

\usepackage{aligned-overset}
\usepackage{amsmath}
\usepackage{amssymb}
\usepackage{bm}
\usepackage[shortlabels]{enumitem}
\usepackage{hyperref}
\usepackage[utf8]{inputenc}
\usepackage{multicol}
\usepackage{mathtools}
\usepackage{physics}
\usepackage{tabularx}
\usepackage{titling}
\usepackage{fancyhdr}
\usepackage{xfrac}
\usepackage[dvipsnames]{xcolor}
\usepackage{pgfplots}

\pgfplotsset{compat = newest}
\usetikzlibrary{intersections}
\usetikzlibrary{patterns}
\usepgfplotslibrary{fillbetween}

\author{Karsten Lehmann\\Mat. Nr 4935758}
\date{WiSe 2021/2022}
\title{Hausaufgaben Blatt 06\\Lineare Algebra - Grundlegende Konzepte}

\setlength{\headheight}{26pt}
\pagestyle{fancy}
\fancyhf{}
\lhead{\thetitle}
\rhead{\theauthor}
\lfoot{\thedate}
\rfoot{Seite \thepage}

\newcommand\Char{\text{char}}
\newcommand\ggT{\text{ggT}}

\begin{document}
\paragraph{Aufgabe 7}
\begin{enumerate}[(a)]
\item Seien $m$ und $n$ natürliche Zahlen mit $m > n$.
  Seien außerdem $k$ und $r$ natürliche Zahlen mit $m = kn + r$ und
  $r \in \qty\big{0, 1, \ldots, n - 1}$.
  Zeigen Sie, dass $\ggT\qty\big(M, n) = \ggT\qty\big(n, r)$

  \subparagraph{Lsg.} Man sagt $a | b$ oder $a$ teilt $b$, falls ein $z$
  in $\mathbb{Z}$ existiert mit $z \cdot a = b$.
  Weiter ist $\ggT\qty\big(n, r)$ das größte $l \in \mathbb{N}$ mit $l$ teilt $n$ und
  $l$ teilt $r$.

  $\Rightarrow \exists \: z_1, z_2 \in \mathbb{Z} \: \colon \:
  n = z_1 \cdot \ggT\qty\big(n, r)$ und $r = z_2 \cdot \ggT\qty\big(n, r)$.

  Nach der Aufgabenstellung sind $m, r \geq 0$ und $\ggT\qty\big(n, r)$ ist nach
  Definition größer als $0$.

  $\Rightarrow z_1, z_2 \in \mathbb{N}$
  \begin{flalign*}
    \Rightarrow m &= k \cdot z_1 \cdot \ggT\qty\big(n, r) +
    z_2 \cdot \ggT\qty\big(n, r) & \\
    &= \ggT\qty\big(n, r) \cdot \underset{\in \mathbb{N}}{\underbrace{\qty\big(z_1 \cdot k + z_2)}}
  \end{flalign*}
  $\Rightarrow \ggT\qty\big(n, r)$ teilt $m$.

  Sei nun $l \in \mathbb{N}$ mit $\ggT\qty\big(n, r) < l < n$ und
  $l$ teilt $m$, sowie $l$ teilt $n$.
  Dann existieren $y_1, y_2 \in \mathbb{Z}$ mit $m = y_1 \cdot l$ und
  $n = y_2 \cdot l$.
  \begin{flalign*}
    \Rightarrow& &y_1 \cdot l &= k \cdot y_2 \cdot l + r && && && && && \\
    &&y_1 \cdot l -  k \cdot y_2 \cdot l &= r \\
    &&\underset{\in \mathbb{Z}}{\underbrace{\qty\big(y_1 - k \cdot y_2)}} \cdot l
    &= r
  \end{flalign*}
  $\Rightarrow l$ teilt $r$

  $\Rightarrow$ Widerspruch ($l$ wäre größer als der \emph{größte} gemeinsame
  Teiler von $r$ und $n$)

  $\Rightarrow$ es gibt keinen größeren Teiler von $m$ und $n$ als
  $\ggT\qty\big(n, r)$.

  $\Rightarrow$ \underline{$\ggT\qty\big(m, n) = \ggT\qty\big(n, r)$}

\item Nutzen Sie (a), um $\ggT\qty\big(1055, 1026)$ zu berechnen.
  Finden Sie außerdem $s, t \in \mathbb{Z}$ mit
  $\ggT\qty\big(1055, 1026) = s \cdot 1055 + t \cdot 1026$.

  \subparagraph{Lsg.} Es ist $1055 = k \cdot 1026 + r$ mit $k = 1$ und
  $r = 29$.

  Nach (a) ist $\ggT\qty\big(1055, 1026) = \ggT\qty\big(1026, 29)$.
  Da $29$ eine Primzahl ist, kommen als Teiler $1$ und $29$ in Frage.
  $1026 / 29 = 35$ Rest $11$.

  $\Rightarrow \ggT\qty\big(1026, 29) = 1 = \ggT\qty\big(1055, 1026)$.

  \newpage
  Nach dem erweiterten euklidischen Algorithmus:
  \begin{flalign*}
    \colorbox{PineGreen!30}{1055} &= 1 \cdot \colorbox{Apricot!50}{1026} + \colorbox{YellowGreen!50}{29} \\
    \colorbox{Apricot!50}{1026} &= 35 \cdot \colorbox{YellowGreen!50}{29} + \colorbox{Salmon!50}{11} \\
    \colorbox{YellowGreen!50}{29} &= 2 \cdot \colorbox{Salmon!50}{11} + \colorbox{Cyan!30}{7} \\
    \colorbox{Salmon!50}{11} &= 1 \cdot \colorbox{Cyan!30}{7} + \colorbox{Orchid!30}{4} \\
    \colorbox{Cyan!30}{7} &= 1 \cdot \colorbox{Orchid!30}{4} + \colorbox{SeaGreen!30}{3} \\
    \colorbox{Orchid!30}{4} &= 1 \cdot \colorbox{SeaGreen!30}{3} + 1 \\
    \colorbox{SeaGreen!30}{3} &= 3 \cdot \underline{1} + 0
  \end{flalign*}
  ... und rückwärts
  \begin{flalign*}
    \underline{1} &= \colorbox{Orchid!30}{4} - \colorbox{SeaGreen!30}{3} \\
    &= \colorbox{Orchid!30}{4} - \qty\big(\colorbox{Cyan!30}{7} - \colorbox{Orchid!30}{4})
    = 2 \cdot \colorbox{Orchid!30}{4} - \colorbox{Cyan!30}{7} \\
    &= 2 \cdot \qty\big(\colorbox{Salmon!50}{11} - \colorbox{Cyan!30}{7}) - \colorbox{Cyan!30}{7}
    = 2 \cdot \colorbox{Salmon!50}{11} - 3 \cdot \colorbox{Cyan!30}{7} \\
    &= 2 \cdot \colorbox{Salmon!50}{11} - 3 \cdot \qty\big(\colorbox{YellowGreen!50}{29} - 2 \cdot \colorbox{Salmon!50}{11})
    = 8 \cdot \colorbox{Salmon!50}{11} - 3 \cdot \colorbox{YellowGreen!50}{29} \\
    &= 8 \cdot \qty\big(\colorbox{Apricot!50}{1026} - 35 \cdot \colorbox{YellowGreen!50}{29}) - 3 \cdot \colorbox{YellowGreen!50}{29}
    = 8 \cdot \colorbox{Apricot!50}{1026} - 283 \cdot \colorbox{YellowGreen!50}{29} \\
    &= 8 \cdot \colorbox{Apricot!50}{1026} - 283 \cdot \qty\big(\colorbox{PineGreen!30}{1055} -  \colorbox{Apricot!50}{1026}) \\
    &= 291 \cdot  \colorbox{Apricot!50}{1026} - 283 \cdot \colorbox{PineGreen!30}{1055}
  \end{flalign*}
  $\Rightarrow \ggT\qty\big(1055, 1026) = s \cdot 1055 + t \cdot 1026$ mit
  $s = -283$ und $t = 291$.

\item Folgern Sie aus (b), dass $\qty[1026]$ in $\mathbb{Z}_{/ 1055}$
  invertierbar ist und bestimmen Sie das multiplikative Inverse von
  $\qty[1026]$ in $\mathbb{Z}_{/ 1055}$.

  \subparagraph{Lsg.} Nach Lemma 2.35 der Vorlesung (\emph{``Sei
    $k \in \mathbb{Z}$. Dann ist $\qty[k] \in \mathbb{Z}_{/ n}$
    genau dann invertierbar, wenn $\ggT\qty\big(k, n) = 1$ gilt''}) und
  (b) ist $\qty[1026]$ in $\mathbb{Z}_{/ 1055}$ invertierbar.

  Das multiplikative Inverse von $1026$ in $\mathbb{Z}_{/ 1055}$ ist $291$.
\end{enumerate}
\newpage
\paragraph{Aufgabe 8} Sei $\qty\big(R, +, \cdot)$ ein Ring.
\begin{enumerate}[(i)]
\item Beweisen Sie Lemma 2.36 (a): Für jedes $z \in \mathbb{Z}$ und jedes
  $a \in R$ gilt $za = \qty\big(z1_R)a$

  \subparagraph{Lsg.} \underline{Behauptung:}
  $P(z) \colon z \cdot a = \qty\big(z \cdot 1_R) \cdot a$.

  \underline{Induktionsanfang:}
  $P(0) \colon 0 \cdot a \overset{\text{Not. 2.11}}= 0_R =
  0_R \cdot a = \qty\big(0 \cdot 1_R) \cdot a$ und \\
  $P(1) \colon 1 \cdot a = 0_R + a = a = 1_R \cdot a =
  \qty\big(1 \cdot 1_R) \cdot a$

  \underline{Induktionsschritt:} Sei nun $P(z)$ für ein beliebiges
  $z \in \mathbb{N}$ wahr.
  Dann ist
  \begin{flalign*}
    \qty\big(z + 1) \cdot a \overset{\text{Not. 2.11}}&= z \cdot a + a
    \overset{\text{Siehe }P(1)}= z \cdot a + 1 \cdot a &\\
    \overset{P(z) \text{ und } P(1)}&= \qty(z \cdot 1_R) \cdot a +
      \qty(1 \cdot 1_R) \cdot a \\
    \overset{\text{Distributivität im Ring}}&= \qty(z \cdot 1_R + 1 \cdot 1_R) \cdot a \\
    \overset{\text{Lemma 2.12 (b)}}&= \qty(\qty\big(z + 1) \cdot 1_R) \cdot a
  \end{flalign*}
  Aus der vollständigen Induktion folgt die Behauptung für $z \in \mathbb{N}$.

  Sei nun $z \in \mathbb{Z}_{< 0}$ beliebig.
  Dann ist $(-z) \in \mathbb{N}$ und es gilt $(-z) \cdot a = \qty\big((-z)1_R)a$.
  Aus Lemma 2.12 (a)  und Lemma 2.22 (a) folgt $-(za) = -\qty(\qty\big(z1_R) a)$.

  $\Rightarrow za = \qty\big(z1_R)a$ für alle $z \in \mathbb{Z}$.

\item Beweisen Sie Lemma 2.40: Wenn $R$ endlich ist, so ist $\Char(R) \ne 0$.
  Insbesondere ist die Charakteristik eine endlichen Körpers immer eine
  Primzahl.

  \subparagraph{Lsg,} Sei $R$ endlich.
  Angenommen $\Char(R) = 0$, dann ist $n \cdot 1_R \ne 0_R$ für alle
  $n \in \mathbb{N}$.
  Da $R$ endlich ist, ist auch $\qty\big{n \cdot 1_R {\big |} n \in \mathbb{N}}$
  endlich.
  Folglich existieren $m > n \in \mathbb{N}$ mit
  \begin{align*}
    m \cdot 1_R &= n \cdot 1_R && {\Big |} + \qty(-\qty\big(n \cdot 1_R)) \\
    m \cdot 1_R + \qty(-\qty\big(n \cdot 1_R)) &= n \cdot 1_R + \qty(-\qty\big(n \cdot 1_R)) \\
    m \cdot 1_R + \qty(-\qty\big(n \cdot 1_R)) &= 0_R && \overset{\text{Lemma 2.12 (a)}}\iff \\
    m \cdot 1_R + \qty\big(- n) \cdot 1_R &= 0_R && \overset{\text{Lemma 2.12 (b)}}\iff \\
    \underset{\in \mathbb{N}}{\underbrace{\qty\big(m - n)}} \cdot 1_R &= 0
  \end{align*}

  $\Rightarrow$ Widerspruch zu $n \cdot 1_R \ne 0_R$ für alle $n \in \mathbb{N}$.

  $\Rightarrow$ \underline{$\Char(R) \ne 0$}.

  Sei nun $\qty\big(R, +, \cdot)$ ein endlicher Körper.
  Dann ist $\qty\big(R, +, \cdot)$ ein endlicher Ring.

  $\Rightarrow \Char(R) \ne 0$

  Aus Lemma 2.39 (a) der Vorlesung (\emph{``Sei $R$ ein Ring mit
    $n \coloneqq \Char(R) \ne 0$. Falls $R$ ein Körper ist, so ist $n$ eine
    Primzahl''}) folgt der zweite Teil der Behauptung.
\end{enumerate}
\end{document}
