\documentclass{scrreprt}

\usepackage{aligned-overset}
\usepackage{amsmath}
\usepackage{amssymb}
\usepackage{bm}
\usepackage[shortlabels]{enumitem}
\usepackage{hyperref}
\usepackage[utf8]{inputenc}
\usepackage{multicol}
\usepackage{mathtools}
\usepackage{physics}
\usepackage{tabularx}
\usepackage{titling}
\usepackage{fancyhdr}
\usepackage{xfrac}
\usepackage[dvipsnames]{xcolor}
\usepackage{pgfplots}

\pgfplotsset{compat = newest}
\usetikzlibrary{intersections}
\usetikzlibrary{patterns}
\usepgfplotslibrary{fillbetween}

\author{Karsten Lehmann\\Mat. Nr 4935758}
\date{WiSe 2021/2022}
\title{Hausaufgaben Blatt 07\\Lineare Algebra - Grundlegende Konzepte}

\setlength{\headheight}{26pt}
\pagestyle{fancy}
\fancyhf{}
\lhead{\thetitle}
\rhead{\theauthor}
\lfoot{\thedate}
\rfoot{Seite \thepage}

\newcommand\Abb{\text{Abb}}

\begin{document}
\paragraph{Aufgabe 6} Geben Sie einen Körper $K$ an, so dass die Vektoren
\[
  \begin{pmatrix}1_K \\ 1_K \\ 0_K \end{pmatrix},
  \begin{pmatrix}1_K \\ 0_K \\ 1_K \end{pmatrix},
  \begin{pmatrix}0_K \\ 1_K \\ 1_K \end{pmatrix},
\]
linear abhängig in $K^3$ sind.
Begründen Sie Ihre Antwort.

\subparagraph{Lsg.} Sei $K = \mathbb{Z}_{/2}$, dann ist
$1_k + 1_k = 0_k$, $0_k + 1_k = 1_k + 0_k = 1_k$ und $0_k + 0_k$.
Seien nun $k_1 = k_2 = k_3$ das neutrale Element der Multiplikation,
dann ist
\[
  k_1 \cdot \begin{pmatrix}1_K \\ 1_K \\ 0_K \end{pmatrix} +
  k_2 \cdot \begin{pmatrix}1_K \\ 0_K \\ 1_K \end{pmatrix} +
  k_3 \cdot \begin{pmatrix}0_K \\ 1_K \\ 1_K \end{pmatrix} =
  \begin{pmatrix}0_k \\ 0_k \\ 0_k\end{pmatrix}
\]
Somit sind die Vektoren in $K^3$ linear abhängig.


\end{document}
