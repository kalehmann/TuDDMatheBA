\documentclass{scrreprt}

\usepackage{aligned-overset}
\usepackage{amsmath}
\usepackage{amssymb}
\usepackage{bm}
\usepackage[shortlabels]{enumitem}
\usepackage{hyperref}
\usepackage[utf8]{inputenc}
\usepackage{multicol}
\usepackage{mathtools}
\usepackage{physics}
\usepackage{tabularx}
\usepackage{titling}
\usepackage{fancyhdr}
\usepackage{xfrac}
\usepackage[dvipsnames]{xcolor}
\usepackage{pgfplots}

\pgfplotsset{compat = newest}
\usetikzlibrary{intersections}
\usetikzlibrary{patterns}
\usepgfplotslibrary{fillbetween}

\author{Karsten Lehmann\\Mat. Nr 4935758}
\date{WiSe 2021/2022}
\title{Hausaufgaben Blatt 08\\Lineare Algebra - Grundlegende Konzepte}

\setlength{\headheight}{26pt}
\pagestyle{fancy}
\fancyhf{}
\lhead{\thetitle}
\rhead{\theauthor}
\lfoot{\thedate}
\rfoot{Seite \thepage}

\newcommand\Abb{\text{Abb}}
\newcommand\hull[1]{\left\langle #1 \right\rangle}

\begin{document}
\paragraph{Aufgabe 5} Sei
\[
  W \coloneqq \qty{
    \begin{pmatrix}x_1 \\ x_2 \\ x_3 \\ x_4\end{pmatrix}
    \, \middle| \,
    x_1 - 3x_2 = 0, x_3 - 2x_4 = 0
  } \subseteq \mathbb{R}^4
\]
\begin{enumerate}[(a)]
\item Finden Sie Vektoren $v_1, v_2 \in \mathbb{R}^4$ mit $\hull{v_1, v_2} = W$.
  \subparagraph{Lsg.}
  \[
    v_1 = \begin{pmatrix}3 \\ 1 \\ 0 \\ 0\end{pmatrix},
    v_2 = \begin{pmatrix}0 \\ 0 \\ 2 \\ 1\end{pmatrix}
  \]
\item Ist $W$ ein Unterraum von $\mathbb{R}^4$?
  Falls ja, bestimmen Sie eine Basis von $W$ und geben Sie die Dimension von
  $W$ an.

  \subparagraph{Lsg.} Es gilt
  \begin{enumerate}[(1)]
  \item Es ist
    $0_{\mathbb{R}^4} = \begin{pmatrix}0 \\ 0 \\ 0 \\ 0\end{pmatrix} \in W$,
    da $0 - 3 \cdot 0 = 0$ und $0 - 2 \cdot 0 = 0$.

  \item Seien $u = \begin{pmatrix}u_1 \\ u_2 \\ u_3 \\ u_4\end{pmatrix},
    v = \begin{pmatrix}v_1 \\ v_2 \\ v_3 \\ v_4\end{pmatrix} \in W$ beliebig.
    Dann ist bereits $v_1 - 3 \cdot v_2 = 0$ und $u_1 - 3 \cdot u_2 = 0$, sowie
    $v_3 - 2 \cdot v_4 = 0$ und $u_3 - 2 \cdot u_4 = 0$.

    Folglich sind
    \begin{flalign*}
      \qty\big(v_1 - 3 \cdot v_2) + \qty\big(u_1 - 3 \cdot u_2)
      &= \qty\big(v_1 - 3 \cdot v_2 + u_1 - 3 \cdot u_2) & \\
      &= \qty\big(v_1 + u_1 - 3 \cdot v_2 - 3 \cdot u_2) \\
      &= \qty\big(v_1 + u_1) - 3 \cdot \qty\big(v_2 + u_2) = 0
    \end{flalign*}
    und
    \begin{flalign*}
      \qty\big(v_3 - 2 \cdot v_4) + \qty\big(u_3 - 2 \cdot u_4)
      &= \qty\big(v_3 - 2 \cdot v_4 + u_3 - 2 \cdot u_4) & \\
      &= \qty\big(v_3 + u_3 - 2 \cdot v_4 - 2 \cdot u_4) \\
      &= \qty\big(v_3 + u_3) - 2 \cdot \qty\big(v_4 + u_4) = 0
    \end{flalign*}

    $\Rightarrow$ \underline{$u + v \in W$}

  \item Sei $u = \begin{pmatrix}u_1 \\ u_2 \\ u_3 \\ u_4\end{pmatrix} \in W$
    und $\lambda \in \mathbb{R}$ beliebig.
    Dann ist $u_1 - 3 \cdot u_2 = 0$ und $u_3 - 2 \cdot u_4 = 0$.
    Folglich sind auch $\lambda \cdot u_1 - 3 \cdot \lambda \cdot u_2 = 0$
    und $\lambda \cdot u_3 - 2 \cdot \lambda \cdot u_4 = 0$.

    $\Rightarrow$ \underline{$\lambda \cdot U \in W$}
  \end{enumerate}

  $\Rightarrow$ \underline{$W$ ist ein Unterraum von $\mathbb{R}^4$}.

  Die Vektoren $v_1, v_2$ aus Teilaufgabe (a) sind offensichtlich linear
  unabhängig und spannen $W$ auf, folglich ist ist $\qty\big{v_1, v_2}$
  eine Basis von $W$.

  Weiter ist $\dim\qty\big(W) = 2$.
\end{enumerate}

\paragraph{Aufgabe 6} Geben Sie einen Körper $K$ an, so dass die Vektoren
\[
  \begin{pmatrix}1_K \\ 1_K \\ 0_K \end{pmatrix},
  \begin{pmatrix}1_K \\ 0_K \\ 1_K \end{pmatrix},
  \begin{pmatrix}0_K \\ 1_K \\ 1_K \end{pmatrix},
\]
linear abhängig in $K^3$ sind.
Begründen Sie Ihre Antwort.

\subparagraph{Lsg.} Sei $K = \mathbb{Z}_{/2}$, dann ist
$1_k + 1_k = 0_k$, $0_k + 1_k = 1_k + 0_k = 1_k$ und $0_k + 0_k$.
Seien nun $k_1 = k_2 = k_3$ gleich dem neutralen Element der Multiplikation
in $K$, dann ist
\[
  k_1 \cdot \begin{pmatrix}1_K \\ 1_K \\ 0_K \end{pmatrix} +
  k_2 \cdot \begin{pmatrix}1_K \\ 0_K \\ 1_K \end{pmatrix} +
  k_3 \cdot \begin{pmatrix}0_K \\ 1_K \\ 1_K \end{pmatrix} =
  \begin{pmatrix}0_k \\ 0_k \\ 0_k\end{pmatrix}
\]
Somit sind die Vektoren in $K^3$ linear abhängig.

\paragraph{Aufgabe 7} Seien $X$ eine nichtleere Menge und sei $K$ ein Körper.
Sei $V \coloneqq \Abb\qty\big(X, K) = \qty\big{f \colon X \to K}$.
Definiere
\begin{flalign*}
  + \colon &V \times V \to V \\
  &\qty\big(f, g) \mapsto f + g \colon X \to K
\end{flalign*}
mit $\qty\big(f + g)\qty\big(x) \coloneqq f\qty\big(x) + g\qty\big(x)$
und
\begin{flalign*}
  \cdot &\colon K \times V \to V \\
  &\qty\big(\lambda, g) \mapsto \lambda \cdot g \colon X \to K
\end{flalign*}
mit $\qty\big(\lambda \cdot g)\qty\big(x) \coloneqq \lambda \cdot g\qty\big(x)$.
Im Folgenden darf ohne Beweis angenommen werden, dass $V$ mit $+$ und $\cdot$
ein Vektorraum über $K$ ist (gezeigt in Aufgabe 7 (a) auf dem Blatt 7).

Für $a \in X$ sei
\[
  \delta_a \colon X \to K, x \mapsto \delta_{ax} \coloneqq
  \begin{cases}
    1 & \text{falls } x = a \\
    0 & \text{falls } x \ne a
  \end{cases}
\]
\newpage
\begin{enumerate}[(a)]
\item Zeigen Sie: Sind $a_1, \ldots, a_n \in X$ paarweise verschieden, so sind
  $\delta_{a_1}, \ldots, \delta_{a_n} \in V$ linear unabhängig.

  \subparagraph{Lsg.} Angenommen $\delta_{a_1}, \ldots, \delta_{a_n} \in V$
  wären linear abhängig, dann existieren $\lambda_1, \ldots \lambda_n \in K$
  mit $\lambda_1 \cdot \delta_{a_1} + \ldots + \lambda_n \cdot \delta_{a_n} = 0$
  und $\lambda_j \ne 0_K$ für mindestens ein $j \leq n$.

  Sei nun $i \ne 0_K \leq n$, dann ist
  \[
    \lambda_1 \cdot \delta_{a_1} + \ldots +
    \lambda_{i - 1} \cdot \delta_{a_{i - 1}} +
    \lambda_{i + 1} \cdot \delta_{a_{i + 1}} +
    \ldots
    \lambda_n \cdot \delta_{a_n} =
    -\lambda_i \cdot \delta_{a_i}
  \]
  Folglich ist
  \[
    \qty\Big(\lambda_1 \cdot \delta_{a_1} + \ldots +
    \lambda_{i - 1} \cdot \delta_{a_{i - 1}} +
    \lambda_{i + 1} \cdot \delta_{a_{i + 1}} +
    \ldots
    \lambda_n \cdot \delta_{a_n})\qty\Big(a_i) =
    -\lambda_i \cdot \qty\Big(\delta_{a_i})\qty\Big(a_i) = -\lambda_i
  \]
  Wären nun alle
  $\delta_{a_1}, \ldots, \delta_{a_{i - 1}}, \delta_{a_{i + 1}}, \delta_{a_n}$
  von $\delta_{a_i}$ verschieden, so wäre
  \[
    \qty\Big(\lambda_1 \cdot \delta_{a_1} + \ldots +
    \lambda_{i - 1} \cdot \delta_{a_{i - 1}} +
    \lambda_{i + 1} \cdot \delta_{a_{i + 1}} +
    \ldots
    \lambda_n \cdot \delta_{a_n})\qty\Big(a_i) = 0_K
  \]
  ein Widerspruch zu $\lambda_i \ne 0_K$.
  $\Rightarrow$ es existieren $j \ne i \leq n$ mit $a_i = a_j$,
  ein Widerspruch zu $a_1, \ldots, a_n$ sind paarweise verschieden.

\item Folgern Sie aus (a), dass die Menge
  $\qty\big{\delta_a \, {\big |} \, a \in X}$ linear unabhängig ist.

  \subparagraph{Lsg.} Angenommen  $\qty\big{\delta_a \, {\big |} \, a \in X}$
  wäre linear abhängig, dann existieren zwei Elemente $a_1 \ne a_2 \in X$, so dass
  $\delta_{a_1}, \delta_{a_2}$ linear abhängig sind - ein Widerspruch zu
  Teilaufgabe (a).

\item Folgern Sie aus (b), dass $V$ nicht endlich-dimensional ist, wenn
  $X$ unendlich ist.

  \subparagraph{Lsg.} Da $X$ unendlich ist, folgt aus (b) die Existenz einer
  unendlichen, linear unabhängigen Teilmenge
  $\qty\big{\delta_a \, {\big |} \, a \in X} \subseteq V$.
  Angenommen $V$ wäre nun endlich-dimensional, dann wäre dies ein Widerspruch
  zu Korollar 3.48 der Vorlesung (\emph{``Angenommen $V$ ist endlich-dimensional,
    dann ist auch jede linear unabhängige Teilmenge von $V$ endlich.''})

\item Sei nun angenommen, dass $X$ endlich ist und sei
  $U \coloneqq \qty\big{\delta_a \,{\big |}\, a \in X}$.
  Zeigen Sie, dass $V = \hull{U}$.
  Begründen Sie, dass $U$ eine Basis von $V$ ist und geben Sie die
  Dimension von $V$ an.

  \subparagraph{Lsg.} Sei $f \colon X \to K$ ein beliebige Abbildung.
  Dann gilt für jedes $x \in X$, dass
  \[
    f\qty\big(x) = \lambda = \lambda \cdot 1 = \lambda \cdot \delta_x\qty\big(x)
  \]
  mit $\lambda \in K$.
  Weiter gilt nach der Definition von $\delta$, dass für
  $x \ne y \in X, \lambda \in K$ gilt  $\lambda \cdot \delta_y\qty\big(x) = 0$.
  Folglich lässt sich $f$ auch durch
  \[
    \lambda_{x_1} \cdot \delta_{x_1} + \ldots + \lambda_{x_n} \cdot \delta_{x_n}
  \]
  mit $\qty\big{x_1, \ldots, x_n} = X$ und $\lambda_{x_i} = f\qty\big(x_i)$ darstellen.

  $\Rightarrow$ $f$ ist eine Linearkombination von Elementen aus $U$.

  Da $f$ beliebig gewählt ist, lässt sich jede Abbildung $g \in V$ als
  Linearkombination von Elementen aus $U$ darstellen.

  $\Rightarrow$ $\hull{U} = V$

  Nach Teilaufgabe (b) ist $U$ linearunabhängig.

  $\Rightarrow$ \underline{$U$ ist eine Basis von $V$}

  Da $X$ endlich ist, ist auch die Basis $U$ endlich.

  $\Rightarrow$ \underline{$X$ ist endlich-dimensional mit
    $\dim\qty\big(X) = \abs{U}$}
\end{enumerate}
\end{document}
