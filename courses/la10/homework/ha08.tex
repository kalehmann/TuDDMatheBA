\documentclass{scrreprt}

\usepackage{aligned-overset}
\usepackage{amsmath}
\usepackage{amssymb}
\usepackage{bm}
\usepackage[shortlabels]{enumitem}
\usepackage{hyperref}
\usepackage[utf8]{inputenc}
\usepackage{multicol}
\usepackage{mathtools}
\usepackage{physics}
\usepackage{tabularx}
\usepackage{titling}
\usepackage{fancyhdr}
\usepackage{xfrac}
\usepackage[dvipsnames]{xcolor}
\usepackage{pgfplots}

\pgfplotsset{compat = newest}
\usetikzlibrary{intersections}
\usetikzlibrary{patterns}
\usepgfplotslibrary{fillbetween}

\author{Karsten Lehmann\\Mat. Nr 4935758}
\date{WiSe 2021/2022}
\title{Hausaufgaben Blatt 08\\Lineare Algebra - Grundlegende Konzepte}

\setlength{\headheight}{26pt}
\pagestyle{fancy}
\fancyhf{}
\lhead{\thetitle}
\rhead{\theauthor}
\lfoot{\thedate}
\rfoot{Seite \thepage}

\newcommand\hull[1]{\left\langle #1 \right\rangle}

\begin{document}
\paragraph{Aufgabe 5} Sei
\[
  W \coloneqq \qty{
    \begin{pmatrix}x_1 \\ x_2 \\ x_3 \\ x_4\end{pmatrix}
    \, \middle| \,
    x_1 - 3x_2 = 0, x_3 - 2x_4 = 0
  } \subseteq \mathbb{R}^4
\]
\begin{enumerate}[(a)]
\item Finden Sie Vektoren $v_1, v_2 \in \mathbb{R}^4$ mit $\hull{v_1, v_2} = W$.
  \subparagraph{Lsg.}
  \[
    v_1 = \begin{pmatrix}3 \\ 1 \\ 0 \\ 0\end{pmatrix},
    v_2 = \begin{pmatrix}0 \\ 0 \\ 2 \\ 1\end{pmatrix}
  \]
\item Ist $W$ ein Unterraum von $\mathbb{R}^4$?
  Falls ja, bestimmen Sie eine Basis von $W$ und geben Sie die Dimension von
  $W$ an.

  \subparagraph{Lsg.} Es gilt
  \begin{enumerate}[(1)]
  \item Es ist
    $0_{\mathbb{R}^4} = \begin{pmatrix}0 \\ 0 \\ 0 \\ 0\end{pmatrix} \in W$,
    da $0 - 3 \cdot 0 = 0$ und $0 - 2 \cdot 0 = 0$.

  \item Seien $u = \begin{pmatrix}u_1 \\ u_2 \\ u_3 \\ u_4\end{pmatrix},
    v = \begin{pmatrix}v_1 \\ v_2 \\ v_3 \\ v_4\end{pmatrix} \in W$ beliebig.
    Dann ist bereits $v_1 - 3 \cdot v_2 = 0$ und $u_1 - 3 \cdot u_2 = 0$, sowie
    $v_3 - 2 \cdot v_4 = 0$ und $u_3 - 2 \cdot u_4 = 0$.

    Folglich sind
    \begin{flalign*}
      \qty\big(v_1 - 3 \cdot v_2) + \qty\big(u_1 - 3 \cdot u_2)
      &= \qty\big(v_1 - 3 \cdot v_2 + u_1 - 3 \cdot u_2) & \\
      &= \qty\big(v_1 + u_1 - 3 \cdot v_2 - 3 \cdot u_2) \\
      &= \qty\big(v_1 + u_1) - 3 \cdot \qty\big(v_2 + u_2) = 0
    \end{flalign*}
    und
    \begin{flalign*}
      \qty\big(v_3 - 2 \cdot v_4) + \qty\big(u_3 - 2 \cdot u_4)
      &= \qty\big(v_3 - 2 \cdot v_4 + u_3 - 2 \cdot u_4) & \\
      &= \qty\big(v_3 + u_3 - 2 \cdot v_4 - 2 \cdot u_4) \\
      &= \qty\big(v_3 + u_3) - 2 \cdot \qty\big(v_4 + u_4) = 0
    \end{flalign*}

    $\Rightarrow$ \underline{$u + v \in W$}

  \item Sei $u = \begin{pmatrix}u_1 \\ u_2 \\ u_3 \\ u_4\end{pmatrix} \in W$
    und $\lambda \in \mathbb{R}$ beliebig.
    Dann ist $u_1 - 3 \cdot u_2 = 0$ und $u_3 - 2 \cdot u_4 = 0$.
    Folglich sind auch $\lambda \cdot u_1 - 3 \cdot \lambda \cdot u_2 = 0$
    und $\lambda \cdot u_3 - 2 \cdot \lambda \cdot u_4 = 0$.

    $\Rightarrow$ \underline{$\lambda \cdot U \in W$}
  \end{enumerate}

  $\Rightarrow$ \underline{$W$ ist ein Unterraum von $\mathbb{R}^4$}.

  Die Vektoren $v_1, v_2$ aus Teilaufgabe (a) sind offensichtlich linear
  unabhängig und spannen $W$ auf, folglich ist ist $\qty\big{v_1, v_2}$
  eine Basis von $W$.

  Weiter ist $\dim\qty\big(W) = 2$.
\end{enumerate}

\paragraph{Aufgabe 6} Geben Sie einen Körper $K$ an, so dass die Vektoren
\[
  \begin{pmatrix}1_K \\ 1_K \\ 0_K \end{pmatrix},
  \begin{pmatrix}1_K \\ 0_K \\ 1_K \end{pmatrix},
  \begin{pmatrix}0_K \\ 1_K \\ 1_K \end{pmatrix},
\]
linear abhängig in $K^3$ sind.
Begründen Sie Ihre Antwort.

\subparagraph{Lsg.} Sei $K = \mathbb{Z}_{/2}$, dann ist
$1_k + 1_k = 0_k$, $0_k + 1_k = 1_k + 0_k = 1_k$ und $0_k + 0_k$.
Seien nun $k_1 = k_2 = k_3$ das neutrale Element der Multiplikation,
dann ist
\[
  k_1 \cdot \begin{pmatrix}1_K \\ 1_K \\ 0_K \end{pmatrix} +
  k_2 \cdot \begin{pmatrix}1_K \\ 0_K \\ 1_K \end{pmatrix} +
  k_3 \cdot \begin{pmatrix}0_K \\ 1_K \\ 1_K \end{pmatrix} =
  \begin{pmatrix}0_k \\ 0_k \\ 0_k\end{pmatrix}
\]
Somit sind die Vektoren in $K^3$ linear abhängig.


\end{document}
