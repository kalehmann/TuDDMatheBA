\documentclass{scrreprt}

\usepackage{aligned-overset}
\usepackage{amsmath}
\usepackage{amssymb}
\usepackage{bm}
\usepackage[shortlabels]{enumitem}
\usepackage{hyperref}
\usepackage[utf8]{inputenc}
\usepackage{multicol}
\usepackage{mathtools}
\usepackage{physics}
\usepackage{tabularx}
\usepackage{titling}
\usepackage{fancyhdr}
\usepackage{xfrac}
\usepackage{pgfplots}

\pgfplotsset{compat = newest}
\usetikzlibrary{intersections}
\usetikzlibrary{patterns}
\usepgfplotslibrary{fillbetween}

\author{Karsten Lehmann\\Mat. Nr 4935758}
\date{WiSe 2021/2022}
\title{Hausaufgaben Blatt 01\\Lineare Algebra - Grundlegende Konzepte}

\setlength{\headheight}{26pt}
\pagestyle{fancy}
\fancyhf{}
\lhead{\thetitle}
\rhead{\theauthor}
\lfoot{\thedate}
\rfoot{Seite \thepage}

\begin{document}
\paragraph{Aufgabe 6} Gegeben seien die Mengen
$A \coloneqq \qty\big{n \in \mathbb{N} {\big |} n^3 > 8n^2}$,
$B \coloneqq \qty\big{2n {\big |} n \in \mathbb{N}, \abs{n - 4} \leq 1}$
und $C \coloneqq \qty\big{z + 10 {\big |} z \in \mathbb{Z}, z^2 < 9}$.
Berechnen Sie die Mengen
\[
  \qty\big(A \cap B) \times \qty\big(B \cap C) \text{ und }
  \mathcal{P}\qty\big(\qty(A \cap B) \times (B \cap C))
\]

\subparagraph{Lsg.}
$A = \qty{9, 10, 11, \ldots}$,
$B = \qty{6, 8, 10}$,
$C = \qty{8, 9, 10, 11, 12}$.

$A \cap B = \qty{10}$, $B \cap C = \qty{8, 10}$.
\[
  \qty\big(A \cap B) \times \qty\big(B \cap C) = \qty\big{(10, 8), (10, 10)}
\]

\[
  \mathcal{P}\qty\big(\qty(A \cap B) \times (B \cap C)) =
  \qty\Big{
    \emptyset,
    \qty\big{\qty(10, 8)}, \qty\big{\qty(10, 10)},
    \qty\big{\qty(10, 8), \qty(10, 10)}
  }
\]

\end{document}