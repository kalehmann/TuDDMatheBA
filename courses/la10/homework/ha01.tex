\documentclass{scrreprt}

\usepackage{aligned-overset}
\usepackage{amsmath}
\usepackage{amssymb}
\usepackage{bm}
\usepackage[shortlabels]{enumitem}
\usepackage{hyperref}
\usepackage[utf8]{inputenc}
\usepackage{multicol}
\usepackage{mathtools}
\usepackage{physics}
\usepackage{tabularx}
\usepackage{titling}
\usepackage{fancyhdr}
\usepackage{xfrac}
\usepackage{pgfplots}

\pgfplotsset{compat = newest}
\usetikzlibrary{intersections}
\usetikzlibrary{patterns}
\usepgfplotslibrary{fillbetween}

\author{Karsten Lehmann\\Mat. Nr 4935758}
\date{WiSe 2021/2022}
\title{Hausaufgaben Blatt 01\\Lineare Algebra - Grundlegende Konzepte}

\setlength{\headheight}{26pt}
\pagestyle{fancy}
\fancyhf{}
\lhead{\thetitle}
\rhead{\theauthor}
\lfoot{\thedate}
\rfoot{Seite \thepage}

\begin{document}
\paragraph{Aufgabe 6} Gegeben seien die Mengen
$A \coloneqq \qty\big{n \in \mathbb{N} {\big |} n^3 > 8n^2}$,
$B \coloneqq \qty\big{2n {\big |} n \in \mathbb{N}, \abs{n - 4} \leq 1}$
und $C \coloneqq \qty\big{z + 10 {\big |} z \in \mathbb{Z}, z^2 < 9}$.
Berechnen Sie die Mengen
\[
  \qty\big(A \cap B) \times \qty\big(B \cap C) \text{ und }
  \mathcal{P}\qty\big(\qty(A \cap B) \times (B \cap C))
\]

\subparagraph{Lsg.}
$A = \qty{9, 10, 11, \ldots}$,
$B = \qty{6, 8, 10}$,
$C = \qty{8, 9, 10, 11, 12}$.

$A \cap B = \qty{10}$, $B \cap C = \qty{8, 10}$.
\[
  \qty\big(A \cap B) \times \qty\big(B \cap C) = \qty\big{(10, 8), (10, 10)}
\]

\[
  \mathcal{P}\qty\big(\qty(A \cap B) \times (B \cap C)) =
  \qty\Big{
    \emptyset,
    \qty\big{\qty(10, 8)}, \qty\big{\qty(10, 10)},
    \qty\big{\qty(10, 8), \qty(10, 10)}
  }
\]

\paragraph{Aufgabe 7} Sei $M$ eine Menge.
Von den folgenden Aussagen beschreiben einige denselben Sachverhalt.
Geben Sie mit kurzer Begründung an, welche das sind.
\begin{multicols}{3}
  \begin{enumerate}[(1)]
  \item $\qty{x} \in M$
  \item $\qty{x} \subseteq M$
  \item $x \in M$
  \item $M \setminus \qty{x} \ne \emptyset$
  \item $\qty{x} \setminus M \ne \emptyset$
  \item $\qty{x} \cap M \ne \emptyset$
  \end{enumerate}
\end{multicols}

\subparagraph{Lsg.} Die Aussagen (2), (3) und (6) beschreiben denselben
Sachverhalt.
\[
  \qty{x} \subseteq M \iff x \in M \iff \qty{x} \cap M \ne \emptyset
\]

Wenn $x$ in $M$ enthalten ist, dann enthält die Schnittmenge von $M$ und
jeder Menge, die $x$ enthält mindestens das Element $x$ und ist somit nicht
leer.

Außerdem heißt eine Menge eine Teilmenge von $M$, falls jedes Element aus dieser
Menge in $M$ enthalten ist.
Somit folgt aus $x$ ist in $M$ enthalten, dass die Menge $\qty{x}$ eine Teilmenge
von $M$ ist.

\paragraph{Aufgabe 8}
\begin{enumerate}[(i)]
\item Geben Sie Beispiele für Mengen $A, B, C, D$ an, für welche die Gleichung
  \[
    \qty(A \setminus B) \cup (C \setminus D) = (A \cup C) \setminus (B \cup D)
  \]
  verletzt ist.

  \subparagraph{Lsg.}
  \begin{itemize}
  \item Sei $A = D = \qty{1, 2}$ und $B = C = \qty{3, 4}$.
    Dann ist $\qty(A \setminus B) \cup (C \setminus D) = \qty{1, 2, 3, 4}$ und
    $(A \cup C) \setminus (B \cup D) = \emptyset$.

  \item Sei $A = \qty{1, 2}$, $B = \qty{3}$, $C =  \qty{2, 3}$ und $D = \qty{1}$.
    Dann ist $\qty(A \setminus B) \cup (C \setminus D) = \qty{1, 2, 3}$ und
    $(A \cup C) \setminus (B \cup D) = \qty{2}$.
  \end{itemize}

\newpage
\item Beweisen sie, dass die Inklusion
  $\qty\big(A \cup C) \setminus \qty\big(B \cup D)
  \subseteq \qty(A \setminus B) \cup \qty(C \setminus D)$ für alle Mengen
  $A, B, C, D$ richtig ist.

  \subparagraph{Lsg.} $x \in A \setminus B \iff x \in A \land \neg (x \in B)$
  \begin{flalign*}
    x \in (A \cup B) \setminus (B \cup D)
    &\iff x \in (A \cup C) \land \neg (x \in B \cup D) & \\
    &\iff (x \in A \lor x \in C) \land \neg (x \in B \lor x \in D) \\
    &\iff (x \in A \land \neg (x \in B \lor x \in D)) \lor (x \in C \land \neg (x \in B \lor x \in D)) \\
    \overset{\text{Abschwächung}}&\Rightarrow (x \in A \land \neg (x \in B)) \lor (x \in C \land \neg (x \in D)) \\
    &\iff x \in (A \setminus B) \lor x \in (C \setminus D) \\
    &\iff x \in (A \setminus B) \cup (C \setminus D)
  \end{flalign*}
\end{enumerate}

\end{document}