\documentclass{scrreprt}

\usepackage{aligned-overset}
\usepackage{amsmath}
\usepackage{amssymb}
\usepackage{bm}
\usepackage[shortlabels]{enumitem}
\usepackage{hyperref}
\usepackage[utf8]{inputenc}
\usepackage{multicol}
\usepackage{mathtools}
\usepackage{physics}
\usepackage{tabularx}
\usepackage{titling}
\usepackage{fancyhdr}
\usepackage{xfrac}
\usepackage[dvipsnames]{xcolor}
\usepackage{pgfplots}

\pgfplotsset{compat = newest}
\usetikzlibrary{intersections}
\usetikzlibrary{patterns}
\usepgfplotslibrary{fillbetween}

\author{Karsten Lehmann\\Mat. Nr 4935758}
\date{WiSe 2021/2022}
\title{Hausaufgaben Blatt 10\\Lineare Algebra - Grundlegende Konzepte}

\setlength{\headheight}{26pt}
\pagestyle{fancy}
\fancyhf{}
\lhead{\thetitle}
\rhead{\theauthor}
\lfoot{\thedate}
\rfoot{Seite \thepage}

\newcommand\Abb{\text{Abb}}
\newcommand\hull[1]{\left\langle #1 \right\rangle}

\begin{document}
\paragraph{Aufgabe 6} Gegeben sind die folgenden Matrizen mit Einträgen aus
dem Körper $\mathbb{Z}/3$.
\[
  A \coloneqq \begin{pmatrix} 2 & 2 \end{pmatrix},
  B \coloneqq \begin{pmatrix} 2 & 0 \\ 1 & 2 \end{pmatrix}
\]
Welche der Produkte $AA$, $AB$, $BA$, $BB$ sind definiert?
Berechnen Sie alle definierten Produkte.
Geben Sie dabei alle Matrixeinträge in der Form $k$ mit $0 \leq k \leq 2$ an.

\subparagraph{Lsg.} Das Produkt $AB$ zweier Matrizen $A$ und $B$ ist definiert,
wenn die Zahl der Spalten von $A$ mit der Zahl der Zeilen von $B$ übereinstimmt.
Somit sind die Produkte
\[
  AB = \begin{pmatrix}
    2 \cdot 2 + 2 \cdot 1 & 2 \cdot 0 + 2 \cdot 2
  \end{pmatrix} = \begin{pmatrix} 0 & 1 \end{pmatrix}
\]
und
\[
  BB = \begin{pmatrix}
    2 \cdot 2 + 0 \cdot 1 & 2 \cdot 0 + 0 \cdot 2 \\
    1 \cdot 2 + 2 \cdot 1 & 1 \cdot 0 + 2 \cdot 2
  \end{pmatrix} = \begin{pmatrix} 1 & 0 \\ 1 & 1 \end{pmatrix}
\]

\paragraph{Aufgabe 7} Sei $K$ ein Körper und sei
\[
  H \coloneqq \qty{
    \begin{pmatrix}
      1_k & a \\
      0_K & 1_K
    \end{pmatrix}
    \,\middle|\,
    a \in K
  }
\]
Zeigen Sie, dass $H$ zusammen mit der üblichen Multiplikation von Matrizen eine
abelsche Gruppe ist.

\subparagraph{Lsg.} Eine Menge $H$ mit einer binären Operation
$\cdot \colon H \times H \to H$ heißt Gruppe, falls gilt
\begin{enumerate}[(1)]
\item $\qty\big(A \cdot B) \cdot C = A \cdot \qty\big(B \cdot C)$ für alle
  $A, B, C \in H$
\item Es existiert ein neutrales Element  $I \in H$, sodass
  $A \cdot I = I \cdot A = A$ für jedes Element $A \in H$
\item Für jedes Element $A \in H$ existiert ein Element $A^{-1} \in H$, so dass
  $A \cdot A^{-1} = I = A^{-1} \cdot A$
\end{enumerate}
Aus Lemma 5.4 der Vorlesung folgt (1).
Weiter folgt aus Lemma 5.6 der Vorlesung (\emph{``Für alle Matrizen
  $A \in \mathcal{M}_{m \times n}\qty\big(K)$ gilt $AI_n = A = I_mA$''}),
dass $A \cdot I_2 = A = I_2 \cdot A$.
Da $I_2 = \begin{pmatrix} 1_k & 0_k \\ 0_k & 1_K \end{pmatrix} \in H$ folgt (2).

Sei nun $a \in K$ beliebig.
Da $K$ ein Körper ist, existiert auch ein Element $-a \in K$ und dann ist
\[
  \begin{pmatrix}
    1_k & a \\
    0_k & 1_k
  \end{pmatrix}
  \cdot
  \begin{pmatrix}
    1_k & -a \\
    0_k & 1_k
  \end{pmatrix}
  =
  \begin{pmatrix}
    1_k \cdot 1_k + a \cdot 0_k & 1_k \cdot -a + a \cdot 1_k \\
    0_k \cdot 1_k + 1_K \cdot 0_k & 0_k \cdot -a + 1_K \cdot 1_k\\
  \end{pmatrix}
  =
  \begin{pmatrix}
    1_k & 0_k \\
    0_k & 1_K \\
  \end{pmatrix}
  =
  \begin{pmatrix}
    1_k & -a \\
    0_k & 1_k
  \end{pmatrix}
  \cdot
  \begin{pmatrix}
    1_k & a \\
    0_k & 1_k
  \end{pmatrix}
\]

$\Rightarrow$ zu jeder Matrix $A \in H$ existiert eine inverse Matrix $A^{-1}$
mit $A \cdot A^{-1} = A^{-1} \cdot A = I_2$

$\Rightarrow$ Jede Matrix $A \in H$ ist invertierbar $\Rightarrow$ (3).

$\Rightarrow$ \underline{$\qty\big(H, \cdot)$ ist eine Gruppe.}

\newpage
Eine Gruppe heißt abelsche Gruppe, wenn für $A, B \in H$ zusätzlich gilt
$A \cdot B = B \cdot A$.
Seien nun $a, b \in K$ beliebig.
Dann ist
\begin{flalign*}
  \begin{pmatrix}
    1_k & a \\
    0_k & 1_k
  \end{pmatrix}
  \cdot
  \begin{pmatrix}
    1_k & b \\
    0_k & 1_k
  \end{pmatrix}
  &=
  \begin{pmatrix}
    1_k \cdot 1_k + a \cdot 0_k & 1_k \cdot b + a \cdot 1_k \\
    0_k \cdot 1_k + 1_K \cdot 0_k & 0_k \cdot b + 1_K \cdot 1_k\\
  \end{pmatrix} \\
  &=
  \begin{pmatrix}
    1_k & b + a  \\
    0_k & 1_k \\
  \end{pmatrix} \\
  \overset{\text{Kommutativität in $K$}}&=
  \begin{pmatrix}
    1_k & a + b \\
    0_k & 1_k \\
  \end{pmatrix} \\
    &=
  \begin{pmatrix}
    1_k \cdot 1_k + b \cdot 0_k & 1_k \cdot a + b \cdot 1_k \\
    0_k \cdot 1_k + 1_K \cdot 0_k & 0_k \cdot a + 1_K \cdot 1_k\\
  \end{pmatrix} \\
  &=
    \begin{pmatrix}
    1_k & b \\
    0_k & 1_k
  \end{pmatrix}
  \cdot
  \begin{pmatrix}
    1_k & a \\
    0_k & 1_k
  \end{pmatrix}
\end{flalign*}

$\Rightarrow$ \underline{$\qty\big{H, \cdot}$ ist eine abelsche Gruppe.}

\paragraph{Aufgabe 8}
\begin{enumerate}[(1)]
\item Zeigen Sie, dass die Abbildung
  \[
    \varphi \colon \mathbb{R}^2 \to \mathbb{R}^2,
    \begin{pmatrix}
      x \\
      y
    \end{pmatrix}
    \mapsto
    \begin{pmatrix}
      x + y \\
      y
    \end{pmatrix}
  \]
  ein Automorphismus ist.

  \subparagraph{Lsg.} Eine Abbildung $\varphi$ zwischen zwei Vektorräumen
  $U, W$ über einem Körper $K$ heißt Automorphismus, falls $U = W$ und
  $\varphi$ bijektiv ist.
  Der erste Teil ist offensichtlich, bleibt also noch zu zeigen, dass
  $\varphi$ bijektiv ist.

  Angenommen $\varphi$ wäre nicht injektiv, dann existieren
  $\begin{pmatrix} x_1 \\ y_1 \end{pmatrix} \ne
  \begin{pmatrix} x_2 \\ y_2 \end{pmatrix} \in \mathbb{R}^2$
  mit $\begin{pmatrix} x_1 + y_1 \\ y_1 \end{pmatrix} =
  \begin{pmatrix} x_2 + y_2 \\ y_2 \end{pmatrix}$.
  Offensichtlich gilt somit $y_1 = y_2$.
  Durch Substitution von $y_2$ und äquivalente Umformung wird aus
  $x_1 + y_1 = x_2 + y_2$ der Term $x_1 = x_2$, ein Widerspruch zu
  $\begin{pmatrix} x_1 \\ y_1 \end{pmatrix} \ne
  \begin{pmatrix} x_2 \\ y_2 \end{pmatrix}$.

  $\Rightarrow$ \underline{$\varphi$ ist injektiv.}

  Sei nun $\begin{pmatrix} x \\ y \end{pmatrix} \in \mathbb{R}^2$ beliebig
  gewählt.
  Dann ist $\begin{pmatrix} x - y \\ y \end{pmatrix}$ ebenfalls ein Element
  aus $\mathbb{R}^2$ und es gilt
  \[
    \varphi\qty(\begin{pmatrix} x - y \\ y \end{pmatrix}) =
    \begin{pmatrix} x \\ y \end{pmatrix}
  \]
  $\Rightarrow$ \underline{$\varphi$ ist surjektiv.}

  $\Rightarrow$ \underline{$\varphi$ ist bijektiv und ein automorphismus.}

\newpage
\item Ist die Abbildung
  \[
    \psi \colon \mathbb{R}^2 \to \mathbb{R}^2,
    \begin{pmatrix}
      x \\
      y
    \end{pmatrix}
    \mapsto
    \begin{pmatrix}
      xy \\
      y
    \end{pmatrix}
  \]
  ebenfalls linear?

  \subparagraph{Lsg.} Eine Abbildung zwischen zwei Vektorräumen
  $U$ und $W$ über einem Körper $K$ heißt linear, falls
  \begin{enumerate}[(i)]
  \item für alle  $u, v \in U$ gilt
    $\psi\qty\big(u) + \psi\qty\big(v) = \psi\qty\big(u + v)$

  \item für alle $\lambda \in K$ und alle $u \in U$ gilt
    $\psi\qty\big(\lambda u) = \lambda \psi\qty\big(u)$.
  \end{enumerate}

  Sei nun $u = \begin{pmatrix} 2 \\ 2 \end{pmatrix} \in \mathbb{R}^2$.
  Dann ist $\psi\qty\big(2 \cdot u) = \begin{pmatrix} 16 \\ 4 \end{pmatrix}$
  und $2 \cdot \psi\qty\big(u) = \begin{pmatrix} 8 \\ 4 \end{pmatrix}$,
  ein Widerspruch zu (ii).

  $\Rightarrow$ \underline{$\psi$ ist nicht linear.}
\end{enumerate}

\end{document}
