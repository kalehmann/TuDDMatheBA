\documentclass{scrreprt}

\usepackage{aligned-overset}
\usepackage{amsmath}
\usepackage{amssymb}
\usepackage{bm}
\usepackage[shortlabels]{enumitem}
\usepackage{hyperref}
\usepackage[utf8]{inputenc}
\usepackage{multicol}
\usepackage{mathtools}
\usepackage{physics}
\usepackage{tabularx}
\usepackage[table]{xcolor}
\usepackage{titling}
\usepackage{fancyhdr}
\usepackage{xfrac}
\usepackage{pgfplots}

\pgfplotsset{compat = newest}
\usetikzlibrary{intersections}
\usetikzlibrary{patterns}
\usepgfplotslibrary{fillbetween}

\author{Karsten Lehmann}
\date{WiSe 2021/2022}
\title{Übungsblatt 07\\Lineare Algebra - Grundlegende Konzepte}

\setlength{\headheight}{26pt}
\pagestyle{fancy}
\fancyhf{}
\lhead{\thetitle}
\rhead{\theauthor}
\lfoot{\thedate}
\rfoot{Seite \thepage}

\newcommand\ccg[1]{\cellcolor{green}{#1}}

\begin{document}
\paragraph{Aufgabe 2}
\begin{enumerate}[(a)]
\item Prüfen Sie, ob die Vektoren
  \[
    \begin{pmatrix}1 \\ 1 \\ 0\end{pmatrix},
    \begin{pmatrix}2 \\ 0 \\ 2\end{pmatrix},
    \begin{pmatrix}0 \\ 2 \\ 3\end{pmatrix}
  \]
  linear unabhängig im $\mathbb{R}$-Vektorraum
  $\mathbb{R}^3$ sind.

  \subparagraph{Lsg.} Vektoren $v_1, \ldots, v_n \in V$ heißen linear abhängig,
  falls $k_1, \ldots, k_n \in K$ existieren, sodass
  $k_1 \cdot v_1 + \ldots + k_n \cdot v_n = 0_V$ mit $k_j \ne 0$ für mindestens
  ein $j \in \qty[1, n]$.
  Die Vektoren $v_1, \ldots, v_n$ heißen \emph{linear unabhängig}, falls sie
  nicht linear abhängig sind.

  Angenommen $x, y, z \in R$ seien so gewählt, dass
  \[
    x \cdot \begin{pmatrix}1 \\ 1 \\ 0\end{pmatrix} +
    y \cdot \begin{pmatrix}2 \\ 0 \\ 2\end{pmatrix} +
    z \cdot \begin{pmatrix}0 \\ 2 \\ 3\end{pmatrix} =
    \begin{pmatrix}0 \\ 0 \\ 0\end{pmatrix}
  \]
  Dann ist die erweiterte Koeffizientenmatrix
  \begin{flalign*}
    \qty(
      \begin{array}{ccc|c}
        1 & 2 & 0 & 0 \\
        1 & 0 & 2 & 0 \\
        0 & 2 & 3 & 0
      \end{array}
    ) &= \qty(
      \begin{array}{ccc|c}
        1 & 2 & 0 & 0 \\
        0 & -2 & 2 & 0 \\
        0 & 2 & 3 & 0
      \end{array}
    ) \\
    &= \qty(
      \begin{array}{ccc|c}
        1 & 2 & 0 & 0 \\
        0 & -2 & 2 & 0 \\
        0 & 0 & 5 & 0
      \end{array}
    ) \\
    &= \qty(
      \begin{array}{ccc|c}
        1 & 2 & 0  & 0 \\
        0 & 1 & -1 & 0 \\
        0 & 0 & 1  & 0
      \end{array}
    ) \\
    &= \qty(
      \begin{array}{ccc|c}
        1 & 2 & 0 & 0 \\
        0 & 1 & 0 & 0 \\
        0 & 0 & 1 & 0
      \end{array}
    ) \\
    &= \qty(
      \begin{array}{ccc|c}
        1 & 0 & 0 & 0 \\
        0 & 1 & 0 & 0 \\
        0 & 0 & 1 & 0
      \end{array}
    )
  \end{flalign*}

  $\Rightarrow$ die erweiterte Koeffizientenmatrix hat eine eindeutige Lösung
  mit $x = y = z = 0$

  $\Rightarrow$ \underline{die Vektoren sind linear unabhängig}
\end{enumerate}
\end{document}