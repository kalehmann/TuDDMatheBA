\documentclass{scrreprt}

\usepackage{aligned-overset}
\usepackage{amsmath}
\usepackage{amssymb}
\usepackage{bm}
\usepackage[shortlabels]{enumitem}
\usepackage{hyperref}
\usepackage[utf8]{inputenc}
\usepackage{multicol}
\usepackage{mathtools}
\usepackage{physics}
\usepackage{tabularx}
\usepackage[table]{xcolor}
\usepackage{titling}
\usepackage{fancyhdr}
\usepackage{xfrac}
\usepackage{pgfplots}
\usepackage{tikz-3dplot}

\pgfplotsset{compat = newest}
\usetikzlibrary{intersections}
\usetikzlibrary{patterns}
\usepgfplotslibrary{fillbetween}

\author{Karsten Lehmann}
\date{WiSe 2021/2022}
\title{Übungsblatt 08\\Lineare Algebra - Grundlegende Konzepte}

\setlength{\headheight}{26pt}
\pagestyle{fancy}
\fancyhf{}
\lhead{\thetitle}
\rhead{\theauthor}
\lfoot{\thedate}
\rfoot{Seite \thepage}

\newcommand\hull[1]{\left\langle #1 \right\rangle}

\begin{document}
\paragraph{Aufgabe 2}
\begin{enumerate}[(a)]
\item Prüfen Sie, ob die Vektoren
  \[
    \begin{pmatrix}1 \\ 1 \\ 0\end{pmatrix},
    \begin{pmatrix}2 \\ 0 \\ 2\end{pmatrix},
    \begin{pmatrix}0 \\ 2 \\ 3\end{pmatrix}
  \]
  linear unabhängig im $\mathbb{R}$-Vektorraum
  $\mathbb{R}^3$ sind.

  \subparagraph{Lsg.} Vektoren $v_1, \ldots, v_n \in V$ heißen linear abhängig,
  falls $k_1, \ldots, k_n \in K$ existieren, sodass
  $k_1 \cdot v_1 + \ldots + k_n \cdot v_n = 0_V$ mit $k_j \ne 0$ für mindestens
  ein $j \in \qty[1, n]$.
  Die Vektoren $v_1, \ldots, v_n$ heißen \emph{linear unabhängig}, falls sie
  nicht linear abhängig sind.

  Angenommen $x, y, z \in R$ seien so gewählt, dass
  \[
    x \cdot \begin{pmatrix}1 \\ 1 \\ 0\end{pmatrix} +
    y \cdot \begin{pmatrix}2 \\ 0 \\ 2\end{pmatrix} +
    z \cdot \begin{pmatrix}0 \\ 2 \\ 3\end{pmatrix} =
    \begin{pmatrix}0 \\ 0 \\ 0\end{pmatrix}
  \]
  Dann ist die erweiterte Koeffizientenmatrix
  \begin{flalign*}
    \qty(
      \begin{array}{ccc|c}
        1 & 2 & 0 & 0 \\
        1 & 0 & 2 & 0 \\
        0 & 2 & 3 & 0
      \end{array}
    ) &= \qty(
      \begin{array}{ccc|c}
        1 & 2 & 0 & 0 \\
        0 & -2 & 2 & 0 \\
        0 & 2 & 3 & 0
      \end{array}
    ) \\
    &= \qty(
      \begin{array}{ccc|c}
        1 & 2 & 0 & 0 \\
        0 & -2 & 2 & 0 \\
        0 & 0 & 5 & 0
      \end{array}
    ) \\
    &= \qty(
      \begin{array}{ccc|c}
        1 & 2 & 0  & 0 \\
        0 & 1 & -1 & 0 \\
        0 & 0 & 1  & 0
      \end{array}
    ) \\
    &= \qty(
      \begin{array}{ccc|c}
        1 & 2 & 0 & 0 \\
        0 & 1 & 0 & 0 \\
        0 & 0 & 1 & 0
      \end{array}
    ) \\
    &= \qty(
      \begin{array}{ccc|c}
        1 & 0 & 0 & 0 \\
        0 & 1 & 0 & 0 \\
        0 & 0 & 1 & 0
      \end{array}
    )
  \end{flalign*}

  $\Rightarrow$ die erweiterte Koeffizientenmatrix hat eine eindeutige Lösung
  mit $x = y = z = 0$

  $\Rightarrow$ \underline{die Vektoren sind linear unabhängig}
\newpage
\item Bestimmen Sie, für welche $b, c \in \mathbb{R}$ die Vektoren
  \[
    v_1 =\begin{pmatrix}1 \\ b\end{pmatrix},
    v_2 = \begin{pmatrix}c \\ 1\end{pmatrix}
  \]
  linear unabhängig im $\mathbb{R}$-Vektorraum $\mathbb{R}^2$ sind.

  \subparagraph{Lsg.} Die Vektoren sind linear unabhängig, falls gilt
  \[
    \lambda_1 \cdot v_! + \lambda_2 \cdot v_2 = 0_{\mathbb{R}^2}
    \iff \lambda_1 = \lambda_2 = 0
  \]
  \begin{flalign*}
    \lambda_1 \cdot 1 + \lambda_2 \cdot c &= 0 \\
    \lambda_1 \cdot b + \lambda_2 \cdot 1 &= 0
    && {\Big |}  - \lambda_1 \cdot b && \\
    \lambda_2 &= - \lambda_1 \cdot b \\
    \lambda_1 + \qty\big(-\lambda_1 \cdot b) \cdot c &= 0 \\
    \lambda_1 - \lambda_1 \cdot b \cdot c &= 0 \\
    \lambda_1 \cdot \qty\big(1 - b \cdot c) &= 0
  \end{flalign*}
  Falls $\lambda_1 \ne 0$, dann muss $b \cdot c = 1$ sein um die Gleichung zu
  erfüllen.

  $\Rightarrow$ \underline{$v_1, v_2$ sind linear unabhängig, falls
    $b \ne \frac{1}{c}$}

\item Finden sie einen Vektor $v \in \mathbb{R}^3$, so dass
  \[
    \hull{v} = \qty{
      \begin{pmatrix}
        x \\
        y \\
        z
      \end{pmatrix}
      \, \middle | \,
      x - 2y = 0, z - x = 0
    }
  \]
  Folgern Sie, dass die obige Menge ein Unterraum von $\mathbb{R}^3$ ist und
  finden Sie eine Basis dieses Unterraums.
  Geben Sie seine Dimension an.

  \subparagraph{Lsg.} Sei $v = \begin{pmatrix} x_v \\ y_v \\ z_v\end{pmatrix}
  = \begin{pmatrix} 2 \\ 1 \\ 2\end{pmatrix}$, dann  ist $x_v - 2 \cdot y_v = 0$
  und $z_v - x_v = 0$.
  Offensichtlich ist auch $\lambda \cdot x_v - 2 \cdot \lambda \cdot y_v = 0$
  und $\lambda \cdot z_v - \lambda \cdot x_v = 0$ für alle
  $\lambda \in \mathbb{R}$.

  \begin{flalign*}
    \Rightarrow &\hull{v} = \qty{
      \begin{pmatrix}
        x \\
        y \\
        z
      \end{pmatrix}
      \, \middle | \,
      x - 2y = 0, z - x = 0
    } &
  \end{flalign*}

  Es ist $0 - 2 \cdot 0 = 0$ und $0 - 0 = 0$, folglich ist
  $0_{\mathbb{R}^3} \in \hull{v}$.

  Seien nun $u, w \in \hull{v}$ beliebig.
  Dann gilt $u_x - 2 \cdot u_y = 0$ und $w_x - 2 \cdot w_y = 0$,
  folglich gilt auch
  $\qty\big(u_x + w_x)- 2 \cdot \qty\big(u_y + w_y) = 0$.
  Ebenso gilt $u_z - u_x = 0$ und $w_z - w_x = 0$, folglich auch
  $\qty\big(u_z + w_z) - \qty\big(u_x + w_x) = 0$.
  Somit ist auch $u + w \in \hull{v}$.

  Seien nun $w \in \hull{v}, \lambda \in \mathbb{R}$ beliebig, dann ist
  $x_w - 2 \cdot y_w = 0$, folglich ist auch
  $\lambda \cdot x_w - 2 \cdot \lambda \cdot y_w = 0$.
  Ebenso ist $z_w - x_w = 0$, folglich ist auch
  $\lambda \cdot z_w - \lambda \cdot x_w = 0$.

  $\Rightarrow$ \underline{$\hull{v}$ ist ein Unterraum}

  Weiter ist
  $v = \begin{pmatrix} 2 \\ 1 \\ 2 \end{pmatrix} \ne 0_{\mathbb{R}^3}$,
  somit ist $\qty\big{v}$ linear unabhängig.

  $\Rightarrow$ \underline{$\qty\big{v}$ ist eine Basis von $\hull{v}$ und
    $\dim \hull{v} = 1$}
\end{enumerate}

\paragraph{Aufgabe 3}
Wir erinnern uns daran, dass $\mathbb{Z}/p$ für jede Primzahl $p$ ein Körper ist.
In dieser Aufgabe betrachten wir Vektorräume über $\mathbb{Z}/p$ für gegebene
Primzahlen $p$.
Im Folgenden schreiben wir $0, 1, \ldots, p - 1$ für die korrespondierenden
Elemente von $\mathbb{Z}/p$, d.h. für $0 \leq k \leq p - 1$ schreiben wir $k$
anstelle von $\qty[k]$.
\begin{enumerate}[(a)]
\item Gegeben seien die folgenden Vektoren in $\qty\big(\mathbb{Z}/5)^3$:
  \[
    a \coloneqq \begin{pmatrix}
      1 \\
      3 \\
      2
    \end{pmatrix},
    b \coloneqq \begin{pmatrix}
      0 \\
      1 \\
      3
    \end{pmatrix},
    c \coloneqq \begin{pmatrix}
      1 \\
      1 \\
      1
    \end{pmatrix}
  \]
  Sind $a, b, c$ linear unabhängig?
  Ist der Vektor $\begin{pmatrix} 1 \\ 0 \\ 2\end{pmatrix}$ in $\hull{a, b, c}$
  enthalten?

  \subparagraph{Lsg.} mittels Gleichnugssystem:
  \begin{flalign*}
    \qty(\begin{array}{ccc|c}
      1 & 0 & 1 & 0 \\
      3 & 1 & 1 & 0 \\
      2 & 3 & 1 & 0
    \end{array}) \overset{\substack{
        R_2 + 2 \cdot R_1 \\
        R_3 + 3 \cdot R_1
    }}&\longrightarrow \qty(\begin{array}{ccc|c}
      1 & 0 & 1 & 0 \\
      0 & 1 & 3 & 0 \\
      0 & 3 & 4 & 0
    \end{array}) \\
    \overset{\substack{
        R_3 + 2 \cdot R_2 \\
    }}&\longrightarrow \qty(\begin{array}{ccc|c}
      1 & 0 & 1 & 0 \\
      0 & 1 & 3 & 0 \\
      0 & 0 & 0 & 0
    \end{array})
  \end{flalign*}
  $\Rightarrow$ es gibt mehr als eine Lösung für
  $\lambda_1 \cdot a + \lambda_2 \cdot b + \lambda_3 \cdot c = 0$.

  $\Rightarrow$ \underline{die Vektoren sind linear abhängig.}

  So lässt sich $c$ als Linearkombination aus $a$ und $b$ darstellen mit
  \[
    \begin{pmatrix} 1 \\ 3 \\ 2 \end{pmatrix} +
    3 \cdot \begin{pmatrix} 0 \\ 1 \\ 3 \end{pmatrix}
    = \begin{pmatrix} 1 \\ 1 \\ 1 \end{pmatrix}
  \]

  \newpage
  Ob der Vektor $\begin{pmatrix} 1 \\ 0 \\ 2\end{pmatrix}$ in $\hull{a, b, c}$
  enthalten ist lässt sich mittels eines Gleichungssystems prüfen:

  \begin{flalign*}
    \qty(\begin{array}{ccc|c}
      1 & 0 & 1 & 1 \\
      3 & 1 & 1 & 0 \\
      2 & 3 & 1 & 2
    \end{array}) \overset{\substack{
        R_2 + 2 \cdot R_1 \\
        R_3 + 3 \cdot R_1
    }}&\longrightarrow \qty(\begin{array}{ccc|c}
      1 & 0 & 1 & 0 \\
      0 & 1 & 3 & 2 \\
      0 & 3 & 4 & 0
    \end{array}) \\
    \overset{\substack{
        R_3 + 2 \cdot R_2 \\
    }}&\longrightarrow \qty(\begin{array}{ccc|c}
      1 & 0 & 1 & 0 \\
      0 & 1 & 3 & 2 \\
      0 & 0 & 0 & 4
    \end{array})
  \end{flalign*}
  Die dritte Reihe ist nun mit
  $\lambda_1 \cdot 0 + \lambda _2 \cdot 0 + \lambda_3 \cdot 0 + 4$
  nicht lösbar.

  $\Rightarrow$ \underline{Der Vektor}
  $\begin{pmatrix} 1 \\ 0 \\ 2\end{pmatrix}$
  \underline{ist nicht in $\hull{a, b, c}$ enthalten}

\item Geben Sie alle Elemente des Vektorraums $\qty\big(\mathbb{Z}/2)^3$ an.
  Visualisieren Sie den Vektorraum durch einen Würfel.

  \subparagraph{Lsg.} Die Elemente sind
  \[
    \begin{pmatrix} 0 \\ 0 \\ 0 \end{pmatrix},
    \begin{pmatrix} 0 \\ 0 \\ 1 \end{pmatrix},
    \begin{pmatrix} 0 \\ 1 \\ 0 \end{pmatrix},
    \begin{pmatrix} 0 \\ 1 \\ 1 \end{pmatrix},
    \begin{pmatrix} 1 \\ 0 \\ 0 \end{pmatrix},
    \begin{pmatrix} 1 \\ 0 \\ 1 \end{pmatrix},
    \begin{pmatrix} 1 \\ 1 \\ 0 \end{pmatrix},
    \begin{pmatrix} 1 \\ 1 \\ 1 \end{pmatrix}
  \]

  \tdplotsetmaincoords{70}{110}
  \begin{center}
    \begin{tikzpicture}[scale=3,tdplot_main_coords]
      \node[
        circle, fill, inner sep=2pt,
        label=above right:$\begin{pmatrix} 0 \\ 0 \\ 0 \end{pmatrix}$
      ] (a) at (0,0,0) {};
      \node[
        circle, fill, inner sep=2pt,
        label=above:$\begin{pmatrix} 0 \\ 0 \\ 1 \end{pmatrix}$
      ] (b) at (0,0,1) {};
      \node[
        circle, fill, inner sep=2pt,
        label=below right:$\begin{pmatrix} 0 \\ 1 \\ 0 \end{pmatrix}$
      ] (c) at (0,1,0) {};
      \node[
        circle, fill, inner sep=2pt,
        label=above right:$\begin{pmatrix} 0 \\ 1 \\ 1 \end{pmatrix}$
      ] (d) at (0,1,1) {};
      \node[
        circle, fill, inner sep=2pt,
        label=below left:$\begin{pmatrix} 1 \\ 0 \\ 0 \end{pmatrix}$
      ] (e) at (1,0,0) {};
      \node[
        circle, fill, inner sep=2pt,
        label=above left:$\begin{pmatrix} 1 \\ 0 \\ 1 \end{pmatrix}$
      ] (f) at (1,0,1) {};
      \node[
        circle, fill, inner sep=2pt,
        label=below:$\begin{pmatrix} 1 \\ 1 \\ 0 \end{pmatrix}$
      ] (h) at (1,1,0) {};
      \node[
        circle, fill, inner sep=2pt,
        label=below right:$\begin{pmatrix} 1 \\ 1 \\ 1 \end{pmatrix}$
      ] (i) at (1,1,1) {};
      \draw [dashed] (a) -- (b);
      \draw [dashed] (a) -- (c);
      \draw [dashed] (a) -- (e);
      \draw (b) -- (d);
      \draw (b) -- (f);
      \draw (c) -- (d);
      \draw (c) -- (h);
      \draw (d) -- (i);
      \draw (e) -- (f);
      \draw (e) -- (h);
      \draw (f) -- (i);
      \draw (h) -- (i);
    \end{tikzpicture}
  \end{center}
\end{enumerate}
\end{document}