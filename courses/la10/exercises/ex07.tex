\documentclass{scrreprt}

\usepackage{aligned-overset}
\usepackage{amsmath}
\usepackage{amssymb}
\usepackage{bm}
\usepackage[shortlabels]{enumitem}
\usepackage{hyperref}
\usepackage[utf8]{inputenc}
\usepackage{multicol}
\usepackage{mathtools}
\usepackage{physics}
\usepackage{tabularx}
\usepackage[table]{xcolor}
\usepackage{titling}
\usepackage{fancyhdr}
\usepackage{xfrac}
\usepackage{pgfplots}

\pgfplotsset{compat = newest}
\usetikzlibrary{intersections}
\usetikzlibrary{patterns}
\usepgfplotslibrary{fillbetween}

\author{Karsten Lehmann}
\date{WiSe 2021/2022}
\title{Übungsblatt 07\\Lineare Algebra - Grundlegende Konzepte}

\setlength{\headheight}{26pt}
\pagestyle{fancy}
\fancyhf{}
\lhead{\thetitle}
\rhead{\theauthor}
\lfoot{\thedate}
\rfoot{Seite \thepage}

\newcommand\ccg[1]{\cellcolor{green}{#1}}

\begin{document}
\paragraph{Aufgabe 2} In den folgenden beiden Teilaufgaben, also in (a) und (b),
geht es um Unterräume des Vektorraumes $\mathbb{R}^n$ für gegebene natürliche
Zahlen $n$.
Dabei ist $\mathbb{R}^n$ mit der üblichen Addition und der üblichen
Skalarmultiplikation versehen.

\begin{enumerate}[(a)]
\item Sei $V \coloneqq \qty{\binom{x}{y} \in \mathbb{R}^2 \,\middle|\,
    x \geq 0, y \geq 0}$.
  Gilt dann
  \begin{enumerate}[label={(a\arabic*)}]
  \item Wenn $u, v \in V$, dann ist auch $u + v \in V$?
  \item Wenn $u \in V$ und $\lambda \in \mathbb{R}$, dann ist auch
    $\lambda u \in V$?

  \item $V$ ist ein Unterraum von $\mathbb{R}^2$
  \end{enumerate}

  \subparagraph{Lsg.} Seien $x_1, x_2, y_1, y_2 \in \mathbb{R}_{\geq 0}$
  beliebig und $u = \binom{x_1}{y_1}, v = \binom{x_2}{y_2}$.
  \begin{enumerate}[label={(a\arabic*)}]
  \item $u + v = \binom{x_1 + x_2}{y_1 + y_2}$.
    Aus $a, b \geq 0 \Rightarrow a + b \geq 0$ folgt, dass $x_1 + x_2 \geq 0$
    und $y_1 + y_2 \geq 0$.

    $\Rightarrow$ \underline{$u + v$ ist in $V$}

  \item Sei $a \in \mathbb{R} > 0$, dann ist $-1 \cdot a < 0$.
    Angenommen $x_1, y_1 > 0$, dann gilt
    $\binom{-1 \cdot x_1}{-1 \cdot y_1} \notin V$.

    $\Rightarrow$ \underline{die Aussage (a2) gilt für $u \ne \binom{0}{0}$ und
      $\lambda = -1$ nicht}.

  \item Eine Teilmenge $W \subset \mathbb{R}^2$ wird Unterraum von
    $\mathbb{R}^2$ genannt, falls
    \begin{enumerate}[(i)]
    \item $0_{\mathbb{R}^2} \in W$
    \item Für $u, v \in W$ gilt $u + v \in W$
    \item Für $\lambda \in \mathbb{R}$ und alle $u \in W$ gilt
      $\lambda \cdot u \in W$
    \end{enumerate}
    Wie in (a2) gezeigt gilt (iii) nicht.

    $\Rightarrow$ \underline{$V$ ist kein Unterraum von $\mathbb{R}^2$}
  \end{enumerate}
\end{enumerate}

\end{document}