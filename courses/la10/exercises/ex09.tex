\documentclass{scrreprt}

\usepackage{aligned-overset}
\usepackage{amsmath}
\usepackage{amssymb}
\usepackage{bm}
\usepackage[shortlabels]{enumitem}
\usepackage{hyperref}
\usepackage[utf8]{inputenc}
\usepackage{multicol}
\usepackage{mathtools}
\usepackage{physics}
\usepackage{tabularx}
\usepackage[table]{xcolor}
\usepackage{titling}
\usepackage{fancyhdr}
\usepackage{xfrac}
\usepackage{pgfplots}
\usepackage{tikz-3dplot}

\pgfplotsset{compat = newest}
\usetikzlibrary{intersections}
\usetikzlibrary{patterns}
\usepgfplotslibrary{fillbetween}

\author{Karsten Lehmann}
\date{WiSe 2021/2022}
\title{Übungsblatt 09\\Lineare Algebra - Grundlegende Konzepte}

\setlength{\headheight}{26pt}
\pagestyle{fancy}
\fancyhf{}
\lhead{\thetitle}
\rhead{\theauthor}
\lfoot{\thedate}
\rfoot{Seite \thepage}

\newcommand\hull[1]{\left\langle #1 \right\rangle}

\begin{document}
\paragraph{Aufgabe 2} Seien
\[
  W \coloneqq \qty{
    \begin{pmatrix} x \\ y \\ z \end{pmatrix} \in \mathbb{R}^3
    \, \middle | \,
    x + y - 5z = 0
  }
\]
und
\[
  U \coloneqq \hull{e_1, e_2} \text{ mit }
  e_1 = \begin{pmatrix} 1 \\ 0 \\ 0 \end{pmatrix}
  \text{ und }
  e_2 = \begin{pmatrix} 0 \\ 1 \\ 0 \end{pmatrix}
\]
\begin{enumerate}[(a)]
\item Finden Sie Vektoren $w_1, w_2 \in \mathbb{R}^3$ mit $W = \hull{w_1, w_2}$.
  Folgern Sie, dass $W$ ein Unterraum von $\mathbb{R}^3$ ist.
  Bestimmen Sie eine Basis von $W$ und geben Sie $\dim_{\mathbb{R}}\qty\big(W)$ an.

  \subparagraph{Lsg.} Seien $W_1 = \begin{pmatrix} 5 \\ 0 \\ 1 \end{pmatrix}$
  und $w_2 = \begin{pmatrix} 0 \\ 5 \\ 1 \end{pmatrix}$.
  Aus Lemma 3.24 der Vorlesung (\emph{``Die Menge $\hull{X}$ ist der kleinste
    Untervektorraum von $V$, der $X$ enthält, dass heißt $X$ ist ein Unter-
    vektorraum.''}) folgt, dass $\hull{w_1, w_2}$ ein Unterraum ist.
  Aus $w_1$ und $w_2$ sind linear unabhängig folgt, dass $\qty\big{w_1, w_2}$
  eine Basis von $\hull{w_1, w_2} = W$ ist.

  $\Rightarrow$ $\dim_{\mathbb{R}}\qty\big(W) = \abs{\qty\big{w_1, w_2}} = 2$

\item Geben Sie $\dim_{\mathbb{R}}\qty\big(U)$ an.

  \subparagraph{Lsg.} Offensichtlich ist
  \[
    k_1 \cdot \begin{pmatrix} 1 \\ 0 \\ 0 \end{pmatrix} +
    k_2 \cdot \begin{pmatrix} 0 \\ 1 \\ 0 \end{pmatrix} = 0
    \iff k_1  = k_2 = 0
  \]
  $\Rightarrow$ $e_1$ und $e_2$ sind linear unabhängig.

  Da $U$ von $e_1$ und $e_2$ aufgespannt wird, ist $\qty\big{e_1, e_2}$ eine
  Basis von $U$.

  $\Rightarrow \dim_{\mathbb{R}}\qty\big(U) = \abs{\qty\big{e_1, e_2}} = 2$

\item Beschreiben Sie $W \cap U$.
  Bestimmen Sie eine Basis von $W \cap U$ und geben Sie
  $\dim_{\mathbb{R}}\qty\big(W \cap U)$ an.

  \subparagraph{Lsg.} $W \cap U = \qty{\begin{pmatrix} x \\ y \\ 0 \end{pmatrix}
    \in \mathbb{R}^3 \, \middle | \, x + y = 0}$.
  Es ist $\qty{\begin{pmatrix} 1 \\ -1 \\ 0 \end{pmatrix}}$ eine Basis von
  $W \cap U$.

  $\Rightarrow$ \underline{$\dim_{\mathbb{R}}\qty\big(W \cap U) = 2$}

\newpage
\item Beweisen Sie mithilfe der Dimensionsformel (Lemma 3.60), dass
  $W + U = \mathbb{R}^3$ gilt.

  \subparagraph{Lsg.} Nach Lemma 3.60 der Vorlesung (\emph{``Sei $V$
    endlich-dimensional und seien $W_1$ und $W_2$ beides Untervektorräume von
    $V$.
    Dann gilt $\dim\qty\big(W_1 + W_2) = \dim\qty\big(W_1) + \dim\qty\big(W_2)
    - \dim\qty\big(W_1 \cap W_2)$''}) folgt $\dim\;ty\big(U + W) = 3$.
  Aus Korollar 3.59 der Vorlesung (\emph{``Sei $V$ endlich-dimensional.
    Sei $W$ ein Untervektorraum von $V$ mit $\dim\qty\big(W) = \dim\qty\big(V)$.
    Dann gilt $W = V$''}) folgt $U + W = \mathbb{R}^3$.
\end{enumerate}

\paragraph{Aufgabe 3} Sei $V$ ein endlich-dimensionaler Vektorraum über einem
Körper $K$ und sei $n \coloneqq \dim_K\qty\big(V)$.
\begin{enumerate}[(a)]
\item Beweisen Sie, dass für jedes $k \in \mathbb{N}$ mit $k \leq n$ ein
  Unterraum $U_K$ von $V$ mit $\dim_k\qty\big(U_K) = k$ existiert.

  \subparagraph{Lsg.} Nach Korollar 3.46 der Vorlesung (\emph{``Falls $V$ ein
    endlich-dimensionaler Vektorraum ist, so existiert eine Basis von $V$ mit
    nur endlich vielen Elementen''}) besitzt $V$ mindestens eine Basis.

  Sei nun $X$ eine Basis von $V$.
  Dann ist $X$ als Basis linear unabhängig und
  $\abs\big{X} = \dim_K\qty\big(V) = n$.
  Für jedes $k \in \mathbb{N}$ mit $k \leq n$ existiert
  eine Teilmenge $Y \subseteq X$ mit $\abs{Y} = k$.

  $Y$ ist als Teilmenge der linear unabhängigen Menge $X$ ebenfalls linear
  unabhängig.
  Weiterhin ist $Y$ die Basis von $U_k = \hull{Y}$ und $U_k$ ist nach Lemma 3.24
  der Vorlesung (\emph{``Die Menge $\hull{X}$ ist der kleinste Unterraum von
    $V$, der $X$ enthält.''}) ein Unterraum von $V$.
  Da $Y$ genau $k$-Elemente hat, ist $U_k$ $k$-dimensional.

\item Geben Sie alle Unterräume von $\qty\big(\mathbb{F}_2)^3$, wobei
  $\mathbb{F}_2 = \mathbb{Z}/2$ sei, nach der Dimension geordnet an.

  \subparagraph{Lsg.} Der Raum $\qty\big(\mathbb{F}_2)^3$ hat folgende
  Unterräume:
  \begin{itemize}
  \item[$3$-dimensional:] $\qty\big(\mathbb{F}_2)^3$
  \item[$2$-dimensional:]
    \[
      \left\langle
        \begin{pmatrix} 0 \\ 0 \\ 1 \end{pmatrix},
        \begin{pmatrix} 0 \\ 1 \\ 0 \end{pmatrix}
      \right\rangle,
      \left\langle
        \begin{pmatrix} 0 \\ 0 \\ 1 \end{pmatrix},
        \begin{pmatrix} 1 \\ 0 \\ 0 \end{pmatrix}
      \right\rangle,
      \left\langle
        \begin{pmatrix} 0 \\ 1 \\ 0 \end{pmatrix},
        \begin{pmatrix} 1 \\ 0 \\ 0 \end{pmatrix}
      \right\rangle,
    \]
    \[
      \left\langle
        \begin{pmatrix} 0 \\ 0 \\ 1 \end{pmatrix},
        \begin{pmatrix} 1 \\ 1 \\ 0 \end{pmatrix}
      \right\rangle,
      \left\langle
        \begin{pmatrix} 0 \\ 1 \\ 0 \end{pmatrix},
        \begin{pmatrix} 1 \\ 0 \\ 1 \end{pmatrix}
      \right\rangle,
      \left\langle
        \begin{pmatrix} 1 \\ 0 \\ 0 \end{pmatrix},
        \begin{pmatrix} 0 \\ 1 \\ 1 \end{pmatrix}
      \right\rangle,
      \left\langle
        \begin{pmatrix} 1 \\ 1 \\ 0 \end{pmatrix},
        \begin{pmatrix} 0 \\ 1 \\ 1 \end{pmatrix}
      \right\rangle
    \]

  \newpage
  \item[$1$-dimensional:]
    \[
      \qty{
        \begin{pmatrix} 0 \\ 0 \\ 0 \end{pmatrix},
        \begin{pmatrix} 0 \\ 0 \\ 1 \end{pmatrix}
      },
      \qty{
        \begin{pmatrix} 0 \\ 0 \\ 0 \end{pmatrix},
        \begin{pmatrix} 0 \\ 1 \\ 0 \end{pmatrix}
      },
      \qty{
        \begin{pmatrix} 0 \\ 0 \\ 0 \end{pmatrix},
        \begin{pmatrix} 0 \\ 1 \\ 1 \end{pmatrix}
      },
      \qty{
        \begin{pmatrix} 0 \\ 0 \\ 0 \end{pmatrix},
        \begin{pmatrix} 1 \\ 0 \\ 0 \end{pmatrix}
      },
    \]
    \[
      \qty{
        \begin{pmatrix} 0 \\ 0 \\ 0 \end{pmatrix},
        \begin{pmatrix} 1 \\ 0 \\ 1 \end{pmatrix}
      },
      \qty{
        \begin{pmatrix} 0 \\ 0 \\ 0 \end{pmatrix},
        \begin{pmatrix} 1 \\ 1 \\ 0 \end{pmatrix}
      },
      \qty{
        \begin{pmatrix} 0 \\ 0 \\ 0 \end{pmatrix},
        \begin{pmatrix} 1 \\ 1 \\ 1 \end{pmatrix}
      }
    \]

  \item[$0$-dimensional:]
    $\hull{\emptyset} = \qty{0_{\qty\big(\mathbb{F}_2)^3}}
    = \qty{\begin{pmatrix} 0 \\ 0 \\ 0 \end{pmatrix}}$
  \end{itemize}
\end{enumerate}
\end{document}