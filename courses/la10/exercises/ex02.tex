\documentclass{scrreprt}

\usepackage{aligned-overset}
\usepackage{amsmath}
\usepackage{amssymb}
\usepackage{bm}
\usepackage[shortlabels]{enumitem}
\usepackage{hyperref}
\usepackage[utf8]{inputenc}
\usepackage{multicol}
\usepackage{mathtools}
\usepackage{physics}
\usepackage{tabularx}
\usepackage{titling}
\usepackage{fancyhdr}
\usepackage{xfrac}
\usepackage{pgfplots}

\pgfplotsset{compat = newest}
\usetikzlibrary{intersections}
\usetikzlibrary{patterns}
\usepgfplotslibrary{fillbetween}

\author{Karsten Lehmann}
\date{WiSe 2021/2022}
\title{Übungsblatt 02\\Lineare Algebra - Grundlegende Konzepte}

\setlength{\headheight}{26pt}
\pagestyle{fancy}
\fancyhf{}
\lhead{\thetitle}
\rhead{\theauthor}
\lfoot{\thedate}
\rfoot{Seite \thepage}

\begin{document}
\paragraph{Aufgabe 1}
\begin{enumerate}[(a)]
\item Sei $M$ die Menge der Einwohner eines Landes.
  Wir definieren Relationen $\sim_1$, $\sim_2$ und $\sim_3$ auf $M$ durch
  \begin{enumerate}[1)]
  \item $x \sim_1 y \iff x$ und $y$ haben ihren ersten Wohnsitz in der selben
    Stadt.

  \item $x \sim_2 y \iff x$ ist die Mutter von $y$.
  \item $x \sim_3 y \iff x$ und $y$ haben einen Altersunterschied von maximal
    $5$ Jahren.
  \end{enumerate}
  Entscheiden Sie für jede der Relationen, welche der Eigenschaften ``reflexiv'',
  ``symmetrisch'' und ``transitiv'' sie hat.

  \subparagraph{Lsg.}
  \begin{enumerate}[1)]
  \item Relation $\sim_1$ ist reflexiv, symmetrisch und transitiv.
  \item Relation $\sim_2$ ist nicht reflexiv, nicht symmetrisch und nicht
    transitiv.

  \item Relation $\sim_3$ ist reflexiv, symmetrisch, allerdings nicht transitiv.
  \end{enumerate}

\item Sei $\sim$ eine Relation auf einer Menge $M$ und $x \in M$, so schreibe
  $A(x)$ für die Menge aller $y \in M$ für welche $x \sim y$ \textbf{nicht} gilt.

  Welche der folgenden 4 Aussagen für eine Äquivalenzrelation $\sim$ auf einer
  Menge $M$ sind immer richtig?
  \begin{enumerate}[(i)]
  \item Wenn $\sim$ nur zwei Äquivalenzklassen hat, so ist $A(x)$ für jedes
    $x \in M$ eine Äquivalenzklasse.

  \item Wenn es ein $x \in M$ gibt, so dass $A(x)$ eine Äquivalenzklasse ist,
    so hat $\sim$ nur zwei Äquivalenzklassen.

  \item Für jedes $x \in M$ ist die Menge $A(x)$ eine Äquivalenzklasse von $\sim$.
  \item Für alle $a_1, a_2, a_3 \in M$ gilt: Wenn nicht $a_1 \sim a_2$ und nicht
    $a_2 \sim a_3$, dann gilt auch nicht $a_1 \sim a_3$.
  \end{enumerate}

  \subparagraph{Lsg.}
  \begin{enumerate}[(i)]
  \item Angenommen $\sim$ hat genau zwei Äquivalenzklassen $\qty[a]$ und
    $\qty[b]$, dann gilt für jedes $x \in M$ entweder $x \sim a$ oder
    $x \sim b$.
    Somit ist für jedes $x \in M \colon A(x) = \qty[a]$ oder $A(x) = \qty[b]$.

  \item Wenn es ein $x \in M$ gibt, für das $A(x)$ eine Äquivalenzklasse ist,
    dann ist $M \setminus A(x) = \qty{y \in M {\big |} x \sim y}$ die zweite
    Äquivalenzklasse.

  \item Sei $M = \mathbb{N}$ und $x \sim y \iff x \mod 3 = y \mod 3$.
    Dann sind $4, 5 \in A(3)$.
    Da $1 = 4 \mod 3 \ne 5 \mod 3 = 2$ ist $A(3)$ keine Äquivalenzklasse von
    $\sim$.
  \item Sei $a_1 = a_3$ und $\neg (a_1 \sim a_2)$.
    Es folgen aus der Reflexivität $a_1 \sim a_3$ und aus der Symmetrie
    $\neg(a_2 \sim a_3)$.
  \end{enumerate}
\end{enumerate}

\newpage
\paragraph{Aufgabe 2} Geben Sie mit kurzer Begründung an, welche der folgenden
Vorschriften eine Äquivalenzrelation auf der Menge $\mathbb{N}$ definieren.
\begin{enumerate}[(i)]
\item $x \sim y \iff x, y$ gerade
\item $x \sim y \iff x - y$ gerade
\item $x \sim y \iff x + y$ gerade
\item $x \sim y \iff x - y$ ungerade
\item $x \sim y \iff x + y$ ungerade
\end{enumerate}

\subparagraph{Lsg.}
\begin{enumerate}[(i)]
\item Sei $x = 3 \in \mathbb{N}$.
  Dann gilt $x \sim x$ \textbf{nicht}.

  $\Rightarrow$ keine Äquivalenzrelation

\item Seien $x, y, z \in \mathbb{N}$ beliebig.
  \begin{itemize}
  \item $x - x = 0$ ist immer gerade $\Rightarrow$ reflexiv
  \item $x - y$ ist gerade $\overset{\text{Kommutativität}}\Rightarrow y - x$
    ist gerade $\Rightarrow$ symmetrisch

  \item Sei $x \sim y$ und $y \sim z$.
    Dann sind entweder $x, y, z$ alle gerade oder $x, y, z$ alle ungerade.
    $\Rightarrow x + z$ ist gerade $\Rightarrow x \sim z \Rightarrow$ transitiv.
  \end{itemize}

  $\Rightarrow$ Äquivalenzrelation

\item Seien $x, y, z \in \mathbb{N}$ beliebig.
  \begin{itemize}
  \item $x + x = 2x$ ist immer gerade $\Rightarrow$ reflexiv
  \item $x + y$ ist gerade $\overset{\text{Kommutativität}}\Rightarrow y + x$
    ist gerade $\Rightarrow$ symmetrisch

  \item Sei $x \sim y$ und $y \sim z$.
    Dann sind entweder $x, y, z$ alle gerade oder $x, y, z$ alle ungerade.
    $\Rightarrow x + z$ ist gerade $\Rightarrow x \sim z \Rightarrow$ transitiv.
  \end{itemize}

  $\Rightarrow$ Äquivalenzrelation

\item Sei $x = 5$, $y = 2$ und $z = 1$.
  Dann gilt $x \sim y$ und $y \sim 1$.
  Allerdings gilt nicht $x \sim z$ - ein Widerspruch zur Transitivität.

  $\Rightarrow$ keine Äquivalenzrelation

\item Sei $x = 3$.
  Dann ist $x + x = 6$ gerade.
  Somit gilt $x \sim x$ \textbf{nicht}.
  Das ist ein Widerspruch zu der Reflexivität.

  $\Rightarrow$ keine Äquivalenzrelation
\end{enumerate}

\end{document}