\documentclass{scrreprt}

\usepackage{aligned-overset}
\usepackage{amsmath}
\usepackage{amssymb}
\usepackage{bm}
\usepackage[shortlabels]{enumitem}
\usepackage{hyperref}
\usepackage[utf8]{inputenc}
\usepackage{multicol}
\usepackage{mathtools}
\usepackage{physics}
\usepackage{tabularx}
\usepackage{titling}
\usepackage{fancyhdr}
\usepackage{xfrac}
\usepackage{pgfplots}

\pgfplotsset{compat = newest}
\usetikzlibrary{intersections}
\usetikzlibrary{patterns}
\usepgfplotslibrary{fillbetween}

\author{Karsten Lehmann}
\date{WiSe 2021/2022}
\title{Übungsblatt 02\\Lineare Algebra - Grundlegende Konzepte}

\setlength{\headheight}{26pt}
\pagestyle{fancy}
\fancyhf{}
\lhead{\thetitle}
\rhead{\theauthor}
\lfoot{\thedate}
\rfoot{Seite \thepage}

\begin{document}
\paragraph{Aufgabe 1}
\begin{enumerate}[(a)]
\item Sei $M$ die Menge der Einwohner eines Landes.
  Wir definieren Relationen $\sim_1$, $\sim_2$ und $\sim_3$ auf $M$ durch
  \begin{enumerate}[1)]
  \item $x \sim_1 y \iff x$ und $y$ haben ihren ersten Wohnsitz in der selben
    Stadt.

  \item $x \sim_2 y \iff x$ ist die Mutter von $y$.
  \item $x \sim_3 y \iff x$ und $y$ haben einen Altersunterschied von maximal
    $5$ Jahren.
  \end{enumerate}
  Entscheiden Sie für jede der Relationen, welche der Eigenschaften ``reflexiv'',
  ``symmetrisch'' und ``transitiv'' sie hat.

  \subparagraph{Lsg.}
  \begin{enumerate}[1)]
  \item Relation $\sim_1$ ist reflexiv, symmetrisch und transitiv.
  \item Relation $\sim_2$ ist nicht reflexiv, nicht symmetrisch und nicht
    transitiv.

  \item Relation $\sim_3$ ist reflexiv, symmetrisch, allerdings nicht transitiv.
  \end{enumerate}

\item Sei $\sim$ eine Relation auf einer Menge $M$ und $x \in M$, so schreibe
  $A(x)$ für die Menge aller $y \in M$ für welche $x \sim y$ \textbf{nicht} gilt.

  Welche der folgenden 4 Aussagen für eine Äquivalenzrelation $\sim$ auf einer
  Menge $M$ sind immer richtig?
  \begin{enumerate}[(i)]
  \item Wenn $\sim$ nur zwei Äquivalenzklassen hat, so ist $A(x)$ für jedes
    $x \in M$ eine Äquivalenzklasse.

  \item Wenn es ein $x \in M$ gibt, so dass $A(x)$ eine Äquivalenzklasse ist,
    so hat $\sim$ nur zwei Äquivalenzklassen.

  \item Für jedes $x \in M$ ist die Menge $A(x)$ eine Äquivalenzklasse von $\sim$.
  \item Für alle $a_1, a_2, a_3 \in M$ gilt: Wenn nicht $a_1 \sim a_2$ und nicht
    $a_2 \sim a_3$, dann gilt auch nicht $a_1 \sim a_3$.
  \end{enumerate}

  \subparagraph{Lsg.}
  \begin{enumerate}[(i)]
  \item Angenommen $\sim$ hat genau zwei Äquivalenzklassen $\qty[a]$ und
    $\qty[b]$, dann gilt für jedes $x \in M$ entweder $x \sim a$ oder
    $x \sim b$.
    Somit ist für jedes $x \in M \colon A(x) = \qty[a]$ oder $A(x) = \qty[b]$.

  \item Wenn es ein $x \in M$ gibt, für das $A(x)$ eine Äquivalenzklasse ist,
    dann ist $M \setminus A(x) = \qty{y \in M {\big |} x \sim y}$ die zweite
    Äquivalenzklasse.

  \item Sei $M = \mathbb{N}$ und $x \sim y \iff x \mod 3 = y \mod 3$.
    Dann sind $4, 5 \in A(3)$.
    Da $1 = 4 \mod 3 \ne 5 \mod 3 = 2$ ist $A(3)$ keine Äquivalenzklasse von
    $\sim$.
  \item Sei $a_1 = a_3$ und $\neg (a_1 \sim a_2)$.
    Es folgen aus der Reflexivität $a_1 \sim a_3$ und aus der Symmetrie
    $\neg(a_2 \sim a_3)$.
  \end{enumerate}
\end{enumerate}

\newpage
\paragraph{Aufgabe 2} Geben Sie mit kurzer Begründung an, welche der folgenden
Vorschriften eine Äquivalenzrelation auf der Menge $\mathbb{N}$ definieren.
\begin{enumerate}[(i)]
\item $x \sim y \iff x, y$ gerade
\item $x \sim y \iff x - y$ gerade
\item $x \sim y \iff x + y$ gerade
\item $x \sim y \iff x - y$ ungerade
\item $x \sim y \iff x + y$ ungerade
\end{enumerate}

\subparagraph{Lsg.}
\begin{enumerate}[(i)]
\item Sei $x = 3 \in \mathbb{N}$.
  Dann gilt $x \sim x$ \textbf{nicht}.

  $\Rightarrow$ keine Äquivalenzrelation

\item Seien $x, y, z \in \mathbb{N}$ beliebig.
  \begin{itemize}
  \item $x - x = 0$ ist immer gerade $\Rightarrow$ reflexiv
  \item $x - y$ ist gerade $\overset{\text{Kommutativität}}\Rightarrow y - x$
    ist gerade $\Rightarrow$ symmetrisch

  \item Sei $x \sim y$ und $y \sim z$.
    Dann sind entweder $x, y, z$ alle gerade oder $x, y, z$ alle ungerade.
    $\Rightarrow x + z$ ist gerade $\Rightarrow x \sim z \Rightarrow$ transitiv.
  \end{itemize}

  $\Rightarrow$ Äquivalenzrelation

\item Seien $x, y, z \in \mathbb{N}$ beliebig.
  \begin{itemize}
  \item $x + x = 2x$ ist immer gerade $\Rightarrow$ reflexiv
  \item $x + y$ ist gerade $\overset{\text{Kommutativität}}\Rightarrow y + x$
    ist gerade $\Rightarrow$ symmetrisch

  \item Sei $x \sim y$ und $y \sim z$.
    Dann sind entweder $x, y, z$ alle gerade oder $x, y, z$ alle ungerade.
    $\Rightarrow x + z$ ist gerade $\Rightarrow x \sim z \Rightarrow$ transitiv.
  \end{itemize}

  $\Rightarrow$ Äquivalenzrelation

\item Sei $x = 5$, $y = 2$ und $z = 1$.
  Dann gilt $x \sim y$ und $y \sim 1$.
  Allerdings gilt nicht $x \sim z$ - ein Widerspruch zur Transitivität.

  $\Rightarrow$ keine Äquivalenzrelation

\item Sei $x = 3$.
  Dann ist $x + x = 6$ gerade.
  Somit gilt $x \sim x$ \textbf{nicht}.
  Das ist ein Widerspruch zu der Reflexivität.

  $\Rightarrow$ keine Äquivalenzrelation
\end{enumerate}

\newpage
\paragraph{Aufgabe 3} Sei $M$ die Menge der Studenten einer Universität.
Geben Sie Beispiele für Relationen auf $M$ an, die
\begin{enumerate}[(1)]
\item reflexiv, symmetrisch und transitiv sind.
\item reflexiv sowie symmetrisch, allerdings nicht transitiv sind.
\item nicht reflexiv, aber symmetrisch und transitiv sind.
\item reflexiv, nicht symmetrisch und transitiv sind.
\item nicht reflexiv, nicht symmetrisch und nicht transitiv sind.
\end{enumerate}

\subparagraph{Lsg.}
\begin{enumerate}[(1)]
\item
  Seien $x, y \in M$ Studenten und $x, y$ genau dann wenn $x, y$ den selben
  Studiengang besuchen.

\item Seien $x,y \in M$ Studenten und $x \sim y$ genau dann wenn $x$ und $y$
  den selben Vornamen haben oder $x$ und $y$ gleich alt sind.

\item Seien $x,y \in M$ Studenten und $x \sim y$ genau dann wenn $1 = 2$.
  Da aus falschen beliebiges folgt, ist diese Relation symmetrisch und transitiv.
  Weiterhin gilt $x \sim x$ \textbf{nicht}, da $1 \ne 2$.

\item Seien $x,y \in M$ Studenten und $x \sim y$ genau dann wenn die
  Matrikelnummer von $y$ größer oder gleich der Matrikelnummer von $x$ ist.
  \begin{itemize}
  \item Es ist $x \sim x$, da die Matrikelnummer größer gleich sich selbst ist
    (\emph{Reflexiv}).
  \item Sei $x \ne y \in M$, $x \sim y$ und die Matrikelnummer von $y$ echt
    größer der Matrikelnummer von $x$.
    Somit gilt \textbf{nicht} $y \sim x$.

    $\Rightarrow$ kein \emph{Symmetrie}.
  \item Seien $x, y, z \in M$ Studenten.
    Wenn $x \sim y$ und $y \sim z$ folgt auch $x \sim z$ (\emph{transitiv}).
  \end{itemize}

\item Seien $x, y$ Studenten und $x \sim y$ genau dann wenn der erste Buchstabe
  des Vornamens von $x$ dem ersten Buchstaben des Nachnamens von $y$ gleicht.
\end{enumerate}

\newpage
\paragraph{Aufgabe 4} Beweisen Sie
\begin{enumerate}[a)]
\item ``Die Relation
$R \coloneqq \qty\big{\qty(m, n) {\big |} m, n \in \mathbb{Z},
m \text{ teilt } n}$ ist eine reflexive und transitive Relation auf
$\mathbb{Z}$.''

\item ``Weiterhin ist
$R_0 \coloneqq \qty\big{(m, n) {\big |} m, n \in \mathbb{N}_0,
  m \text{ teilt } n}$ eine Ordnung auf $\mathbb{N}_0$''.
\end{enumerate}

\subparagraph{Lsg.}
\begin{enumerate}[a)]
\item Sei $m \in \mathbb{Z}$ beliebig.
Es gilt $m$ teilt $m$, falls ein $z \in \mathbb{Z}$ existiert mit
$m \cdot z = m$.

$\Rightarrow R$ ist reflexiv mit $m \cdot 1 = m$. \\

Seien $x, y, z \in \mathbb{Z}$ und $(x, y), (y, z) \in R$.

  $\Rightarrow \exists z_1, z_2 \in \mathbb{Z} \colon y = x \cdot z_1 \land z = y \cdot z_2$

  $\Rightarrow z = x \cdot z_1 \cdot z_2$

  Für alle $z_1, z_2 \in \mathbb{Z}$ gilt $z_1 \cdot z_2 \in \mathbb{Z}$.

  $\Rightarrow (x, z) \in R$

  $\Rightarrow R$ ist \emph{transitiv}. \\

  Allerdings ist $R$ nicht antisymmetrisch, da $x = -2, y = 2 \in \mathbb{Z}$
  und $x \:|\: y$ sowie $y \:|\: x$, aber $x \ne y$.

\item Eine Ordnung ist eine Relation, die reflexiv, antisymmetrisch und transitiv ist.
  Die \emph{Reflexivität} und \emph{Transitivität} wurde schon in a) gezeigt.

  \underline{Antisymmetrie:} Seien $x, y \in \mathbb{N}_0$ und $x \:|\: y$ sowie
  $y \:|\: x$.
  Dann existieren $z_1, z_2 \in \mathbb{Z}$ mit $x = z_1 \cdot y$ und
  $y = z_2 \cdot x$.

  $\Rightarrow y = z_1 \cdot z_2 \cdot y$

  $\Rightarrow z_1 \cdot z_2 = 1 \Rightarrow z_1 = z_2 = 1$

  $\Rightarrow x = y$

  $\Rightarrow R_0$ ist antisymmetrisch.
\end{enumerate}

\end{document}