\documentclass{article}

\usepackage{amsmath}
\usepackage{amssymb}
\usepackage[ngerman]{babel}
\usepackage[shortlabels]{enumitem}
\usepackage{genealogytree}
\usepackage{hyperref}
\usepackage[utf8]{inputenc}
\usepackage{mathtools}
\usepackage{tabularx}
\usepackage{textcomp}
\usepackage{tikz}
\usetikzlibrary{positioning}
\usetikzlibrary{shapes}
\usepackage{xcolor}
\definecolor{light-gray}{gray}{.9}

\author{Karsten Lehmann}
\date{WiSe 2020}
\title{Übung Programmieren für Mathematiker (PR10)}

\begin{document}


\maketitle

\newpage

\begin{tikzpicture}[scale=.5, every node/.style={scale=.5}]
  \node[draw, rectangle, rounded corners] at (0,0) (start) {Start};
  \node[draw, diamond, below = of start] (durst) {Durst?};
  \node[align = left, draw, rectangle, right = of durst] (gb) {Inhalte Geldbörse \\- Preis der Besorgung \\= Rest};
  \node[draw, diamond, below = of gb] (bp) {Rest $\geq$ Bierpreis};
  \node[align=left, draw, rectangle, below = of bp] (wirtschaft) {Nächste \\Wirtschaft\\suchen};
  \node[align = center, draw, diamond, below = of wirtschaft] (wfrage) {Wirtschaft\\annehmbar?};
  \node[align=left, draw, rectangle, below = of wfrage] (bier) {Bier\\bestellen,\\Bier\\trinken};
  \node[align = center, draw, diamond, below = of bier] (durst_frage) {Noch\\Durst?};
  \node[align = left, draw, rectangle, right = of durst_frage] (rest) {verbleibender Rest\\- Bierpreis\\= neuer Rest};
  \node[align = center, draw, diamond, below = of rest] (rest_frage) {Rest $\geq$\\Bierpreis?};
  \node[align = center, draw, diamond, right = of rest_frage] (auto_frage) {Auto-\\fahrer?};  
  \node[align = center, draw, diamond, below = of auto_frage] (pegel_frage) {0.5 \textperthousand\\erreicht?};

  \node[align = left, draw, rectangle] at (durst_frage |- pegel_frage) (zahlen) {Bier\\bezahlen};
  \coordinate (p1) at (start |- zahlen);
  \coordinate (p2) at (durst_frage |- rest_frage);
  \coordinate (p3) at (durst |- bp);
  \coordinate[right = of wfrage] (p4);
  \coordinate[below = .5cm of bp] (p5);
  \coordinate[right = .5cm of auto_frage] (p6);
  \coordinate[below = .5cm of wfrage] (p7);
  \node[draw, rectangle, rounded corners, below = of p1] (ende) {Ende};

  \draw[-latex] (start) -> (durst);
  \draw[-latex] (durst) -> node[above] {Nein} (gb);
  \draw[-latex] (gb) -> (bp);
  \draw[-latex] (bp) -> node[left] {ja} (wirtschaft);
  \draw[-latex] (wirtschaft) -> (wfrage);
  \draw[-latex] (wfrage) -> node[left] {ja} (bier);
  \draw[-latex] (bier) -> (durst_frage);
  \draw[-latex] (durst_frage) -> node[above] {ja} (rest);
  \draw[-latex] (rest) -> (rest_frage);
  \draw[-latex] (rest_frage) -> node[above] {ja} (auto_frage);
  \draw[-latex] (auto_frage) -> node[left] {ja} (pegel_frage);
  \draw[-latex] (pegel_frage) -> node[above, xshift=3cm] {ja} (zahlen);
  \draw[-latex] (durst) -> node[left, yshift=15cm] {nein} (ende); 
  \draw[-latex] (zahlen) -> (p1); 
  \draw[-latex] (durst_frage) -> node[left, yshift=3cm] {nein} (zahlen);
  \draw[-latex] (rest_frage) -> node[above] {nein} (p2); 
  \draw[-latex] (bp) -> node[above] {nein} (p3);
  \draw (wfrage) -- (p4);
  \draw[-latex] (p4) |- node[above] {Nein} (p5);
  
  \draw (pegel_frage) -| node[below] {Nein} (p6);
  \draw (auto_frage) -- node[above] {Nein} (p6);
  \draw[-latex] (p6) |- (p7);
\end{tikzpicture}

\section*{Variablen}

\begin{tabularx}{\textwidth}{l X}
Durst? & LOGICAL, Wert im Algorithmus nicht definiert, eventuell durch Funktion berechnet \\
Inhalt\_Geldboerse & INTEGER (in Cent) oder REAL (in Euro), Konstante \\
Preis\_der\_Besorgung & INTEGER oder REAL (siehe oben) \\
Rest & INTEGER oder REAL (siehe oben) \\
Bierpreis & INTEGER oder REAL (siehe oben), Konstante \\
Wirtschaft\_annehmbar & LOGICAL \\
Autofahrer & LOGICAL, Konstante \\
\end{tabularx}

\end{document}