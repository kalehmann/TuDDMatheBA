\documentclass{article}

\usepackage{aligned-overset}
\usepackage{amsmath}
\usepackage{amssymb}
\usepackage{bbm}
\usepackage[shortlabels]{enumitem}
\usepackage{genealogytree}
\usepackage{hyperref}
\usepackage[utf8]{inputenc}
\usepackage{interval}
\intervalconfig{
  soft open fences
}
\usepackage{listings}
\usepackage{mathtools}
\usepackage{physics}
\usepackage{tikz}
\usetikzlibrary{positioning}
\usepackage{xcolor}
\definecolor{light-gray}{gray}{.9}

\author{Karsten Lehmann}
\date{03.11.2020}
\title{Sonderübung Programmieren für Mathematiker}

\begin{document}


\maketitle
\newpage

\section*{Grundlegende Struktur eines FORTRAN-Programms}

\begin{lstlisting}{Fortran}
  PROGRAM taschenrechner
    IMPLICIT NONE

    INTEGER :: a, b, ergebnis

    READ (*,*) a, b
    ergebnis = a + b
    WRITE(*,*) " a + b = ", ergebnis
  END PROGRAM
\end{lstlisting}

\textbf{IMPLICIT NONE}
\begin{itemize}
\item IMPLICIT NONE Statement sollte verwendet werden
\item wenn IMPLICIT NONE nicht verwendet, wird der Variable \emph{zufällig} (nach Lage
  des Buchstaben im Alphabet) ein Datentyp zugewiesen ($\Rightarrow$ Chaos)
\end{itemize}

\textbf{Maximale Zeilenlänge}
\begin{itemize}
\item beträgt 132 Zeichen
\item \textbf{Achtung} Es werden auch nur die ersten 132 Zeichen ausgewertet, wenn diese syntaktisch korrekt sind,
  kompiliert das Programm dennoch (jedoch nur die ersten 132 Zeichen).
\item mit \& kann man Zeilen fortführen, jedoch maximal 40 mal.
\end{itemize}

\textbf{Groß-/Kleinschreibung}
\begin{itemize}
\item Seit dem Fortran 90 Standard ist die Sprache case insensitive
\end{itemize}

\textbf{Kommentare}
\begin{itemize}
\item Werden mit dem ! eingeleitet
\end{itemize}

\textbf{Variablen}

Werden deklariert mit

\begin{lstlisting}{fortran}
  DATENTYP :: variable1, variabl2, ...
\end{lstlisting}

\begin{tabular}{l l}
  INTEGER   & ganze Zahlen \\
  REAL      & Gleitkommazahlen \\
  CHARACTER & Zeichenketten \\
  LOGICAL & Boolean (Werte: .TRUE. und .FALSE.)\\
\end{tabular}

\end{document}