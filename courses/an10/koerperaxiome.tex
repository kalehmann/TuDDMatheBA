\documentclass{article}
\usepackage{aligned-overset}
\usepackage{amsmath}
\usepackage{amssymb}
\usepackage{amsthm}
\usepackage{bm}
\usepackage[shortlabels]{enumitem}
\usepackage{hyperref}
\usepackage[utf8]{inputenc}
\usepackage{mathtools}
\usepackage{physics}
\usepackage{pgfplots}
\usepackage{titling}
\usepackage{fancyhdr}
\usepackage{xfrac}

\newtheorem*{definition}{Definition}
\newtheorem*{satz}{Satz}

\renewcommand{\thesubsection}{\arabic{subsection}}

\author{Karsten Lehmann}
\date{WiSe 2020}
\title{Körperaxiome}

\pagestyle{fancy}
\fancyhf{}
\lhead{\thetitle}
\rhead{\theauthor}
\lfoot{\thedate}
\rfoot{Seite \thepage}

\begin{document}
\section*{Körperaxiome}

\subsection*{Algebraische Strukturen}

\begin{definition}[Abelsche Gruppe]
  \label{def:abel}
  Eine Abelsche Gruppe $(G, *)$ ist ein paar aus einer Menge $G$ und einer Verknüpfung $*$, so dass gilt:
  \begin{itemize}
  \item $(a * b) * c = a * (b * c)$ (Assoziativität)
  \item $a * e = e * a = a$ (Existenz eines neutralen Elementes $e$)
  \item es existiert für jedes $a \in G$ ein $a^{-1} \in G$, so dass $a * a^{-1} = e$
    (Existenz eines Inversen Elementes)
  \item $a * b = b * a$ (Kommutativität)
  \end{itemize}
\end{definition}

\begin{definition}[Ring]
  \label{def:ring}
  Ein Ring $(R, +, \cdot)$ ist eine Menge $R$ mit zwei Verknüpfungen $+$ und $\cdot$, so dass gilt:
  \begin{itemize}
  \item $(R, +)$ ist eine \hyperref[def:abel]{Abelsche Gruppe}
  \item $a \cdot b + a \cdot c = a \cdot (b + c)$ (Distributivität)
  \end{itemize}
\end{definition}

\begin{definition}[Trivialer Ring]
  \label{def:trivring}
  Der Triviale Ring $(\{ 0 \}, +, \cdot)$ ist ein \hyperref[def:ring]{Ring} mit der Zahl Null als
  einzigem Element und den beiden trivialen Verknüpfungen $+ \colon 0 + 0 = 0$ und
  $\cdot \colon 0 \cdot 0 = 0$.
\end{definition}

\begin{definition}[Körper]
  Ein Körper $(K, +, \cdot)$ ist ein \hyperref[def:trivring]{nicht-trivialer Ring}, in dem zusätzlich gilt:
  \begin{itemize}
  \item $a \cdot b = b \cdot a$ (Kommutativität)
  \item $a \cdot e = e \cdot a = a$ (Existenz eines multiplikativen Neutral-/Einzelelementes)
  \item $\forall a \in K, a \ne 0 \colon \exists \; a^{-1} \in K \colon a \cdot a^{-1} = 0$
    (Existenz eines multiplikativen inversen Elementes)
  \end{itemize}
\end{definition}

\pagebreak

\subsection*{Schranken}

\begin{definition}[Obere Schranke]
  $K$ ist ein angeordneter Körper. $M \subseteq K \land M \ne \emptyset$.
  Ein Element $s \in K$ ist \textbf{obere Schranke} von $M$, wenn $\forall x \in M \colon x \leq s$. 
\end{definition}

\begin{definition}[untere Schranke]
  $K$ ist ein angeordneter Körper. $M \subseteq K \land M \ne \emptyset$.
  Ein Element $s \in K$ ist \textbf{untere Schranke} von $M$, wenn $\forall x \in M \colon x \geq s$. 
\end{definition}

\begin{definition}[Maximum]
  $K$ ist ein angeordneter Körper. $M \subseteq K \land M \ne \emptyset$.
  Ein Element $s \in K$ heißt Maximum von $M$ oder $\max M$, falls $s$ eine obere Schranke von $M$
  ist und $s \in M$.
\end{definition}

\begin{definition}[Minimum]
  $K$ ist ein angeordneter Körper. $M \subseteq K \land M \ne \emptyset$.
  Ein Element $s \in K$ heißt Minimum von $M$ oder $\min M$, falls $s$ eine untere Schranke von $M$
  ist und $s \in M$.
\end{definition}

\begin{definition}[Supremum]
  $K$ ist ein angeordneter Körper. $M \subseteq K \land M \ne \emptyset$.
  Ein Element $s \in K$ heißt Supremum von $M$ oder $\sup M$, falls $s$ eine obere Schranke von $M$
  ist und $s \leq \overset\_s$ für alle oberen Schranken $\overset\_s$ von $M$.
\end{definition}

\begin{definition}[Infimum]
  $K$ ist ein angeordneter Körper. $M \subseteq K \land M \ne \emptyset$.
  Ein Element $s \in K$ heißt Infimum von $M$ oder $\inf M$, falls $s$ eine untere Schranke von $M$
  ist und $s \geq \overset\_s$ für alle unteren Schranken $\overset\_s$ von $M$.
\end{definition}


\pagebreak

\subsection*{Körperaxiome}

Zusammengefasst ergeben sich somit in einem Körper $(K, +, \cdot)$ mit den beiden Verknüpfungen $+$
und $\cdot$ die folgenden Rechenregeln:

\begin{enumerate}[label=(A\arabic*)]
\item \label{a1} $\forall a,b,c \in K \colon (a + b) + c = a + (b + c)$
  (Additive Assoziativität)
\item \label{a2} $\exists \; 0 \in K \colon \forall a \in K \colon a + 0 = a$
  (Additives neutrales Element)
\item \label{a3} $\forall a \in K \exists \; (-a) \in K \colon a + (-a) = 0$
  (Additives Inverses)
\item \label{a4} $\forall a,b \in K \colon a + b = b + a$
  (Additive Kommutativität)
\item \label{a5} $\forall a,b,c \in K \colon (a \cdot b) \cdot c = a \cdot (b \cdot c)$
  (Multiplikative Assoziativität)
\item \label{a6} $\exists \; 1 \in K \colon \forall a \in K \colon a \cdot 1 = a$
  (Multiplikatives neutrales Element)
\item \label{a7} $\forall a \in K, a \ne 0 \: \exists \; a^{-1} \in K \colon a \cdot a^{-1} = 1$
  (Multiplikatives Inverses)
\item \label{a8} $\forall a,b \in K \colon a \cdot b = b \cdot a$
  (Multiplikative Kommutativität)
\item \label{a9} $\forall a,b,c \in K \colon a \cdot b + a \cdot c = a \cdot (b + c)$
  (Distributivität)
\end{enumerate}

Eine Menge $K$, welche mit zwei Verknüpfungen von $K$ nach $K$, $+\colon K \times K \to K$ und
$\cdot\colon K \times K \to K$ die Axiome (A1) bis (A9) erfüllt ist somit ein Körper.

\subsubsection*{Beispiele für Körper}

Unendliche Körper sind zum Beispiel die Reellen Zahlen $\mathbb{R}$ oder die komplexen Zahlen $\mathbb{C}$.

Ein endlicher Körper mit zwei Elementen ist zum Beispiel die Menge $\{ 0, 1\}$ mit folgenden Verknüpfungen:

\begin{itemize}
\item $0 + 0 = 0$
\item $0 + 1 = 1$
\item $1 + 0 = 1$
\item $1 + 1 = 0$
\item $0 \cdot 0 = 0$
\item $1 \cdot 0 = 0$
\item $0 \cdot 1 = 0$
\item $1 \cdot 1 = 1$
\end{itemize}

\pagebreak

\subsection*{Anordnungsaxiome}

Sei $(K,+,\cdot)$ ein Körper. Falls es eine Teilmenge $P \subseteq K$ gibt, sodass die folgenden 3 Axiome
gelten:

\begin{enumerate}[label=(A\arabic*)]
\setcounter{enumi}{9}
\item \label{a10} $\forall a \in K \colon a = 0 \lor a \in P \lor (-a) \in P$ (Trichotomiegesetz)
\item \label{a11} $\forall a, b \in P \colon a + b \in P$
\item \label{a12} $\forall a, b \in P \colon a \cdot b \in P$
\end{enumerate}

dann heißt $(K,+,\cdot)$ auch \textbf{angeordneter Körper}.

\subsubsection*{Das Vollständigkeitsaxiom}

\begin{enumerate}[label={(A\arabic*)}]
\setcounter{enumi}{12}
\item  \label{a13}
  In $\mathbb{R}$ besitzt jede nichtleere, nach oben beschränkte Menge ein Supremum \textbf{und}
  jede nichtleere, nach unten beschränkte Menge ein Infimum
\end{enumerate}

\pagebreak

\section*{Anwendung}

\subsection{Eindeutigkeit des additiven Neutralelementes}

Angenommen es gäbe zwei neutrale Elemente der Addition $a$ und $b$.
Dann gilt $\forall x \in K \colon x + a = x \land x + b = x$ und es folgt:

\begin{align*}
  a &= a \\
  \overset{\text{\hyperref[a2]{(A2)}, $b$ ist neutrales Element}}&= a + b \\
  \overset{\text{\hyperref[a4]{(A4)}, Kommutativität der Addition}}&= b + a \\
  \overset{\text{\hyperref[a4]{(A2)}, $a$ ist neutrales Element}}&= b \\
\end{align*}

Folglich ist das neutrale Element der Addition eindeutig.

\subsection{Eindeutigkeit des multiplikativen Neutralelementes}

Angenommen es gäbe zwei neutrale Elemente der Multiplikation $a$ und $b$.
Dann gilt $\forall x \in K \colon x \cdot a = x \land x \cdot b = x$ und es folgt:

\begin{align*}
  a &= a \\
  \overset{\text{\hyperref[a6]{(A6)}, $b$ ist neutrales Element}}&= a \cdot b \\
  \overset{\text{\hyperref[a8]{(A8)}, Kommutativität der Multiplikation}}&= b \cdot a \\
  \overset{\text{\hyperref[a4]{(A6)}, $a$ ist neutrales Element}}&= b \\
\end{align*}

Folglich ist das neutrale Element der Addition eindeutig.

\pagebreak

\subsection{Eindeutigkeit des additiven Inversen}
\label{eindeutigkeit_add_inv}

Angenommen für ein Element $a \in K$ gäbe es zwei additive Inverse Element $b$ und $c$ mit
$a + b = 0 \land a + c = 0$.

\begin{align*}
  b &= b \\ 
  \overset{\text{\hyperref[a2]{(A2), $0$ ist neutrales Element der Addition}}}&= b + 0 \\
  \overset{\text{Annahme $a + c = 0$}}&= b + (a + c) \\
  \overset{\text{\hyperref[a1]{(A1)}}}&= (b + a) + c \\
  \overset{\text{\hyperref[a4]{(A4)}, Kommutativität der Addition}}&= (a + b) + c \\
  \overset{\text{Annahme $a + b = 0$}}&= 0 + c \\
  \overset{\text{\hyperref[a4]{(A4)}, Kommutativität der Addition}}&= c + 0 \\
  \overset{\text{\hyperref[a2]{(A2), $0$ ist neutrales Element der Addition}}}&= c \\
\end{align*}

Somit ist das additive Inverse eindeutig bestimmt.

\subsection{Eindeutigkeit des multiplikativen Inversen}
\label{eindeutigkeit_mul_inv}

Angenommen für ein Element $a \in K$ gäbe es zwei multiplikative Inverse Elemente $b$ und $c$ mit
$a \cdot b = 1 \land a \cdot c = 1$.

\begin{align*}
  b &= b \\
  \overset{\text{\hyperref[a6]{(A6), $1$ ist neutrales Element der Multiplikation}}}&= b \cdot 1 \\
  \overset{\text{Annahme $a \cdot c = 1$}}&= b \cdot (a \cdot c) \\
  \overset{\text{\hyperref[a5]{(A5)}}}&= (b \cdot a) \cdot c \\
  \overset{\text{\hyperref[a8]{(A8)}, Kommutativität der Multiplikation}}&= (a \cdot b) \cdot c \\
  \overset{\text{Annahme $a \cdot b = 1$}}&= 1 \cdot c \\
  \overset{\text{\hyperref[a8]{(A8)}, Kommutativität der Multiplikation}}&= c \cdot 1 \\
  \overset{\text{\hyperref[a6]{(A6), $1$ ist neutrales Element der Multiplikation}}}&= c \\
\end{align*}

Somit ist das multiplikative Inverse eindeutig bestimmt.

\subsection{Existenz der Differenz von b und a}

Für jedes $a \in \mathbb{R}$ und jedes $b \in \mathbb{R}$ gibt es ein $x \in \mathbb{R}$
mit $a + x = b$.

Angenommen $x$ wäre gleich $b - a$, dann gilt

\begin{align*}
  a + x &= a + x \\
  a + x \overset{\text{Annahme}}&= a + (b - a) \\
  a + x \overset{\hyperref[a4]{(A4)}}&= a + ((-a) + b) \\
  a + x \overset{\hyperref[a1]{(A1)}}&= (a + (-a)) + b \\
  a + x \overset{\hyperref[a3]{(A3)}}&= 0 + b \\
  a + x \overset{\hyperref[a4]{(A4)}}&= b + 0 \\
  a + x \overset{\hyperref[a2]{(A2)}}&= b
\end{align*}

\subsection{Existenz des Quotienten von b und a}
Für jedes $a \in \mathbb{R}$ und jedes $b \in \mathbb{R}$ gibt es ein $x \in \mathbb{R}$
mit $a \cdot x = b$

Angenommen $x$ wäre gleich $\frac{b}{a}$, dann gilt

\begin{align*}
  a \cdot x &= a \cdot x \\
  a \cdot x \overset{\text{Annahme}}&= a \cdot \frac{b}{a} \\ 
  a \cdot x \overset{\hyperref[a8]{(A8)}}&= (a \cdot a^{-1}) \cdot b \\ 
  a \cdot x \overset{\hyperref[a7]{(A7)}}&= 1 \cdot b \\
  a \cdot x \overset{\hyperref[a6]{(A6)}}&= b
\end{align*}

\subsection{Das additive Inverse des additiven Inversen eines Elementes ist das Element selbst}
\label{add_add_inv}

\begin{align*}
  (-a) + a \overset{\hyperref[a4]{(A4)}}&= a + (-a) \\
  \overset{\hyperref[a2]{(A2)}}&= 0
\end{align*}

Damit ist $a$ ein additives inverses Element von $(-a)$.
Somit ist $(-(-a)) = a$

\subsection{Das multiplikative Inverse des multiplikativen Inversen eines Elementes ist das Element selbst}

\begin{align*}
  a^{-1} \cdot a \overset{\hyperref[a8]{(A8)}}&= a \cdot a^{-1} \\
  \overset{\hyperref[a7]{(A7)}}&= 0 \\
\end{align*}

Damit ist $a$ ein multiplikatives inverses Element von $a^{-1}$.
Somit ist
\[
  (a^{-1})^{-1} = a
\]

\subsection{Wenn $ab = 0$ impliziert, dass  $a = 0$ oder $b = 0$}

\begin{itemize}
\item[``$\Rightarrow$''] $a \cdot 0 = 0 \cdot a =  0$
  \label{ab0:right}
  \begin{align*}
    0 &= 0 \\
      \overset{\hyperref[a3]{(A3)}}&= a \cdot 0 - a \cdot 0 \\
      \overset{\hyperref[a2]{(A2)}}&= a \cdot (0 + 0) - a \cdot 0 \\
      \overset{\hyperref[a9]{(A9)}}&= (a \cdot 0 + a \cdot 0) - a \cdot 0 \\
      \overset{\hyperref[a1]{(A1)}}&= a \cdot 0 + (a \cdot 0 - a \cdot 0) \\
      \overset{\hyperref[a3]{(A3)}}&= a \cdot 0 + 0 \\
      \overset{\hyperref[a2]{(A2)}}&= a \cdot 0 \\
      \overset{\hyperref[a8]{(A8)}}&= 0 \cdot a \\
  \end{align*}
\item[``$\Leftarrow$'']
  Vorausgesetzt $a \cdot b = 0$.

  \begin{enumerate}[label=\textbf{\arabic*. Fall}]
  \item $a = 0$. In diesem Fall ist die Annahme direkt bestätigt.
  \item $a \ne 0$.
    \begin{align*}
      b &= b \\
      \overset{\hyperref[a6]{(A6)}}&= b \cdot 1 \\
      \overset{\hyperref[a7]{(A7)}}&= b \cdot (a \cdot a^{-1})  \\      
      \overset{\hyperref[a5]{(A5)}}&= (b \cdot a) \cdot a^{-1}  \\
      \overset{\text{Voraussetzung}}&= 0 \cdot a^{-1}  \\
      \overset{\hyperref[ab0:right]{\text{Implikation }\Rightarrow}}&= 0 
    \end{align*}
  \end{enumerate}
\end{itemize}

\subsection{Das additive Inverse der Summe zweier Elemente ist die Summe der beiden Inversen}

Es sei zu beweisen, dass $-(a + b) = (-a) + (-b)$.

\begin{align*}
  (a + b) + ((-a) + (-b)) \overset{\hyperref[a1]{(A1)}}&= (a + (-a)) + (b + (-b)) \\
                          \overset{\hyperref[a3]{(A3)}}&= 0 + 0 \\
                          \overset{\hyperref[a2]{(A2)}}&= 0
\end{align*}

Somit ist $(-a) + (-b)$ ein additives Inverses von $(a + b)$. Aus der
\hyperref[eindeutigkeit_add_inv]{Eindeutigkeit des additiven Inversen}
folgt, dass $-(a + b) = (-a) + (-b)$.

\subsection{Das multiplikative Inverse des Produktes zweier Elemente ist das Produkt der beiden Inversen}

Es sei zu beweisen, dass $(a \cdot b)^{-1} = a^{-1} \cdot b^{-1}$.

\begin{align*}
  (a \cdot b) \cdot (a^{-1} \cdot b^{-1}) \overset{\hyperref[a5]{(A5)}}&= (a \cdot a^{-1}) \cdot (b \cdot b^{-1}) \\
                                         \overset{\hyperref[a7]{(A7)}}&= 1 \cdot 1 \\
                                         \overset{\hyperref[a6]{(A6)}}&= 1
\end{align*}

Somit ist $a^{-1} \cdot b^{-1}$ ein multiplikatives Inverses von $(a \cdot b)$. Aus der
\hyperref[eindeutigkeit_mul_inv]{Eindeutigkeit des multiplikativen Inversen}
folgt, dass $(a \cdot b)^{-1} = a^{-1} \cdot b^{-1}$.

\subsection{Das Produkt eines Elementes mit dem additiven Inversen eines weiteren Elementes ist das additive Inverse des Produktes beider Elemente}
\label{prod_add_inv}

Es sei zu beweisen, dass $a \cdot (-b) = -(a \cdot b)$.

\begin{align*}
  a \cdot b + a \cdot (-b) \overset{\hyperref[a9]{(A9)}}&= a \cdot (b + (-b)) \\
                           \overset{\hyperref[a3]{(A3)}}&= a \cdot 0 \\
                           \overset{\hyperref[ab0:right]{a \cdot 0 = 0}}&= 0
\end{align*}

Somit ist $a \cdot (-b)$ ein additives Inverses von $a \cdot b$. Aus der
\hyperref[eindeutigkeit_add_inv]{Eindeutigkeit des additiven Inversen}
folgt, dass $a \cdot (-b) = -(a \cdot b)$.


\subsection{Für jedes Element verschieden von 0 ist das multiplikative Inverse des additiven Inversen gleich dem additiven Inversen des multiplikativen Inversen}

Es sei zu beweisen, dass $(-a)^{-1} = -(a^{-1})$.

\begin{align*}
  (-a) \cdot -(a^{-1}) \overset{\hyperref[prod_add_inv]{(a \cdot (-b) = -(a \dot b))}}&= -((-a) \cdot a^{-1}) \\
                       \overset{\hyperref[a8]{(A8)}}&= -(a^{-1} \cdot (-a)) \\
                       \overset{\hyperref[prod_add_inv]{(a \cdot (-b) = -(a \dot b))}}&= -(-(a^{-1} \cdot a)) \\
                       \overset{\hyperref[a8]{(A8)}}&= -(-(a \cdot a^{-1})) \\
                       \overset{\hyperref[a7]{(A7)}}&= -(-(1) \\
                       \overset{\hyperref[add_add_inv]{-(-(a)) = a}}&= 1
\end{align*}

Somit ist $-(a^{-1})$ ein multiplikatives Inverses von $(-a)$. Aus der
\hyperref[eindeutigkeit_mul_inv]{Eindeutigkeit des multiplikativen Inversen}
folgt, dass $(-a)^{-1} = -(a^{-1})$.

\subsection{1. Binomische Formel}

Es sei zu beweisen, dass $(a + b)^2 = a^2 + 2ab + b^2$.

\begin{align*}
  (a + b)^2 \overset{a^2 = a \cdot a}&= (a + b) \cdot (a + b) \\
            \overset{\hyperref[a9]{(A(9)}}&= (a + b) \cdot a + (a + b) \cdot b \\
            \overset{\hyperref[a8]{(A8)}}&= a \cdot (a + b) + b \cdot (a + b) \\
            \overset{\hyperref[a9]{(A9)}}&= a \cdot a + a \cdot b + b \cdot a + b \cdot b \\
            \overset{\hyperref[a8]{(A8)}}&= a \cdot a + a \cdot b + a \cdot b + b \cdot b \\
            \overset{a \cdot a = a^2}&= a^2 + a \cdot b + a \cdot b + b^2 \\
            \overset{a + a = 2a}&= a^2 + 2ab + b^2
\end{align*}

\end{document}