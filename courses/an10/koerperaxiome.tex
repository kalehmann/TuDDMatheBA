\documentclass{article}
\usepackage{aligned-overset}
\usepackage{amsmath}
\usepackage{amssymb}
\usepackage{amsthm}
\usepackage{bm}
\usepackage[shortlabels]{enumitem}
\usepackage{hyperref}
\usepackage[utf8]{inputenc}
\usepackage{mathtools}
\usepackage{physics}
\usepackage{pgfplots}
\usepackage{titling}
\usepackage{fancyhdr}
\usepackage{xfrac}

\newtheorem*{definition}{Definition}
\newtheorem*{satz}{Satz}

\author{Karsten Lehmann}
\date{WiSe 2020}
\title{Körperaxiome}

\pagestyle{fancy}
\fancyhf{}
\lhead{\thetitle}
\rhead{\theauthor}
\lfoot{\thedate}
\rfoot{Seite \thepage}

\begin{document}
\section*{Körperaxiome}

\subsection*{Algebraische Strukturen}

\begin{definition}[Abelsche Gruppe]
  \label{def:abel}
  Eine Abelsche Gruppe $(G, *)$ ist ein paar aus einer Menge $G$ und einer Verknüpfung $*$, so dass gilt:
  \begin{itemize}
  \item $(a * b) * c = a * (b * c)$ (Assoziativität)
  \item $a * e = e * a = a$ (Existenz eines neutralen Elementes $e$)
  \item es existiert für jedes $a \in G$ ein $a^{-1} \in G$, so dass $a * a^{-1} = e$
    (Existenz eines Inversen Elementes)
  \item $a * b = b * a$ (Kommutativität)
  \end{itemize}
\end{definition}

\begin{definition}[Ring]
  \label{def:ring}
  Ein Ring $(R, +, \cdot)$ ist eine Menge $R$ mit zwei Verknüpfungen $+$ und $\cdot$, so dass gilt:
  \begin{itemize}
  \item $(R, +)$ ist eine \hyperref[def:abel]{Abelsche Gruppe}
  \item $a \cdot b + a \cdot c = a \cdot (b + c)$ (Distributivität)
  \end{itemize}
\end{definition}

\begin{definition}[Trivialer Ring]
  \label{def:trivring}
  Der Triviale Ring $(\{ 0 \}, +, \cdot)$ ist ein \hyperref[def:ring]{Ring} mit der Zahl Null als
  einzigem Element und den beiden trivialen Verknüpfungen $+ \colon 0 + 0 = 0$ und
  $\cdot \colon 0 \cdot 0 = 0$.
\end{definition}

\begin{definition}[Körper]
  Ein Körper $(K, +, \cdot)$ ist ein \hyperref[def:trivring]{nicht-trivialer Ring}, in dem zusätzlich gilt:
  \begin{itemize}
  \item $a \cdot b = b \cdot a$ (Kommutativität)
  \item $a \cdot e = e \cdot a = a$ (Existenz eines multiplikativen Neutral-/Einzelelementes)
  \item $\forall a \in K, a \ne 0 \colon \exists \; a^{-1} \in K \colon a \cdot a^{-1} = 0$
    (Existenz eines multiplikativen inversen Elementes)
  \end{itemize}
\end{definition}

\pagebreak

\subsection*{Körperaxiome}

Zusammengefasst ergeben sich somit in einem Körper $(K, +, \cdot)$ mit den beiden Verknüpfungen $+$
und $\cdot$ die folgenden Rechenregeln:

\begin{enumerate}[label=(A\arabic*)]
\item $\forall a,b,c \in K \colon (a + b) + c = a + (b + c)$ (Additive Assoziativität)
\item $\exists \; 0 \in K \colon \forall a \in K \colon a + 0 = a$ (Additives neutrales Element)
\item $\forall a \in K \exists \; (-a) \in K \colon a + (-a) = 0$ (Additives Inverses)
\item $\forall a,b \in K \colon a + b = b + a$ (Additive Kommutativität)
\item $\forall a,b,c \in K \colon (a \cdot b) \cdot c = a \cdot (b \cdot c)$ (Multiplikative Assoziativität)
\item $\exists \; 1 \in K \colon \forall a \in K \colon a \cdot 1 = a$ (Multiplikatives neutrales Element)
\item $\forall a \in K, a \ne 0 \: \exists \; a^{-1} \in K \colon a \cdot a^{-1} = 1$ (Multiplikatives Inverses)
\item $\forall a,b \in K \colon a + b = b + a$ (Multiplikative Kommutativität)
\item $\forall a,b,c \in K \colon a \cdot b + a \cdot c = a \cdot (b + c)$ (Distributivität)
\end{enumerate}

Eine Menge $K$, welche mit zwei Verknüpfungen von $K$ nach $K$, $+\colon K \times K \to K$ und
$\cdot\colon K \times K \to K$ die Axiome (A1) bis (A9) erfüllt ist somit ein Körper.

\subsubsection*{Beispiele für Körper}

Unendliche Körper sind zum Beispiel die Reellen Zahlen $\mathbb{R}$ oder die komplexen Zahlen $\mathbb{C}$.

Ein endlicher Körper mit zwei Elementen ist zum Beispiel die Menge $\{ 0, 1\}$ mit folgenden Verknüpfungen:

\begin{itemize}
\item $0 + 0 = 0$
\item $0 + 1 = 1$
\item $1 + 0 = 1$
\item $1 + 1 = 0$
\item $0 \cdot 0 = 0$
\item $1 \cdot 0 = 0$
\item $0 \cdot 1 = 0$
\item $1 \cdot 1 = 1$
\end{itemize}

\pagebreak

\section*{Anordnungsaxiome}

Sei $(K,+,\cdot)$ ein Körper. Falls es eine Teilmenge $P \subseteq K$ gibt, sodass die folgenden 3 Axiome
gelten:

\begin{enumerate}[label=(A\arabic*)]
\setcounter{enumi}{9}
\item $\forall a \in K \colon a = 0 \lor a \in P \lor (-a) \in P$ (Trichotomiegesetz)
\item $\forall a, b \in P \colon a + b \in P$
\item $\forall a, b \in P \colon a \cdot b \in P$
\end{enumerate}

dann heißt $(K,+,\cdot)$ auch \textbf{angeordneter Körper}.

\end{document}