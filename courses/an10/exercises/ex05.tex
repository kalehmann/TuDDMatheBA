\documentclass{scrreprt}

\usepackage{aligned-overset}
\usepackage{amsmath}
\usepackage{amssymb}
\usepackage{bm}
\usepackage[shortlabels]{enumitem}
\usepackage{hyperref}
\usepackage[utf8]{inputenc}
\usepackage{multicol}
\usepackage{mathtools}
\usepackage{physics}
\usepackage{tabularx}
\usepackage{titling}
\usepackage{fancyhdr}
\usepackage{xfrac}
\usepackage{pgfplots}

\pgfplotsset{compat = newest}
\usetikzlibrary{intersections}
\usetikzlibrary{patterns}
\usepgfplotslibrary{fillbetween}

\author{Karsten Lehmann}
\date{WiSe 2021/2022}
\title{Übungsblatt 05\\Analysis - Grundlegende Konzepte}

\setlength{\headheight}{26pt}
\pagestyle{fancy}
\fancyhf{}
\lhead{\thetitle}
\rhead{\theauthor}
\lfoot{\thedate}
\rfoot{Seite \thepage}

\begin{document}
\paragraph{21. Zeigen Sie,} dass die Abbildung
$I \colon \mathbb{Z} \to \overline{\mathbb{Z}}$ gemäß
\[
  I(k) \coloneqq \qty\big[\qty(k, 0)],
  I\qty\big(\qty(-k)) \coloneqq \qty\big[\qty(0, k)],
  k \in \mathbb{N}
\]
ein Isomorphismus ist und die Ordnung erhält, dass heißt für alle
$m, k \in \mathbb{N}$ mit $m \leq k$ gilt $I(m) \leq I(k)$.

\subparagraph{Lsg.} Nach Definition der Vorlesung ist
\[
  \overline{\mathbb{Z}} \coloneqq \qty\Big{\qty\big[\qty(n, n')]
    \: {\Big |} \: n, n' \in \mathbb{N}}
\]

und

\[
  \qty\big[(n, n')] \coloneqq \qty\Big{(a, a') \: {\Big |} \:
    a, a' \in \mathbb{N} \colon \qty\big((n, n'), (a, a')) \in Q}
\]

mit

\[
  Q \coloneqq \qty\Big{\qty\big((n_1, n_1'), (n_2, n_2')) \in
    (\mathbb{N} \times \mathbb{N}) \times (\mathbb{N} \times \mathbb{N})
  \: {\Big |} \: n_1 + n_2' = n_1' + n_2}
\]

also gilt auch kurz
\[
  \qty\big[(n, n')] \coloneqq \qty\Big{(a, a') \in \mathbb{N} \times \mathbb{N}
  \: {\Big |} \: n + a' = n' + a}
\]

Ein Isomorphismus ist eine \emph{bijektive Abbildung}, welche die Struktur
erhält. \\

\underline{Zu zeigen:} ist nun, dass $I$ sowohl \emph{injektiv} als auch
\emph{surjektiv} ist.
\begin{itemize}
\item[$I$ ist injektiv:] Sei $k \in \mathbb{N}$.
  Dann ist
  \begin{flalign*}
    I(k) &= \qty\big[(k, 0)]
    = \qty\big{\qty(n, n') \in \mathbb{N} \times \mathbb{N} \: {\big |} \:
      k + n' = 0 + n} & \\
    &= \qty\big{\qty(n, n') \in \mathbb{N} \times \mathbb{N} \: {\big |} \:
      k = n - n'}
  \end{flalign*}
  \begin{flalign*}
    I((-k)) &= \qty\big[(0, k)] = \qty\big{\qty(n, n') \in \mathbb{N} \times \mathbb{N} \: {\big |} \:
      0 + n' = k + n} & \\
    &= \qty\big{\qty(n, n') \in \mathbb{N} \times \mathbb{N} \: {\big |} \:
      k = n' - n}
  \end{flalign*}
  Insbesondere ist $I(0) = \qty\big{(n, n') \in \mathbb{N} \times \mathbb{N}
    \: {\big |} \: n = n'}$.

  Seien $k, l \in \mathbb{N}$ beliebig.
  Angenommen es gilt
  \begin{itemize}
  \item[$I(k) = I(l)$:] Dann gilt $\qty\big[(k, 0)] = \qty\big[(l, 0)]$.
    Es folgt $\qty\big((k, 0), (l, 0)) \in Q$.

    Also $k + 0 = 0 + l \Rightarrow k = l$

  \item[$I(k) = I((-l))$:]  Dann gilt $\qty\big[(k, 0)] = \qty\big[(0, l)]$.
    Es folgt $\qty\big((k, 0), (0, l)) \in Q$.

    Also $k + l = 0 + 0 \overset{k, l \in \mathbb{N}}\Rightarrow k = l = 0$.
  \item[$I((-k)) = I(l)$:] analog zu $I(k) = I((-l))$
  \item[$I((-k)) = I((-l))$:] Dann gilt $\qty\big[(0, k)] = \qty\big[(0, l)]$.
    Es folgt $\qty\big((0, k), (0, l)) \in Q$.

    Also $0 + k = l + 0 \Rightarrow k = l$
  \end{itemize}
  $\Rightarrow I$ ist injektiv.

\newpage
\item[$I$ ist surjektiv:] Seien $n, n' \in \mathbb{N}$ beliebig.
  Die Abbildung $I$ ist surjektiv, genau dann wenn ein $k \in \mathbb{N}$
  existiert mit
  $I(k) = \qty\big[(n, n')]$ oder $I((-k)) = \qty\big[(n, n')]$.

  \begin{minipage}[t]{.45\textwidth}
    Sei $n \geq n'$.
    Dann ist $\qty\big((k, 0), (n, n')) \in Q$, da
    $k + n' = 0 + n \Rightarrow k = n - n' \in \mathbb{N}$.

    $I(k) = \qty\big[(n, n')]$
  \end{minipage}
  \hfill
  \vrule
  \hfill
  \begin{minipage}[t]{.45\textwidth}
    Sei $n < n'$.
    Dann ist $\qty\big((0, k), (n, n')) \in Q$, da
    $0 + n' = k + n \Rightarrow k = n' - n \in \mathbb{N}$.

    $I((-k)) = \qty\big[(n, n')]$
  \end{minipage}

  $\Rightarrow$ für alle $\qty\big[(n, n')] \in \overline{\mathbb{Z}}$
  existiert ein $k \in \mathbb{N}$ mit $I(k) = \qty\big[(n, n')]$ oder
  $I((-k)) = \qty\big[(n, n')]$.

  $\Rightarrow I$ ist surjektiv.

  $\Rightarrow I$ ist bijektiv. \\

  Weiter ist zu zeigen, dass $I$ ein Isomorphismus bezüglich der Ordnung ist.
  Auf $\overline{\mathbb{Z}}$ ist die Ordnung durch die Relation
  \[
    R \coloneqq \qty\big{\qty\big(\qty[(m, m')], \qty[(n, n')])
      \in \overline{\mathbb{Z}} \times \overline{\mathbb{Z}}
      \: {\Big |} \:
      \qty[(m, m')] \leq \qty[(n, n')]
    }
  \]
  mit
  $\qty\big[(m, m')] \leq \qty\big[(n, n')] \iff m + n' \leq_{\mathbb{N}} m' + n$
  gegeben.

  Seien $k \leq l \in \mathbb{N}$, dann
  \begin{flalign*}
    k &\leq_{\mathbb{N}} l &\\
    k + 0 &\leq_{\mathbb{N}} l + 0 \\
    k + 0 &\leq_{\mathbb{N}} 0 + l
  \end{flalign*}
  $\Rightarrow \qty\big[(k, 0)] \leq_{\overline{\mathbb{Z}}} \qty\big[(l, 0)]$

  $\Rightarrow I(k) \leq_{\overline{\mathbb{Z}}} I(l)$

  $\Rightarrow I$ ist ein Isomorphismus bezüglich der Ordnung.
\end{itemize}

\end{document}