\documentclass{scrreprt}

\usepackage{aligned-overset}
\usepackage{amsmath}
\usepackage{amssymb}
\usepackage{bm}
\usepackage[shortlabels]{enumitem}
\usepackage{hyperref}
\usepackage[utf8]{inputenc}
\usepackage{multicol}
\usepackage{marvosym}
\usepackage{mathtools}
\usepackage{physics}
\usepackage{tabularx}
\usepackage{titling}
\usepackage{fancyhdr}
\usepackage{xfrac}
\usepackage{pgfplots}

\pgfplotsset{compat = newest}
\usetikzlibrary{intersections}
\usetikzlibrary{patterns}
\usepgfplotslibrary{fillbetween}

\author{Karsten Lehmann}
\date{WiSe 2021/2022}
\title{Übungsblatt 10\\Analysis - Grundlegende Konzepte}

\setlength{\headheight}{26pt}
\pagestyle{fancy}
\fancyhf{}
\lhead{\thetitle}
\rhead{\theauthor}
\lfoot{\thedate}
\rfoot{Seite \thepage}

\newcommand\diam{\text{diam}}

\begin{document}
\paragraph{50. Sei $\qty\big(X, d)$ ein metrischer Raum und $M \subset X$
  nicht leer und beschränkt.}
Zeigen Sie, dass
\[
  \diam\qty\big(M) = \diam\qty\big(\overline{M})
\]

\subparagraph{Lsg.} $\diam\qty\big(M)$ ist definiert als
$\sup \qty\big{d\qty\big(x, y) {\big |} x, y \in M}$ und
$\overline{M}$ (oder $\text{cl}M$) ist definiert als $M \cup \partial M$.
Es gilt
\[
  \diam\qty\big(M) = \diam\qty\big(\overline{M}) \iff
  \diam\qty\big(M) \leq \diam\qty\big(\overline{M}) \land
  \diam\qty\big(M) \geq \diam\qty\big(\overline{M})
\]
\begin{itemize}
\item[``$\leq$''] Angenommen $\diam\qty\big(M) > \diam\qty\big(\overline{M})$,
  dann existieren zwei Elemente $x, y \in M$ mit $d\qty\big(x, y) >
  \sup\qty\big{d\qty\big(a, b) {\big |} a, b \in \overline{M}}$.

  $\Rightarrow$ Widerspruch, da $x, y$ dann auch in der Vereinigung
  $M \cup \partial M$ - was $\overline{M}$ entspricht - enthalten sein
  müssen.

  $\Rightarrow$ \underline{$\diam\qty\big(M) \leq \diam\qty\big(\overline{M})$}

\item[``$\geq$''] Seien $x, y \in \overline{M}$ und $\epsilon > 0$ beliebig.
  Dann existieren $a, b \in M$ mit $d\qty\big(x, a) < \frac{\epsilon}{2}$
  und $d\qty\big(y, b) < \frac{\epsilon}{2}$.
  (Anderenfalls hätten $x$ oder $y$ eine Umgebung ohne Punkte von $M$,
  $\Rightarrow x$ oder $y$ sind äußere Punkte von $M$, \Lightning\,
  zu $x, y$ sind Randpunkte von $M$)

  \begin{flalign*}
    d\qty\big(x, y) \overset{\bigtriangleup\text{-Ungl}}
    &\leq d\qty\big(x, a) + d\qty\big(b, y) & \\
    \overset{d\qty\big(a, b) > 0}&\leq
    d\qty\big(x, a) + d\qty\big(b, y) + d\qty\big(a, b) \\
    &\leq d\qty\big(x, a) + d\qty\big(b, y) +
    \underset{= \diam(M)}{\underbrace{\sup\qty\big{d\qty\big(a, b) {\big |} a, b \in M}}} \\
    d\qty\big(x, y)
    \overset{d\qty\big(x, a) = d\qty\big(y, b) = \frac{\epsilon}{2}}&\leq
    \epsilon + \diam\qty\big(M)
  \end{flalign*}
  Da $x, y \in \partial M$  beliebig gewählt sind, folgt
  \[
    \underset{= \diam\qty\big(x, y)}
    {\underbrace{\sup\qty\big{d\qty\big(x, y)} {\big |} x, y \in M}}
    \leq \diam\qty\big(M) + \epsilon
  \]
  Da außerdem $\epsilon$ beliebig gewählt ist, folgt
  \underline{$\diam\qty\big(\overline{M}) \leq \diam\qty\big(M)$}.
\end{itemize}

$\Rightarrow \diam\qty\big(M) = \diam\qty\big(\overline{M})$

\end{document}