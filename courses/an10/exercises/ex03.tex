\documentclass{scrreprt}

\usepackage{aligned-overset}
\usepackage{amsmath}
\usepackage{amssymb}
\usepackage{bm}
\usepackage[shortlabels]{enumitem}
\usepackage{hyperref}
\usepackage[utf8]{inputenc}
\usepackage{multicol}
\usepackage{mathtools}
\usepackage{physics}
\usepackage{tabularx}
\usepackage{titling}
\usepackage{fancyhdr}
\usepackage{xfrac}
\usepackage{pgfplots}

\pgfplotsset{compat = newest}
\usetikzlibrary{intersections}
\usetikzlibrary{patterns}
\usepgfplotslibrary{fillbetween}

\author{Karsten Lehmann}
\date{WiSe 2021/2022}
\title{Übungsblatt 03\\Analysis - Grundlegende Konzepte}

\setlength{\headheight}{26pt}
\pagestyle{fancy}
\fancyhf{}
\lhead{\thetitle}
\rhead{\theauthor}
\lfoot{\thedate}
\rfoot{Seite \thepage}

\begin{document}
\paragraph{11. Es sei $M$ eine Menge.}
Für Mengen $A, B \in \mathcal{P}(M)$, das heißt für $A, B \subset M$ betrachten
wir die Relation
\[
  A \leq B \iff A \subset B
\]
Beweisen Sie, dass $\leq$ eine Ordnungsrelation auf $\mathcal{P}(M)$ ist.

\subparagraph{Lsg.} Eine Relation $R \subset A \times A$ heißt
\emph{Ordnungsrelation} auf $A$, falls für alle $a, b, c \in A$ gilt:
\begin{enumerate}[1)]
\item $(a, a) \in R$
\item $(a, b), (b, a) \in R \Rightarrow a = b$
\item $(a, b), (b, c) \in R \Rightarrow (a, c) \in \mathbb{R}$
\end{enumerate}

Sei nun $R$
\[
  R = \qty\Big{(a, b) \in \mathcal{P}(M) \times \mathcal{P}(M) \: {\Big |} \: a \subset b}
\]
\begin{enumerate}[1)]
\item Erfüllt, da $a \subset a \: \forall \: a \in P(M)$
\item $(a, b) \in R \Rightarrow a \subset b$ und
  $(b, a) \in R \Rightarrow b \subset a$.
  Aus der Vorlesung Lineare Algebra ist bekannt, dass
  \[
    a = b \iff a \subset b \land b \subset a
  \]
\item $(a, b) \in R$ bedeutet, dass jedes Element aus $a$ auch ein Element von
  $b$ ist.
  $(b, c) \in R$ bedeutet, dass jedes Element aus $b$ auch ein Element von $c$
  ist.
  $\Rightarrow$ jedes Element von $a$ ist in $b$ und somit auch in $c$ enthalten.
\end{enumerate}

\subparagraph{12. Es seien $M$ eine nichtleere Menge}, $Q \subset M \times M$ eine
Äquivalenzrelation auf $M$ und für $x \in M$ bezeichne $\qty[x]$ die
Äquivalenzklasse von $x$ bezüglich $Q$.
Zeigen Sie:
\begin{enumerate}[(a)]
\item Für alle $x \in M$ gilt $x \in \qty[x]$
\item Für alle $x, y \in M$ gilt
  \[
    (x, y) \in Q \iff [x] = [y]
  \]
\item Für alle $x, y \in M$ gilt entweder $\qty[x] = \qty[y]$ oder
  $\qty[x] \cap \qty[y] = \emptyset$.
\end{enumerate}

\subparagraph{Lsg.} Die Menge $[x] = \qty\big{y \in M {\big |} (x, y) \in Q}$
heißt \emph{Äquivalenzklasse} von $x \in M$ bezüglich $Q$.
\begin{enumerate}[(a)]
\item $Q$ ist eine Äquivalenzrelation.
  Also gilt für alle $x \in M \colon (x, x) \in Q$.
  Somit gilt auch $x \in \qty[x]$.

\newpage
\item Sei $(x, y) \in Q$.
  \begin{itemize}
  \item[``$\Rightarrow$'']
    Aus der Definition der Äquivalenzklasse folgt
    $y \in \qty[x]$.

    Für jedes $a \in \qty[x]$ gilt $\qty(x, a) \in Q$.
    Mit $(y, x) \in Q$ folgt mit der Transitivität der Äquivalenzrelation
    $\qty(y, a) \in Q \Rightarrow a \in \qty[y] \Rightarrow \qty[x] \subset \qty[y]$

    Für jedes $b \in \qty[y]$ gilt $\qty(y, b) \in Q$.
    Mit $\qty(x, y) \in Q$ folgt mit der Transitivität der Äquivalenzrelation
    $\qty(x, b) \in Q \Rightarrow b \in \qty[x] \Rightarrow \qty[y] \subset \qty[x]$

  \item[``$\Leftarrow$''] Sei $\qty[x] = \qty[y]$.
    Aus (a) folgt $y\in \qty[x]$.
    Nach Definition der Äquivalenzklasse folgt $\qty(x, y) \in Q$.

  \end{itemize}
\item Sei $\qty[x] = \qty[y]$.
  Aus (a) folgt $\qty{x, y} \subset \qty[x] \cap \qty[y]$.

  Sei nun $\qty[x] \ne \qty[y]$.
  Angenommen es existiert $a \in \qty[x] \cap \qty[y]$, dann
  folgt $(x, a), (y, a) \in Q$.
  Mit der Symmetrie und Transitivität der Äquivalenzrelation folgt
  $(x, y) \in Q$.
  Aus (b) folgt $\qty[x] = \qty[y]$ - ein Widerspruch.

  $\Rightarrow \qty[x] \cap \qty[y] = \emptyset \iff \qty[x] \ne \qty[y]$
\end{enumerate}

\end{document}