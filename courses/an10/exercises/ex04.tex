\documentclass{scrreprt}

\usepackage{aligned-overset}
\usepackage{amsmath}
\usepackage{amssymb}
\usepackage{bm}
\usepackage[shortlabels]{enumitem}
\usepackage{hyperref}
\usepackage[utf8]{inputenc}
\usepackage{multicol}
\usepackage{mathtools}
\usepackage{physics}
\usepackage{tabularx}
\usepackage{titling}
\usepackage{fancyhdr}
\usepackage{xfrac}
\usepackage{pgfplots}

\pgfplotsset{compat = newest}
\usetikzlibrary{intersections}
\usetikzlibrary{patterns}
\usepgfplotslibrary{fillbetween}

\author{Karsten Lehmann}
\date{WiSe 2021/2022}
\title{Übungsblatt 04\\Analysis - Grundlegende Konzepte}

\setlength{\headheight}{26pt}
\pagestyle{fancy}
\fancyhf{}
\lhead{\thetitle}
\rhead{\theauthor}
\lfoot{\thedate}
\rfoot{Seite \thepage}

\begin{document}
\paragraph{16. Untersuchen Sie die folgenden Abbildungen auf Injektivität,
  Surjektivität und Bijektivität:}

\begin{enumerate}[(a)]
\item $f \colon \mathbb{N} \to \mathbb{N}, n \mapsto 2n + 5$

  \subparagraph{Lsg.}
  Sei $2x + 5 = 2y + 5$.
  Durch äquivalente Umformung erhält man
  \begin{flalign*}
    2x + 5 &= 2y + 5 && {\Big |} - 5 && &&\\
    2x &= 2y &&{\Big |} :2 \\
    x &= y
  \end{flalign*}
  Somit ist die Abbildung injektiv.

  Sei $y \in \mathbb{N} = 3$.
  Dann ist $f^{-1}(3) = -1$ und $-1 \notin \mathbb{N}$.
  Somit ist $f$ weder surjektiv noch bijektiv.

\item $f \colon \mathbb{N} \to \mathbb{N}, n \mapsto n \cdot n$

  \subparagraph{Lsg.}
  Seien $x,y \in \mathbb{N}$ und $x \cdot x = y \cdot y$.
  Somit ist $x = \pm y$.
  Da $-y \notin \mathbb{N}$ folgt $x = y$.
  Somit ist die Abbildung injektiv.

  Sei nun $x \in \mathbb{N} = 2$.
  Es existiert kein Element $y \in \mathbb{N}$ mit $y \cdot y = 2$, da
  $\sqrt{2} \notin \mathbb{N}$.
  Somit ist $f$ weder surjektiv noch bijektiv.

\item $f \colon \mathbb{N} \times \mathbb{N} \to \mathbb{N}$ mit
  $(m, n) \mapsto n \cdot m$.

  \subparagraph{Lsg.}
  Seien $\qty(x, y) \in \mathbb{N} \times \mathbb{N}$ und $x \ne y$.
  Da die Multiplikation kommutativ ist, folgt $f(x, y) = f(y, x)$.
  Somit ist $f$ nicht injektiv.

  Sei $x \in \mathbb{N}$ beliebig.
  Da $x \cdot 1 = x$, existiert für jedes Element aus der Zielmenge
  ein Paar $\qty(x, 1) \in \mathbb{N} \times \mathbb{N}$ mit $f(x, 1) = x$.
  Somit ist die Abbildung $f$ surjektiv, aber nicht bijektiv.

\item $f \colon \mathcal{P}\qty\big(\qty{\emptyset, \qty{\emptyset},
    \qty{\emptyset, \qty{\emptyset}}}) \to \mathbb{N}$ mit
  $M \mapsto \#(M)$, wobei $\#(M)$ die Anzahl der Elemente in $M$ angibt.

  \subparagraph{Lsg.}
  Die Mengen $\qty{\emptyset}$ und $\qty{\qty{\emptyset}}$ sind Elemente aus
  $\mathcal{P}\qty\big(\qty{\emptyset, \qty{\emptyset}, \qty{\emptyset,
      \qty{\emptyset}}})$.
  Außerdem ist $\#(\qty{\emptyset}) = \#(\qty{\qty{\emptyset}}) = 1$.
  Somit ist $f$ nicht injektiv.

  Die Menge $\qty{\emptyset, \qty{\emptyset}, \qty{\emptyset, \qty{\emptyset}}}$
  enthält $3$ Elemente.
  Demzufolge umfasst auch jedes Element der Potenzmenge dieser Menge maximal $3$
  Elemente.
  Daraus folgend existiert kein Element
  $x \in \mathcal{P}\qty\big(\qty{\emptyset, \qty{\emptyset}, \qty{\emptyset,
      \qty{\emptyset}}})$ mit $f(x) = 4$ oder $f(x) = 5$.

  Damit ist $f$ auch nicht surjektiv und nicht bijektiv.
\end{enumerate}

\newpage
\paragraph{17. Beweisen Sie,} dass $\mathbb{N}$ und
$2\mathbb{N} \coloneqq \qty\big{ 2n {\big |} n \in \mathbb{N}}$ bezüglich der
Addition isomorph sind.
Zeigen Sie außerdem, dass $\mathbb{N}$ und $2\mathbb{N}$ bezüglich der
Multiplikation nicht isomorph sind.

\subparagraph{Lsg.} Ein \emph{Isomorphismus} einer Struktur ist eine
bijektive Abbildung zwischen zwei Mengen, welche die auf beiden Mengen
vorhandene Struktur erhält.

Die Struktur eines Rechenoperators $*$ zwischen zwei Mengen $M$ und $N$
bleibt für eine Abbildung $I \colon M \to N$ erhalten, falls
\begin{itemize}
\item $*_M \colon M \times M \to M$ definiert ist (Operation auf der ersten Menge)
\item $*_N \colon N \times N \to N$ definiert ist (Operation auf der zweiten Menge)
\item $\forall x, y \in M \colon I(x \; *_M \; y) = I(x) \; *_N \; I(y)$ (Strukturerhaltung)
\end{itemize}

Für diese Abbildung heißen die Mengen \emph{isomorph} bezüglich des Operators. \\

\noindent
\underline{Gegeben:}
\[
  I \colon \mathbb{N} \to 2\mathbb{N}, n \mapsto 2n
\]
\underline{Zu zeigen:}
\[
  \forall a, b \in \mathbb{N} \colon I(a + b) = I(a) + I(b)
\]

Seien $a, b \in \mathbb{N}$ beliebig.
\begin{flalign*}
  I(a + b) \overset{\text{Def. von $I$}}&= 2(a + b) & \\
  \overset{\text{Distributivität}}&= 2a + 2b \\
  \overset{\text{Def. von $I$}}&= I(a) + I(b)
\end{flalign*}
Da $a, b$ beliebig sind $\mathbb{N}$ und $2\mathbb{N}$ für die Addition
isomorph. \\

\noindent
Angenommen $\mathbb{N}$ und $2\mathbb{N}$ wären für die Multiplikation isomorph.
Dann müsste für alle $a, b \in \mathbb{N}$ gelten: $I(a \cdot b) = I(a) \cdot I(b)$
\[
  I(a \cdot b) = 2(a \cdot b) \ne 2a \cdot 2b = I(a) \cdot I(b)
\]
Das ist ein Widerspruch, somit sind $N$ und $2\mathbb{N}$ für die Multiplikation
nicht isomorph.
\end{document}