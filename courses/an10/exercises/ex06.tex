\documentclass{scrreprt}

\usepackage{aligned-overset}
\usepackage{amsmath}
\usepackage{amssymb}
\usepackage{bm}
\usepackage[shortlabels]{enumitem}
\usepackage{hyperref}
\usepackage[utf8]{inputenc}
\usepackage{multicol}
\usepackage{mathtools}
\usepackage{physics}
\usepackage{tabularx}
\usepackage{titling}
\usepackage{fancyhdr}
\usepackage{xfrac}
\usepackage{pgfplots}

\pgfplotsset{compat = newest}
\usetikzlibrary{intersections}
\usetikzlibrary{patterns}
\usepgfplotslibrary{fillbetween}

\author{Karsten Lehmann}
\date{WiSe 2021/2022}
\title{Übungsblatt 06\\Analysis - Grundlegende Konzepte}

\setlength{\headheight}{26pt}
\pagestyle{fancy}
\fancyhf{}
\lhead{\thetitle}
\rhead{\theauthor}
\lfoot{\thedate}
\rfoot{Seite \thepage}

\begin{document}
\paragraph{26. Zeigen Sie Folgendes:}
\begin{enumerate}[a)]
\item Es sei $K$ ein angeordneter Körper.
  Für $x, y \in K$ definieren wir
  \[
    \max\qty\big{x, y} \coloneqq
    \begin{cases}
      x & \text{falls } x \geq y \\
      y & \text{falls } x < y \\
    \end{cases}
    \text{ und }
    \min\qty\big{x, y} \coloneqq
    \begin{cases}
      x & \text{falls } x \leq y \\
      y & \text{falls } x > y
    \end{cases}
  \]
  Dann gilt
  \begin{enumerate}[(1)]
  \item $\max\qty\big{x, y} = \frac{1}{2} \qty\big(x + y + \abs{x - y})$
  \item $\min\qty\big{x, y} = \frac{1}{2} \qty\big(x + y - \abs{x - y})$
  \item $\max\qty\big{x, -x} = \abs{x}$
  \end{enumerate}

  \subparagraph{Lsg.}
  \begin{enumerate}[(1)]
  \item
    \begin{minipage}[t]{.4\textwidth}
      Sei $x \geq y$,  dann ist $x - y \geq 0$
      $\Rightarrow \abs{x - y} = x - y$.

      \begin{flalign*}
        \max\qty\big{x, y} &= \frac{1}{2} \qty\big(x + y + \abs{x - y}) \\
        &= \frac{1}{2} \qty\big(x + y + x - y) \\
        &= \frac{1}{2} \cdot 2 \cdot x = x
      \end{flalign*}
    \end{minipage}
    \hfill
    \vrule
    \hfill
    \begin{minipage}[t]{.4\textwidth}
      Sei $x < y$, dann ist $x - y < 0$
      $\Rightarrow \abs{x - y} = -(x - y)$

      \begin{flalign*}
        \max\qty\big{x, y} &= \frac{1}{2} \qty\big(x + y + \abs{x - y}) \\
        &= \frac{1}{2} \qty\big(x + y + (-(x - y))) \\
        &= \frac{1}{2} \cdot 2 \cdot y = y
      \end{flalign*}
    \end{minipage}

  \item
    \begin{minipage}[t]{.4\textwidth}
      Sei $x \leq y$,  dann ist $x - y \leq 0$
      $\Rightarrow \abs{x - y} =  -(x - y)$.

      \begin{flalign*}
        \max\qty\big{x, y} &= \frac{1}{2} \qty\big(x + y - \abs{x - y}) \\
        &= \frac{1}{2} \qty\big(x + y - (-(x - y))) \\
        &= \frac{1}{2} \qty\big(x + y + x - y) \\
        &= \frac{1}{2} \cdot 2 \cdot x = x
      \end{flalign*}
    \end{minipage}
    \hfill
    \vrule
    \hfill
    \begin{minipage}[t]{.4\textwidth}
      Sei $x > y$, dann ist $x - y > 0$
      $\Rightarrow \abs{x - y} = x - y)$

      \begin{flalign*}
        \max\qty\big{x, y} &= \frac{1}{2} \qty\big(x + y - \abs{x - y}) \\
        &= \frac{1}{2} \qty\big(x + y - (x - y)) \\
        &= \frac{1}{2} \cdot 2 \cdot y = y
      \end{flalign*}
    \end{minipage}

  \item
    \begin{minipage}[t]{.4\textwidth}
      Sei $x < 0$,  dann ist $\max\qty\big{x, -x} = -x$
      und $\abs{x} = -x$

      $\Rightarrow \max\qty\big{x, -x} = \abs{x}$.
    \end{minipage}
    \hfill
    \vrule
    \hfill
    \begin{minipage}[t]{.4\textwidth}
      Sei $x \geq 0$,  dann ist $\max\qty\big{x, -x} = x$
      und $\abs{x} = x$

      $\Rightarrow \max\qty\big{x, -x} = \abs{x}$.
    \end{minipage}
  \end{enumerate}
\end{enumerate}
\end{document}