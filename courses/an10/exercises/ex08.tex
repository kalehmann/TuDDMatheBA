\documentclass{scrreprt}

\usepackage{aligned-overset}
\usepackage{amsmath}
\usepackage{amssymb}
\usepackage{bm}
\usepackage[shortlabels]{enumitem}
\usepackage{hyperref}
\usepackage[utf8]{inputenc}
\usepackage{multicol}
\usepackage{mathtools}
\usepackage{physics}
\usepackage{tabularx}
\usepackage{titling}
\usepackage{fancyhdr}
\usepackage{xfrac}
\usepackage{pgfplots}

\pgfplotsset{compat = newest}
\usetikzlibrary{intersections}
\usetikzlibrary{patterns}
\usepgfplotslibrary{fillbetween}

\author{Karsten Lehmann}
\date{WiSe 2021/2022}
\title{Übungsblatt 08\\Analysis - Grundlegende Konzepte}

\setlength{\headheight}{26pt}
\pagestyle{fancy}
\fancyhf{}
\lhead{\thetitle}
\rhead{\theauthor}
\lfoot{\thedate}
\rfoot{Seite \thepage}

\begin{document}
\paragraph{37. Entscheiden Sie,} welche der folgenden Teilmengen von
$\mathbb{R}$ offen und welche abgeschlossen sind.
\begin{multicols}{3}
  \begin{enumerate}[(a)]
  \item $\emptyset$
  \item $\mathbb{R}$
  \item $\qty\big(0, 1) \cup \qty\big[2, 3]$
  \item $\qty\big(0, 1) \cup {\big [}1, 3{\big )}$
  \item $\qty{\frac{1}{n} \: \middle | \: n \in \mathbb{N}_{> 0}}$
  \item $\qty{\frac{1}{n} \: \middle | \: n \in \mathbb{N}_{> 0}}
    \cup \qty\big{0}$
  \end{enumerate}
\end{multicols}

\subparagraph{Lsg.} Eine Menge $M \subseteq \mathbb{R}$ heißt offen, falls
für jedes Element $x \in M$ ein $r > 0$ existiert, so dass die Kugel um $x$
mit dem Radius $r$ ein Teilmenge von $M$ ist
($B_r\qty\big(x) \subseteq M$).
Dabei ist $B_r\qty\big(x)$ definiert als
$\qty\big{y \in M {\big |} d(x, y) < r}$ und für $\mathbb{R}$ ist die
Standardmetrik definiert als $d\qty\big(x, y) = \abs{x - y}$.

Weiterhin heißt eine Menge $M$ abgeschlossen, falls deren Komplement
$\mathbb{R} \setminus M$ offen ist.

\begin{enumerate}[(a)]
\item die leere Menge ist offen, da sie kein Element besitzt für welches die
  Bedingung für eine offene Menge nicht gilt.

  Allerdings ist $\mathbb{R} \setminus \emptyset = \mathbb{R}$ ebenfalls offen,
  da für alle $x \in \mathbb{R}$ ist zum Beispiel \\
  $B_1\qty\big(x) \subset \mathbb{R}$.

  $\Rightarrow \emptyset$ ist offen und abgeschlossen zugleich.

\item $\mathbb{R}$ ist wie schon in (a) gezeigt offen.
  Allerdings ist wie ebenfalls schon in (a) gezeigt auch das Komplement
  $\mathbb{R} \setminus \mathbb{R} = \emptyset$ offen.

  $\Rightarrow \mathbb{R}$ ist offen und abgeschlossen zugleich.

\item Sei $x = 3$.
  Dann gilt für jedes $r > 0$, dass $x + \frac{r}{2} \in B_r(x)$, da
  $d(x, x + \frac{r}{2}) = \abs{x - x + \frac{r}{2}} = \frac{r}{2} < r$.
  Allerdings ist $x + \frac{r}{2} \notin \qty\big(0, 1) \cup \qty\big[2, 3]$.

  $\Rightarrow B_r\qty\big(x) \nsubseteq \qty\big(0, 1) \cup \qty\big[2, 3]$

  $\Rightarrow \qty\big(0, 1) \cup \qty\big[2, 3]$ ist nicht offen.

  Das Komplement von $\qty\big(0, 1) \cup \qty\big[2, 3]$ ist
  $\mathbb{R} \setminus \qty\big(0, 1) \cup \qty\big[2, 3] =
  {\big (}-\infty, 0{\big ]} \cup {\big [}1, 2 {\big )} \cup
  \qty\big(3, \infty)$

  Sei nun $x = 0$.
  Dann gilt für jedes $r > 0$, dass $\frac{r}{2} \in B_r(x)$.
  Allerdings ist $\frac{r}{2} \notin {\big (}-\infty, 0{\big ]} \cup
  {\big [}1, 2 {\big )} \cup \qty\big(3, \infty)$.

  $\Rightarrow$ das Komplement von $\qty\big(0, 1) \cup \qty\big[2, 3]$ ist
  nicht offen.

  $\Rightarrow \qty\big(0, 1) \cup \qty\big[2, 3]$ ist weder offen noch
  abgeschlossen.

\item $\qty\big(0, 1) \cup {\big [}1, 3 {\big )} = \qty\big(0, 3)$.
  Sei nun $x \in \qty\big(0, 3)$ beliebig und
  $r = \min\qty{\frac{x}{2}, \frac{3 - x}{2}}$.
  Dann ist $r > 0$ und \\
  \begin{minipage}[t]{.4\textwidth}
    Fall 1: $x < \frac{3}{2}$
    \begin{flalign*}
      B_{\sfrac{x}{2}}\qty\big(x) &=
      \qty{y \in \mathbb{R} \middle | \abs{x - y} < \frac{x}{2}} \\
      &= \qty(\frac{x}{2}, \frac{3x}{2}) \subset \qty\big(0, 3)
    \end{flalign*}
  \end{minipage}
  \hfill
  \vrule
  \hfill
  \begin{minipage}[t]{.4\textwidth}
    Fall 2: $x \geq \frac{3}{2}$
    \begin{flalign*}
      B_{\frac{3 - x}{2}} \qty\big(x) &=
      \qty{y \in \mathbb{R} \middle | \abs{x - y} < \frac{3 - x}{2}} \\
      &= \qty(\frac{3x - 3}{2}, \frac{x + 3}{2}) \subset \qty\big(0, 3)
    \end{flalign*}
  \end{minipage}
  $\Rightarrow \qty\big(0, 3)$ ist offen.

  Das Komplement $\mathbb{R} \setminus \qty\big(0, 3) =
  {\big (} -\infty, 0 {\big ]} \cup {\big [} 3, \infty {\big )}$ ist nicht
  offen (da zum Beispiel für $x = 3$ kein $r > 0$ mit $B_r\qty(x) \subseteq
  {\big (} -\infty, 0 {\big ]} \cup {\big [} 3, \infty {\big )}$ existiert).

  $\Rightarrow \qty\big(0, 3)$ ist nicht abgeschlossen.

\item $\qty{\frac{1}{n} \: \middle | \: n \in \mathbb{N}_{>0}}$ ist nicht offen,
  da $B_r\qty\big(x)$ für $x = 1$ und jedes $r > 0$ Elemente zwischen
  $1$ und dem nächste Element $\frac{1}{2}$ enthält.

  In dem Komplement
  $\mathbb{R} \setminus \qty{\frac{1}{n} \: \middle | \: n \in \mathbb{N}_{>0}}$
  ist auch die $0$ enthalten.
  Nun findet sich für jedes $r > 0$ ein $n \in \mathbb{N}_{> 0}$ mit
  $\frac{1}{n} \in B_r\qty\big(0)$.

  $\Rightarrow B_r\qty\big(0)$ ist für jedes $r > 0$ nicht in dem
  Komplement von $\qty{\frac{1}{n} \: \middle | \: n \in \mathbb{N}_{>0}}$
  enthalten.

  $\Rightarrow \qty{\frac{1}{n} \: \middle | \: n \in \mathbb{N}_{>0}}$ ist
  nicht abgeschlossen.

\item $\qty{\frac{1}{n} \: \middle | \: n \in \mathbb{N}_{>0}} \cup \qty\big{0}$
  ist nicht offen (siehe (f)).

  Das Komplement von
  $\qty{\frac{1}{n} \: \middle | \: n \in \mathbb{N}_{>0}} \cup \qty\big{0}$
  hat die Form
  \[
    \qty\big(1, \infty) \cup \qty(1, \frac{1}{2}) \cup
    \qty(\frac{1}{2}, \frac{1}{3}) \cup \ldots
    \cup \qty(-\infty, 0)
  \]
  - ist also als Vereinigung offener Intervalle ebenfalls offen.

  $\Rightarrow \qty{\frac{1}{n} \: \middle | \: n \in \mathbb{N}_{>0}}
  \cup \qty\big{0}$ ist abgeschlossen.
\end{enumerate}

Entscheiden Sie des Weiteren, welche der folgenden Teilmengen von $\mathbb{R}^2$
offen und welche abgeschlossen sind.
\begin{multicols}{3}
  \begin{enumerate}[(a)]
  \item $\qty\big(0, 1)^2$
  \item $\qty\big[0, 1] \times \qty\big{0}$
  \item $\qty\big[0, 1]^2$
  \item $\overset{\infty}{\underset{n = 1}\bigcup} \qty\big(n, n + 1)^2$
  \item $\qty\big(0, 1) \times \qty{0}$
  \item $\mathbb{R}^2 \setminus \qty\big(0, 1)^2$
  \end{enumerate}
\end{multicols}
Bestimmen Sie außerdem das Innere, die Randpunkte und die
Häufungspunkte der jeweiligen Mengen.

\subparagraph{Lsg.} \:\\

\begin{tabular}{l|c|c|c|c|c}
  & Offen? & Abgeschlossen? & Inneres & Randpunkte & Häufungspunkte \\
  \hline
  (a) & Ja & Nein & $\qty\big(0, 1)^2$ &
    $\qty\big[0, 1]^2 \setminus \qty\big(0, 1)^2$ & $\qty\big[0, 1]^2$ \\
  (b) & Nein & Ja & $\emptyset$ & $\qty\big[0, 1] \times \qty\big{0}$ &
    $\emptyset$ \\
  (c) & Nein & Ja & $\qty\big(0, 1)^2$ &
    $\qty\big[0, 1]^2 \setminus \qty\big(0, 1)^2$ & $\qty\big[0, 1]^2$ \\
  (d) & Ja & Nein &
    $\overset{\infty}{\underset{n = 1}\bigcup} \qty\big(n, n + 1)^2$ &
    $\overset{\infty}{\underset{n = 1}\bigcup} \qty\big[n, n + 1]^2
      \setminus \overset{\infty}{\underset{n = 1}\bigcup} \qty\big(n, n + 1)^2$ &
    $\overset{\infty}{\underset{n = 1}\bigcup} \qty\big[n, n + 1]^2$\\
  (e) & Nein & Nein & $\emptyset$ &
    $\qty\big[0, 1] \times \qty\big{0}$ & $\emptyset$ \\
  (f) & Nein & Ja & $\mathbb{R}^2 \setminus \qty\big[0, 1]^2$ &
    $\qty\big[0, 1]^2 \setminus \qty\big(0, 1)^2$ &
    $\mathbb{R}^2 \setminus \qty\big(0, 1)^2$
\end{tabular}

\end{document}