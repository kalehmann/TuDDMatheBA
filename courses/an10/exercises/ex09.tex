\documentclass{scrreprt}

\usepackage{aligned-overset}
\usepackage{amsmath}
\usepackage{amssymb}
\usepackage{bm}
\usepackage[shortlabels]{enumitem}
\usepackage{hyperref}
\usepackage[utf8]{inputenc}
\usepackage{multicol}
\usepackage{mathtools}
\usepackage{physics}
\usepackage{tabularx}
\usepackage{titling}
\usepackage{fancyhdr}
\usepackage{xfrac}
\usepackage{pgfplots}

\pgfplotsset{compat = newest}
\usetikzlibrary{intersections}
\usetikzlibrary{patterns}
\usepgfplotslibrary{fillbetween}

\author{Karsten Lehmann}
\date{WiSe 2021/2022}
\title{Übungsblatt 09\\Analysis - Grundlegende Konzepte}

\setlength{\headheight}{26pt}
\pagestyle{fancy}
\fancyhf{}
\lhead{\thetitle}
\rhead{\theauthor}
\lfoot{\thedate}
\rfoot{Seite \thepage}

\begin{document}
\paragraph{41. Sei $\qty\big{a_n}_{n \in \mathbb{N}}$ eine Folge in
  $\mathbb{R}$, $a \in \mathbb{R}$.}
Beweisen Sie die Äquivalenz folgender Aussagen:
\begin{enumerate}[(i)]
\item $\qty\big{a_n}_{n \in \mathbb{N}}$ konvergiert gegen $a$.
\item $\qty\big{a_n}_{n \in \mathbb{N}}$ ist beschränkt und hat $a$ als einzigen
  Häufungswert.
\item jede Teilfolge von $\qty\big{a_n}_{n \in \mathbb{N}}$ hat eine Teilfolge,
  die gegen $a$ konvergiert.
\end{enumerate}

\subparagraph{Lsg.}
\begin{itemize}
\item[``(i) $\Rightarrow$ (ii)''] Sei $\qty\big{a_n}_{n \in \mathbb{N}}$ eine
  konvergente Folge in $\mathbb{R}$, nach Satz 9.7 der Vorlesung
  (\emph{``Sei $\qty\big(X, d)$ ein metrischer Raum, dann ist jede konvergente
    Folge $\qty\big{x_n}$ beschränkt''}) gilt:
  \underline{$\qty\big{a_n}$ ist beschränkt}.

  Weiterhin ist nach Folgerung 9.1 der Vorlesung (\emph{``Für jede Folge
    $\qty\big{x_n}$ gilt $x = \underset{n \to \infty}\lim x_n \iff$ jede Kugel
    $B_{\epsilon}\qty\big(x)$ enthält fast alle $x_n$''}) gilt:
  \underline{$a$ ist ein Häufungswert von $\qty\big{a_n}$}.

  Angenommen $b \ne a$ wäre nun ein weiterer Häufungswert von $\qty\big{a_n}$
  und $\epsilon = \frac{d\qty\big(a, b)}{3}$.
  Da $\qty\big{a_n}$ gegen $a$ konvergiert, existiert $n_0 \in \mathbb{N}$
  so, dass für alle $n > n_0$ gilt $d\qty\big(x_n, a) < \epsilon$.
  Angenommen es gelte nun für ein $m > n_0$, dass
  $d\qty\big(x_m, b) < \epsilon$, dann
  \begin{flalign*}
    d\qty\big(a, b) < d\qty\big(x_m, a) + d\qty\big(x_m, b) &< 2\epsilon \\
    d\qty\big(a, b) < \frac{2}{3}d\qty\big(a, b)
  \end{flalign*}
  $\Rightarrow$ ein Widerspruch, dass heißt es existiert für jedes
  $\epsilon > 0$ ein $n_0 \in \mathbb{N}$, so dass alle $x_n$ mit $n > n_0$
  nicht in der Umgebung von $b$ liegen.

  $\Rightarrow$ es liegen nur endlich viele $x_n$ in Umgebung von $b$

  $\Rightarrow$ \underline{$a$ ist einziger Häufungswert von $\qty\big{a_n}$}.

\item[``(ii) $\Rightarrow$ (i)''] $\qty\big{a_n}_{n \in \mathbb{N}}$ ist
  beschränkt und hat $a$ als einzigen Häufungswert.

  Angenommen $\qty\big{a_n}$ würde nicht gegen $a$ konvergieren, dann existiert
  für alle $\epsilon > 0$ und jedes $n_0 \in \mathbb{N}$ ein $k > n_0$, so dass
  $\abs\big{a_k - a } > \epsilon$.
  Sei nun $\qty\big{a_k}$ die Folge dieser Elemente für ein beliebiges
  $\epsilon$.
  Dann ist $\qty\big{a_k}$ als Teilfolge der beschränkten Folge $\qty\big{a_n}$
  ebenfalls beschränkt.
  Nach Theorem 9.29 (\emph{``Bolzano-Weierstraß  - jede beschränkte Folge in
    $\mathbb{R}$ hat mindestens eine konvergente Teilfolge.''}) besitzt
  $\qty\big{a_k}$ eine konvergente Teilfolge $\qty\big{a_{k_l}}_l$ mit
  $a_{k_l} \overset{l \to \infty}\longrightarrow b$.

  Es folgt aus Satz 9.12 der Vorlesung (\emph{``$\gamma$ ist Häufungswert der
    Folge $\qty\big{x_n} \iff$ es gibt eine Teilfolge $\qty\big{x_{n_k}}_k$ mit
    $x_{n_k} \overset{k \to \infty}\longrightarrow \gamma$''}), dass
  $b$ ein Häufungswert von $\qty\big{a_k}$ und $\qty\big{a_n}$ ist.
  Per Definition von $\qty\big{a_k}$ liegt kein Element der Folge in
  der Umgebung $B_{\frac{\epsilon}{2}}(a)$, es folgt $b \ne a$

  $\Rightarrow$ Widerspruch zu $a$ ist einziger Häufungswert.

  $\Rightarrow$ \underline{$\qty\big{a_n}$ konvergiert gegen $a$}.

\item[``(ii) $\Rightarrow$ (iii)''] Sei $\qty\big{a_{n_k}}_{k \in \mathbb{N}}$
  eine beliebige Teilfolge von $\qty\big{a_n}$.
  Als Teilfolge einer beschränkten Folge ist $\qty\big{a_{n_k}}$ beschränkt.

  Nach Bolzano-Weierstraß hat $\qty\big{a_{n_k}}$ eine konvergente Teilfolge.
  Diese wird im folgenden als $\qty\big{b_n}$ mit
  $b_n \overset{n \to \infty}\longrightarrow b$ bezeichnet.

  Angenommen $b \ne a$.
  Nach Satz 9.19 der Vorlesung ist $b$ als Grenzwert der Teilfolge
  $\qty\big{b_n}$ ein Häufungswert von $\qty\big{a_n}$.

  $\Rightarrow$ Widerspruch zu der Voraussetzung ``$a$ ist einziger
  Häufungswert''

  $\Rightarrow b = a$

  $\Rightarrow$ \underline{jede beliebige Teilfolge von $\qty\big{a_n}$ besitzt
    mindestens eine Teilfolge, die gegen $a$ konvergiert.}

\item[``(iii) $\Rightarrow$ (ii)''] Jede Teilfolge einer Teilfolge von
  $\qty\big{a_n}_{n \in \mathbb{N}}$ ist auch eine Teilfolge von
  $\qty\big{a_n}_{n \in \mathbb{N}}$.
  Nach Voraussetzung existiert somit eine Teilfolge von
  $\qty\big{a_n}_{n \in \mathbb{N}}$, die gegen $a$ konvergiert.
  Aus Satz 9.12 der Vorlesung folgt, dass $a$ ein Häufungswert von
  $\qty\big{a_n}_{n \in \mathbb{N}}$ ist.

  Angenommen $\qty\big{a_n}$ hätte nun einen weiteren Häufungswert $b \ne a$,
  dann existiert eine Teilfolge $\qty\big{a_{n_k}}_{k \in \mathbb{N}}$ mit
  $a_{n_k} \overset{k \to \infty}\longrightarrow b$.

  Da jede Folge eine Teilfolge von sich selbst ist, müsste
  $\qty\big{a_{n_k}}_{k \in \mathbb{N}}$ als Teilfolge einer Teilfolge
  von $\qty\big{a_n}_{n \in \mathbb{N}}$ somit gleichzeitig gegen
  $a \ne b$ konvergieren - ein Widerspruch zur Eindeutigkeit des
  Grenzwertes.

  $\Rightarrow$ \underline{$\qty\big{a_n}_{n \in \mathbb{N}}$ hat $a$ als
    einzigen Häufungswert}.

  Angenommen $\qty\big{a_n}_{n \in \mathbb{N}}$ wäre nicht beschränkt.
  Sei dann $\qty\big{a_{n_k}}_{k \in \mathbb{N}}$ als Teilfolge so gewählt,
  dass $a_{n_1} = a_1$ und $d\qty\big(a_{n_{k + 1}}, a) > d\qty\big(a_{n_k}, a)$.
  Angenommen es existiert für ein $a_{n_k}$ kein Nachfolger, dann wären $a_{n_k}$
  und $-\qty\big(a_{n_k})$ Schranken von $\qty\big{a_n}$ - ein Widerspruch zur
  Annahme ``$\qty\big{a_n}_{n \in \mathbb{N}}$ ist unbeschränkt''.

  Sei nun $r = \frac{d\qty\big(a_1, a)}{2}$.
  Dann enthält $B_r\qty\big(a)$ per Definition von
  $\qty\big{a_{n_k}}_{k \in \mathbb{N}}$ kein Element aus
  der Folge.

  $\Rightarrow$ kein Element aus $\qty\big{a_{n_k}}_{k \in \mathbb{N}}$ liegt
  in der Umgebung von $a$

  $\Rightarrow$ kein Element einer Teilfolge von
  $\qty\big{a_{n_k}}_{k \in \mathbb{N}}$ liegt in der Umgebung von $a$

  $\Rightarrow$ keine Teilfolge von $\qty\big{a_{n_k}}_{k \in \mathbb{N}}$
  konvergiert gegen $a$.

  $\Rightarrow$ Widerspruch zur Annahme \emph{``jede Teilfolge von
    $\qty\big{a_n}_{n \in \mathbb{N}}$ hat eine Teilfolge, die gegen $a$
    konvergiert.''}

  $\Rightarrow$ \underline{$\qty\big{a_n}_{n \in \mathbb{N}}$ ist beschränkt.}
\end{itemize}

\end{document}