\documentclass{scrreprt}

\usepackage{aligned-overset}
\usepackage{amsmath}
\usepackage{amssymb}
\usepackage{bm}
\usepackage[shortlabels]{enumitem}
\usepackage{hyperref}
\usepackage[utf8]{inputenc}
\usepackage{multicol}
\usepackage{mathtools}
\usepackage{physics}
\usepackage{tabularx}
\usepackage{titling}
\usepackage{fancyhdr}
\usepackage{xfrac}
\usepackage{pgfplots}

\pgfplotsset{compat = newest}
\usetikzlibrary{intersections}
\usetikzlibrary{patterns}
\usepgfplotslibrary{fillbetween}

\author{Karsten Lehmann}
\date{WiSe 2021/2022}
\title{Übungsblatt 02\\Analysis - Grundlegende Konzepte}

\setlength{\headheight}{26pt}
\pagestyle{fancy}
\fancyhf{}
\lhead{\thetitle}
\rhead{\theauthor}
\lfoot{\thedate}
\rfoot{Seite \thepage}

\begin{document}
\paragraph{6. Es werden die folgenden Aussagen betrachtet:}
\begin{enumerate}[label={(A\arabic*)}]
\item ``Jede:r Studierende kann beim Hören jedes Musikstücks mindestens
  eine Hausaufgabe lösen.''
\item ``Es gibt (mindestens) eine:n Studierende:n und (mindestens) ein
  Musikstück, sodass diese:r Studierende beim Hören dieses Stückes alle
  Aufgaben lösen kann.''
\end{enumerate}

Drücken Sie die beiden Aussagen mit den Quantoren $\forall$ und $\exists$ aus.
Beschreiben Sie dazu die Studierenden, Musikstücke und Hausaufgaben durch Mengen.
Negieren Sie anschließend die beiden Aussagen.
Dabei soll die Negation von (A1) nicht mit ``Nicht zu jedem'' beginnen,
und die Negation von (A2) nicht mit ``Es gibt keine''.
Formulieren Sie im Anschluss die negierten Aussagen wieder in natürlicher
Sprache.

\subparagraph{Lsg.} Sei $S$ die Menge der Schüler, $M$ die Menge der Musikstücke
und $H$ die Menge der Hausaufgaben.
Weiterhin beschreibt die Aussage $P(s, m, h)$, dass der Schüler $s$
beim Hören des Musikstückes $m$ die Hausaufgabe $h$ lösen kann.

\begin{enumerate}[label={(A\arabic*})]
\item $\forall \, s \in S \colon \forall \, m \in M \colon \exists \, h \in H \colon P(s, m, h)$

  \textbf{Negation}: $\exists \, s \in S \colon \exists \, m \in M \colon \forall h \in H \colon \neg P(s, m, h)$

  ``Es gibt mindestens einen Schüler, für den mindestens ein Musikstück
  existiert, bei dem der Schüler nicht in der Lage ist, irgendeine Hausaufgabe
  aus dem Kurs zu lösen.''

\item $\exists \, s \in S \colon \exists \, m \in M \colon \forall \, h \in H \colon P(s, m, h)$

  \textbf{Negation}: $\forall \, s \in S \exists \, h \in H \colon \colon \forall \, m \in M \colon \neg P(s, m, h)$

  ``Für alle Schüler existiert (mindestens) eine Hausaufgabe, die der Schüler
  beim Hören von Musik (egal welches Stück) nicht lösen kann.''
\end{enumerate}

\paragraph{7. Sei $A$ eine Menge von Parteien, $B$ eine Menge von Wahlversprechen
  und $C$ die Menge der Wahlen.}
Weiterhin ist $P(a, b, c)$ die Aussage
\begin{center}
  ``Partei $a$ löst das Wahlversprechen $b$ zur Wahl $c$ nicht ein.'' 
\end{center}

Diskutieren Sie den Unterschied der folgenden Aussagen:
\renewcommand{\theequation}{\arabic{equation}}
\begin{align}
  \forall \, a \in A \colon \exists \, b \in B \colon \forall \, c \in C &\colon P(a, b, c) \\
  \forall \, a \in A \colon \forall \, c \in C \colon \exists \, b \in B &\colon P(a, b, c)
\end{align}

\subparagraph{Lsg.}
Die erste Aussage besagt, dass jede Partei mindestens ein Wahlversprechen hat,
welches noch bei keiner Wahl eingehalten wurde.
Die zweite Aussage hingegen besagt, dass jede Partei bei jeder Wahl mindestens
ein Wahlversprechen bricht.

\end{document}