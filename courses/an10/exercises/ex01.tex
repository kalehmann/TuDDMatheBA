\documentclass{scrreprt}

\usepackage{aligned-overset}
\usepackage{amsmath}
\usepackage{amssymb}
\usepackage{bm}
\usepackage[shortlabels]{enumitem}
\usepackage{hyperref}
\usepackage[utf8]{inputenc}
\usepackage{multicol}
\usepackage{mathtools}
\usepackage{physics}
\usepackage{tabularx}
\usepackage{titling}
\usepackage{fancyhdr}
\usepackage{xfrac}
\usepackage{pgfplots}

\pgfplotsset{compat = newest}
\usetikzlibrary{intersections}
\usetikzlibrary{patterns}
\usepgfplotslibrary{fillbetween}

\author{Karsten Lehmann}
\date{WiSe 2021/2022}
\title{Übungsblatt 01\\Analysis - Grundlegende Konzepte}

\setlength{\headheight}{26pt}
\pagestyle{fancy}
\fancyhf{}
\lhead{\thetitle}
\rhead{\theauthor}
\lfoot{\thedate}
\rfoot{Seite \thepage}

\begin{document}

\paragraph{1. Welche der folgenden Aussagen uber Element- und Teilmengenbeziehungen sind wahr?}

\begin{enumerate}[a)]
\item $a \in \qty{a, b}$

  \subparagraph{Lsg.} Diese Aussage ist wahr, das Element $a$ ist in der Menge
  mit den Elementen $a$ und $b$ enthalten.

\item $\qty{b, a, \emptyset} \subset \qty{a, b}$

  \subparagraph{Lsg.} Man sagt, dass eine Menge $N$ eine Teilmenge von der Menge
  $M$ ist, falls für jedes $n$ in $N$ gilt: $n \in M$.
  Die Leere Menge ($\emptyset$) ist zwar eine Teilmenge von $\qty{a, b}$ -
  allerdings kein Element aus $\qty{a, b}$.
  Somit ist diese Aussage falsch.

\item $\emptyset \in \emptyset$

  \subparagraph{Lsg.} Diese Aussage ist falsch.
  Die leere Menge ist zwar eine Teilmenge der leeren Menge, allerdings kein
  Element der leeren Menge.

\item $\emptyset \in \qty{a, b}$

  \subparagraph{Lsg.} Diese Aussage ist falsch.
  Die leere Menge ist zwar eine Teilmenge der Menge $\qty{a, b}$, allerdings kein
  Element der Menge $\qty{a, b}$.

\item $\emptyset \subset \qty{b} \in \qty\big{\qty{a}, \qty{b}}$

  \subparagraph{Lsg.} Diese Aussage ist wahr.
  Die leere Menge ist Teilmenge jeder anderen Menge und die Menge mit dem
  Element $b$ ist auch in $\qty\big{\qty{a}, \qty{b}}$ enthalten.

\item $\qty{a} \subset \qty\big{\qty{a}, \emptyset}$

  \subparagraph{Lsg.} Diese Aussage ist falsch, da das Element $a$ nicht in
  der Menge $\qty\big{\qty{a}, \emptyset}$ enthalten ist.
\end{enumerate}

\end{document}