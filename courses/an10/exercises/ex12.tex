\documentclass{scrreprt}

\usepackage{aligned-overset}
\usepackage{amsmath}
\usepackage{amssymb}
\usepackage{bm}
\usepackage[shortlabels]{enumitem}
\usepackage{hyperref}
\usepackage[utf8]{inputenc}
\usepackage{multicol}
\usepackage{marvosym}
\usepackage{mathtools}
\usepackage{pdflscape}
\usepackage{physics}
\usepackage{tabularx}
\usepackage{titling}
\usepackage{fancyhdr}
\usepackage{xfrac}
\usepackage{pgfplots}

\pgfplotsset{compat = newest}
\usetikzlibrary{intersections}
\usetikzlibrary{patterns}
\usepgfplotslibrary{fillbetween}

\author{Karsten Lehmann}
\date{WiSe 2021/2022}
\title{Übungsblatt 12\\Analysis - Grundlegende Konzepte}

\setlength{\headheight}{26pt}
\pagestyle{fancy}
\fancyhf{}
\lhead{\thetitle}
\rhead{\theauthor}
\lfoot{\thedate}
\rfoot{Seite \thepage}

\newcommand\diam{\text{diam}}

\begin{document}
\paragraph{61. Sei $M \subset \mathbb{R}$} und $f \colon M \to \mathbb{R}$
eine streng monoton wachsende Funktion.
Zeigen Sie, dass die Umkehrfunktion $f^{-1} \colon \mathcal{R}\qty\big(f) \to M$
existiert und streng monoton ist.

\subparagraph{Lsg.} Eine Funktion $f \colon M \to \mathbb{R}$ heißt streng
monoton wachsend, falls für alle $x < y \in M$ gilt, dass
$f\qty\big(x) < f\qty\big(y)$.
Damit eine Funktion $f$ eine Umkehrfunktion besitzt muss die Funktion $f$
bijektiv sein.
Angenommen $f$ wäre nicht bijektiv, dann ist $f$ nicht injektiv und/oder nicht
surjektiv.

\begin{itemize}
\item Angenommen $f$ wäre nicht injektiv, dann existiert $x \ne y$ mit
  $f\qty\big(x) = f\qty\big(y)$.

  Aus $x \ne y$ folgt $x > y$ oder $x < y$.
  Da $f$ streng monoton wachsend folgt $f\qty\big(x) < f\qty\big(y)$
  oder $f\qty\big(x) > f\qty\big(y)$ - ein Widerspruch zu
  $f\qty\big(x) = f\qty\big(y)$.

  $\Rightarrow$ \underline{$f$ ist injektiv.}

\item Angenommen $f$ wäre nicht surjektiv, dann existiert ein
  $y \in \mathcal{P}\qty\big(f)$, so dass für alle $x \in M$ gilt:
  $f\qty\big(x) \ne y$.

  $\Rightarrow$ Widerspruch zur Definition von $\mathcal{R}\qty\big(f)$
  (des Wertebereiches).

  $\Rightarrow$ \underline{$f$ ist surjektiv.}
\end{itemize}
$\Rightarrow$ $f$ ist bijektiv. \\
$\Rightarrow$ \underline{Die Umkehrfunktion von $f$ existiert.} \\
Angenommen $f^{-1}$ wäre nicht streng monoton, dann existieren
$x < y < z\in \mathcal{R}\qty\big(f)$, so dass
$f^{-1}\qty\big(x) \leq f^{-1}\qty\big(y)$ und
$f^{-1}\qty\big(z) \leq f^{-1}\qty\big(y)$.
Seien nun $a = f^{-1}\qty\big(x), b = f^{-1}\qty\big(y)$ und
$c = f^{-1}\qty\big(z)$. \\
Dann ist $a \leq b$ und $c \leq b$, aber
$f\qty\big(a) < f\qty\big(b) <f\qty\big(c)$ - ein Widerspruch zu
``$f$ ist streng monoton wachsend''.

$\Rightarrow$ \underline{$f^{-1}$ ist streng monoton.}

\end{document}