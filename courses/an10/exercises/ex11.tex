\documentclass{scrreprt}

\usepackage{aligned-overset}
\usepackage{amsmath}
\usepackage{amssymb}
\usepackage{bm}
\usepackage[shortlabels]{enumitem}
\usepackage{hyperref}
\usepackage[utf8]{inputenc}
\usepackage{multicol}
\usepackage{marvosym}
\usepackage{mathtools}
\usepackage{pdflscape}
\usepackage{physics}
\usepackage{tabularx}
\usepackage{titling}
\usepackage{fancyhdr}
\usepackage{xfrac}
\usepackage{pgfplots}

\pgfplotsset{compat = newest}
\usetikzlibrary{intersections}
\usetikzlibrary{patterns}
\usepgfplotslibrary{fillbetween}

\author{Karsten Lehmann}
\date{WiSe 2021/2022}
\title{Übungsblatt 11\\Analysis - Grundlegende Konzepte}

\setlength{\headheight}{26pt}
\pagestyle{fancy}
\fancyhf{}
\lhead{\thetitle}
\rhead{\theauthor}
\lfoot{\thedate}
\rfoot{Seite \thepage}

\newcommand\diam{\text{diam}}

\begin{document}
\begin{landscape}
\paragraph{55. Bestimmen Sie die Partialsummen der folgenden Reihen} und
entscheiden Sie über die Folge der Partialsummen ob die Reihen konvergieren
oder divergieren.
\begin{enumerate}[a)]
\item $\underset{n = 1}{\overset{\infty}{\sum}}
  \frac{1}{\qty\big(3n - 1)\qty\big(3n + 2)}$

  \subparagraph{Lsg.} Die Folge $\qty\big{s_m}_{m \in \mathbb{N}}$ der
  Partialsummen ist:
  \begin{small}
    \begin{flalign*}
      s_m &\coloneqq \sum_{n = 1}^m \frac{1}{\qty\big(3n - 1)\qty\big(3n + 2)}
      = \sum_{n = 1}^m \frac{1}{\qty\big(3n - 1)\qty\big(3(n + 1) - 1)} &\\
      &= \frac{1}{3} \sum_{n = 1}^m \frac{
        \qty\Big(3\qty\big(n + 1) - 1) - \qty\Big(3n - 1)
      }{
        \qty\big(3n - 1)\qty\big(3(n + 1) - 1)
      } \\
      &= \frac{1}{3} \sum_{n = 1}^m \frac{1}{3n - 1} -
        \frac{1}{3\qty\big(n + 1) - 1}  \\
      &= \frac{1}{3} \cdot \qty(\frac{1}{3 - 1} +
        \underset{0}{\underbrace{
          \qty(-\frac{1}{3 \cdot 2 - 1} + \frac{1}{3 \cdot 2 - 1})
        }} +
        \underset{0}{\underbrace{
          \qty(-\frac{1}{3 \cdot 3 - 1} + \frac{1}{3 \cdot 3 - 1})
        }}  +
        \ldots +
        \underset{0}{\underbrace{
          \qty(-\frac{1}{3 \cdot n} + \frac{1}{3 \cdot n - 1})
        }} -
        \frac{1}{3 \cdot \qty\big(n + 1) - 1}
      ) \\
      &= \frac{1}{3} \cdot \qty(
        \frac{1}{2} - \frac{1}{3 \cdot \qty\big(n + 1) - 1}
      ) \\
      &= \frac{1}{6} - \frac{1}{3\qty\big(3n + 2)}
    \end{flalign*}
  \end{small}
  $\Rightarrow s_m \overset{m \to \infty}\longrightarrow \frac{1}{6}$, somit
  konvergiert die Reihe.
\end{enumerate}
\end{landscape}

\newpage
\begin{enumerate}[(a)]
\setcounter{enumi}{1}
\item $\underset{n = 1}{\overset{\infty}{\sum}} \frac{(-7)^n}{9^n}$

  \subparagraph{Lsg.} Die Folge $\qty\big{s_m}_{m \in \mathbb{N}}$ der
  Partialsummen ist
  \begin{flalign*}
    s_m &\coloneqq \sum_{n = 1}^m \frac{\qty\big(-7)^n}{9^n} & \\
    &= \sum_{n = 1}^m \qty(\frac{-7}{9})^n
    && \abs{\frac{-7}{9}} < 1 \Rightarrow \text{ geometrische Reihe} \\
    &= \frac{1 - \qty(\frac{-7}{9})^{n + 1}}{1 - \frac{-7}{9}} - \qty(\frac{-7}{9})^0 \\
    &= \frac{1 - \qty(\frac{-7}{9})^{n + 1}}{\frac{16}{9}}  - 1\\
    &= -\frac{\frac{7}{9} - \qty(\frac{-7}{9})^{n + 1}}{\frac{16}{9}} \\
    &= - \frac{7 - 9 \cdot \qty(\frac{-7}{9})^{n + 1}}{16}
  \end{flalign*}
  $\Rightarrow s_m \overset{m \to \infty}\longrightarrow -\frac{7}{16}$,
  die Reihe konvergiert.
\end{enumerate}

\paragraph{56. Es sei $\qty\big{a_n}_{n \in \mathbb{N}}$ definiert durch
  $a_n \coloneqq \frac{\qty\big(-1)^n}{\sqrt{n + 1}}$ für alle
  $n \in \mathbb{N}$.}
Zeigen Sie, dass die Reihe $\underset{n = 0}{\overset{\infty}{\sum}} a_n$
konvergiert, aber das Cauchy-Produkt
\[
  \sum_{n = 0}^{\infty}\sum_{k = 0}^n a_k a_{n - k}
\]
der Reihe mit sich selbst nicht konvergiert.
Was geht schief?

\subparagraph{Lsg.} Es ist $\displaystyle a_n \coloneqq \qty\big(-1)^n \cdot
\frac{n \cdot \frac{1}{n}}{n \cdot \sqrt{\frac{1}{n} + \frac{1}{n^2}}}
= \qty\big(-1)^n \cdot \frac{\frac{1}{n}}{\sqrt{\frac{1}{n} + \frac{1}{n^2}}}$
und $\frac{\frac{1}{n}}{\sqrt{\frac{1}{n} + \frac{1}{n^2}}}
\overset{n \to \infty}\longrightarrow \frac{0}{\sqrt{1}} = 0$.

$\Rightarrow$ \underline{die Reihe}
$\underset{n = 0}{\overset{\infty}{\sum}} a_n$
\underline{konvergiert nach dem Leibnitzkriterium für alternierende Reihen in
  $\mathbb{R}$}.

\newpage
\begin{flalign*}
  \sum_{n = 0}^{\infty}\sum_{k = 0}^n a_k a_{n - k} &=
  \sum_{n = 0}^{\infty}\sum_{k = 0}^n \frac{\qty\big(-1)^k}{\sqrt{k + 1}} \cdot
  \frac{\qty\big(-1)^{n - k}}{\sqrt{n - k + 1}} & \\
  &= \sum_{n = 0}^{\infty}\sum_{k = 0}^n
  \frac{\qty\big(-1)^{n}}{\sqrt{k + 1} \cdot \sqrt{n - k + 1}} \\
   &= \sum_{n = 0}^{\infty}\sum_{k = 0}^n
  \frac{\qty\big(-1)^{n}}{\sqrt{\qty\big(k + 1) \cdot \qty\big(n - k + 1)}}
\end{flalign*}

Nach Satz 7.1 der Vorlesung ist das geometrische Mittel stets kleiner oder
gleich dem arithmetischem Mittel für $n \in \mathbb{N}$ und
$x_1, \ldots, x_n \in \mathbb{R}_{>  0}$:
\begin{equation}
  \tag{*}
  \sqrt[n]{x_1 \cdot \ldots \cdot x_n} \leq \frac{1}{2} \cdot
  \qty\Big(x_1 + \ldots + x_n)
\end{equation}
Es folgt
\begin{flalign*}
  \abs{\frac{\qty\big(-1)^n}{\sqrt{\qty\big(k + 1) \cdot \qty\big(n - k + 1)}}}
  &\geq \frac{1}{\frac{1}{2}\qty\Big(\qty\big(k + 1) + \qty\big(n - k + 1))} &\\
  &= \frac{2}{\qty\big(k + 1) + \qty\big(n - k + 1)} \\
  &= \frac{2}{n + 2}
  &\geq \frac{1}{n + 2}
\end{flalign*}
Aus Beispiel 12.4 der Vorlesung ist bereits bekannt, dass
$\displaystyle \sum_{k = 1}^{\infty}\frac{1}{k}$ gegen $\infty$ divergiert.
Folglich divergiert auch
$\displaystyle \sum_{k = 3}^{\infty}\frac{1}{k} =
\sum_{n = 1}^{\infty}\frac{1}{n + 2}$.

$\Rightarrow$ \underline{Nach dem Majorantenkriterium divergiert die Reihe}
$\displaystyle \sum_{n = 0}^{\infty}\sum_{k = 0}^{n}a_ka_{n - k}$
\underline{ gegen $\infty$.}

Was geht schief?
Die Reihe $\displaystyle \sum_{n = 0}^{\infty}a_n$ konvergiert nicht absolut.
Folglich divergieren manche Umordnungen der Reihe (oder konvergieren) gegen
andere Werte.

\end{document}