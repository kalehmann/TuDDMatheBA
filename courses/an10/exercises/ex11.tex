\documentclass{scrreprt}

\usepackage{aligned-overset}
\usepackage{amsmath}
\usepackage{amssymb}
\usepackage{bm}
\usepackage[shortlabels]{enumitem}
\usepackage{hyperref}
\usepackage[utf8]{inputenc}
\usepackage{multicol}
\usepackage{marvosym}
\usepackage{mathtools}
\usepackage{pdflscape}
\usepackage{physics}
\usepackage{tabularx}
\usepackage{titling}
\usepackage{fancyhdr}
\usepackage{xfrac}
\usepackage{pgfplots}

\pgfplotsset{compat = newest}
\usetikzlibrary{intersections}
\usetikzlibrary{patterns}
\usepgfplotslibrary{fillbetween}

\author{Karsten Lehmann}
\date{WiSe 2021/2022}
\title{Übungsblatt 11\\Analysis - Grundlegende Konzepte}

\setlength{\headheight}{26pt}
\pagestyle{fancy}
\fancyhf{}
\lhead{\thetitle}
\rhead{\theauthor}
\lfoot{\thedate}
\rfoot{Seite \thepage}

\newcommand\diam{\text{diam}}

\begin{document}
\begin{landscape}
\paragraph{55. Bestimmen Sie die Partialsummen der folgenden Reihen} und
entscheiden Sie über die Folge der Partialsummen ob die Reihen konvergieren
oder divergieren.
\begin{enumerate}[a)]
\item $\underset{n = 1}{\overset{\infty}{\sum}}
  \frac{1}{\qty\big(3n - 1)\qty\big(3n + 2)}$

  \subparagraph{Lsg.} Die Folge $\qty\big{s_m}_{m \in \mathbb{N}}$ der
  Partialsummen ist:
  \begin{small}
    \begin{flalign*}
      s_m &\coloneqq \sum_{n = 1}^m \frac{1}{\qty\big(3n - 1)\qty\big(3n + 2)}
      = \sum_{n = 1}^m \frac{1}{\qty\big(3n - 1)\qty\big(3(n + 1) - 1)} &\\
      &= \frac{1}{3} \sum_{n = 1}^m \frac{
        \qty\Big(3\qty\big(n + 1) - 1) - \qty\Big(3n - 1)
      }{
        \qty\big(3n - 1)\qty\big(3(n + 1) - 1)
      } \\
      &= \frac{1}{3} \sum_{n = 1}^m \frac{1}{3n - 1} -
        \frac{1}{3\qty\big(n + 1) - 1}  \\
      &= \frac{1}{3} \cdot \qty(\frac{1}{3 - 1} +
        \underset{0}{\underbrace{
          \qty(-\frac{1}{3 \cdot 2 - 1} + \frac{1}{3 \cdot 2 - 1})
        }} +
        \underset{0}{\underbrace{
          \qty(-\frac{1}{3 \cdot 3 - 1} + \frac{1}{3 \cdot 3 - 1})
        }}  +
        \ldots +
        \underset{0}{\underbrace{
          \qty(-\frac{1}{3 \cdot n} + \frac{1}{3 \cdot n - 1})
        }} -
        \frac{1}{3 \cdot \qty\big(n + 1) - 1}
      ) \\
      &= \frac{1}{3} \cdot \qty(
        \frac{1}{2} - \frac{1}{3 \cdot \qty\big(n + 1) - 1}
      ) \\
      &= \frac{1}{6} - \frac{1}{3\qty\big(3n + 2)}
    \end{flalign*}
  \end{small}
  $\Rightarrow s_m \overset{m \to \infty}\longrightarrow \frac{1}{6}$, somit
  konvergiert die Reihe.
\end{enumerate}
\end{landscape}

\end{document}