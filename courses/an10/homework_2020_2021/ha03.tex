\documentclass{article}

\usepackage{aligned-overset}
\usepackage{amsmath}
\usepackage{amssymb}
\usepackage{bbm}
\usepackage[shortlabels]{enumitem}
\usepackage{genealogytree}
\usepackage{hyperref}
\usepackage[utf8]{inputenc}
\usepackage{interval}
\intervalconfig{
  soft open fences
}
\usepackage{mathtools}
\usepackage{pgfplots}
\usepackage{physics}
\usepackage{tikz}
\usetikzlibrary{positioning}
\usepackage{xcolor}
\definecolor{light-gray}{gray}{.9}

\author{Karsten Lehmann}
\date{17.11.2020}
\title{Hausaufgabe 03 Analysis - Grundlegende Konzepte}

\begin{document}


\maketitle
\newpage

\section*{Hausaufgabe 1}
\begin{itemize}
\item
  \[
    s_k = \sup A_k
  \]
  
  \begin{align*}
    B &\coloneqq \bigcap_{k = 1}^n A_K \\
      &= \{ a \in \mathbb{R} | \forall k = 1, \ldots, n \colon  a \in A_k \} \\
  \end{align*}
  
  In der Aufgabe 2 der Übung wurde bereits bewiesen, dass
  
  \[
    A = \bigcup_{k = 1}^n A_k
  \]
  
  nach oben beschränkt ist mit
  \begin{itemize}
  \item der Definition von $s$ : $\forall k = 1, \ldots, n \colon s_k \leq s$
  \item der Definition von $s_k$ : $\forall a \in A_k \colon a \leq s_k$
  \end{itemize}

  Aus den Anordnungsaxiomen folgt nun $\forall k = 1, \ldots, n, \forall a \in A_k \colon a \leq s$. Da der Durchschnitt zweier Mengen eine
  Teilmenge der Vereinigung ist (und die Vereinigung nach oben beschränkt ist) folgt aus 1.3.3 (h) der Vorlesung, dass
  
  \[
    \bigcap_{k = 1}^n A_K \ne \subseteq \bigcup_{k = 1}^n A_k
  \]
  beschränkt ist. \\
\item
  Man betrachte die Menge aller Suprema $s_k$ der Mengen $A_k$ für $k = 1$ bis $n$, insbesondere das kleinste Element dieser Menge $s = \min \{ s_1, \ldots, s_n \}$.
  Im Durchschnitt aller Mengen $A_k$ kann kein Element echt größer als $s$ sein, denn ansonsten wäre dieses Element
  auch in der Menge $A_k$ enthalten aus der $s$ entnommen wurde und $s$ wäre nicht Supremum dieser Menge, ein Widerspruch.
  Folglich ist $s$ eine obere Schranke des Durchschnitts der Mengen $A_k$. Somit gilt
  \[
    \sup B \leq \min \{ s_1, \ldots, s_n \}
  \]

  \newpage
  \emph{Folgt aus der Beschränktheit von $B$ auch die Beschränktheit jeder der Mengen $A_k, k = 1, \ldots, n$?}

  Beweis durch Widerspruch:
  
  Sei $A_1 \subseteq \mathbb{R} = \{ 1 \}$ eine beschränkte Menge und $A_2 \subseteq \mathbb{R} = \mathbb{R}_{> 0}$
  eine nach oben unbeschränkte Menge, dann ist
  \[
    B = \bigcap_{k = 1}^{2} A_k = A_1
  \]
  $A_1$ (und damit auch $B$) ist nach oben beschränkt, $A_2$ jedoch nicht. Ein Widerspruch zu
  \[
    B \text{ ist beschränkt} \Rightarrow \forall k = 1, \ldots, n \colon A_k \text{ ist beschränkt}
  \]

\item \emph{Ist die Menge $\overset{\infty}{\underset{k = 1}{\bigcap}} A_k$ beschränkt?}

  Der Durchschnitt der Mengen $A_k$ von $k = 1$ bis $k = \infty$ ist definiert als
  \[
    \{ a | a \in A_1 \land a \in A_2 \land \ldots \land a \in A_\infty \}
  \]

  Der Definition kann entnommen werden, dass der Durchschnitt eine Teilmenge jedes $A_k$ von $k = 1$ bis $k = \infty$
  ist. Nach der Aufgabenstellung ist jede Menge $A_k$ nach oben beschränkt und nach 1.3.3 (h) aus der Vorlesung
  ist jede Teilmenge einer nach oben beschränkten Menge ebenfalls nach oben beschränkt.
  
\end{itemize}

\newpage
\section*{Hausaufgabe 2}

Nach Definition ist das Supremum einer Menge größer oder gleich jedem Element der Menge.
Sei nun $s = \sup A$, dann schreiben wir

\begin{align*}
  \forall a \in A      &\colon a \leq s                     && | : s \\
                       &\colon \frac{a}{s} \leq 1           && | : a \\
                       &\colon \frac{1}{s} \leq \frac{1}{a} && (\text{Definition von $A^{-1}$}) \\
  \forall b \in A^{-1} &\colon \frac{1}{s} \leq b \\
\end{align*}

Somit ist $\frac{1}{s}$ eine untere Schranke von $A^{-1}$ für $s = \sup A$.
Angenommen $\frac{1}{s}$ \textbf{ist nicht} Infimum von $A^{-1}$, dann
existiert eine Zahl, die kleiner oder gleich jedem Element von $A^{-1}$,
aber größer als $\frac{1}{s}$ ist.

\begin{align*}
  \forall \epsilon > 0, \epsilon \ne s \colon s &> s - \epsilon && | :(s - \epsilon) \\
  \frac{s}{s - \epsilon} &> 1 && | :s\\
  \frac{1}{s - \epsilon} &> \frac{1}{s} \\
\end{align*}
\[
  \exists \epsilon > 0, \epsilon \ne s \colon \forall a \in A \colon \frac{1}{s - \epsilon} \leq \frac{1}{a} 
\]

Den Ausdruck $\frac{1}{s - \epsilon} \leq \frac{1}{a}$ kann man umschreiben zu

\begin{align*}
  \frac{1}{s - \epsilon} &\leq \frac{1}{a} && | *a \\
  \frac{a}{s - \epsilon} &\leq 1           && | *(s - \epsilon)\\
  a &\leq s - \epsilon 
\end{align*}

Dies ist ein Widerspruch dazu, dass $s$ Supremum von $A$ ist. Also ist $\frac{1}{s}$ Infimum von $A^{-1}$.
Da $\frac{1}{s} * s = 1$ folgt $\frac{1}{s} = (\sup A)^{-1}$

\newpage
\section*{Hausaufgabe 3}


\begin{enumerate}[a)]
\item
  \emph{Behauptung}
  \[
    A(m) \colon (a^n)^m = a^{n * m} = (a^m)^n
  \]
  \emph{Induktionsanfang}
  \[
    A(1) \colon (a^n)^1 = a^{n * 1} = (a^1)^n (= a^n) 
  \]
  Diese Aussage ist wahr. \\
  \emph{Induktionsschritt}
  \begin{align*}
    A(m + 1) \colon &(a^n)^m * a^n = a^{n * m} * a^{n} = (a^m * a) ^ n  &&  \text{(Anwendung der Definition der Potenz)}\\
                    &(a^n)^{m + 1} = a^{n * m + n} = (a^{m + 1})^n      && \text{(A9)} \\
                    &(a^n)^{m + 1} = a^{n * (m + 1)} = (a^{m + 1})^n
  \end{align*}

\item
    \emph{Behauptung}
  \[
    B(n) \colon a^n * b^n = (a * b)^n
  \]
  \emph{Induktionsanfang}
  \[
    B(1) \colon a^1 * b^1 = (a * b)^1 
  \]
  Diese Aussage ist wahr. \\
  \emph{Induktionsschritt}
  \begin{align*}
    B(n + 1) \colon &a^n * a * b^n * b = (a * b)^n * (a * b) &&  \text{(Anwendung der Definition der Potenz)}\\
                    &a^{n + 1} * b^{n + 1} = (a * b)^{n + 1}
  \end{align*}
\end{enumerate}

\end{document}