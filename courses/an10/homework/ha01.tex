\documentclass{scrreprt}

\usepackage{aligned-overset}
\usepackage{amsmath}
\usepackage{amssymb}
\usepackage{bm}
\usepackage[shortlabels]{enumitem}
\usepackage{hyperref}
\usepackage[utf8]{inputenc}
\usepackage{multicol}
\usepackage{mathtools}
\usepackage{physics}
\usepackage{tabularx}
\usepackage{titling}
\usepackage{fancyhdr}
\usepackage{xfrac}
\usepackage{pgfplots}

\pgfplotsset{compat = newest}
\usetikzlibrary{intersections}
\usetikzlibrary{patterns}
\usepgfplotslibrary{fillbetween}

\author{Karsten Lehmann (Übungsgruppe 1)\\Mat. Nr 4935758}
\date{WiSe 2021/2022}
\title{Hausaufgaben Blatt 03\\Analysis - Grundlegende Konzepte}

\setlength{\headheight}{26pt}
\pagestyle{fancy}
\fancyhf{}
\lhead{\thetitle}
\rhead{\theauthor}
\lfoot{\thedate}
\rfoot{Seite \thepage}

\begin{document}
\paragraph{14. Beweisen Sie die de Morgan'schen Regeln
  für Mengen $A, B \subset M$ und logische Ausagen $p$ und $q$}

\begin{enumerate}[(a)]
\item $\neg \qty\big(p \lor q) = \neg p \land \neg q$
  \subparagraph{Lsg.} durch Wahrheitstabelle
  \begin{center}
    \begin{tabular}{c | c | c | c}
      $p$ & $q$ & $\neg \qty\big(p \lor q)$ & $\neg p \land \neg q$ \\
      \hline
      W & W & F & F \\
      W & F & F & F \\
      F & W & F & F \\
      F & F & W & W
    \end{tabular}
  \end{center}

\item $\neg \qty\big(p \land q) = \neg p \lor \neg q$

  \subparagraph{Lsg.} durch Wahrheitstabelle
  \begin{center}
    \begin{tabular}{c | c | c | c}
      $p$ & $q$ & $\neg \qty\big(p \land q)$ & $\neg p \lor \neg q$ \\
      \hline
      W & W & F & F \\
      W & F & W & W \\
      F & W & W & W \\
      F & F & W & W
    \end{tabular}
  \end{center}
\item $\qty\big(A \cup B)^C = A^C \cap B^C$

  \subparagraph{Lsg.}
  \begin{flalign*}
    x \in \qty\big(A \cup B) &\iff x \in A \lor x \in B & \\
    x \in \qty\big(A \cup B)^C &\iff \neg \qty\big(x \in A \lor x \in B) \\
    \overset{\text{Siehe (a)}}&\iff \neg \qty\big(x \in A) \land \neg \qty\big(x \in B) \\
    &\iff x \in A^C \land x \in B^C \\
    &\iff x \in \qty\big(A^C \cap B^C)
  \end{flalign*}
\item $\qty\big(A \cap B)^C = A^C \cup B^C$

  \subparagraph{Lsg.}
  \begin{flalign*}
    x \in \qty\big(A \cap B) &\iff x \in A \land x \in B & \\
    x \in \qty\big(A \cap B)^C &\iff \neg \qty\big(x \in A \land x \in B) \\
    \overset{\text{Siehe (b)}}&\iff \neg \qty\big(x \in A) \lor \neg \qty\big(x \in B) \\
    &\iff x \in A^C \lor x \in B^C \\
    &\iff x \in \qty\big(A^C \cup B^C)
  \end{flalign*}
\end{enumerate}
\newpage
Es sei nun $\mathcal{N} \subset \mathcal{P}(M)$ ein nichtleeres Mengensystem
(also eine Menge von Mengen).
Beweisen Sie auch für diesen Fall die de Morgan'schen Regeln:
\begin{enumerate}[(a)]
\setcounter{enumi}{4}
\item $\qty(\underset{N \in \mathcal{N}}\bigcup N)^C
  = \underset{N \in \mathcal{N}}\bigcap N^C$
  \subparagraph{Lsg.}
  \begin{flalign*}
    x \in \bigcup_{N \in \mathcal{N}}N
    &\iff \exists N \in \mathcal{N} \colon x \in N & \\
    x \in \qty(\bigcup_{N \in \mathcal{N}}N)^C
    &\iff \forall N \in \mathcal{N} \colon \neg \qty\big(x \in N) \\
    &\iff \forall N \in \mathcal{N} \colon x \in N^C \\
    &\iff x \in \bigcap_{N \in \mathcal{N}} N^C
  \end{flalign*}

\item $\qty(\underset{N \in \mathcal{N}}\bigcap N)^C
  = \underset{N \in \mathcal{N}}\bigcup N^C$.

  \subparagraph{Lsg.}
  \begin{flalign*}
    x \in \bigcap_{N \in \mathcal{N}}N
    &\iff \forall N \in \mathcal{N} \colon x \in N & \\
    x \in \qty(\bigcap_{N \in \mathcal{N}}N)^C
    &\iff \exists N \in \mathcal{N} \colon \neg \qty\big(x \in N) \\
    &\iff \exists N \in \mathcal{N} \colon x \in N^C \\
    &\iff x \in \bigcup_{N \in \mathcal{N}} N^C
  \end{flalign*}

\end{enumerate}

\end{document}