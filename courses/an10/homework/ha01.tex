\documentclass{scrreprt}

\usepackage{aligned-overset}
\usepackage{amsmath}
\usepackage{amssymb}
\usepackage{bm}
\usepackage[shortlabels]{enumitem}
\usepackage{hyperref}
\usepackage[utf8]{inputenc}
\usepackage{multicol}
\usepackage{mathtools}
\usepackage{physics}
\usepackage{tabularx}
\usepackage{titling}
\usepackage{fancyhdr}
\usepackage{xfrac}
\usepackage{pgfplots}

\pgfplotsset{compat = newest}
\usetikzlibrary{intersections}
\usetikzlibrary{patterns}
\usepgfplotslibrary{fillbetween}

\author{Karsten Lehmann (Übungsgruppe 1)\\Mat. Nr 4935758}
\date{WiSe 2021/2022}
\title{Hausaufgaben Blatt 03\\Analysis - Grundlegende Konzepte}

\setlength{\headheight}{26pt}
\pagestyle{fancy}
\fancyhf{}
\lhead{\thetitle}
\rhead{\theauthor}
\lfoot{\thedate}
\rfoot{Seite \thepage}

\begin{document}
\paragraph{14. Beweisen Sie die de Morgan'schen Regeln
  für Mengen $A, B \subset M$ und logische Ausagen $p$ und $q$}

\begin{enumerate}[(a)]
\item $\neg \qty\big(p \lor q) = \neg p \land \neg q$
  \subparagraph{Lsg.} durch Wahrheitstabelle
  \begin{center}
    \begin{tabular}{c | c | c | c}
      $p$ & $q$ & $\neg \qty\big(p \lor q)$ & $\neg p \land \neg q$ \\
      \hline
      W & W & F & F \\
      W & F & F & F \\
      F & W & F & F \\
      F & F & W & W
    \end{tabular}
  \end{center}

\item $\neg \qty\big(p \land q) = \neg p \lor \neg q$

  \subparagraph{Lsg.} durch Wahrheitstabelle
  \begin{center}
    \begin{tabular}{c | c | c | c}
      $p$ & $q$ & $\neg \qty\big(p \land q)$ & $\neg p \lor \neg q$ \\
      \hline
      W & W & F & F \\
      W & F & W & W \\
      F & W & W & W \\
      F & F & W & W
    \end{tabular}
  \end{center}
\item $\qty\big(A \cup B)^C = A^C \cap B^C$

  \subparagraph{Lsg.}
  \begin{flalign*}
    x \in \qty\big(A \cup B) &\iff x \in A \lor x \in B & \\
    x \in \qty\big(A \cup B)^C &\iff \neg \qty\big(x \in A \lor x \in B) \\
    \overset{\text{Siehe (a)}}&\iff \neg \qty\big(x \in A) \land \neg \qty\big(x \in B) \\
    &\iff x \in A^C \land x \in B^C \\
    &\iff x \in \qty\big(A^C \cap B^C)
  \end{flalign*}
\item $\qty\big(A \cap B)^C = A^C \cup B^C$

  \subparagraph{Lsg.}
  \begin{flalign*}
    x \in \qty\big(A \cap B) &\iff x \in A \land x \in B & \\
    x \in \qty\big(A \cap B)^C &\iff \neg \qty\big(x \in A \land x \in B) \\
    \overset{\text{Siehe (b)}}&\iff \neg \qty\big(x \in A) \lor \neg \qty\big(x \in B) \\
    &\iff x \in A^C \lor x \in B^C \\
    &\iff x \in \qty\big(A^C \cup B^C)
  \end{flalign*}
\end{enumerate}
\newpage
Es sei nun $\mathcal{N} \subset \mathcal{P}(M)$ ein nichtleeres Mengensystem
(also eine Menge von Mengen).
Beweisen Sie auch für diesen Fall die de Morgan'schen Regeln:
\begin{enumerate}[(a)]
\setcounter{enumi}{4}
\item $\qty(\underset{N \in \mathcal{N}}\bigcup N)^C
  = \underset{N \in \mathcal{N}}\bigcap N^C$
  \subparagraph{Lsg.}
  \begin{flalign*}
    x \in \bigcup_{N \in \mathcal{N}}N
    &\iff \exists N \in \mathcal{N} \colon x \in N & \\
    x \in \qty(\bigcup_{N \in \mathcal{N}}N)^C
    &\iff \forall N \in \mathcal{N} \colon \neg \qty\big(x \in N) \\
    &\iff \forall N \in \mathcal{N} \colon x \in N^C \\
    &\iff x \in \bigcap_{N \in \mathcal{N}} N^C
  \end{flalign*}

\item $\qty(\underset{N \in \mathcal{N}}\bigcap N)^C
  = \underset{N \in \mathcal{N}}\bigcup N^C$.

  \subparagraph{Lsg.}
  \begin{flalign*}
    x \in \bigcap_{N \in \mathcal{N}}N
    &\iff \forall N \in \mathcal{N} \colon x \in N & \\
    x \in \qty(\bigcap_{N \in \mathcal{N}}N)^C
    &\iff \exists N \in \mathcal{N} \colon \neg \qty\big(x \in N) \\
    &\iff \exists N \in \mathcal{N} \colon x \in N^C \\
    &\iff x \in \bigcup_{N \in \mathcal{N}} N^C
  \end{flalign*}
\end{enumerate}

\newpage
\paragraph{15. Es sei $M$ eine Menge.}
Eine Zerlegung $\mathcal{Z} \in \mathcal{P}(M)$ von $M$ ist ein Mengensystem
(also eine Menge von Mengen) mit folgenden Eigenschaften:
\begin{enumerate}[(i)]
\item Für alle $Z \in \mathcal{Z}$ gilt $Z \ne \emptyset$
\item Für alle $Z_1, Z_2 \in \mathcal{Z}$ mit $Z_1 \ne Z_2$ gilt
  $Z_1 \cap Z_2 = \emptyset$.
\item $M = \underset{Z \in \mathcal{Z}}\bigcup Z$
\end{enumerate}

Beweisen Sie die folgenden Aussagen:
\begin{enumerate}[(a)]
\item Ist $R \subset M \times M$ eine Äquivalenzrelation auf $M$, so ist die
  Menge der Äquivalenzklassen von $R$ eine Zerlegung von $M$.

  \subparagraph{Lsg.} Seien $a, b$ zwei Äquivalenzklassen aus der Menge aller
  Äquivalenzklassen von $R$.
  \begin{enumerate}[(i)]
  \item Sei $a' \in M$ und $a$ die Äquivalenzklasse von $a'$ bezüglich $R$,
    also $[a'] = a$.

    Dann ist $a = \qty{x \in M {\Big |} (a', x) \in R}$.
    $R$ ist als Äquivalenzrelation \emph{reflexiv}, somit folgt aus
    $a' \in M$ auch $\qty(a',a') \in R$.

    $\Rightarrow a' \in a$

    $\Rightarrow a \ne \emptyset$

  \item Sei $a \ne b$ und $a$ die Äquivalenzklasse von $a'$ sowie $b$ die
    Äquivalenzklasse von $b'$.
    Angenommen es gibt ein Element $x \in a \cap b$.
    Dann gilt nach Definition der Äquivalenzklasse
    \[
      (a', x), (b', x) \in R
    \]
    Da $R$ als Äquivalenzrelation \emph{symmetrisch} ist gilt auch
    \[
      (a', x), (x, b') \in R
    \]
    Aus der \emph{Transitivität} von $R$ folgt $(a', b') \in R$, somit
    ist $a' \sim b'$ und $a = b$.
    Das ist ein Widerspruch, daraus folgend kann kein $x \in a \cap b$
    existieren.

    $\Rightarrow a \cap b = \emptyset$

  \item Jedes Element $x \in M$ besitzt eine Äquivalenzklasse.
    Wie in (i) bereits gezeigt gilt immer $x \in \qty[x]$.
    Somit existiert für jedes Element aus $M$ eine Äquivalenzklasse, welche
    dieses Element enthält.
    Weiterhin enthält jede Äquivalenzklasse (von Elementen aus $M$) auch nur
    Elemente aus $M$.
    Somit ist die Vereinigung aller Äquivalenzklassen von $R$ gleich der Menge
    $M$.
  \end{enumerate}

\newpage
\item Ist $\mathcal{Z} \in \mathcal{P}(M)$ eine Zerlegung von $M$, so gibt es
  eine Äquivalenzrelation $R \subset M \times M$, für die $\mathcal{Z}$ die
  Menge der Äquivalenzklassen von $R$ ist.

  \subparagraph{Lsg.} Sei
  \[
    R = \qty\Big{ (a, b) \in M \times M {\Big |}
      \exists Z \in \mathcal{Z} \colon a \in Z \land b \in Z}
  \]
  \begin{enumerate}[1)]
  \item Sei $x \in M$.
    Da die Vereinigung von $\mathcal{Z}$ gleich $M$ ist,
    existiert ein $Z \in \mathcal{Z}$ mit $x \in Z$

    $\Rightarrow (x, x) \in R$

    Somit ist $R$ \emph{reflexiv}.

  \item Sei $(x, y) \in R$.
    Nach der Definition von $R$ existiert somit ein Element
    $Z_1 \in \mathcal{Z}$ mit $x \in Z_1 \land y \in Z_1$.

    $\Rightarrow$ es gilt auch $y \in Z_1 \land x \in Z_1$

    $\Rightarrow$ $(y, x) \in R$

    Somit ist $R$ \emph{symmetrisch}.

  \item Seien $(x, y), (y, z) \in R$.
    Nach der Definition von $R$ existiert somit ein Element
    $Z_1 \in \mathcal{Z}$ mit $x \in Z_1 \land y \in Z_1$.
    Weiterhin existiert ein Element $Z_2 \in \mathcal{Z}$ mit
    $y \in Z_2 \land z \in Z_2$.

    Da $\mathcal{Z}$ eine Zerlegung ist, sind zwei unterschiedliche Elemente aus
    $\mathcal{Z}$ immer disjunkt.

    Aus $y \in Z_1$ und $y \in Z_2$ folgt somit $Z_1 = Z_2$.

    $\Rightarrow x, y, z \in Z_1$.

    $\Rightarrow (x, z) \in R$

    Somit ist $R$ \emph{transitiv}.
  \end{enumerate}

  $\Rightarrow R$ ist eine Äquivalenzrelation.

  Sei nun $Z \in \mathcal{Z}$.
  Dann ist für jedes Element $x \in Z$ nach Definition von $R$
  \[
    \qty[x] = \qty{y \in M {\Big |} (x, y) \in R}
    = \qty{y \in M {\Big |} \exists Z' \in \mathcal{Z} \colon (x, y) \in Z'} = Z
  \].
  Somit ist ein jedes Element aus $\mathcal{Z}$ die Äquivalenzklasse all seiner
  Elemente bezüglich $R$.

  $\Rightarrow \mathcal{Z}$ ist die Menge der Äquivalenzklassen von $R$.
\end{enumerate}

\end{document}