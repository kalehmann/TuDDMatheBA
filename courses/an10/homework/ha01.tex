\documentclass{article}

\usepackage{aligned-overset}
\usepackage{amsmath}
\usepackage{amssymb}
\usepackage{bbm}
\usepackage[shortlabels]{enumitem}
\usepackage{genealogytree}
\usepackage{hyperref}
\usepackage[utf8]{inputenc}
\usepackage{interval}
\intervalconfig{
  soft open fences
}

\usepackage{mathtools}
\usepackage{physics}
\usepackage{tikz}
\usetikzlibrary{positioning}
\usepackage{xcolor}
\definecolor{light-gray}{gray}{.9}

\author{Lisa Würzebesser (4981474) HA-02 \\ Lukas Kamratzki (3752330) HA-04 \\ Karsten Lehmann (4935758) HA-01}
\date{01.11.2020}
\title{Hausaufgabe 01 Analysis - Grundlegende Konzepte}

\begin{document}


\maketitle
\newpage

\section*{Hausaufgabe 1}

\begin{enumerate}[(a)]
\item Beweisen Sie, dass für alle $a, b \in \mathbb{R}$ gilt: $(a^{-1})^{-1} = a$ \\

  $(a^{-1})^{-1}$ bezeichnet das multiplikative Inverse von $a^{-1}$.

  \begin{align*}
    1  &= a * a^{-1}  &\text{(Laut dem Axiom (A7))}\\
       \overset{\text{(A8)}}&{=} a^{-1} * a  \\
       &\Rightarrow a \text{ ist ein inverses von } a^{-1}
  \end{align*}
  \\
  Sei $a \in \mathbb{R}$ und seien $a^{-1}, b \in \mathbb{R}$ zwei multiplikative Inverse Elemente von $a$.
  Dann gilt

  \begin{align*}
    a^{-1} \underset{\substack{1 \text{ ist ein} \\ \text{neutrales Element}}}&{=} a^{-1} * 1 \\
           \underset{\substack{\text{Die } 1 \text{ lässt sich auch als Produkt aus } a \text{ und dem}\\ \text{multiplikativen inversen Element } b \text{ beschreiben}}}&{=} a^{-1} * (a * b) \\
           \overset{\text{(A5)}}&{=} (a^{-1} * a) * b \\
           \overset{\text{(A8)}}&{=} (a * a^{-1}) * b \\
           \overset{\text{(A7)}}&{=} 1 * b \\
    a^{-1} \underset{\substack{1 \text{ ist ein} \\ \text{neutrales Element}}}&{=} b \\
  \end{align*}

  Somit ist gezeigt, dass es auch nur genau ein multiplikatives Inverses gibt. Weiterhin ist $a$ das einzige multiplikative Inverse von $a^{-1}$ und es gilt
  \[
    (a^{-1})^{-1} = a
  \]

\item Beweisen Sie, dass für alle $a, b \in \mathbb{R}$ gilt: $-(a + b) = (-a) + (-b)$ \\

  \begin{align*}
    (a + b) + ((-a) + (-b)) &= (a + b) + ((-a) + (-b)) \\
                            \overset{\text{(A1)}}&{=} a + (b + (-b)) + (-a) \\
                            \overset{\text{(A3)}}&{=} a + 0 + (-a) \\
                            \overset{\text{(A2)}}&{=} a + (-a) \\
                            \overset{\text{(A3)}}&{=} 0 \\
    (a + b) + ((-a) + (-b)) &= 0 \\
  \end{align*}

  Also ist $(-a) + (-b)$ ein additives inverses Element von $a + b$. Aus Lemma (b)
  (\emph{``Es gibt nur ein additives Inverses''}) folgt:

  \[
    -(a + b) = (-a) + (-b)
  \]

\item Beweisen Sie, dass für alle $a, b \in \mathbb{R}, a \ne 0$ gilt: $(-a)^{-1} = -a^{-1}$ \\

  \begin{align*}
    (-a) * (-a^{-1}) &= (-a) * (-a^{-1}) \\
                     \overset{\text{(A5)}}&{=} (-(-a)) * a^{-1} \\
                     \overset{\text{(e)}}&{=} a * a^{-1} \\
                     \overset{\text{(A7)}}&{=} 1\\
    (-a) * -a^{-1} &= 1 \\
  \end{align*}

  Somit ist $-(a^{-1})$ ein multiplikatives Inverses von $(-a)$ und aus Lemma (b)
  (\emph{``Es gibt nur ein additives Inverses''}) folgt:
  \[
    (-a)^{-1} = -a^{-1}
  \]

\item Beweisen Sie, dass für alle $a, b \in \mathbb{R}, a \ne 0$ gilt: $a * x = b$ \\

  Sei $x = a^{-1} * b$, dann gilt
  \begin{align*}
    a * x &= a * (a^{-1} * b) \\
          \overset{\text{(A5)}}&{=} (a * a^{-1}) * b \\
          \overset{\text{(A7)}}&{=} 1 * b \\
          \overset{\text{(A8)}}&{=} b * 1 \\
          \overset{\text{(A6)}}&{=} b\\
    a * x = b \\
  \end{align*}

  Daraus folgt, es existiert \textbf{mindestens} ein $x \in \mathbb{R}$, für welches gilt $a * x = b$.
  Nun soll gezeigt werden, dass auch nur \textbf{genau ein} $x$ existiert, für welches diese Behauptung
  gilt. \\

  Sei $y \in \mathbb{R}$ ein weiteres Element, so dass $a * y = b$ und $x = a^{-1} * b$. Dann gilt

  \begin{align*}
    x &= a^{-1} * b \\
      &= a^{-1} * (a * y) \\
      &\overset{\text{(A5)}}= (a^{-1} * a) * y \\
      &\overset{\text{(A8)}}= (a * a^{-1}) * y \\
      &\overset{\text{(A7)}}= 1 * y \\
      &\overset{\text{(A5)}}= y * 1 \\
      &\overset{\text{(A6)}}= y \\
   x  &= y
  \end{align*}

  Somit existiert nur \textbf{genau ein} $x \in \mathbb{R}$, für welches gilt $a * x = b$.
\end{enumerate}

\section*{Hausaufgabe 2}

In der Menge $\mathbb{R}$ der reellen Zahlen werden zwei Operatoren $\oplus, \odot \colon \mathbb{R} \times \mathbb{R} \to \mathbb{R}$ erklärt durch

\[
  a \oplus b = a + b + 1 \text{ und } a \odot b = a + b + a * b
\]

Zeigen Sie, dass für $(\mathbb{R}, \oplus, \odot)$ die Axiome (A1) bis (A9) aus der Vorlesung erfüllt sind.

\begin{enumerate}[label=(A\arabic*)]
%% A1
\item
  \begin{align*}
    a \oplus (b \oplus c) &= a + (b \oplus c) + 1 \\
                          &= (a \oplus b) + c + 1 \\
                          &= a + (b + c + 1) + 1 \\
                          &\overset{\text{(A1)}}= (a + b + c) + 1 + 1 \\
    \\
    (a \oplus b) \oplus c &= (a \oplus b) + c + 1 \\
                          &= (a + b + 1) + c + 1 \\
    \overset{\text{(A1)}}&{=} (a + b + c) + 1 + 1 \\
  \end{align*}

  Somit ist $a \oplus (b \oplus c) = (a \oplus b) \oplus c$

%% A2 
\item Sei $\mathbb{O} \in \mathbb{R}$ neutrales Element der Operation $\oplus$, dann soll gelten:

  $\exists \mathbb{O} \forall a \in \mathbb{R} \colon a \oplus \mathbb{O} = a$

  \begin{align*}
    a \oplus \mathbb{O} &= a \\
    a + \mathbb{O} + 1 &= a\\
    a + (-a) + \mathbb{O} + 1 \underset{\substack{\text{Addition des Inversen von} \\ a \text{ auf beiden Seiten}}}&{=} a + (-a) \\
    \mathbb{O} + 1 = 0 \\
  \end{align*}

  Somit existiert ein neutrales Element $\mathbb{O} \in \mathbb{R}$ als inverses Element von $1$ für welches gilt
  $a \oplus \mathbb{R} = a$.

%% A3
\item $\forall a \in \mathbb{R} \exists (-a=)' \in \mathbb{R} \colon a \oplus (-a)' = 0$ \\

  \begin{align*}
    a \oplus (-a)' &= 0 \\
    a + (-a)' + 1 &= 0 \\ 
    a + (-a) + (-a)' + 1 \underset{\substack{\text{Addition des Inversen von} \\ a \text{ auf beiden Seiten}}}&{=} (-a)\\
    (-a)' + 1 &= (-a) \\
    (-a)' + (-1) \underset{\substack{\text{Addition des Inversen von} \\ 1 \text{ auf beiden Seiten}}}&{=} (-a) + (-1)\\
    (-a)' &= (-a) + (-1) \\
  \end{align*}

  Somit existiert ein inverses Element $(-a)' \in \mathbb{R}$ als $(-a) + (-1)$ für welches gilt
  $a \oplus (-a)' = 0$ \\

  Alternativ für $\forall a \in \mathbb{R} \exists (-a=)' \in \mathbb{R} \colon a \oplus (-a)' = \mathbb{O}$

  \begin{align*}
    a \oplus (-a)' &= \mathbb{O} \\
    a + (-a)' + 1 &= -1 \\ 
    a + (-a) + (-a)' + 1 \underset{\substack{\text{Addition des Inversen von} \\ a \text{ auf beiden Seiten}}}&{=} (-1) + (-a)\\
    (-a)' + 1 &= (-1) + (-a) \\
    (-a)' + (-1) \underset{\substack{\text{Addition des Inversen von} \\ 1 \text{ auf beiden Seiten}}}&{=} (-1) + (-a) + (-1)\\
    (-a)' &= (-a) + (-2) \\
  \end{align*}

%% A4
\item
  \begin{align*}
    a \oplus b &= b \oplus a \\
    a + b + 1 &= b + a + 1 \\
    a + b + 1 &\overset{\text{(A4)}}&{=} a + b + 1 \\
  \end{align*}

%% A5
\item $\forall a,b,c \in \mathbb{R} \colon (a \odot b) \odot c = a \odot (b \odot c)$
  \begin{align*}
    (a \odot b) \odot c &= a \odot (b \odot c) \\
    (a \odot b) + c + (a \odot b) * c &= a + (b \odot c) + a * (b \odot c) \\
    (a + b + a * b) + c + (a + b + a * b) * c &= a + (b + c + b * c) + a * (b + c + b * c) \\
    (a + b + a * b) + c + a * c + b * c + a * b * c \overset{\text{(A9)}}&{=} a + (b + c + b * c) + a * b + a * c + a* b * c \\
    (a + b + c) + a * b + a * c + b * c + a * b * c \overset{\text{(A9)}}&{=} (a + b + c) + b * c + a * b + a * c + a* b * c \\
    (a + b + c) + a * b + a * c + b * c + a * b * c \overset{\text{(A4)}}&{=} (a + b + c) + a * b + a * c + b * c + a * b * c \\
  \end{align*}

%% A6
\item $\exists \mathit{1} \in \mathbb{R} \forall a \in \mathbb{R} \colon a \odot \mathit{1} = a$ \\

  \begin{align*}
    a \odot \mathit{1} &= a \\
    a + \mathit{1} + a * \mathit{1} &= a \\
    a + \mathit{1} * (1 + a) \overset{\text{(A9)}}&{=} a \\
    a + (-a) + \mathit{1} * (1 + a) \underset{\substack{\text{Addition des Inversen von} \\ a \text{ auf beiden Seiten}}}&{=} a + (-a) \\
    \mathit{1} * (1 - a) &= 0 \\
    \mathit{1} * (1 + a) * (1 + a)^{-1} \underset{\substack{\text{Multiplikation des Inversen von} \\ (1+a) \text{ auf beiden Seiten}}}&{=} 0 \\
    \mathit{1} * 1 &= 0 \\
    \mathit{1} &= 0 \\
  \end{align*}

  Somit existiert ein neutrales Element $\mathit{1} \in \mathbb{R}$ als $0$ für welches gilt
  $a \odot \mathit{1} = a$

%% A7
\item $\forall a \in \mathbb{R}, a \ne 0 \exists \mathbbm{a}^{-1} \in \mathbb{R} \colon a \odot \mathbbm{a}^{-1} = 1$
  
  \begin{align*}
    a \odot \mathbbm{a}^{-1} &= 1 \\
    a + \mathbbm{a}^{-1} + a * \mathbbm{a}^{-1} &= 1 \\
    a + \mathbbm{a}^{-1} * (1 + a) \overset{\text{(A9)}}&{=} 1 \\
    a + (-a) + \mathbbm{a}^{-1} * (1 + a) \underset{\substack{\text{Addition des Inversen von} \\ a \text{ auf beiden Seiten}}}&{=} 1 + (-a) \\
    \mathbbm{a}^{-1} * (1 + a) &= 1 - a \\
    \mathbbm{a}^{-1} * (1 + a) * (1 + a)^{-1} \underset{\substack{\text{Multiplikation des Inversen von} \\ (1+a) \text{ auf beiden Seiten}}}&{=} (1 - a) * (1 + a)^{-1} \\
    \mathbbm{a}^{-1} &= (1 - a) * (1 + a)^{-1} \\
  \end{align*}

  Somit existiert ein inverses Element $\mathbbm{a}^{-1} \in \mathbb{R}$ als $(1 + a) * (1 + a)^{-1}$ für welches gilt
  $a \odot \mathbbm{a}^{-1} = 1$ \\

  Alternativ für $\forall a \in \mathbb{R}, a \ne 0 \exists \mathbbm{a}^{-1} \in \mathbb{R} \colon a \odot \mathbbm{a}^{-1} = \mathit{1}$

  \begin{align*}
    a \odot \mathbbm{a}^{-1} &= \mathit{1} \\
    a \odot \mathbbm{a}^{-1} &= 0 \\
    a + \mathbbm{a}^{-1} + a * \mathbbm{a}^{-1} &= 0 \\
    a + \mathbbm{a}^{-1} * (1 + a) \overset{\text{(A9)}}&{=} 0 \\
    a + (-a) + \mathbbm{a}^{-1} * (1 + a) \underset{\substack{\text{Addition des Inversen von} \\ a \text{ auf beiden Seiten}}}&{=} (-a) \\
    \mathbbm{a}^{-1} * (1 + a) &= 1 - a \\
    \mathbbm{a}^{-1} * (1 + a) * (1 + a)^{-1} \underset{\substack{\text{Multiplikation des Inversen von} \\ (1+a) \text{ auf beiden Seiten}}}&{=} (-a) * (1 + a)^{-1} \\
    \mathbbm{a}^{-1} &= (-a) * (1 + a)^{-1} \\
  \end{align*}

%% A8
\item $\forall a,b \in \mathbb{R} \colon a \odot b = b \odot a$
  \begin{align*}
    a \odot b &= b \odot a \\
    a + b + a * b &= b + a + b * a \\
    a + b + a * b \overset{\text{(A4)}}&{=} a + b + b * a \\
    a + b + a * b \overset{\text{(A8)}}&{=} a + b + a * b \\
  \end{align*}

%% A9
\item $\forall a,b,c \in \mathbb{R} \colon (a \oplus b) \odot c = a \odot c \oplus b \odot c$

  \begin{align*}
    (a \oplus b) \odot c &= a \odot c \oplus b \odot c \\
    (a + b + 1) \odot c &= a \odot c + b \odot c + 1 \\
    (a + b + 1) + c + (a + b + 1) * c &= a + c + a * c + b + c + b * c + 1 \\
    (a + b + 1) + c + a * c + b * c + c \overset{\text{(A9)}}&{=} a + c + a * c + b + c + b * c + 1 \\
    a + b + c + c + a * c + b * c + 1 \overset{\text{(A4)}}&{=} a + b + c + c + a * c + b * c + 1 \\
  \end{align*}
\end{enumerate}

\section*{Hausaufgabe 3}

\begin{align*}
  A &\coloneqq \{ x \colon x \in \mathbb{R}, \abs{x - 1} < 2 \} \\
  B &\coloneqq \{ x \colon x \in \mathbb{R}, x > -8 \} \\
  C &\coloneqq \{ x \colon x \in \mathbb{R}, -8 <x \leq 1 \} \\
\end{align*}

\textbf{Stellen Sie die Menge $A$ als Intervall dar!} \\

\begin{minipage}[t]{.45\textwidth}
  \textbf{Fall 1}: $x \geq 1$ \\
  \begin{align*}
    x - 1 &< 2 & | +1\\
    x &< 3 \\
  \end{align*}
\end{minipage}
\hfill
\vrule
\hfill
\begin{minipage}[t]{.45\textwidth}
  \textbf{Fall 2}: $x < 1$ \\
  \begin{align*}
    -(x - 1) &< 2 \\
    -x + 1) &< 2 &&| -1 \\
    -x &< 1 &&| * (-1) \\
    x &> -1 \\
  \end{align*}
\end{minipage}

\[
  A = \interval[open]{-1}{3}
\]

\textbf{Stellen Sie die Menge $B$ als Intervall dar!} \\

\[
  B = \interval[open]{-8}{\infty}
\]

\textbf{Stellen Sie die Menge $C$ als Intervall dar!} \\

\[
  C = \interval[open left]{-8}{1}
\]

\textbf{Bestimmen Sie $A \cap B, A \cup B, A \setminus B, A \cap C, B \setminus C$}

\begin{align*}
  A \cap B &= \interval[open]{-1}{3} \cap \interval[open]{-8}{\infty} \\
           &= \interval[open]{-1}{3} \\
           &= A \\
  \\
  A \cup B &= \interval[open]{-1}{3} \cup \interval[open]{-8}{\infty} \\
           &= \interval[open]{-8}{\infty} \\
           &= B \\
  \\
  A \setminus B &= \interval[open]{-1}{3} \setminus \interval[open]{-8}{\infty} \\
           &= \emptyset \\
  \\
  A \cap C &= \interval[open]{-1}{3} \cap \interval[open left]{-8}{1} \\
           &= \interval[open left]{-1}{1} \\
  \\
  B \setminus C &= \interval[open]{-8}{\infty} \setminus \interval[open left]{-8}{1} \\
           &= \interval[open]{1}{\infty} \\
\end{align*}

\end{document}