\documentclass{scrreprt}

\usepackage{aligned-overset}
\usepackage{amsmath}
\usepackage{amssymb}
\usepackage{bm}
\usepackage[shortlabels]{enumitem}
\usepackage{hyperref}
\usepackage[utf8]{inputenc}
\usepackage{multicol}
\usepackage{mathtools}
\usepackage{physics}
\usepackage{tabularx}
\usepackage{titling}
\usepackage{fancyhdr}
\usepackage{xfrac}
\usepackage{pgfplots}

\pgfplotsset{compat = newest}
\usetikzlibrary{intersections}
\usetikzlibrary{patterns}
\usepgfplotslibrary{fillbetween}

\author{Karsten Lehmann (Übungsgruppe 1)\\Mat. Nr 4935758}
\date{WiSe 2021/2022}
\title{Hausaufgaben Blatt 11\\Analysis - Grundlegende Konzepte}

\setlength{\headheight}{26pt}
\pagestyle{fancy}
\fancyhf{}
\lhead{\thetitle}
\rhead{\theauthor}
\lfoot{\thedate}
\rfoot{Seite \thepage}

\begin{document}
\paragraph{58. Untersuchen Sie die folgenden Reihen auf Konvergenz:}
\begin{enumerate}[(a)]
\item $\displaystyle \sum_{n = 1}^{\infty} \qty\big(-1)^nn$

  \subparagraph{Lsg.}Sei $\qty\big{x_n}_{n \in \mathbb{N}}$ eine Reihe mit
  $x_n \coloneqq \qty\big(-1)^nn$.
  Angenommen $\sum_n x_n$ konvergiert, dann folgt aus Folgerung 12.2 der
  Vorlesung (\emph{``Sei $X$ ein normierter Raum und $\qty\big{x_k}$ eine Folge
    in $X$, dann folgt aus der Konvergenz von $\sum_k x_k$, dass
    $x_k \overset{k \to \infty}\longrightarrow 0$''}), dass
  $\displaystyle \lim_{n \to \infty} x_n = 0$, allerdings gilt für alle
  $n \in \mathbb{N}$, dass $d\qty\big(x_n, 0) > \frac{1}{2}$, ein Widerspruch
  zur Konvergenz.

  $\Rightarrow$ \underline{Die Reihe }
  $\displaystyle \sum_{n = 1}^{\infty} \qty\big(-1)^n n$
  \underline{ divergiert.}

\item $\displaystyle \sum_{n = 1}^{\infty} \qty\Big(\sqrt[n]{n} - 1)^n$

  \subparagraph{Lsg.} Aus Satz 5.20 (4) der Vorlesung (\emph{``Für alle
    $a, \epsilon \in \mathbb{R}_{> 0}$ existiert ein $n_0 \in \mathbb{N}$, so
    dass $\abs\big{\sqrt[n]{a} - 1} < \epsilon$ für alle $n > n_0$.''}) folgt,
  dass für jedes $q \in \qty\big(0, 1)$ ein $k_0 \in \mathbb{N}$ existiert,
  so dass für alle $k > k_0$ gilt:
  \begin{flalign*}
    \sqrt[k]{\norm{\qty\big(\sqrt[k]{k} - 1)^k}} &=
    \sqrt[k]{\abs{\qty\big(\sqrt[k]{k} - 1)^k}} &\\
    \overset{\sqrt[k]{k} > 1}&= \sqrt[k]{\qty\big(\sqrt[k]{k} - 1)^k}
    = \sqrt[k]{k} - 1 \\
    &\leq \abs{\sqrt[k]{k} - 1} < q
  \end{flalign*}

  $\Rightarrow$ \underline{Nach dem Wurzelkriterium konvergiert die Reihe }
  $\displaystyle \sum_{n = 1}^{\infty} \qty\Big(\sqrt[n]{n} - 1)^n$
  \underline{ absolut.}

\item $\displaystyle \sum_{n = 1}^{\infty} \frac{\sqrt{n + 1}}{\sqrt{n^5 + n^2 - 1}}$

  \subparagraph{Lsg.} Es ist
  \begin{flalign*}
    \norm{\frac{\sqrt{n + 1}}{\sqrt{n^5 + n^2 - 1}}}
    < \norm\Bigg{
      \frac{1}{\sqrt{n^4} \cdot \sqrt{n + \frac{1}{n^2} - \frac{1}{n^4}}}
    }
    = \norm\Bigg{\frac{1}{
      n^2 \cdot \underset{> 1}{\underbrace{\textstyle
          \sqrt{n + \frac{1}{n^2} - \frac{1}{n^4}}
        }}
    }}
    < \frac{1}{n^2}
  \end{flalign*}

  und $\displaystyle \sum_{n = 1}^{\infty} \frac{1}{n^2}$ konvergiert nach
  Beispiel 12.6 der Vorlesung.

  $\Rightarrow$ \underline{Nach dem Majorantenkriterium konvergiert die Reihe }
  $\displaystyle \norm{\frac{\sqrt{n + 1}}{\sqrt{n^5 + n^2 - 1}}}$
  \underline{ absolut.}

\newpage
\item $\displaystyle \sum_{n = 1}^{\infty} \frac{1}{7^n} \binom{3n}{n}$

  \subparagraph{Lsg.} Es ist $\binom{3n}{n} =
  \frac{\qty\big(3n)!}{n!\qty\big(3n - n)!} =
  \frac{\qty\big(3n)!}{n!\qty\big(2n)!}$.
  Weiter ist $\frac{\qty\big(3n)!}{n!\qty\big(2n)!}$ als Produkt und Quotient
  positiver Zahlen ebenfalls positiv, somit gilt
  $\norm{\frac{\qty\big(3n)!}{n!\qty\big(2n)!}}
  = \abs{\frac{\qty\big(3n)!}{n!\qty\big(2n)!}}
  = \frac{\qty\big(3n)!}{n!\qty\big(2n)!}$.
  Sei nun $a_n \coloneqq \frac{1}{7^n} \binom{3n}{n}$, dann ist
  \begin{flalign*}
    \frac{\norm{a_{n + 1}}}{\norm{a_n}}
    &= \frac{
      \norm{
        \frac{
          \qty\big(3(n + 1))!
        }{
          7^{n + 1}\qty\big(n + 1)!\qty\big(2(n + 1))!
        }
      }
    }{\norm{\frac{\qty\big(3n)!}{7^nn!\qty\big(2n)!}}} \\
    &= \frac{
      \frac{
        \qty\big(3(n + 1))!
      }{
        7^{n + 1}\qty\big(n + 1)!\qty\big(2(n + 1))!
      }
    }{\frac{\qty\big(3n)!}{7^nn!\qty\big(2n)!}} \\
    &= \frac{\qty\big(3(n + 1))!}{7^{n + 1}\qty\big(n + 1)!\qty\big(2(n + 1))!}
    \cdot \frac{7^nn!\qty\big(2n)!}{\qty\big(3n)!} \\
    &= \frac{\qty\big(3(n + 1))!}{7\qty\big(n + 1)!\qty\big(2(n + 1))!}
    \cdot \frac{n!\qty\big(2n)!}{\qty\big(3n)!} \\
    &= \frac{
      \qty\big(3n + 3) \cdot
      \qty\big(3n + 2) \cdot
      \qty\big(3n + 1) \cdot
      \qty\big(3n)!
    }{
      7 \cdot
      \qty\big(n + 1) \cdot
      n! \cdot
      \qty\big(2n + 2) \cdot
      \qty\big(2n + 1) \cdot
      \qty\big(2n)!
    } \cdot \frac{n!\qty\big(2n)!}{\qty\big(3n)!} \\
    &= \frac{
      \qty\big(3n + 3) \cdot \qty\big(3n + 2) \cdot \qty\big(3n + 1)
    }{
      7 \cdot \qty\big(n + 1) \cdot \qty\big(2n + 2) \cdot \qty\big(2n + 1)
    } \\
     &= \frac{
      3 \cdot \qty\big(n + 1) \cdot \qty\big(3n + 2) \cdot \qty\big(3n + 1)
    }{
      7 \cdot \qty\big(n + 1) \cdot \qty\big(2n + 2) \cdot \qty\big(2n + 1)
    } \\
    &= \frac{
      3 \cdot \qty\big(3n + 2) \cdot \qty\big(3n + 1)
    }{
      7 \cdot \qty\big(2n + 2) \cdot \qty\big(2n + 1)
    } \\
    &= \frac{27n^2 + 27n + 6}{28n^2 + 42n + 14} \\
    &= \frac{27}{28} \cdot \underset{< 1}{\underbrace{
        \qty(\frac{n^2 + n +\frac{2}{9}}{n^2 + \frac{3}{2}n + \frac{1}{2}})
      }}
  \end{flalign*}

  $\Rightarrow$ \underline{Die Reihe } $\sum_k \qty\big{x_k}$
  \underline{ konvergiert nach dem Quotientenkriterium absolut mit }
  $\frac{\norm{x_k + 1}}{\norm{x_k}} \leq \frac{27}{28} < 1$.
\end{enumerate}

\newpage
\paragraph{59. Sei $q \in \mathbb{R}, \abs{q} < 1$.}
Zeigen Sie
\[
  \sum_{n = 0}^{\infty} \qty\big(n + 1)q^n = \qty(\sum_{k = 0}^{\infty} q^k)^2
\]

\subparagraph{Lsg.} Nach Beispiel 12.3 der Vorlesung ist
$\displaystyle \sum_{k = 0}^{\infty} q^k = \frac{1}{1 - q}$ mit
$\abs{q} < 1$.
Weiter ist $\sqrt[n]{\norm{q^n}} = \norm{q} < 1 \, \forall \, n \in \mathbb{N}$.

$\Rightarrow$ \underline{Die Reihe} $\displaystyle \sum_{k = 0}^{\infty} q^k$
\underline{konvergiert nach dem Wurzelkriterium absolut.}

\begin{flalign*}
  \qty(\sum_{k = 0}^{\infty} q^k)^2
  &= \sum_{k = 0}^{\infty} q^k \cdot \sum_{n = 0}^{\infty} q^n \\
  &= \sum_{n = 0}^{\infty}\sum_{k = 0}^n q^k \cdot q^{n - k} \\
  &= \sum_{n = 0}^{\infty}\sum_{k = 0}^n q^{k + n - k} \\
  &= \sum_{n = 0}^{\infty}\sum_{k = 0}^n q^n \\
  &= \sum_{n = 0}^{\infty} \qty\big(1 + n) q^n
\end{flalign*}

\paragraph{60. Überprüfen Sie,} ob man mit dem Quotientenkriterium die
Konvergenz oder Divergenz der Reihe $\displaystyle \sum_{n = 0}^{\infty} a_n$
mit
\[
  a_{2k} \coloneqq \qty(\frac{1}{2})^{2k}
  \text{ und }
  a_{2k + 1} \coloneqq \qty(\frac{1}{3})^{2k + 1}
  \text{ für } k \in \mathbb{N}
\]
sicherstellen kann.
Gelingt dies mit dem Wurzelkriterium?

\subparagraph{Lsg.} Sei $n \in \mathbb{N}$ beliebig.

\begin{minipage}{0.4\textwidth}
  Angenommen $2$ teilt $n$, dann ist
  \begin{flalign*}
    \frac{\norm{a_{n + 1}}}{\norm{a_n}} &=
    \frac{\norm{\qty(\frac{1}{3})^{n + 1}}}{\norm{\qty(\frac{1}{2})^n}}
    = \frac{\norm{\frac{1}{3^{n + 1}}}}{\norm{\frac{1}{2^n}}} & \\
    &= \norm{\frac{2^n}{3^{n + 1}}}
    = \norm{\qty(\frac{2}{3})^n \cdot \frac{1}{3}} < 1
  \end{flalign*}
\end{minipage}
\hfill
\vrule
\hfill
\begin{minipage}{0.4\textwidth}
  Angenommen $2$ teilt $n$ nicht, dann ist
  \begin{flalign*}
    \frac{\norm{a_{n + 1}}}{\norm{a_n}} &=
    \frac{\norm{\qty(\frac{1}{2})^n}}{\norm{\qty(\frac{1}{3})^{n + 1}}}
    = \frac{\norm{\frac{1}{2^n}}}{\norm{\frac{1}{3^{n + 1}}}} & \\
    &= \norm{\frac{3^{n + 1}}{2^n}}
    = \norm{\qty(\frac{3}{2})^n \cdot 3} > 1
  \end{flalign*}
\end{minipage}

In der Vorlesung wurde das Quotientenkriterium in zwei Formen eingeführt:
\begin{enumerate}[(1)]
\item Wenn für fast $n \in \mathbb{N}$ gilt, dass
  $\frac{\norm{a_{n + 1}}}{\norm{a_n}} \leq q < 1$, dann konvergiert
  $\sum_n a_n$ absolut.

\item Wenn für fast $n \in \mathbb{N}$ gilt, dass
  $\frac{\norm{a_{n + 1}}}{\norm{a_n}} \geq 1$, dann divergiert
  $\sum_n \norm{a_n}$.
\end{enumerate}
Die Folge fällt in keine der beiden Formen, somit lässt sich mit dem
Quotientenkriterium keine Aussage über die Konvergenz der Reihe treffen.

\begin{minipage}{0.4\textwidth}
  Angenommen $2$ teilt $n$, dann ist
  \begin{flalign*}
    \sqrt[n]{\norm{a_{n}}} &=
    \sqrt[n]{\norm{\qty(\frac{1}{2})^n}} &\\
    &= \sqrt[n]{\norm{\frac{1}{2^n}}} \\
    &= \sqrt[n]{\frac{1}{\abs{2^n}}} = \frac{1}{2}
  \end{flalign*}
\end{minipage}
\hfill
\vrule
\hfill
\begin{minipage}{0.4\textwidth}
  Angenommen $2$ teilt $n$ nicht, dann ist
  \begin{flalign*}
    \sqrt[n]{\norm{a_n}}
    &= \sqrt[n]{\norm{\qty(\frac{1}{3})^n}} & \\
    &= \sqrt[n]{\frac{1}{\abs{3^n}}} & \\
    &= \frac{1}{3}
  \end{flalign*}
\end{minipage}

$\Rightarrow \sqrt[n]{\norm{a_n}} \leq \frac{1}{2} < 1 \, \forall \,
n \in \mathbb{N}$

$\Rightarrow$ \underline{Die Reihe }
$\displaystyle \sum_n \norm{a_n}$
\underline{ konvergiert nach dem Wurzelkriteriem absolut}.
\end{document}
