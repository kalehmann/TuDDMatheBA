\documentclass{scrreprt}

\usepackage{aligned-overset}
\usepackage{amsmath}
\usepackage{amssymb}
\usepackage{bm}
\usepackage[shortlabels]{enumitem}
\usepackage{hyperref}
\usepackage[utf8]{inputenc}
\usepackage{multicol}
\usepackage{mathtools}
\usepackage{physics}
\usepackage{tabularx}
\usepackage{titling}
\usepackage{fancyhdr}
\usepackage{xfrac}
\usepackage{pgfplots}

\pgfplotsset{compat = newest}
\usetikzlibrary{intersections}
\usetikzlibrary{patterns}
\usepgfplotslibrary{fillbetween}

\author{Karsten Lehmann (Übungsgruppe 1)\\Mat. Nr 4935758}
\date{WiSe 2021/2022}
\title{Hausaufgaben Blatt 06\\Analysis - Grundlegende Konzepte}

\setlength{\headheight}{26pt}
\pagestyle{fancy}
\fancyhf{}
\lhead{\thetitle}
\rhead{\theauthor}
\lfoot{\thedate}
\rfoot{Seite \thepage}

\begin{document}
\paragraph{28.} Es sei $K$ ein angeordneter Körper und $1_K$ das neutrale Element
der Multiplikation.
Die Abbildung $f \colon \mathbb{Q} \to K$ sei wie in der Vorlesung definiert
durch
\[
  f(q) = f\qty(\frac{m}{n}) \coloneqq \frac{m1_k}{n1_K} \coloneqq
  \qty\big(m1_K) \cdot \qty\big(n1_K)^{-1}
  \text{ für } q = \frac{m}{n} \in \mathbb{Q}
\]

\underline{Beweisen Sie:} Für alle $p, q \in \mathbb{Q}$ gilt
\[
  p \leq q \iff f(p) \leq f(q)
\]

\subparagraph{Lsg.} Seien $p$ und $q$ beliebig.
Die Repräsentanten $\frac{m_p}{n_p}$ (von $p$) und $\frac{m_q}{n_q}$ (von $q$)
seien so gewählt, dass $n_p \cdot 1_K > 0$ und $n_q \cdot 1_K > 0$ sind.
Es folgt, dass $\qty\big(n_p \cdot 1_K)^{-1} > 0$ und
$\qty\big(n_q \cdot 1_K)^{-1} > 0$.

\begin{flalign*}
  p \leq q \iff && m_p \cdot n_q &\leq m_q \cdot n_p
  && {\Big |} \cdot \qty(1_K)^2 > 0 \\
  && \qty(1_K)^2 \cdot m_p \cdot n_q &\leq \qty(1_K)^2 \cdot m_q \cdot n_p \\
  && \qty(m_p1_k) \cdot \qty(n_q1_K) &\leq \qty(m_q1_K) \cdot \qty(n_p1_K)
  && {\Big |} \cdot \qty(n_q1_K)^{-1} \\
  && \qty(m_p1_k) \cdot \qty(n_q1_K) \cdot \qty(n_q1_K)^{-1}
  &\leq \qty(m_q1_K) \cdot \qty(n_p1_K) \cdot \qty(n_q1_K)^{-1} \\
  && \qty(m_p1_k) \cdot 1 &\leq \qty(m_q1_K) \cdot \qty(n_p1_K)
  \cdot \qty(n_q1_K)^{-1} && {\Big |} \cdot \qty(n_p1_K)^{-1} \\
  && \qty(m_p1_k) \cdot \qty(n_p1_K)^{-1} &\leq \qty(m_q1_K) \cdot \qty(n_q1_K)^{-1}
  && \iff f(p) \leq f(q)
\end{flalign*}

\paragraph{29.} Es sei $K$ ein angeordneter Körper und für $n \in \mathbb{N}$
seien $x_1, \ldots, x_n \in K$ gegeben, die alle gleiches Vorzeichen haben und
$x_i \geq -1$ erfüllen.
Zeigen Sie, dass dann die folgende Ungleichung gilt:
\[
  \prod_{i = 1}^n \qty\big(1 + x_i) \geq 1 + \sum_{i = 1}^n x_i
\]

\subparagraph{Lsg.}
\underline{Behauptung:}
\[
  P(n) \colon \prod_{i = 1}^n \qty\big(1 + x_i) \geq 1 + \sum_{i = 1}^n x_i
\]
\underline{Induktionsanfang:}
\begin{align*}
  P(0) &\colon \prod_{i = 1}^0 \qty\big(1 + x_i) \geq 1 + \sum_{i = 1}^0 x_i
       \colon && 1 \geq 0 && \\
  P(1) &\colon \prod_{i = 1}^1 \qty\big(1 + x_i) \geq 1 + \sum_{i = 1}^1 x_i
       \colon && 1 + x_1 \geq 1 + x_1
\end{align*}
\underline{Induktionsschritt:} Sei $P(n)$ für ein beliebiges $n \in \mathbb{N}$
wahr.
\begin{flalign*}
  \prod_{i = 1}^n \qty\big(1 + x_i) &\geq 1 + \sum_{i = 1}^n x_i
  && {\Big |} \cdot \qty(1 + x_{n + 1}) \geq 0 \\
  \qty(1 + x_{n + 1}) \cdot \prod_{i = 1}^n \qty\big(1 + x_i) &\geq
  \qty(1 + x_{n + 1}) \cdot \qty(1 + \sum_{i = 1}^n x_i) \\
  \prod_{i = 1}^{n + 1} &\geq 1 +
  \underset{\makebox[0pt]{$\geq 0$}}{\underbrace{x_{n + 1}  \sum_{i = 1}^n x_i}} +
  \sum_{i = 1}^n x_i + x_{n + 1} && {\Big |} \text{ Abschätzung nach unten}\\
  &\geq 1 +  \sum_{i = 1}^n x_i + x_{n + 1} \\
  \prod_{i = 1}^{n + 1} &\geq 1 +  \sum_{i = 1}^{n + 1} x_i
\end{flalign*}

Nach der vollständigen Induktion folgt die Behauptung.

\paragraph{30.} Es seien $K$ ein archimedisch angeordneter Körper sowie
$a \in K$ mit $a > 0$.
Welche der angegebenen Intervallfolgen sind Intervallschachtelungen?
\begin{enumerate}[(a)]
\item $\qty{\qty[0_K, \frac{1}{n}a]}_{n \in \mathbb{N}_{>0}}$
\item $\qty{\qty[0_K, \frac{(n + 1) 1_K}{na}]}_{n \in \mathbb{N}_{>0}}$
\item $\qty{\qty[\frac{n}{n + 1}a, \frac{n + 1}{n} a]}_{n \in \mathbb{N}_{>0}}$
\end{enumerate}

\subparagraph{Lsg.}
\begin{enumerate}[(a)]
\item Ist eine Intervallschachtelung, da
  \begin{enumerate}[(1)]
  \item $\qty[0_k, \frac{1}{n}a] \ne \emptyset$ für alle $n \in \mathbb{N}$
  \item $\forall \; n \in \mathbb{N} \colon
    \frac{1}{n + 1} < \frac{1}{n}
    \Rightarrow \qty[0_k, \frac{1}{n + 1}a] \subset \qty[0_K, \frac{1}{n}a]$

  \item $\forall \epsilon > 0, a \in K \; \exists \; n \in \mathbb{N}
    \colon \frac{1}{n}a - 0_K< \epsilon$ mit $n > \frac{a}{\epsilon + 0_K}$
    und ein solches $n$ existiert, da $K$ ein archimedisch angeordneter Körper
    ist.
  \end{enumerate}

\newpage
\item Sei $\epsilon \in K$ so gewählt, dass
  $\epsilon \in \qty(0, \frac{1_K}{a} - 0_K)$.
  Angenommen es existiert ein $n \in \mathbb{N}_{> 0}$ mit
  $\frac{\qty(n + 1)1_K}{na} - 0_K < \epsilon$, dann wäre
  $\frac{n + 1}{n} < 1$ - ein Widerspruch.

  $\Rightarrow$ keine Intervallschachtelung.

\item Ist eine Intervallschachtelung, da
  \begin{enumerate}[(1)]
  \item $\frac{n}{n + 1}a < \frac{n + 1}{n}a$ für alle $n \in \mathbb{N}_{> 0}$
    $\Rightarrow \qty[\frac{n}{n + 1}a, \frac{n + 1}{n} a] \ne \emptyset$

  \item $\frac{n + 1}{n + 2}a > \frac{n}{n + 1}a$ und
    $\frac{n + 2}{n + 1}a < \frac{n + 1}{n}a$
    $\Rightarrow \qty[\frac{n + 1}{n + 2}a, \frac{n + 2}{n + 1} a]
    \subset \qty[\frac{n}{n + 1}a, \frac{n + 1}{n} a]$

  \item Sei
    \begin{flalign*}
      \frac{n + 1}{n}a - \frac{n}{n + 1}a &< \epsilon \\
      \frac{n + 1}{n} - \frac{n}{n + 1} =
      \qty(\frac{n + 1}{n} - 1) + \qty(1 - \frac{n}{n + 1})
      &< \frac{\epsilon}{a} && {\Big |} \; \frac{1}{n} >\frac{1}{n + 1} \\
      \qty(\frac{n + 1}{n} - 1) + \qty(1 - \frac{n}{n + 1})
      < 2 \qty(\frac{n + 1}{n} - 1) &< \frac{\epsilon}{a} \\
      \frac{2n + 2}{n} &< \frac{\epsilon}{a} + 2 \\
      \frac{2a}{\epsilon} &< n
    \end{flalign*}

    $\forall \epsilon > 0, a \in K \; \exists \; n \in \mathbb{N}
    \colon \frac{n + 1}{n}a - \frac{n}{n + 1}a < \epsilon$ mit
    $n > \frac{2a}{\epsilon}$.
    Da $K$ archimedisch angeordnet ist, existiert ein solches $n$ auch.
  \end{enumerate}
\end{enumerate}

\end{document}
