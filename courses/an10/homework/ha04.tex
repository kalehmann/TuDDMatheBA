\documentclass{scrreprt}

\usepackage{aligned-overset}
\usepackage{amsmath}
\usepackage{amssymb}
\usepackage{bm}
\usepackage[shortlabels]{enumitem}
\usepackage{hyperref}
\usepackage[utf8]{inputenc}
\usepackage{multicol}
\usepackage{mathtools}
\usepackage{physics}
\usepackage{tabularx}
\usepackage{titling}
\usepackage{fancyhdr}
\usepackage{xfrac}
\usepackage{pgfplots}

\pgfplotsset{compat = newest}
\usetikzlibrary{intersections}
\usetikzlibrary{patterns}
\usepgfplotslibrary{fillbetween}

\author{Karsten Lehmann (Übungsgruppe 1)\\Mat. Nr 4935758}
\date{WiSe 2021/2022}
\title{Hausaufgaben Blatt 06\\Analysis - Grundlegende Konzepte}

\setlength{\headheight}{26pt}
\pagestyle{fancy}
\fancyhf{}
\lhead{\thetitle}
\rhead{\theauthor}
\lfoot{\thedate}
\rfoot{Seite \thepage}

\begin{document}
\paragraph{28.} Es sei $K$ ein angeordneter Körper und $1_K$ das neutrale Element
der Multiplikation.
Die Abbildung $f \colon \mathbb{Q} \to K$ sei wie in der Vorlesung definiert
durch
\[
  f(q) = f\qty(\frac{m}{n}) \coloneqq \frac{m1_k}{n1_K} \coloneqq
  \qty\big(m1_K) \cdot \qty\big(n1_K)^{-1}
  \text{ für } q = \frac{m}{n} \in \mathbb{Q}
\]

\underline{Beweisen Sie:} Für alle $p, q \in \mathbb{Q}$ gilt
\[
  p \leq q \iff f(p) \leq f(q)
\]

\subparagraph{Lsg.} Seien $p$ und $q$ beliebig.
Die Repräsentanten $\frac{m_p}{n_p}$ (von $p$) und $\frac{m_q}{n_q}$ (von $q$)
seien so gewählt, dass $n_p \cdot 1_K > 0$ und $n_q \cdot 1_K > 0$ sind.
Es folgt, dass $\qty\big(n_p \cdot 1_K)^{-1} > 0$ und
$\qty\big(n_q \cdot 1_K)^{-1} > 0$.

\begin{flalign*}
  p \leq q \iff && m_p \cdot n_q &\leq m_q \cdot n_p
  && {\Big |} \cdot \qty(1_K)^2 > 0 \\
  && \qty(1_K)^2 \cdot m_p \cdot n_q &\leq \qty(1_K)^2 \cdot m_q \cdot n_p \\
  && \qty(m_p1_k) \cdot \qty(n_q1_K) &\leq \qty(m_q1_K) \cdot \qty(n_p1_K)
  && {\Big |} \cdot \qty(n_q1_K)^{-1} \\
  && \qty(m_p1_k) \cdot \qty(n_q1_K) \cdot \qty(n_q1_K)^{-1}
  &\leq \qty(m_q1_K) \cdot \qty(n_p1_K) \cdot \qty(n_q1_K)^{-1} \\
  && \qty(m_p1_k) \cdot 1 &\leq \qty(m_q1_K) \cdot \qty(n_p1_K)
  \cdot \qty(n_q1_K)^{-1} && {\Big |} \cdot \qty(n_p1_K)^{-1} \\
  && \qty(m_p1_k) \cdot \qty(n_p1_K)^{-1} &\leq \qty(m_q1_K) \cdot \qty(n_q1_K)^{-1}
  && \iff f(p) \leq f(q)
\end{flalign*}

\end{document}
