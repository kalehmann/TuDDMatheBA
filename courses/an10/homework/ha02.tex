\documentclass{scrreprt}

\usepackage{aligned-overset}
\usepackage{amsmath}
\usepackage{amssymb}
\usepackage{bm}
\usepackage[shortlabels]{enumitem}
\usepackage{hyperref}
\usepackage[utf8]{inputenc}
\usepackage{multicol}
\usepackage{mathtools}
\usepackage{physics}
\usepackage{tabularx}
\usepackage{titling}
\usepackage{fancyhdr}
\usepackage{xfrac}
\usepackage{pgfplots}

\pgfplotsset{compat = newest}
\usetikzlibrary{intersections}
\usetikzlibrary{patterns}
\usepgfplotslibrary{fillbetween}

\author{Karsten Lehmann (Übungsgruppe 1)\\Mat. Nr 4935758}
\date{WiSe 2021/2022}
\title{Hausaufgaben Blatt 04\\Analysis - Grundlegende Konzepte}

\setlength{\headheight}{26pt}
\pagestyle{fancy}
\fancyhf{}
\lhead{\thetitle}
\rhead{\theauthor}
\lfoot{\thedate}
\rfoot{Seite \thepage}

\begin{document}
\paragraph{19. Seien $X, Y, Z$ Mengen} und $f \colon X \to Y$ sowie
$g \colon Y \to Z$ Abbildungen.
Im Folgenden betrachten wir ihre Verknüpfungen $g \circ f \colon X \to Z$.
Beweisen Sie folgende Aussagen:
\begin{enumerate}[(a)]
\item Sind $f$ und $g$ injektiv, so ist $g \circ f$ injektiv.

  \subparagraph{Lsg.} Seien $a, b \in X$ so gewählt, dass
  $(g \circ f)(a) = (g \circ f)(b)$.
  Die Verknüpfung $(g \circ f)$ ist injektiv, wenn nun $x = y$ folgt.

  $(g \circ f)(x) = g(f(x)) = g(f(y)) = (g \circ f)(y)$.
  Da $g$ injektiv ist, folgt $f(x) = f(y)$.
  Da $f$ ebenfalls injektiv ist, folgt $x = y$.

  $\Rightarrow g \circ f$ ist injektiv.

\item Sind $f$ und $g$ surjektiv, so ist $g \circ f$ surjektiv.

  \subparagraph{Lsg.} Sei $z \in Z$.
  Da $g$ surjektiv ist, existiert $y \in Y$ mit $g(y) = z$.
  Da $f$ ebenfalls surjektiv ist, existiert $x \in X$
  $f(x) = y$ und $g(f(x)) = (g \circ f)(x) = z$.

  $\Rightarrow g \circ f$ ist surjektiv.

\item Sind $g \circ f$ injektiv, so ist $f$ injektiv.

  \subparagraph{Lsg.} Angenommen es existieren $a \ne b \in X$ mit
  $f(a) = f(b)$.
  Daraus folgt $g(f(a)) = (g \circ f)(a) = (g \circ f)(b) = g(f(b))$
  - ein Widerspruch zu ``$g \circ f$ ist injektiv''.

  $\Rightarrow$ es existieren keine $a \ne b \in X$ mit $f(a) = f(b)$

  $\Rightarrow f$ ist injektiv.

\item Sind $g \circ f$ surjektiv, so ist $g$ surjektiv.

  \subparagraph{Lsg.} Da $g \circ f$ surjektiv ist, existiert für jedes
  $z \in Z$ ein $x \in X$ mit $f(x) = y \in Y$ und $g(y) = g(f(x)) = z$.

  Angenommen $g$ wäre nicht surjektiv,
  dann existiert ein $z \in Z$ mit $\forall y \in Y \colon g(y) \ne z$ - ein
  Widerspruch.

  $\Rightarrow g$ ist surjektiv.
\end{enumerate}

\end{document}