\documentclass{scrreprt}

\usepackage{aligned-overset}
\usepackage{amsmath}
\usepackage{amssymb}
\usepackage{bm}
\usepackage[shortlabels]{enumitem}
\usepackage{hyperref}
\usepackage[utf8]{inputenc}
\usepackage{multicol}
\usepackage{mathtools}
\usepackage{physics}
\usepackage{tabularx}
\usepackage{titling}
\usepackage{fancyhdr}
\usepackage{xfrac}
\usepackage{pgfplots}

\pgfplotsset{compat = newest}
\usetikzlibrary{intersections}
\usetikzlibrary{patterns}
\usepgfplotslibrary{fillbetween}

\author{Karsten Lehmann (Übungsgruppe 1)\\Mat. Nr 4935758}
\date{WiSe 2021/2022}
\title{Hausaufgaben Blatt 04\\Analysis - Grundlegende Konzepte}

\setlength{\headheight}{26pt}
\pagestyle{fancy}
\fancyhf{}
\lhead{\thetitle}
\rhead{\theauthor}
\lfoot{\thedate}
\rfoot{Seite \thepage}

\begin{document}
\paragraph{19. Seien $X, Y, Z$ Mengen} und $f \colon X \to Y$ sowie
$g \colon Y \to Z$ Abbildungen.
Im Folgenden betrachten wir ihre Verknüpfungen $g \circ f \colon X \to Z$.
Beweisen Sie folgende Aussagen:
\begin{enumerate}[(a)]
\item Sind $f$ und $g$ injektiv, so ist $g \circ f$ injektiv.

  \subparagraph{Lsg.} Seien $a, b \in X$ so gewählt, dass
  $(g \circ f)(a) = (g \circ f)(b)$.
  Die Verknüpfung $(g \circ f)$ ist injektiv, wenn nun $x = y$ folgt.

  $(g \circ f)(x) = g(f(x)) = g(f(y)) = (g \circ f)(y)$.
  Da $g$ injektiv ist, folgt $f(x) = f(y)$.
  Da $f$ ebenfalls injektiv ist, folgt $x = y$.

  $\Rightarrow g \circ f$ ist injektiv.

\item Sind $f$ und $g$ surjektiv, so ist $g \circ f$ surjektiv.

  \subparagraph{Lsg.} Sei $z \in Z$.
  Da $g$ surjektiv ist, existiert $y \in Y$ mit $g(y) = z$.
  Da $f$ ebenfalls surjektiv ist, existiert $x \in X$
  $f(x) = y$ und $g(f(x)) = (g \circ f)(x) = z$.

  $\Rightarrow g \circ f$ ist surjektiv.

\item Sind $g \circ f$ injektiv, so ist $f$ injektiv.

  \subparagraph{Lsg.} Angenommen es existieren $a \ne b \in X$ mit
  $f(a) = f(b)$.
  Daraus folgt $g(f(a)) = (g \circ f)(a) = (g \circ f)(b) = g(f(b))$
  - ein Widerspruch zu ``$g \circ f$ ist injektiv''.

  $\Rightarrow$ es existieren keine $a \ne b \in X$ mit $f(a) = f(b)$

  $\Rightarrow f$ ist injektiv.

\item Sind $g \circ f$ surjektiv, so ist $g$ surjektiv.

  \subparagraph{Lsg.} Da $g \circ f$ surjektiv ist, existiert für jedes
  $z \in Z$ ein $x \in X$ mit $f(x) = y \in Y$ und $g(y) = g(f(x)) = z$.

  Angenommen $g$ wäre nicht surjektiv,
  dann existiert ein $z \in Z$ mit $\forall y \in Y \colon g(y) \ne z$ - ein
  Widerspruch.

  $\Rightarrow g$ ist surjektiv.
\end{enumerate}

\newpage
\paragraph{20. Beweisen Sie,} dass Addition und Multiplikation auf den
natürlichen Zahlen kommutativ sind, dass also für alle $m, n \in \mathbb{N}$
gilt:
\begin{enumerate}[(a)]
\item $m + 0 = 0 + m$ und $m + 1$ und $1 + m$

  \subparagraph{Lsg.} durch vollständig Induktion nach $m$. \\
  \begin{minipage}[t]{.45\textwidth}
    \underline{Behauptung:} $m + 0 = 0 + m$ \\
    \underline{Induktionsanfang:} $0 + 0 = 0 + 0$ \\
    \underline{Induktionsschritt:}  Sei $m + 0 = 0 + m$ für
    ein beliebiges $m \in \mathbb{N}$ wahr.
    \begin{flalign*}
      v(m) + 0 &\overset{\text{Def. Addition}}= v(m) = v(m + 0) \\
      \overset{\text{Ind. Vor.}}&= v(0 + m) \\
      &= 0 + v(m)
    \end{flalign*}

    $\Rightarrow$ Nach vollständiger Induktion ist $m + 0 = 0 + m$
  \end{minipage}
  \hfill
  \vrule
  \hfill
  \begin{minipage}[t]{.45\textwidth}
    \underline{Behauptung:} $m + 1 = 1 + m$ \\
    \underline{Induktionsanfang:} $0 + 1 = 1 + 0$ wurde bereits auf der linken
    Seite gezeigt. \\
    \underline{Induktionsschritt:}  Sei $m + 1 = 1 + m$ für
    ein beliebiges $m \in \mathbb{N}$ wahr.
    \begin{flalign*}
      v(m) + 1 \overset{v(m) = m + 1}&= (m + 1) + 1 \\
      \overset{\text{Ind. Vor.}}&= (1 + m) + 1 \\
      \overset{\text{Assozitivität}}&= 1 + (m + 1) \\
      &= 1 + v(m)
    \end{flalign*}

    $\Rightarrow$ Nach vollständiger Induktion ist \\
    $m + 1 = 1 + m$
  \end{minipage}

\item $m + n = n + m$

  \subparagraph{Lsg.} durch vollständige Induktion nach $m$:
  Sei $n \in \mathbb{N}$ beliebig.

  \underline{Behauptung:} $m + n = n + m$ \\
  \underline{Induktionsanfang:} $0 + n = n + 0$ und $1 + n = n + 1$
  sind nach (a) wahr. \\
  \underline{Induktionsschritt:} Sei $m + n = n + m$ für ein beliebiges
  $m \in \mathbb{N}$ wahr.
  \begin{flalign*}
    v(m) + n &= (1 + m) + n \\
    \overset{\text{(a)}}&= (m + 1) + n \\
    \overset{\text{Assoziativität}}&= m + (1 + n) \\
    \overset{\text{(a)}}&= m + (n + 1) \\
    \overset{\text{Assoziativität}}&= (m + n) + 1 \\
    \overset{\text{Ind. Vor.}}&= (n + m) + 1 \\
    \overset{\text{Assoziativität}}&= n + (m + 1) \\
    &= n + v(m)
  \end{flalign*}

  $\Rightarrow$ Nach vollständiger Induktion ist \\
  $m + n = n + m$

\newpage
\item $n \cdot m = m \cdot n$

  \subparagraph{Lsg.} Analog der Addition: \\
  \begin{minipage}[t]{.45\textwidth}
    \underline{Behauptung:} $n \cdot 0 = 0 \cdot n$ \\
    \underline{Induktionsanfang:} $0 \cdot 0 = 0 \cdot 0$ \\
    \underline{Induktionsschritt:}  Sei $n \cdot 0 = 0 \cdot n$ für
    ein beliebiges $n \in \mathbb{N}$ wahr.
    \begin{flalign*}
      v(n) \cdot 0 &= 0 = 0 + 0 \\
      &= n \cdot 0 + 0 \\
      \overset{\text{Ind. Vor.}}&= 0 \cdot n + 0 \\
      &= 0 \cdot v(n)
    \end{flalign*}

    $\Rightarrow$ Nach vollständiger Induktion ist $n \cdot 0 = 0 \cdot n$
  \end{minipage}
  \hfill
  \vrule
  \hfill
  \begin{minipage}[t]{.45\textwidth}
    \underline{Behauptung:} $n \cdot 1 = n = 1 \cdot n$ \\
    \underline{Induktionsanfang:} $0 \cdot 1 = 0 = 1 \cdot 0$
    nach der Definition der Multiplikation und der linken Seite. \\
    \underline{Induktionsschritt:}  Sei $n \cdot 1 = n = 1 \cdot n$ für
    ein beliebiges $n \in \mathbb{N}$ wahr.
    \begin{flalign*}
      1 \cdot v(n) &= 1 \cdot n + 1 \\
      \overset{\text{Ind. Vor.}}&= v(n) \\
      &= v(0) \cdot 0 + v(n) \\
      &= v(n) \cdot v(0) = v(n) \cdot 1
    \end{flalign*}

    $\Rightarrow$ Nach vollständiger Induktion ist \\
    $n \cdot 1 = n = 1 \cdot n$
  \end{minipage}

  Weiterhin ist zu zeigen, dass $v(m) \cdot n = m \cdot n + n$.
  Sei $m \in \mathbb{N}$ beliebig:

  \underline{Induktionsanfang:} $v(m) \cdot 0 = 0 = m \cdot 0 + 0$
  Auch $v(m) \cdot 1 = v(m) = m + 1 = m \cdot 1 + 1$ ist wahr.

  \underline{Induktionsschritt:} Sei $v(m) \cdot n = m \cdot n + n$ für ein
  beliebiges $n \in \mathbb{N}$ wahr.
  \begin{flalign*}
    v(m) \cdot v(n) \overset{\text{Def. Multiplikation}}&= v(m) \cdot n + v(m) && \\
    \overset{\text{Ind. Vor.}}&= m \cdot n + n + v(m) \\
    \overset{n + (m + 1) = m + (n + 1)}&= m \cdot n + m + v(n) \\
    \overset{\text{Def. Multiplikation}}&= m \cdot v(n) + v(n)
  \end{flalign*}
  $\Rightarrow$ Nach der vollständigen Induktion ist
  $v(m) \cdot n = m \cdot n + n$. \\

  Schließlich ist die Kommutativität der Multiplikation zu zeigen:

  \underline{Behauptung}: $m \cdot n = n \cdot m$

  \underline{Induktionsanfang:} Sei $m \in \mathbb{N}$ beliebig.
  Wie bereits gezeigt sind $m \cdot 0 = 0 \cdot m$ und $m \cdot 1 = 1 \cdot m$
  wahr.

  \underline{Induktionsschritt:}
  \begin{flalign*}
    m \cdot v(n) \overset{\text{Def. Multiplikation}}&= m \cdot n + m \\
    \overset{\text{Ind. Vor.}}&= n \cdot m + m \\
    &= v(n) \cdot m
  \end{flalign*}
  $\Rightarrow$ nach der vollständigen Induktion folgt die Behauptung.
\end{enumerate}

\end{document}