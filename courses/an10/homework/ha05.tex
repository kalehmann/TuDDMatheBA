\documentclass{scrreprt}

\usepackage{aligned-overset}
\usepackage{amsmath}
\usepackage{amssymb}
\usepackage{bm}
\usepackage[shortlabels]{enumitem}
\usepackage{hyperref}
\usepackage[utf8]{inputenc}
\usepackage{multicol}
\usepackage{mathtools}
\usepackage{physics}
\usepackage{tabularx}
\usepackage{titling}
\usepackage{fancyhdr}
\usepackage{xfrac}
\usepackage{pgfplots}

\pgfplotsset{compat = newest}
\usetikzlibrary{intersections}
\usetikzlibrary{patterns}
\usepgfplotslibrary{fillbetween}

\author{Karsten Lehmann (Übungsgruppe 1)\\Mat. Nr 4935758}
\date{WiSe 2021/2022}
\title{Hausaufgaben Blatt 07\\Analysis - Grundlegende Konzepte}

\setlength{\headheight}{26pt}
\pagestyle{fancy}
\fancyhf{}
\lhead{\thetitle}
\rhead{\theauthor}
\lfoot{\thedate}
\rfoot{Seite \thepage}

\begin{document}
\paragraph{36.} Es seien $n \in \mathbb{N}_{> 0}$ und
$x_1, \ldots, x_n \in \mathbb{R}_{> 0}$.
Beweisen Sie:
\[
  \qty(\sum_{k = 1}^n \frac{x_k}{n})^2 \leq \sum_{k = 1}^n \frac{x_k^2}{n}
\]

\subparagraph{Lsg.} Seien $x, y \in \mathbb{R}_{> 0}$ beliebig.
Es gilt $(x - y)^2 \geq 0$.
Weiterhin ist \\
$\qty\big(x - y)^2 = x^2 + y^2 - 2xy$.
Es folgt
\begin{equation}
  \tag{*}
  x^2 + y^2 \geq 2xy
\end{equation}
\begin{flalign*}
  &&\qty(\sum_{k = 1}^n \frac{x_k}{n})^2 &\leq \sum_{k = 1}^n \frac{x_k^2}{n} \\
  &&\qty(\frac{1}{n})^2 \qty(\sum_{k = 1}^n x_k)^2
  &\leq \frac{1}{n} \sum_{k = 1}^n x_k^2 && {\Big |} \cdot n^2 \\
  \text{\underline{Behauptung:}} &&
  P(n) \colon \qty(\sum_{k = 1}^n x_k)^2 &\leq n \cdot \sum_{k = 1}^n x_k^2
\end{flalign*}
\underline{Induktionsanfang:} $P(1) \colon x_1^2 \leq x_1^2$ \\
\underline{Induktionsschritt:} Sei $P(n)$ für ein beliebiges $n \in \mathbb{N}$
wahr.
\begin{flalign*}
  P(n + 1) \colon \qty(\sum_{k = 1}^{n + 1} x_k)^2
  &= \qty(\sum_{k = 1}^n x_k + x_{n + 1})^2 \\
  &= \qty(\sum_{k = 1}^n x_k)^2 +
    \qty(\sum_{k = 1}^n 2x_kx_{n + 1}) + x_{n + 1}^2 \\
  \overset{\text{Siehe (*)}}&\leq \qty(\sum_{k = 1}^n x_k)^2 +
    \qty(\sum_{k = 1}^n x_k^2 + x_{n + 1}^2) + x_{n + 1}^2 \\
  \overset{\text{Induktionsvorraussetzung}}&\leq n \cdot
    \sum_{k = 1}^n x_k^2 +
    \qty(\sum_{k = 1}^n x_k^2 + x_{n + 1}^2) + x_{n + 1}^2 \\
  &= n \cdot \sum_{k = 1}^n x_k^2 + \sum_{k = 1}^n x_k^2 +
    \sum_{k = 1}^n x_{n + 1}^2 + x_{n + 1}^2 \\
  &= \qty\big(n + 1) \cdot \sum_{k = 1}^n x_k^2 +
    \sum_{k = 1}^{n + 1} x_{n + 1}^2
    = \qty\big(n + 1) \cdot \sum_{k = 1}^n x_k^2 +
    \qty\big(n + 1) \cdot x_{n + 1}^2 \\
  &= \qty\big(n + 1) \cdot \sum_{k = 1}^{n + 1} x_k^2
\end{flalign*}

$\Rightarrow$ nach der vollständigen Induktion folgt die Behauptung.

\end{document}
