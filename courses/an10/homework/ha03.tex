\documentclass{scrreprt}

\usepackage{aligned-overset}
\usepackage{amsmath}
\usepackage{amssymb}
\usepackage{bm}
\usepackage[shortlabels]{enumitem}
\usepackage{hyperref}
\usepackage[utf8]{inputenc}
\usepackage{multicol}
\usepackage{mathtools}
\usepackage{physics}
\usepackage{tabularx}
\usepackage{titling}
\usepackage{fancyhdr}
\usepackage{xfrac}
\usepackage{pgfplots}

\pgfplotsset{compat = newest}
\usetikzlibrary{intersections}
\usetikzlibrary{patterns}
\usepgfplotslibrary{fillbetween}

\author{Karsten Lehmann (Übungsgruppe 1)\\Mat. Nr 4935758}
\date{WiSe 2021/2022}
\title{Hausaufgaben Blatt 05\\Analysis - Grundlegende Konzepte}

\setlength{\headheight}{26pt}
\pagestyle{fancy}
\fancyhf{}
\lhead{\thetitle}
\rhead{\theauthor}
\lfoot{\thedate}
\rfoot{Seite \thepage}

\begin{document}
\paragraph{24.} Es sei $\overline{\mathbb{Z}}$ definiert wie in der Vorlesung.
Zeigen Sie:
\begin{enumerate}[(a)]
\item Die Multiplikation auf $\overline{\mathbb{Z}}$ ist unabhängig vom
  Repräsentanten.

\item Die Relation $\leq$ auf $\overline{\mathbb{Z}}$ ist unabhängig vom
  Repräsentanten und eine Ordnungsrelation.
\end{enumerate}

\subparagraph{Lsg.}
\begin{enumerate}[(a)]
\item Die Multiplikation auf $\overline{\mathbb{Z}}$ ist definiert mit
  \[
    \overline{m} \cdot \overline{n} = \qty\big[(m, m')] \cdot
    \qty\big[(n, n')] =
    \qty\big[(m \cdot n + m' \cdot n', m \cdot n' + m' \cdot n)]
  \]

  Seien $(a, a') \sim (m, m')$ und $(b, b') \sim (n, n')$.
  Dass heißt $\overline{a} = \overline{m}$ und $\overline{b} = \overline{n}$.
  \begin{equation}
    \tag{*}
    a + m' = a' + m \quad b + n' = b' + n
  \end{equation}
  \begin{flalign*}
    \colorbox{yellow}{an + a'n' +} &\colorbox{yellow}{mb' + m'b} +
      \colorbox{green!30}{mn + m'n + mn' + m'n'} & \\
    &= n \cdot (a + m') + n' \cdot (a' + m) + m \cdot (b' + n) + m'
      \cdot (b + n')\\
    \overset{(*)}&= n \cdot (a' + m) + n' \cdot (a + m') + m \cdot
      (b + n') + m' \cdot (b' + n) \\
    &= a'n + nm + n'a + n'm' + mb + mn' + m'b' + m'n \\
    &= \colorbox{yellow}{mb + m'b' + an' + a'n} +
      \colorbox{green!30}{mn + m'n + mn' + m'n'}
  \end{flalign*}
  \begin{flalign*}
    & \colorbox{yellow}{\text{$\Rightarrow$}} \colorbox{orange!30}{an + a'n'} +
      \colorbox{blue!20}{mb' + m'b} = \colorbox{blue!20}{mb + m'b'} +
      \colorbox{orange!30}{an' + a'n} & \\
    & \Rightarrow \qty\big(
        \colorbox{orange!30}{an + a'n'},
        \colorbox{orange!30}{an' + a'n}
      ) \sim \qty\big(
        \colorbox{blue!20}{mb + m'b'},
        \colorbox{blue!20}{mb' + m'b}
      ) \\
    & \Rightarrow (a, a') \cdot (n, n') = (b, b') \cdot (m, m')
  \end{flalign*}

\item Es gilt $\overline{m} \leq \overline{n} \iff m + n' \leq_{\mathbb{N}} m' + n$.
  Seien nun $(a, a') \sim (m, m')$ und $(b, b') \sim (n, n')$.
  Angenommen $\overline{m} \leq \overline{n}$, dann
  \begin{flalign*}
    m + n' &\leq m' + n & \\
    a + m' + m + n' &\leq m' + n + a' + m & \\
    a + n' &\leq a' + n  && \Rightarrow a \leq n & \\
    a + n' + b' + n &\leq a' + n + b + n' \\
    a + b' &\leq a' + b && \Rightarrow a \leq b
  \end{flalign*}

  Die Relation ``$\leq$'' heißt auf $\overline{\mathbb{Z}}$ ein
  Ordnungsrelation, falls
  \begin{enumerate}[(1)]
  \item $\forall \overline{a} \in \overline{\mathbb{Z}} \colon
    \overline{a} \leq \overline{a}$

  \item $\forall \; \overline{a}, \overline{b} \in \overline{\mathbb{Z}} \colon
    \overline{a} = \overline{b} \iff \overline{a} \leq \overline{b} \land
    \overline{b} \leq \overline{a}$

  \item $\forall \overline{a}, \overline{b}, \overline{c} \in \overline{\mathbb{Z}} \colon
    \overline{a} \leq \overline{b} \land \overline{b} \leq \overline{c}
    \Rightarrow \overline{a} \leq \overline{c}$
  \end{enumerate}

  \newpage
  \begin{enumerate}[(1)]
  \item Sei $\overline{a} \in \overline{\mathbb{Z}}$ beliebig.
    \[
      \overline{a} \leq \overline{a} \iff a + a' \leq_{\mathbb{N}} a' + a
    \]
    Aus der Kommutativität der natürlichen Zahlen folgt die Annahme, somit ist
    ``$\leq$'' unter $\overline{\mathbb{Z}}$ \emph{reflexiv}.

  \item Seien $\overline{a}, \overline{b} \in \overline{\mathbb{Z}}$ beliebig.
    \begin{itemize}
    \item[``$\Rightarrow$''] Angenommen $\overline{a} = \overline{b}$.
      Dann gilt $(a, a') \sim (b, b')$ und $a + b' = a' + b$.

      Somit sind $a + b' \leq a' + b$ und $b + a' \leq b' +a$ wahr.

      $\Rightarrow \overline{a} \leq \overline{b} \land
      \overline{b} \leq \overline{a}$

    \item[``$\Leftarrow$''] Angenommen es gilt
      $\overline{a} \leq \overline{b} \land \overline{b} \leq \overline{a}$.
      Dann fogt $a + b' \leq_{\mathbb{N}} a' + b$ und
      $b + a' \leq_{\mathbb{N}} b' +a$.
      Da ``$\leq$'' unter $\mathbb{N}$ eine Ordnungsrelation ist, folgt \\
      $a + b' = a' + b$.

      $\Rightarrow (a' a') \sim (b, b')$

      $\Rightarrow \overline{a} = \overline{b}$
    \end{itemize}

  \item Seien $\overline{a}, \overline{b}, \overline{c} \in
    \overline{\mathbb{Z}}$ und $\overline{a} \leq \overline{b}$ sowie
    $\overline{b} \leq \overline{c}$.
    \begin{flalign*}
      a + b' &\leq a' + b & \\
      a + b' + b + c' &\leq a' + b + b' + c \\
      a + c' &\leq a' + c && \Rightarrow \overline{a} \leq \overline{c} &
    \end{flalign*}
  \end{enumerate}

  $\Rightarrow$ ``$\leq$'' ist Ordnung unter $\overline{\mathbb{Z}}$.
\end{enumerate}

\paragraph{25. Zeigen Sie, dass $\mathbb{Q}$ ein Körper ist.}

\:\\
\underline{Hinweis:} Es kann ohne Beweis angenommen werden, dass die
Rechenoperationen auf $\mathbb{Q}$ wohldefiniert, dass heißt unabhängig vom
Repräsentanten sind.
Des Weiteren können Sie ohne Beweis die Kommutativität und Assoziativität
für beide Rechenoperationen auf $\mathbb{Z}$, sowie die Distributivität
annehmen.

\subparagraph{Lsg.}
Eine Menge $K$ heißt Körper, falls man auf $K$ Addition und Multiplikation hat
mit
\begin{enumerate}[(1)]
\item Es gibt neutrale Elemente $0 \in K$ und $1 \in K_{\ne 0}$
\item Addition und Multiplikation sind jeweils kommutativ und assoziativ
\item Addition und Multiplikation sind distributiv
\item Es gibt Inverse, dass heißt $\forall k \in K$ existiert $-k$,
  $\forall k \in K_{\ne 0}$ existiert $k^{-1}$
\end{enumerate}

\newpage
\subparagraph{(1) Neutrale Elemente}
Die Addition auf $\mathbb{Q}$ ist definiert als
\[
  \qty[\frac{m}{m'}] + \qty[\frac{n}{n'}] = \qty[\frac{mn' + m'n}{m'n'}]
\]
Ein neutrales Element $0 \in \mathbb{Q}$ der Addition existiert, falls
$\forall m \in \mathbb{Q} \colon m + 0 = 0 + m = m$.
Sei nun $0 = \qty[\frac{0}{1}]$ und $m = \qty[\frac{m}{m'}] \in \mathbb{Q}$
beliebig.
Dann ist
\[
  \qty[\frac{m}{m'}] + \qty[\frac{0}{1}] =
  \qty[\frac{m \cdot 1 + m' \cdot 0}{m' \cdot 1}]
  = \qty[\frac{m}{m'}]
  = \qty[\frac{1 \cdot m + 0}{1 \cdot m'}]
  = \qty[\frac{1 \cdot m + 0 \cdot m'}{1 \cdot m'}]
  = \qty[\frac{0}{1}] + \qty[\frac{m}{m'}]
\]

$\Rightarrow 0 = \qty[\frac{0}{1}]$ ist neutrales Element der Addition auf
$\mathbb{Q}$.

Die Multiplikation auf $\mathbb{Q}$ ist definiert als
\[
  \qty[\frac{m}{m'}] \cdot \qty[\frac{n}{n'}] = \qty[\frac{mn}{m'n'}]
\]

Ein neutrales Element $1 \in \mathbb{Q}$ der Multplikation existiert, falls
$\forall m \in \mathbb{Q} \colon m \cdot 1 = 1 \cdot m = m$.
Sei nun $1 = \qty[\frac{1}{1}]$ und $\qty[\frac{m}{m'}] \in \mathbb{Q}$
beliebig.
Dann ist
\[
  \qty[\frac{m}{m'}] + \qty[\frac{1}{1}] =
  \qty[\frac{m \cdot 1}{m' \cdot 1}]
  = \qty[\frac{m}{m'}]
  = \qty[\frac{1 \cdot m}{1 \cdot m'}]
  = \qty[\frac{1}{1}] + \qty[\frac{m}{m'}]
\]

$\Rightarrow 1 = \qty[\frac{1}{1}]$ ist neutrales Element der Multiplikation auf
$\mathbb{Q}$.

\subparagraph{(2) Assoziativität und Kommutativität für Addition und
  Multiplikation}

\:\\
Seien $\qty[\frac{m}{m'}], \qty[\frac{n}{n'}], \qty[\frac{o}{o'}] \in \mathbb{Q}$ beliebig.
Dann ist
\begin{flalign*}
  \qty[\frac{m}{m'}] + \qty(\qty[\frac{n}{n'}] + \qty[\frac{o}{o'}])
  &= \qty[\frac{m}{m'}] + \qty[\frac{no' + n'o}{n'o'}] \\
  &= \qty[\frac{mn'o' + m' \cdot \qty\big(no' + n'o)}{m'n'o'}] \\
  &= \qty[\frac{mn'o' + m'no' + m'n'o}{m'n'o'}] \\
  &= \qty[\frac{\qty\big(mn' + m'n) \cdot o' + m'n'o}{m'n'o'}] \\
  &= \qty[\frac{mn' + m'n}{m'n'}] + \qty[\frac{o}{o'}] \\
  &= \qty(\qty[\frac{m}{m'}] + \qty[\frac{n}{n'}]) + \qty[\frac{o}{o'}]
\end{flalign*}
$\Rightarrow$ die Addition ist assoziativ unter $\mathbb{Q}$.

\newpage
\begin{flalign*}
  \qty[\frac{m}{m'}] \cdot \qty(\qty[\frac{n}{n'}] \cdot \qty[\frac{o}{o'}])
  &= \qty[\frac{m}{m'}] + \qty[\frac{no}{n'o'}] \\
  &= \qty[\frac{m \cdot \qty(no)}{m' \cdot \qty(n'o')}] \\
  \overset{\text{Assoziativität auf $\mathbb{Z}$}}&=
  \qty[\frac{\qty(mn) \cdot o}{\qty(m'n') \cdot o'}] \\
  &= \qty[\frac{mn}{m'n'}] \cdot \qty[\frac{o}{o'}] \\
  &= \qty(\qty[\frac{m}{m'}] \cdot \qty[\frac{n}{n'}]) \cdot \qty[\frac{o}{o'}]
\end{flalign*}

$\Rightarrow$ die Multiplikation ist assoziativ unter $\mathbb{Q}$.

\begin{flalign*}
  \qty[\frac{m}{m'}] + \qty[\frac{n}{n'}] \overset{\text{Def. Add.}}&=
  \qty[\frac{mn' + m'n}{m'n'}] & \\
  \overset{\text{Kommutativität in $\mathbb{Z}$}}&= \qty[\frac{nm' + n'm}{n'm'}] \\
  &= \qty[\frac{n}{n'}] + \qty[\frac{m}{m'}]
\end{flalign*}
$\Rightarrow$ Addition ist kommutativ.

\begin{flalign*}
  \qty[\frac{m}{m'}] \cdot \qty[\frac{n}{n'}] \overset{\text{Def. Mul.}}&=
  \qty[\frac{mn}{m'n'}] & \\
  \overset{\text{Kommutativität in $\mathbb{Z}$}}&= \qty[\frac{nm}{n'm'}] \\
  &= \qty[\frac{n}{n'}] \cdot \qty[\frac{m}{m'}]
\end{flalign*}
$\Rightarrow$ Multiplikation ist kommutativ.

\subparagraph{(3) Distributivität unter $\mathbb{Q}$}

\:\\
Seien $\qty[\frac{m}{m'}], \qty[\frac{n}{n'}], \qty[\frac{o}{o'}] \in
\mathbb{Q}$.
Dann gilt
\begin{flalign*}
  \qty[\frac{m}{m'}] \cdot \qty(\qty[\frac{n}{n'}] + \qty[\frac{o}{o'}])
  &= \qty[\frac{m}{m'}] \cdot \qty[\frac{no' + n'o}{n'o'}]
  = \qty[\frac{m \cdot \qty\big(no' + n'o)}{m' \cdot \qty\big(n'o')}] \\
  &= \qty[\frac{mno' + mn'o}{m'n'o'}] \\
  &= \qty[\frac{mno'}{m'n'o'} + \frac{mn'o}{m'n'o'}]
  = \qty[\frac{mn}{m'n'} + \frac{mo}{m'o'}] \\
  &= \qty[\frac{mn}{m'n'}] + \qty[\frac{mo}{m'o'}] \\
  &= \qty[\frac{m}{m'}] \cdot \qty[\frac{n}{n'}] + \qty[\frac{m}{m'}] \cdot \qty[\frac{o}{o'}]
\end{flalign*}
$\Rightarrow$ Addition und Multiplikation sind distributiv unter $\mathbb{Z}$.

\newpage
\subparagraph{(4) Inverse Elemente}

\:\\
Für die Addition existieren inverse Elemente, falls
$\forall \: m \in \mathbb{Q} \: \exists \: -m \in \mathbb{Q} \colon m + (-m)
= 0$.
Sei nun $\qty[\frac{m}{m'}] \in \mathbb{Q}$ beliebig.
Für jede ganze Zahl $m \in \mathbb{Z}$ existiert ein inverses Element $-m$
bezüglich der Addition mit $m + (-m) = 0$.

\[
  \qty[\frac{m}{m'}] + \qty[\frac{(-m)}{m'}] = \qty[\frac{mm' + m'(-m)}{m'm'}]
  = \qty[\frac{mm' - mm'}{m'm'}] = \qty[\frac{0}{m'm'}] = 0
\]

$\Rightarrow$ Für jedes Element in $\mathbb{Q}$ existiert ein inverses
Element bezüglich der Addition.


Für die Multiplikation existieren inverse Elemente, falls
$\forall \: m \in \mathbb{Q} \: \exists \: m^{-1} \in \mathbb{Q}$ gilt: \\
$m \cdot m^{-1} = 1$.
Sei nun $m = \qty[\frac{m}{m'}] \in \mathbb{Q}_{\ne 0}$ beliebig.
Definiere $m^{-1}$ als $\qty[\frac{m'}{m}]$.
\[
  m \cdot m^{-1} = \qty[\frac{m}{m'}] \cdot \qty[\frac{m'}{m}]
  = \qty[\frac{mm'}{m'm}]
  \sim \qty[\frac{1}{1}]
\]

$\Rightarrow$ Für jedes Element in $\mathbb{Q}_{\ne 0}$ existiert ein inverses
Element für die Multiplikation.

\end{document}
