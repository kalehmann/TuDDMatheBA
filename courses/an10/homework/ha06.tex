\documentclass{scrreprt}

\usepackage{aligned-overset}
\usepackage{amsmath}
\usepackage{amssymb}
\usepackage{bm}
\usepackage[shortlabels]{enumitem}
\usepackage{hyperref}
\usepackage[utf8]{inputenc}
\usepackage{multicol}
\usepackage{mathtools}
\usepackage{physics}
\usepackage{tabularx}
\usepackage{titling}
\usepackage{fancyhdr}
\usepackage{xfrac}
\usepackage{pgfplots}

\pgfplotsset{compat = newest}
\usetikzlibrary{intersections}
\usetikzlibrary{patterns}
\usepgfplotslibrary{fillbetween}

\author{Karsten Lehmann (Übungsgruppe 1)\\Mat. Nr 4935758}
\date{WiSe 2021/2022}
\title{Hausaufgaben Blatt 08\\Analysis - Grundlegende Konzepte}

\setlength{\headheight}{26pt}
\pagestyle{fancy}
\fancyhf{}
\lhead{\thetitle}
\rhead{\theauthor}
\lfoot{\thedate}
\rfoot{Seite \thepage}

\begin{document}
\paragraph{40. Beweisen Sie die folgenden Aussagen:}
\begin{enumerate}[(a)]
\item Es sei $M$ eine beliebige nicht-leere Menge und
  $f, g \colon M \to \mathbb{R}$ nach oben beschränkte Funktionen.
  Zeigen sie die Ungleichung
  \[
    \sup\qty{f(x) + g(x) \: {\Big |} \: x \in M}
    \leq
    \sup\qty{f(x) \: {\Big |} \: x \in M} +
    \sup\qty{g(x) \: {\Big |} \: x \in M}
  \]

  \subparagraph{Lsg.} Das Supremum ist als obere Schranke einer Menge größer
  oder gleich jedem Element der Menge.
  Sei nun $F \coloneqq \qty{f(x) \: {\Big |} \: x \in M}$ und
  $G \coloneqq \qty{g(x) \: {\Big |} \: x \in M}$.
  Dann $\sup F \geq f(x) \: \forall \: x \in M$ und
  $\sup G \geq g(x) \: \forall \: x \in M$.
  \begin{flalign*}
    \sup\qty{f(x) + g(x) \: {\Big |} \: x \in M}
    &\leq \sup\qty{\sup F + g(x) \: {\Big |} \: x \in M} \\
    &\leq \sup\qty{\sup F + \sup G \: {\Big |} \: x \in M} = \sup F + \sup G \\
    \overset{\text{Def. } F \text{ und } G}&=
    \sup\qty{f(x) \: {\Big |} \: x \in M} +  \sup\qty{g(x) \: {\Big |} \: x \in M}
  \end{flalign*}
\end{enumerate}

\paragraph{42. Sei $X \coloneqq \mathbb{R}^2$.}
Skizzieren Sie

\begin{enumerate}[(a)]
\item für $p \in \qty\big{1, 2, \infty}$ die abgeschlossene Kugel vom Radius
  $r = 1$ um den Punkt $x = \qty\big(1, 1)$ bezüglich der $p$-Norm.

\item die abgeschlossene Kugel vom Radius $r_1 = \frac{1}{2}$ und $r_2 = 2$
  um den Punkt $x = \qty\big(0, 0)$ bezüglich der diskreten Metrik im
  $\mathbb{R}^2$.

  \subparagraph{Lsg.} Die diskrete Metrik ist definiert als
  $d\qty\big(x, y) \coloneqq \begin{cases}
    0 & x = y \\
    1 & x \ne y
  \end{cases}$.
  Somit ist
  \[
    B_{\frac{1}{2}}\qty\big((0, 0))
    = \qty{y \in \mathbb{R}^2 \: {\Big |} \: d\qty\big(x, y) < \frac{1}{2}}
    = \qty\big{\qty(0, 0)}
  \]
  und
  \[
    B_2\qty\big((0, 0))
    = \qty{y \in \mathbb{R}^2 \: {\Big |} \: d\qty\big(x, y) < 2}
    = \mathbb{R}^2
  \]
\end{enumerate}

\end{document}
