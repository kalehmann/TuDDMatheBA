\documentclass{scrreprt}

\usepackage{aligned-overset}
\usepackage{amsmath}
\usepackage{amssymb}
\usepackage{bm}
\usepackage[shortlabels]{enumitem}
\usepackage{hyperref}
\usepackage[utf8]{inputenc}
\usepackage{multicol}
\usepackage{mathtools}
\usepackage{physics}
\usepackage{tabularx}
\usepackage{titling}
\usepackage{fancyhdr}
\usepackage{xfrac}
\usepackage{pgfplots}

\pgfplotsset{compat = newest}
\usetikzlibrary{intersections}
\usetikzlibrary{patterns}
\usepgfplotslibrary{fillbetween}

\author{Karsten Lehmann (Übungsgruppe 1)\\Mat. Nr 4935758}
\date{WiSe 2021/2022}
\title{Hausaufgaben Blatt 08\\Analysis - Grundlegende Konzepte}

\setlength{\headheight}{26pt}
\pagestyle{fancy}
\fancyhf{}
\lhead{\thetitle}
\rhead{\theauthor}
\lfoot{\thedate}
\rfoot{Seite \thepage}

\begin{document}
\paragraph{40. Beweisen Sie die folgenden Aussagen:}
\begin{enumerate}[(a)]
\item Es sei $M$ eine beliebige nicht-leere Menge und
  $f, g \colon M \to \mathbb{R}$ nach oben beschränkte Funktionen.
  Zeigen sie die Ungleichung
  \[
    \sup\qty{f(x) + g(x) \: {\Big |} \: x \in M}
    \leq
    \sup\qty{f(x) \: {\Big |} \: x \in M} +
    \sup\qty{g(x) \: {\Big |} \: x \in M}
  \]

  \subparagraph{Lsg.} Das Supremum ist als obere Schranke einer Menge größer
  oder gleich jedem Element der Menge.
  Sei nun $F \coloneqq \qty{f(x) \: {\Big |} \: x \in M}$ und
  $G \coloneqq \qty{g(x) \: {\Big |} \: x \in M}$.
  Dann $\sup F \geq f(x) \: \forall \: x \in M$ und
  $\sup G \geq g(x) \: \forall \: x \in M$.
  \begin{flalign*}
    \sup\qty{f(x) + g(x) \: {\Big |} \: x \in M}
    &\leq \sup\qty{\sup F + g(x) \: {\Big |} \: x \in M} \\
    &\leq \sup\qty{\sup F + \sup G \: {\Big |} \: x \in M} = \sup F + \sup G \\
    \overset{\text{Def. } F \text{ und } G}&=
    \sup\qty{f(x) \: {\Big |} \: x \in M} +  \sup\qty{g(x) \: {\Big |} \: x \in M}
  \end{flalign*}
\end{enumerate}
\end{document}
