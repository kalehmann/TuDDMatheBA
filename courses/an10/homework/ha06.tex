\documentclass{article}
\usepackage{aligned-overset}
\usepackage{amsmath}
\usepackage{amssymb}
\usepackage{bm}
\usepackage[shortlabels]{enumitem}
\usepackage{hyperref}
\usepackage[utf8]{inputenc}
\usepackage{mathtools}
\usepackage{physics}
\usepackage{titling}
\usepackage{fancyhdr}
\usepackage{xfrac}

\author{Albina Oscherowa \\ Karsten Lehmann}
\date{WiSe 2020}
\title{Hausaufgabe 06 Analysis - Grundlegende Konzepte}

\pagestyle{fancy}
\fancyhf{}
\lhead{\thetitle}
\rhead{\theauthor}
\lfoot{\thedate}
\rfoot{Seite \thepage}

\begin{document}
\section*{Hausaufgabe 1}

Zeigen Sie unter Verwendung entsprechender Definitionen, dass die Folge $(a_n)_{n \in \mathbb{N}}$ mit
$a_n \coloneqq \frac{n^3 + 5n}{3n^3 - 6}, n \in \mathbb{N}$ gegen $\frac{1}{3}$ konvergiert und eine
Cauchy-Folge ist. \\

\noindent
Dann in der Vorlesung unter 2.1.2 bereits bewiesen wurde, dass alle konvergenten Folgen auch Cauchy-Folgen sind,
muss nur die Konvergenz bewiesen werden. Sei $\frac{1}{3}$ der Grenzwert der Folge $(a_n)$. Dann gilt

\[
  \forall \epsilon > 0 \exists n_0 \in \mathbb{N} \forall n \in \mathbb{N}_{\geq n_0} \colon \abs{a_n - \frac{1}{3}} < \epsilon
\]
Es gilt
\begin{align*}
  \abs{\frac{n^3 + 5n}{3n^3 - 6} - \frac{1}{3}} &= \abs{\frac{n^3 + 5n}{3n^3 - 6} - \frac{1(n^3 - 2)}{3(n^3 - 2)}} \\
                                                &= \abs{\frac{n^3 + 5n}{3n^3 - 6} - \frac{n^3 - 2}{3n^3 - 6}} \\
                                                &= \abs{\frac{5n + 2}{3n^3 - 6}} \\
  \text{Sei nun $n_0 \geq 2$} \\
                                                &= \frac{5n + 2}{3n^3 - 6} \\
                                                &= \frac{5n + 2}{n^3 + 2n^3  - 6} && | 2n^3 - 6 \text{ ist größer als 0 für } n_o \geq 2 \\
                                                &< \frac{5n + 2}{n^3} \\
                                                &= \frac{5 + \sfrac{2}{n}}{n^2} && \frac{2}{n} \leq 1 \text{ für alle } n_0 \geq 2 \\
                                                &= \frac{5 + 1}{n^2}  \\
                                                &\leq \frac{6}{n^2} \\
                                                &< \frac{6}{n} \\
                                                &< \epsilon
\end{align*}

für alle $n \in \mathbb{N}_{\geq n_0}$, wenn $n_0 \in \mathbb{N}$ so gewählt, dass $n_0 \geq 2 \land \frac{6}{n} < \epsilon$.
Damit ist der Grenzwert tatsächlich $\frac{1}{3}$\\

\newpage
\section*{Hausaufgabe 2}

Geben Sie jeweils zwei reelle Nullfolgen $(a_n)_{n \in \mathbb{N}}, (b_n)_{n \in \mathbb{N}}$ mit $b_n \ne 0$ für
alle $n \in \mathbb{N}$ an, so dass die Folge $(c_n)_{n \in \mathbb{N}}$ mit $c_n \coloneqq \frac{a_n}{b_n}$

\begin{enumerate}[a)]
\item eine Nullfolge ist
  \[
    a_n \coloneqq 0, b_n \coloneqq \frac{1}{n}
  \]
  \[
    \frac{a_n}{b_n} = \frac{0}{\sfrac{1}{n}} = 0 
  \]
  
\item gegen einen von Null verschiedenen Wert konvergiert
  \[
    a_n \coloneqq \frac{2}{n}, b_n \coloneqq \frac{1}{n}
  \]
  \[
    \frac{a_n}{b_n} = \frac{\sfrac{2}{n}}{\sfrac{1}{n}} = \frac{2}{n} \cdot \frac{n}{1} = \frac{2n}{n} = 2
  \]
  
\item bestimmt divergiert
  \[
    a_n \coloneqq \frac{1}{n}, b_n \coloneqq \frac{1}{n^2}
  \]

  \[
    \frac{a_n}{b_n} = \frac{\sfrac{1}{n}}{\sfrac{1}{n^2}} = \frac{1}{n} \cdot \frac{n^2}{1} = \frac{n^2}{n} = n
  \]
  
\item unbestimmt divergiert
  \[
    a_n \coloneqq \frac{(-1)^n}{n}, b_n \coloneqq \frac{1}{n}
  \]

  \[
    \frac{a_n}{b_n} = \frac{\sfrac{(-1)^n}{n}}{\sfrac{1}{n}} = \frac{(-1)^n}{n} \cdot \frac{n}{1} = \frac{(-1)^nn}{n} = (-1)^n
  \]

  
\end{enumerate}

\newpage
\section*{Hausaufgabe 3}

Untersuchen Sie die angegebenen Folgen $(a_n)_{n \in \mathbb{N}}$ auf Konvergenz und bestimmen Sie gegebenenfalls deren
Grenzwerte:

\begin{enumerate}[a)]
\item $a_n \coloneqq \text{sgn}\left(-\frac{1}{n^2}\right)^n$ \\
  Per Definition ist $n > 0$, somit ist auch $n^2 > 0$ und $\frac{1}{n^2} > 0$.
  Damit ist $\text{sgn}\left(-\frac{1}{n^2}\right) = -1$ und die Folge lässt sich als $(-1)^n$ beschreiben.

  \[
    a_n \coloneqq \begin{cases}
      -1 & n \text{ ist ungerade} \\
      1  & n \text{ ist gerade} \\
    \end{cases}
  \]

  Damit ist die Menge divergent und besitzt keinen Grenzwert.
  
\item $a_n \coloneqq \frac{\sfrac{(2+\sfrac{1}{n})}{n}}{1+\sfrac{1}{n}}$ \\
  $1 + \frac{1}{n}$ konvergiert gegen $1$ (Vorlesung 2.1.4 (a), (b), 2.1.5 (b)).
  Somit konvergiert auch $2 + \frac{1}{n}$ gegen $2$ und $\frac{2}{0}$ konvergiert gegen $0$.
  Damit konvergiert die Folge gegen $\frac{0}{1} = 0$.
\item $a_n \coloneqq \sqrt[3]{n+1} - \sqrt[3]{n}$ \\
  \begin{align*}
    \sqrt[3]{n+1} - \sqrt[3]{n} &= \sqrt[3]{n \cdot \left(1 + \frac{1}{n}\right)} - \sqrt[3]{n} \\
                                &= \sqrt[3]{n} \cdot \sqrt[3]{1 + \frac{1}{n}} - \sqrt[3]{n}
  \end{align*}
  Die Folge $\frac{1}{n}$ konvergiert gegen $0$ (Vorlesung 2.1.4), somit konvergiert $\sqrt[3]{1 + \frac{1}{n}}$
  gegen $\sqrt[3]{1} = 1$. Sei nun $\underset{n\to\infty}\lim \sqrt[3]{n} = b$, dann konvergiert die Folge gegen
  $b \cdot 1 - b = 0$.
\end{enumerate}

\newpage
\section*{Hausaufgabe 4}

Sei $A \subseteq \mathbb{R}$ eine nichtleere Menge. Beweisen Sie, dass es eine Folge $(a_n)$ aus $A$ gibt mit
$\underset{n\to\infty}\lim a_n = \inf A$.

\noindent
Für $\inf A$ gilt

\begin{enumerate}[(i)]
\item $\forall a \in A \colon a \geq \inf A$
\item $\forall \epsilon > 0 \exists a \in A \colon a \leq \inf A + \epsilon$.
\end{enumerate}
Angenommen es gibt nun ein $b \in A$ mit $a > b$ und $\epsilon = a - b > 0$, dann gilt nach (ii):

\begin{minipage}[t]{.45\textwidth}
  \textbf{Fall 1}: $a < \inf A + \epsilon$:
  \begin{align*}
    a &< \inf A + (a - b) \\
    a &< \inf A + a - b && | -a \\
    0 &< \inf A - b     && | +b \\
    b &< \inf A \\
  \end{align*}\
  Ein Widerspruch zu (i)
\end{minipage}
\begin{minipage}[t]{.45\textwidth}
  \textbf{Fall 2}: $a = \inf A + \epsilon$
  \begin{align*}
    a &= \inf A + (a - b) \\
    a &= \inf A + a - b && | -a \\
    0 &= \inf A - b     && | +b \\
    b &= \inf A \\
  \end{align*}
  Damit ist $a > \inf A + \frac{a - b}{2}$, ein Widerspruch zu (ii)
\end{minipage}
\\
Aus (ii) folgt:
\begin{align*}
  a &> \inf A - \epsilon && | - \inf A \\
  -\inf A + a &> \epsilon && | \cdot (-1) \\
  \inf A - a &< \epsilon \\
  \abs{a - \inf A} &< \epsilon \\ 
\end{align*}
Sei nun $(a_n)$ die Folge der Elemente aus $A$, geordnet nach der Größe beginnend mit dem größten Element.
Da kein Element in $A$ echt kleiner als $a$ ist, gilt für diese Folge:

\[
  \forall \epsilon > 0 \exists n_0 \in \mathbb{N}_{\leq \abs{A}} \exists n \in \left(\mathbb{N}_{\geq n_0} \setminus \mathbb{N}_{> \abs{A}}\right) \colon \abs{a_n - \inf A} < \epsilon
\]

Somit konvergiert die Folge $(a_n)$ gegen $\inf A$


\end{document}