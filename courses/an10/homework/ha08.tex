\documentclass{scrreprt}

\usepackage{aligned-overset}
\usepackage{amsmath}
\usepackage{amssymb}
\usepackage{bm}
\usepackage[shortlabels]{enumitem}
\usepackage{hyperref}
\usepackage[utf8]{inputenc}
\usepackage{multicol}
\usepackage{mathtools}
\usepackage{physics}
\usepackage{tabularx}
\usepackage{titling}
\usepackage{fancyhdr}
\usepackage{xfrac}
\usepackage{pgfplots}

\pgfplotsset{compat = newest}
\usetikzlibrary{intersections}
\usetikzlibrary{patterns}
\usepgfplotslibrary{fillbetween}

\author{Karsten Lehmann (Übungsgruppe 1)\\Mat. Nr 4935758}
\date{WiSe 2021/2022}
\title{Hausaufgaben Blatt 10\\Analysis - Grundlegende Konzepte}

\setlength{\headheight}{26pt}
\pagestyle{fancy}
\fancyhf{}
\lhead{\thetitle}
\rhead{\theauthor}
\lfoot{\thedate}
\rfoot{Seite \thepage}

\begin{document}
\paragraph{53. Sei $\mathbb{R}^2$ mit der euklidischen Metrik versehen.}
Entscheiden Sie, ob die folgenden Teilmengen von $\mathbb{R}^2$ kompakt sind.
\begin{enumerate}[(a)]
\item $\qty\big{\qty\big(n, n) \,{\big |}\, n \in \mathbb{N}}$

  \subparagraph{Lsg.} Sei
  $M \coloneqq \qty\big{\qty\big(n, n) \,{\big |}\, n \in \mathbb{N}}$
  und $\mathcal{U} \coloneqq
  \qty{B_{\frac{1}{3}}\qty\big(\qty(n, n)) \,\middle|\, n \in \mathbb{N}}$.
  Dann besteht $\mathcal{U}$ aus offenen Teilmengen von $\mathbb{R}^2$ und
  für jedes Element $m \in M$ findet sich ein $U \in \mathcal{U}$ mit
  $m \in U$, somit ist $\mathcal{U}$ eine offen Überdeckung von $M$.

  Sei nun $\tilde{\mathcal{U}}$ eine beliebige endliche Teilmenge von $U$.
  Dann umfasst jedes Element $P \in \tilde{\mathcal{U}}$ genau ein Element
  $m \in M$ (, denn
  $d\qty\big((n, n), (n + 1, n + 1)) = \sqrt{2} > \frac{1}{3}$).

  $\Rightarrow \tilde{\mathcal{P}}$ enthält nur endlich viele Elemente aus $M$

  $\Rightarrow M \not\subset \bigcup \tilde{\mathcal{U}}$

  $\Rightarrow$ die offene Überdeckung $U$ enthält keine endliche
  Überdeckung von $M$.

  $\Rightarrow$ \underline{$M$ ist nicht kompakt}.

\item $\qty{\qty(\frac{1}{n}, \frac{1}{n}) \,\middle |\, n \in \mathbb{N}_{> 0}}$

  \subparagraph{Lsg.} Sei $\qty\big{x_n}$ eine Folge in $\mathbb{R}^2$ mit
  $x_n \coloneqq \qty(\frac{1}{n}, \frac{1}{n})$.
  Dann gilt $x_n \overset{n \to \infty}\longrightarrow \qty\big(0, 0)$.
  Nach Satz 10.1 (1) der Vorlesung (\emph{``Sei $\qty\big{x_n}$ eine Folge im
    metrischen Raum $\qty\big(X, d)$, dann gilt
    $x_n \to x \Rightarrow \qty\big{x_n}$ ist Cauchy-Folge.''}).
  ist $\qty\big{x_n}$ ein Cauchy-Folge.

  Nun ist jedes Folgenglied von $\qty\big{x_n}$ in
  $\qty{\qty(\frac{1}{n}, \frac{1}{n}) \,\middle |\, n \in \mathbb{N}_{> 0}}$
  enthalten, allerdings ist $\qty\big(0, 0)$ nicht in der Menge enthalten.

  $\Rightarrow \qty\big{x_n}$ (und jede Teilfolge von $\qty\big{x_n}$)
  konvergiert gegen einen Wert außerhalb der Menge.

  $\Rightarrow
  \qty{\qty(\frac{1}{n}, \frac{1}{n}) \,\middle |\, n \in \mathbb{N}_{> 0}}$
  ist nicht folgenkompakt.

  Nach Theorem 11.1 der Vorlesung (\emph{``Sei $\qty\big(X, d)$ metrischer Raum
    und $M \subset X$, dann folgt $M$ kompakt $\iff M$ folgenkompakt''})
  ist $\qty{\qty(\frac{1}{n}, \frac{1}{n}) \,\middle |\, n \in \mathbb{N}_{> 0}}$
  \underline{nicht kompakt}.

\item $\qty\big{x \in \mathbb{R}^2 \,{\big |}\, \abs{x}_2 \leq 1}$

  \subparagraph{Lsg.} Sei $M \coloneqq
  \qty\big{x \in \mathbb{R}^2 \,{\big |}\, \abs{x}_2 \leq 1}$ und
  $x \in \mathbb{R}^2 \setminus M$ beliebig.
  Nun gilt $\abs{x}_2 > 1$, anderenfalls wäre $x \in M$.
  Es sei nun $\epsilon = \frac{d\qty\big(1, x)}{2}$, dann ist
  $B_{\epsilon}\qty\big(x) \subset \mathbb{R}^2 \setminus M$.

  $\Rightarrow \mathbb{R}^2 \setminus M$ ist offen

  $\Rightarrow M$ ist abgeschlossen.

  Außerdem ist $M$ abgeschlossen (vgl. Aufgabe 42 auf Blatt 08).

  Nach Theorem 11.3 der Vorlesung (\emph{``Satz von Heine-Borel: Sei
    $X = \mathbb{R}^n$ mit beliebiger Norm und $M \subset X$.
    Dann gilt $M$ ist kompakt $\iff M$ ist abgeschlossen und beschränkt.''})
  folgt, dass \underline{$M$ kompakt ist}.
\end{enumerate}

\newpage
\paragraph{54. Sei $X \coloneqq \qty\big(0, \infty) \subset \mathbb{R}$}.
\begin{enumerate}[(a)]
\item Laut Aufgabe 41 ist die Abbildung
  $d \colon X \times X \to \big[0, \infty \big)$, welche für $x, y \in X$
  durch
  \[
    d\qty\big(x, y) \coloneqq \abs{\frac{1}{x} - \frac{1}{y}}
  \]
  definiert ist, eine Metrik auf $X$.
  Zeigen Sie, dass die Folge $\qty\big{n}_{n \in \mathbb{N}_{> 0}}$ eine
  Cauchy-Folge bezüglich der Metrik $d$ ist.

  \subparagraph{Lsg.} Sei $\epsilon > 0$ beliebig.
  Dann findet sich ein $N \in \mathbb{N}$ mit $\frac{1}{N} < \epsilon$.
  Somit gilt für $m, n > N$, dass
  \[
    \abs{\frac{1}{m} - \frac{1}{n}} < \frac{1}{N} < \epsilon
  \]

  $\Rightarrow$ \underline{$\qty\big{n}_{n \in \mathbb{N}_{> 0}}$ ist eine
    Cauchy-Folge}

\item Ist der metrische Raum $\qty\big(X, d)$ vollständig?

  \subparagraph{Lsg.} Die Folge $\qty\big{n}_{n \in \mathbb{N}_{> 0}}$
  ist eine Cauchy-Folge (siehe (a)).
  Allerdings ist $\underset{n \to \infty}\lim n = \infty \notin X$.

  $\Rightarrow \qty\big{n}_{n \in \mathbb{N}_{> 0}}$ konvergiert in $X$ nicht.

  $\Rightarrow$ \underline{$\qty\big(X, d)$ ist nicht vollständig}
\end{enumerate}

\end{document}
