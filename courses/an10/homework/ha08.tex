\documentclass{article}
\usepackage{aligned-overset}
\usepackage{amsmath}
\usepackage{amssymb}
\usepackage{bm}
\usepackage[shortlabels]{enumitem}
\usepackage{hyperref}
\usepackage[utf8]{inputenc}
\usepackage{mathtools}
\usepackage{physics}
\usepackage{tabularx}
\usepackage{titling}
\usepackage{fancyhdr}
\usepackage{xfrac}

\author{Albina Oscherowa \\ Karsten Lehmann}
\date{WiSe 2020}
\title{Hausaufgabe 08 Analysis - Grundlegende Konzepte}

\pagestyle{fancy}
\fancyhf{}
\lhead{\thetitle}
\rhead{\theauthor}
\lfoot{\thedate}
\rfoot{Seite \thepage}

\begin{document}
\section*{Hausaufgabe 1}

Sei $x \in \mathbb{R}$. Wir betrachten die durch $x_n \coloneqq x^n$ definierte reelle Folge
$\left(x_n\right)_{n\in\mathbb{N}}$. Untersuchen Sie in Abhängigkeit von $x$, ob die Folge
konvergiert, bestimmt divergiert oder divergiert. Es bezeichne $\mathcal{H}(x)$ die Menge der
Häufungswerte der Folge. Bestimmen Sie $\mathcal{H}(x)$, $\underset{n \to \infty}\limsup x_n$
und $\underset{n \to \infty}\liminf x_n$.
Bitte stellen Sie Ihre Antworten in Tabellenform dar:

\begin{small}
\begin{tabularx}{\textwidth}{X|X|X|X|lr}
Bedingung an $x$ & Verhalten von $\left(x_n\right)_{n\ in \mathbb{N}}$ & Häufungswerte & $\inf \mathcal{H}$ & $\sup \mathcal{H}$ \\
\hline
$x < -1$         & divergiert          & $\{-\infty,  +\infty\}$ & $+\infty$ & $-\infty$ & \hyperref[beg:1]{(1)} \\
$x = -1$         & divergiert          & $\{ -1, 1 \}$           & $1$       & $-1$      & \hyperref[beg:2]{(2)} \\
$-1 < x < 1$     & konvergent          & $\{ 0 \}$               & $0$       & $0$       & \hyperref[beg:3]{(3)} \\
$x = 1$          & konvergent          & $\{ 1 \}$               & $1$       & $1$       & \hyperref[beg:4]{(4)} \\
$x > 1$          & divergiert bestimmt & $\{ +\infty \}$               & $+\infty$ & $+\infty$ \\
\end{tabularx}
\end{small}

\noindent
Geben Sie kurze Begründungen für Ihre Antworten an.

\begin{enumerate}[(1)]
\item
  \label{beg:1} Alternative Form ist $(-1)^n \cdot \abs{x}^n$ und für $x < -1$ divergiert $\abs{x}^n$ bestimmt, allerdings alterniert die Folge
\item
  \label{beg:2} $(-1)^n = \begin{cases}-1 & \text{falls } n \mod 2 = 1\\ 1 & \text{falls } n \mod 2 = 0 \end{cases}$
  für alle $n \in \mathbb{N}$
\item
  \label{beg:3} Proposition 2.1.4 (d) der Vorlesung
\item
  \label{beg:4} Für alle $n \in \mathbb{N}$ ist $1^n = 1$
\end{enumerate}

\section*{Hausaufgabe 2}

Sei $\left(a_n\right)$ eine positive, monoton wachsende Zahlenfolge in $\mathbb{R}$.
Beweisen Sie folgende Aussage: Ist $\left(a_n\right)$ beschränkt, so konvergiert die Reihe
$\overset{\infty}{\underset{k=1}\sum} \left(\frac{a_{k+1}}{a_k} - 1\right)$.

Nach Proposition 2.2.4 der Vorlesung ist $(a_n)$ ist beschränkt als beschränkte und monoton wachsende Folge konvergent.
Sei nun $\lim_{n\to\infty} (a_n) = a$, dann ist auch $\lim_{n\to\infty}(a_{n+1}) = a$.
Da $(a_n)$ positiv und monoton wachsend ist, gilt $a > 0$.
Somit folgt aus Proposition 2.1.5 (e) der Vorlesung, dass
$\underset{n\to\infty}\lim \frac{a_{n+1}}{a_n} = \frac{a}{a} = 1$ und
$\underset{n\to\infty}\lim \frac{a_{n+1}}{a_n} - 1 = 0$.

\section*{Hausaufgabe 3}

Untersuchen Sie die folgenden Reihen $\overset{\infty}{\underset{k=1}\sum} a_k$ auf Konvergenz, absolute Konvergenz, bzw. Divergenz.

\begin{enumerate}[a)]
\item $a_k \coloneqq \frac{(k!)^2}{(2k)!}$
  \begin{align*}
    \frac{\sfrac{((k+1)!)^2}{(2(k+1))!}}{\sfrac{(k!)^2}{(2k)!}} &= \frac{((k+1)!)^2}{(2(k+1))!} \cdot \frac{(2k)!}{(k!)^2} \\
                                                                &= \frac{((k+1) \cdot k!)^2}{(2k+2)\cdot(2k+1)\cdot(2k)!} \cdot \frac{(2k)!}{(k!)^2} \\
                                                                &= \frac{(k+1)^2 \cdot k!^2}{(2k+2)\cdot(2k+1)} \cdot \frac{1}{(k!)^2} \\
                                                                &= \frac{(k+1)^2}{(2k+2)\cdot(2k+1)} \\
                                                                &= \frac{k+1}{4k+2} \\
                                                                &< 1
  \end{align*}
  Somit folgt die absolute Konvergenz aus dem Quotientenkriterium.
\item $a_k \coloneqq \frac{1}{4^k} - \frac{(-1)^k}{3^k}$
\item $a_k \coloneqq (-1)^{k+1} \frac{\sqrt{k+\frac{1}{k}}}{k}$
  \[
    \lim_{k\to\infty} \frac{\sqrt{k+\frac{1}{k}}}{k} = 0
  \]

  Somit konvergiert die Reihe nach dem Leibnitz-Kriterium
\end{enumerate}

\end{document}