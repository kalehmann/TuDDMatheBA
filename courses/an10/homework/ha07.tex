\documentclass{scrreprt}

\usepackage{aligned-overset}
\usepackage{amsmath}
\usepackage{amssymb}
\usepackage{bm}
\usepackage[shortlabels]{enumitem}
\usepackage{hyperref}
\usepackage[utf8]{inputenc}
\usepackage{multicol}
\usepackage{mathtools}
\usepackage{physics}
\usepackage{tabularx}
\usepackage{titling}
\usepackage{fancyhdr}
\usepackage{xfrac}
\usepackage{pgfplots}

\pgfplotsset{compat = newest}
\usetikzlibrary{intersections}
\usetikzlibrary{patterns}
\usepgfplotslibrary{fillbetween}

\author{Karsten Lehmann (Übungsgruppe 1)\\Mat. Nr 4935758}
\date{WiSe 2021/2022}
\title{Hausaufgaben Blatt 09\\Analysis - Grundlegende Konzepte}

\setlength{\headheight}{26pt}
\pagestyle{fancy}
\fancyhf{}
\lhead{\thetitle}
\rhead{\theauthor}
\lfoot{\thedate}
\rfoot{Seite \thepage}

\begin{document}
\paragraph{47. Es seien $\qty\big{a_n}_{n \in \mathbb{N}}$ und
  $\qty\big{b_n}_{n \in \mathbb{N}}$ Folgen in $\mathbb{R}$.}
Beweisen oder widerlegen Sie:
\begin{enumerate}[(a)]
\item Ist $a$ Häufungspunkt der Menge $\qty\big{a_n}_{n \in \mathbb{N}}$,
  so ist $a$ Häufungswert der Folge $\qty\big{a_n}_{n \in \mathbb{N}}$.

  \subparagraph{Lsg.} Die Menge $\qty\big{a_n}_{n \in \mathbb{N}}$ ist definiert
  als $\qty\big{a_n {\big |} n \in \mathbb{N}}$.
  Dass heißt jedes Element der Menge ist ein Glied der Folge
  $\qty\big{a_n}_{n \in \mathbb{N}}$.

  Nun ist $a$ Häufungspunkt der Menge
  $\qty\big{a_n {\big |} n \in \mathbb{N}}$, genau dann wenn
  für jedes $\epsilon > 0$ die offene Kugel $B_{\epsilon}\qty\big(a)$
  unendlich viele Elemente aus $\qty\big{a_n {\big |} n \in \mathbb{N}}$
  enthält.

  $\Rightarrow$ unendlich viele Glieder der Folge
  $\qty\big{a_n}_{n \in \mathbb{N}}$ liegen in der Umgebung von $a$

  $\Rightarrow a$ ist Häufungswert der Folge.

\item Hat $\qty\big{a_n}_{n \in \mathbb{N}}$ nur endlich viele Häufungswerte,
  so ist $\qty\big{a_n}_{n \in \mathbb{N}}$ beschränkt.

  \subparagraph{Lsg.} Sei $a_n \coloneqq n$.
  Sei nun $\gamma \in \mathbb{R}$ beliebig und $\epsilon = \frac{1}{3}$.
  Dann enthält $B_{\epsilon}\qty\big(\gamma)$ maximal ein $a_n$.

  $\Rightarrow \gamma$ ist kein Häufungswert von $\qty\big{a_n}_{n \in \mathbb{N}}$

  $\Rightarrow$ da $\gamma$ beliebig ist, entspricht die Menge der Häufungswerte
  von $\qty\big{a_n}_{n \in \mathbb{N}}$ gleich der leeren Menge.

  Weiterhin findet sich für jedes $x \in \mathbb{R}$ ein $n \in \mathbb{N}$ mit
  $n > x$.

  $\Rightarrow \qty\big{a_n}_{n \in \mathbb{N}}$ ist nicht beschränkt.

  $\Rightarrow$ die Aussage ist falsch.

\item Sind $\qty\big{a_n}_{n \i}$ und $\qty\big{b_n}_{n \in \mathbb{N}}$
  beschränkt, so gilt
  \[
    \limsup_{n \to \infty} \qty\big(a_n + b_n) =
    \limsup_{n \to \infty} a_n + \limsup_{n \to \infty} b_n
  \]

  \subparagraph{Lsg.} Sei $a_n \coloneqq \qty\big(-1)^n$ und
  $b_n \coloneqq -\qty\Big(\qty\big(-1)^n)$.
  Dann ist $\underset{n \to \infty}{\limsup} \; a_n = 1 =
  \underset{n \to \infty}{\limsup} \; b_n$.
  Allerdings ist $a_n + b_n = 0$ und
  $\underset{n \to \infty}{\limsup} \; a_n + b_n = 0$.

  $\Rightarrow$ die Aussage ist falsch.
\end{enumerate}

\newpage
\paragraph{48. Es seien $\qty\big{a_n}_{n \in \mathbb{N}}$ und
  $\qty\big{b_n}_{n \in \mathbb{N}}$ Folgen in $\mathbb{R}$.}
Beweisen Sie:
\begin{enumerate}[(a)]
\item Es gelte $a_n \to a$ und $b_n \to b$.
  Dann folgt $a_n + b_n \to a + b$.

  \subparagraph{Lsg.} Sei $\epsilon > 0$ beliebig.
  Dann ist $\delta = \frac{\epsilon}{2} > 0$.
  Da die beiden Folgen nach der Voraussetzung konvergieren,
  existieren $n_0, n_1 \in \mathbb{N}$, so dass für alle
  $m \in \mathbb{N}_{> n_0}$ und $n \in \mathbb{N}_{> n_1}$ gilt
  $d\qty\big(a_m, a) < \delta$ und $d\qty\big(b_n, b) < \delta$.

  Sei nun $n_2 = \max\qty\big{n_0, n_1}$.
  Dann gilt für alle $n \in \mathbb{N}_{> n_2}$
  \begin{flalign*}
    d\qty\big(a_n, a) &< \delta && \text{und} && \\
    d\qty\big(b_n, b) &< \delta \\
    \Rightarrow d\qty\big(a_n, a) + d\qty\big(b_n, b) &< 2\delta \\
    \abs\big{a_n - a} + \abs\big{b_n - b} &< 2\delta \\
    \abs\big{a_n + b_n - \qty\big(a + b)}
    \overset{\bigtriangleup\text{-Ungl.}}< \abs\big{a_n - a} + \abs\big{b_n - b}
    &< 2\delta \\
    d\qty\big(a_n + b_n, a + b) &< 2\delta \overset{\text{Def.} \delta}= \epsilon
  \end{flalign*}
  $\Rightarrow a_n + b_n \overset{n \to \infty}\longrightarrow a + b$

\item Ist $\qty\big{a_n}_{n \in \mathbb{N}}$ eine Nullfolge und
  $\qty\big{b_n}_{n \in \mathbb{N}}$ beschränkt, so gilt
  $a_nb_n \to 0$.

  \subparagraph{Lsg.} Da $\qty\big{b_n}_{n \in \mathbb{N}}$ beschränkt ist,
  existiert ein $M \in \mathbb{R}_{> 0}$, so dass für alle
  $n \in \mathbb{N}$ gilt $\abs{b_n} < M$.
  \begin{flalign*}
    \Rightarrow \abs{b_n} \cdot \abs{a_n} &< M \cdot \abs{a_n} & \\
    \abs{b_n \cdot a_n} &< \abs{M \cdot a_n} \\
    d\qty\big(b_n \cdot a_n, 0) &< d\qty\big(M \cdot a_n, 0)
  \end{flalign*}

  Sei nun $\epsilon > 0$ beliebig und $\delta > 0$ so gewählt, dass
  $\delta = \frac{\epsilon}{M}$.
  Da $a$ eine Nullfolge ist, existiert ein $n_0 \in \mathbb{N}$, so dass
  für alle $n \in \mathbb{N}_{> n_0}$ gilt:
  \begin{flalign*}
    d\qty\big(a_n, 0) &< \delta = \frac{\epsilon}{M}
    && {\Big |} \; \cdot M && \\
    M \cdot d\qty\big(a_n, 0) &< \epsilon \\
    \abs\big{M \cdot a_n - M \cdot 0} &< \epsilon \\
    d\qty\big(a_nb_n, 0) < d\qty\big(M \cdot a_n, 0) &< \epsilon
  \end{flalign*}

  $\Rightarrow a_nb_n \overset{n \to \infty}\longrightarrow 0$
\end{enumerate}

\newpage
\subparagraph{49. Untersuchen Sie,} ob die Folgen konvergent sind und berechnen
Sie gegebenenfalls deren Grenzwert:
\begin{enumerate}[(a)]
\item $\qty\Big{\sqrt{n^2 + 10} - n}_{n \in \mathbb{N}}$

  \subparagraph{Lsg.}
  \begin{flalign*}
    \sqrt{n^2 + 10} - n &= \qty\Big(\sqrt{n^2 + 10} - n) \cdot 1 & \\
    &= \qty\Big(\sqrt{n^2 + 10} - n) \frac{\sqrt{n^2 + 10} + n}{\sqrt{n^2 + 10} + n} \\
    &= \frac{\qty\Big(\sqrt{n^2 + 10} - n) \cdot \qty\Big(\sqrt{n^2 + 10} - n)}{\sqrt{n^2 + 10} - n} \\
    &= \frac{\qty\Big(\sqrt{n^2 + 10})^2 - n^2}{\sqrt{n^2 + 10} + n} \\
    &= \frac{n^2 + 10 - n^2}{\sqrt{n^2 + 10} + n} \\
    &= \frac{10}{\sqrt{n^2 \cdot \qty(1 + \frac{10}{n^2})} + n} \\
    &= \frac{10}{\sqrt{n^2} \cdot \sqrt{1 + \frac{10}{n^2}} + n} \\
    &= \frac{10}{n \cdot \qty(\sqrt{1 + \frac{10}{n^2}} + 1)}
    \overset{n \to \infty}\longrightarrow 0
  \end{flalign*}
\item $\qty{\frac{n!}{n^n}}_{n \in \mathbb{N}}$

  \subparagraph{Lsg.}
  \begin{flalign*}
    \frac{n!}{n^n} &= \frac{1}{n} \cdot
    \underset{\leq 1}{\underbrace{\frac{2}{n}}} \cdot
    \underset{\leq 1}{\underbrace{\frac{3}{n}}} \cdot
    \underset{\leq 1}{\underbrace{\frac{4}{n}}} \cdot
    \ldots \cdot \underset{\leq 1}{\underbrace{\frac{n}{n}}} \\
    0 < \frac{n!}{n^n} &\leq \frac{1}{n} \overset{n \to \infty}\longrightarrow 0
  \end{flalign*}

\newpage
\item $\qty{\frac{1 + 2^k + \ldots + n^k}{n^{k + 1}}}_{n \in \mathbb{N}}$,
  wobei $k \in \mathbb{N}$ eine feste natürliche Zahl ist.

  \subparagraph{Lsg.}
  \begin{flalign*}
    \frac{1 + 2^k + \ldots + n^k}{n^{k + 1}}
    = \frac{\sum_{i = 1}^n i^k}{n^{k + 1}}
  \end{flalign*}
  Sei nun $\qty\big{a_n} \coloneqq \sum_{i = 1}^n i^k$ und
  $\qty\big{b_n} \coloneqq n^{k + 1}$.
  Dann ist $\qty\big{b_n}$ offensichtlich streng monoton wachsend.

  Nach dem Satz von Stolz (Satz 9.34 der Vorlesung) ist
  $\underset{n \to \infty}\lim\frac{a_n}{b_n} = \underset{n \to \infty}\lim
  \frac{a_{n + 1} - a_n}{b_{n + 1} - b_n}$.
  Also
  \begin{flalign*}
    \lim_{n \to \infty} \frac{\sum_{i = 1}^n i^k}{n^{k + 1}}
    &= \lim_{n \to \infty} \frac{\sum_{i = 1}^{n + 1} i^k - \sum_{i = 1}^n i^k}
      {\qty\big(n + 1)^{k + 1} - n^{k + 1}} \\
    &= \lim_{n \to \infty}\frac{\qty\big(n + 1)^k}
      {\qty\big(n + 1)^{k + 1} - n^{k + 1}} \\
    &= \lim_{n \to \infty}\frac{n^{k} + \qty\big(k + 1)n^{k-1} + \ldots + 1}
      {n^{k + 1} + \qty\big(k + 1)n^k + \ldots + 1 - n^{k + 1}} \\
    &= \lim_{n \to \infty}\frac{n^{k} + \qty\big(k + 1)n^{k-1} + \ldots + 1}
      {\qty\big(k + 1)n^k + \ldots + 1} \\
    &= \frac{1}{k + 1}
  \end{flalign*}
\end{enumerate}
\end{document}
