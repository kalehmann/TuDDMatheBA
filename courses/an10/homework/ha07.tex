\documentclass{scrreprt}

\usepackage{aligned-overset}
\usepackage{amsmath}
\usepackage{amssymb}
\usepackage{bm}
\usepackage[shortlabels]{enumitem}
\usepackage{hyperref}
\usepackage[utf8]{inputenc}
\usepackage{multicol}
\usepackage{mathtools}
\usepackage{physics}
\usepackage{tabularx}
\usepackage{titling}
\usepackage{fancyhdr}
\usepackage{xfrac}
\usepackage{pgfplots}

\pgfplotsset{compat = newest}
\usetikzlibrary{intersections}
\usetikzlibrary{patterns}
\usepgfplotslibrary{fillbetween}

\author{Karsten Lehmann (Übungsgruppe 1)\\Mat. Nr 4935758}
\date{WiSe 2021/2022}
\title{Hausaufgaben Blatt 09\\Analysis - Grundlegende Konzepte}

\setlength{\headheight}{26pt}
\pagestyle{fancy}
\fancyhf{}
\lhead{\thetitle}
\rhead{\theauthor}
\lfoot{\thedate}
\rfoot{Seite \thepage}

\begin{document}
\paragraph{47. Es seien $\qty\big{a_n}_{n \in \mathbb{N}}$ und
  $\qty\big{b_n}_{n \in \mathbb{N}}$ Folgen in $\mathbb{R}$.}
Beweisen oder widerlegen Sie:
\begin{enumerate}[(a)]
\item Ist $a$ Häufungspunkt der Menge $\qty\big{a_n}_{n \in \mathbb{N}}$,
  so ist $a$ Häufungswert der Folge $\qty\big{a_n}_{n \in \mathbb{N}}$.

  \subparagraph{Lsg.} Die Menge $\qty\big{a_n}_{n \in \mathbb{N}}$ ist definiert
  als $\qty\big{a_n {\big |} n \in \mathbb{N}}$.
  Dass heißt jedes Element der Menge ist ein Glied der Folge
  $\qty\big{a_n}_{n \in \mathbb{N}}$.

  Nun ist $a$ Häufungspunkt der Menge
  $\qty\big{a_n {\big |} n \in \mathbb{N}}$, genau dann wenn
  für jedes $\epsilon > 0$ die offene Kugel $B_{\epsilon}\qty\big(a)$
  unendlich viele Elemente aus $\qty\big{a_n {\big |} n \in \mathbb{N}}$
  enthält.

  $\Rightarrow$ unendlich viele Glieder der Folge
  $\qty\big{a_n}_{n \in \mathbb{N}}$ liegen in der Umgebung von $a$

  $\Rightarrow a$ ist Häufungswert der Folge.

\item Hat $\qty\big{a_n}_{n \in \mathbb{N}}$ nur endlich viele Häufungswerte,
  so ist $\qty\big{a_n}_{n \in \mathbb{N}}$ beschränkt.

  \subparagraph{Lsg.} Sei $a_n \coloneqq n$.
  Sei nun $\gamma \in \mathbb{R}$ beliebig und $\epsilon = \frac{1}{3}$.
  Dann enthält $B_{\epsilon}\qty\big(\gamma)$ maximal ein $a_n$.

  $\Rightarrow \gamma$ ist kein Häufungswert von $\qty\big{a_n}_{n \in \mathbb{N}}$

  $\Rightarrow$ da $\gamma$ beliebig ist, entspricht die Menge der Häufungswerte
  von $\qty\big{a_n}_{n \in \mathbb{N}}$ gleich der leeren Menge.

  Weiterhin findet sich für jedes $x \in \mathbb{R}$ ein $n \in \mathbb{N}$ mit
  $n > x$.

  $\Rightarrow \qty\big{a_n}_{n \in \mathbb{N}}$ ist nicht beschränkt.

  $\Rightarrow$ die Aussage ist falsch.

\item Sind $\qty\big{a_n}_{n \i}$ und $\qty\big{b_n}_{n \in \mathbb{N}}$
  beschränkt, so gilt
  \[
    \limsup_{n \to \infty} \qty\big(a_n + b_n) =
    \limsup_{n \to \infty} a_n + \limsup_{n \to \infty} b_n
  \]

  \subparagraph{Lsg.} Sei $a_n \coloneqq \qty\big(-1)^n$ und
  $b_n \coloneqq -\qty\Big(\qty\big(-1)^n)$.
  Dann ist $\underset{n \to \infty}{\limsup} \; a_n = 1 =
  \underset{n \to \infty}{\limsup} \; b_n$.
  Allerdings ist $a_n + b_n = 0$ und
  $\underset{n \to \infty}{\limsup} \; a_n + b_n = 0$.

  $\Rightarrow$ die Aussage ist falsch.
\end{enumerate}

\end{document}
