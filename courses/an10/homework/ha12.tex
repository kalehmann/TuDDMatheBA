\documentclass{scrreprt}

\usepackage{aligned-overset}
\usepackage{amsmath}
\usepackage{amssymb}
\usepackage{bm}
\usepackage[shortlabels]{enumitem}
\usepackage{hyperref}
\usepackage[utf8]{inputenc}
\usepackage{multicol}
\usepackage{mathtools}
\usepackage{physics}
\usepackage{pdflscape}
\usepackage{tabularx}
\usepackage{titling}
\usepackage{fancyhdr}
\usepackage{xfrac}
\usepackage[dvipsnames]{xcolor}
\usepackage{pgfplots}

\pgfplotsset{compat = newest}
\usetikzlibrary{intersections}
\usetikzlibrary{patterns}
\usepgfplotslibrary{fillbetween}

\author{Karsten Lehmann (Übungsgruppe 1)\\Mat. Nr 4935758}
\date{WiSe 2021/2022}
\title{Hausaufgaben Blatt 14\\Analysis - Grundlegende Konzepte}

\setlength{\headheight}{26pt}
\pagestyle{fancy}
\fancyhf{}
\lhead{\thetitle}
\rhead{\theauthor}
\lfoot{\thedate}
\rfoot{Seite \thepage}

\begin{document}
\paragraph{86. Sei $b \in \mathbb{R}$ mit $b < 0$.}
Beweisen oder widerlegen Sie die folgenden Aussagen:
\begin{enumerate}[(a)]
\item Für $n \in \mathbb{N}$ seien $f_n \colon \qty\big(0, 1) \to \mathbb{R}$
  definiert durch $f_n\qty\big(x) \coloneqq x^n$.
  Dann gilt
  \[
    f_n \to 0 \text{ gleichmäßig.}
  \]
  \subparagraph{Lsg.} Eine Folge von Funktionen $\qty\big{f_n}$ konvergiert
  \emph{gleichmäßig} gegen eine Funktion $f \colon X \to Y$ auf $M \subseteq X$,
  falls
  \[
    \forall \epsilon > 0 \,\exists\, n_0 \in \mathbb{N}
    \,\forall\, n \geq n_0 \,\forall\, x \in M \colon
    d\qty\big(f_n(x), f(x)) < \epsilon
  \]
  Sei nun $\epsilon = \frac{1}{3}$ und $n \in \mathbb{N}_{\geq 1}$ beliebig.
  Aus Satz 5.20 der Vorlesung
  (\emph{``Für $r > 0$ gilt $a < b \iff a^r < b^r$''}) folgt
  $x^{\frac{1}{n}} < 1^{\frac{1}{n}} = 1$ für $x \in \qty\big(0, 1)$
  - insbesondere ist $\qty(\frac{1}{2})^{\frac{1}{n}} \in \qty\big(0, 1)$.

  Somit ist nun $d\qty(f_n\qty(\frac{1}{2}), 0) =
  \abs{\qty(\qty(\frac{1}{2})^{\frac{1}{n}})^n - 0} = \frac{1}{2} > \epsilon$,
  ein Widerspruch.

\end{enumerate}

\end{document}
