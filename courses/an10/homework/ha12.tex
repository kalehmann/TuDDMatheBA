\documentclass{scrreprt}

\usepackage{aligned-overset}
\usepackage{amsmath}
\usepackage{amssymb}
\usepackage{bm}
\usepackage[shortlabels]{enumitem}
\usepackage{hyperref}
\usepackage[utf8]{inputenc}
\usepackage{multicol}
\usepackage{mathtools}
\usepackage{physics}
\usepackage{pdflscape}
\usepackage{tabularx}
\usepackage{titling}
\usepackage{fancyhdr}
\usepackage{xfrac}
\usepackage[dvipsnames]{xcolor}
\usepackage{pgfplots}

\pgfplotsset{compat = newest}
\usetikzlibrary{intersections}
\usetikzlibrary{patterns}
\usepgfplotslibrary{fillbetween}

\author{Karsten Lehmann (Übungsgruppe 1)\\Mat. Nr 4935758}
\date{WiSe 2021/2022}
\title{Hausaufgaben Blatt 14\\Analysis - Grundlegende Konzepte}

\setlength{\headheight}{26pt}
\pagestyle{fancy}
\fancyhf{}
\lhead{\thetitle}
\rhead{\theauthor}
\lfoot{\thedate}
\rfoot{Seite \thepage}

\begin{document}
\paragraph{86. Sei $b \in \mathbb{R}$ mit $b < 0$.}
Beweisen oder widerlegen Sie die folgenden Aussagen:
\begin{enumerate}[(a)]
\item Für $n \in \mathbb{N}$ seien $f_n \colon \qty\big(0, 1) \to \mathbb{R}$
  definiert durch $f_n\qty\big(x) \coloneqq x^n$.
  Dann gilt $f_n \to 0$ gleichmäßig.
  \subparagraph{Lsg.} Eine Folge von Funktionen $\qty\big{f_n}$ konvergiert
  \emph{gleichmäßig} gegen eine Funktion $f \colon X \to Y$ auf $M \subseteq X$,
  falls
  \[
    \forall\, \epsilon > 0 \,\exists\, n_0 \in \mathbb{N}
    \,\forall\, n \geq n_0 \,\forall\, x \in M \colon
    d\qty\big(f_n(x), f(x)) < \epsilon
  \]
  Sei nun $\epsilon = \frac{1}{3}$ und $n \in \mathbb{N}_{\geq 1}$ beliebig.
  Aus Satz 5.20 (2) der Vorlesung
  (\emph{``Für $r > 0$ gilt $a < b \iff a^r < b^r$''}) folgt
  $x^{\frac{1}{n}} < 1^{\frac{1}{n}} = 1$ für $x \in \qty\big(0, 1)$.
  Somit ist nun $d\qty(f_n\qty(\qty(\frac{1}{2})^{\frac{1}{n}}), 0) =
  \abs{\qty(\qty(\frac{1}{2})^{\frac{1}{n}})^n - 0} = \frac{1}{2} > \epsilon$,
  ein Widerspruch.

\item Für $n \in \mathbb{N}$ seien
  $g_n \colon \qty(0, \frac{1}{2}) \to \mathbb{R}$ definiert durch
  $g_n\qty\big(x) \coloneqq x^n$.
  Dann gilt $g_n \to 0$ gleichmäßig.
  \subparagraph{Lsg.} Aus Beispiel 9.10 der Vorlesung folgt
  $\displaystyle \lim_{n \to \infty} \qty(\frac{1}{2})^n = 0$, dass
  heißt
  \[
    \forall\, \epsilon > 0 \,\exists\, n_0 \in \mathbb{N}
    \,\forall\, n \geq n_0 \colon \abs{\qty(\frac{1}{2})^n - 0} =
    \abs{\qty(\frac{1}{2})^n} < \epsilon
  \]
  Sei $x \in \qty(0, \frac{1}{2})$ beliebig gewählt.
  Wieder aus Satz 5.20 (2) der Vorlesung folgt $x^n < \qty(\frac{1}{2})^n$.
  Also gilt
  \[
    \forall\, \epsilon > 0 \,\exists\, n_0 \in \mathbb{N}
    \,\forall\, n \geq n_0 \colon \abs{x^n - 0} =
    \abs{x^n} < \epsilon \; \qty(\forall\, x \in \qty(0, \frac{1}{2}))
  \]
\end{enumerate}

\paragraph{87. Beweisen Sie die folgenden Aussagen}
\begin{enumerate}[(a)]
\item Sei $f \colon \mathbb{R} \to \mathbb{R}, x_0 \in \mathbb{R}, f(x) > 0$
  für alle $x \in \mathbb{R} \setminus \qty\big{x_0}$, dann gilt
  \[
    \underset{x \ne x_0}{\lim_{x \to x_0}} f\qty\big(x) = 0 \iff
    \underset{x \ne x_0}{\lim_{x \to x_0}} \frac{1}{f\qty\big(x)} = \infty
  \]
  \begin{small}
  \subparagraph{Lsg.} Es ist
  \begin{flalign*}
    \underset{x \ne x_0}{\lim_{x \to x_0}} f\qty\big(x) = 0
    \overset{\epsilon-\delta-\text{Krit.}}&\iff
    \forall\,\epsilon > 0 \,\exists\, \delta > 0 \colon
    \abs\big{x - x_0} < \delta \Rightarrow f\qty\big(x) < \epsilon \\
    \overset{\text{Satz 5.4 (6)}}&\iff
    \forall\,\epsilon > 0 \,\exists\, \delta > 0 \colon
    \abs\big{x - x_0} < \delta \Rightarrow \frac{1}{f\qty\big(x)} > \frac{1}{\epsilon} \\
    &\iff
    \forall\,L > 0 \,\exists\, \delta > 0 \colon
    \abs\big{x - x_0} < \delta \Rightarrow \frac{1}{f\qty\big(x)} > L \\
    &\iff \underset{x \ne x_0}{\lim_{x \to x_0}} \frac{1}{f\qty\big(x)} = \infty
  \end{flalign*}
\end{small}

\item Sei $D \subset \mathbb{R}, x_0 \in \mathbb{R}$ ein Häufungspunkt von $D$,
  $x_0 \notin D$.
  Seien $c \in \mathbb{R}$ und $f, g \colon D \to \mathbb{R}$ mit
  $\displaystyle \lim_{x \to x_0} f\qty\big(x) = \infty$ und $g\qty\big(x) > c$
  für alle $x \in D$.
  Dann gilt
  $\displaystyle \lim_{x \to x_0}\qty\Big(f\qty\big(x) + g\qty\big(x)) = \infty$.

  \subparagraph{Lsg.} Sei $h \colon D \to \mathbb{R}, x \mapsto c$ eine Funktion.
  Dann ist $\displaystyle \lim_{x \to x_0} h(x) = c$.
  Nach Satz 14.4 (1) der Vorlesung folgt
  $\displaystyle \lim_{x \to x_0}\qty\big(f\qty\big(x) + h\qty\big(x)) = \infty$.
  Da $h\qty\big(x) > g\qty\big(x)$ für alle $x \in D$ folgt weiter, dass
  $f\qty\big(x) + g\qty\big(x) \geq f\qty\big(x) + h\qty\big(x)$ für alle
  $x \in D$.

  $\Rightarrow \displaystyle \lim_{x \to x_0}\qty\big(f\qty\big(x) +
  g\qty\big(x)) = \infty$
\end{enumerate}

\paragraph{88.}
\begin{enumerate}[(a)]
\item Untersuchen Sie, ob für $x \in \mathbb{R}$ mit $x \to 0$ die Beziehung
  $\cos\qty\big(x) = o\qty\big(x)$ gilt.
  \subparagraph{Lsg.} Man schreibt $f\qty\big(x) = o\qty\big(x)$ für $x \to x_0$,
  wenn $\displaystyle \underset{x \ne x_0}{\lim_{x \to x_0}}
  \frac{\norm{f\qty\big(x)}}{x} = 0$.
  Nun ist $\displaystyle \cos\qty\big(x) =
  \sum_{k = 0}^{\infty} \qty\big(-1)^k \frac{x^{2k}}{2k!} =
  1 - \frac{x^2}{2!} + \frac{x^4}{4!} - \frac{x^6}{6!} \pm \ldots$

  Somit ist $\displaystyle \underset{x \ne 0}{\lim_{x \to 0}}
  \frac{\cos\qty\big(x)}{x} = \underset{x \ne 0}{\lim_{x \to 0}}
  \frac{\abs{1 + x \cdot \qty(-\frac{x}{2!} + \frac{x^3}{4!} - \ldots)}}{x}
  = \underset{x \ne 0}{\lim_{x \to 0}} \frac{1}{\abs{x}}$
  und nach Beispiel 14.6 der Vorlesung ist
  $\lim_{x \to 0} \frac{1}{\abs{x}} = \infty$.

  $\Rightarrow$ \underline{die Beziehung gilt nicht.}

\item Es sei $x_0 \in \mathbb{R}$.
  Beweisen Sie, dass für $x \in \mathbb{R}$ mit $x \to x_0$ die Beziehung
  \[
    \qty\big(x - x_0)^2 \cdot o\qty\big(\qty\big(x - x_0)) =
    o\qty\Big(\qty\big(x - x_0)^3)
  \]
  gilt.
  \subparagraph{Lsg.} Es ist
  $\displaystyle \underset{x \ne x_0}{\lim_{x \to x_0}}
  \frac{\qty\big(x - x_0)^2}{\qty\big(x - x_0)^2} = 1$
  und per Definition von ``klein $o$'' außerdem
  $\displaystyle \underset{x \ne x_0}{\lim_{x \to x_0}}
  \frac{\abs{o\qty\big(x - x_0)}}{x - x_0} = 0$.
  Aus Satz 14.4 (1) der Vorlesung folgt
  $\displaystyle \underset{x \ne x_0}{\lim_{x \to x_0}}
  \qty(\frac{\qty\big(x - x_0)^2}{\qty\big(x - x_0)^2} \cdot
  \frac{\abs{o\qty\big(x - x_0)}}{x - x_0}) =
  \underset{x \ne x_0}{\lim_{x \to x_0}}
  \frac{\abs{\qty\big(x - x_0)^2 \cdot o\qty\big(x - x_0)}}{\qty\big(x - x_0)^3}
  = 0$.

  $\Rightarrow$ \underline{
    $\qty\big(x - x_0)^2 \cdot o\qty\big(\qty\big(x - x_0))$ ist ``klein $o$''
    von $o\qty\Big(\qty\big(x - x_0)^3)$}
\end{enumerate}
\end{document}
