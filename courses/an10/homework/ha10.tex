\documentclass{scrreprt}

\usepackage{aligned-overset}
\usepackage{amsmath}
\usepackage{amssymb}
\usepackage{bm}
\usepackage[shortlabels]{enumitem}
\usepackage{hyperref}
\usepackage[utf8]{inputenc}
\usepackage{multicol}
\usepackage{mathtools}
\usepackage{physics}
\usepackage{tabularx}
\usepackage{titling}
\usepackage{fancyhdr}
\usepackage{xfrac}
\usepackage{pgfplots}

\pgfplotsset{compat = newest}
\usetikzlibrary{intersections}
\usetikzlibrary{patterns}
\usepgfplotslibrary{fillbetween}

\author{Karsten Lehmann (Übungsgruppe 1)\\Mat. Nr 4935758}
\date{WiSe 2021/2022}
\title{Hausaufgaben Blatt 12\\Analysis - Grundlegende Konzepte}

\setlength{\headheight}{26pt}
\pagestyle{fancy}
\fancyhf{}
\lhead{\thetitle}
\rhead{\theauthor}
\lfoot{\thedate}
\rfoot{Seite \thepage}

\begin{document}
\paragraph{64. Sei $s_n \coloneqq \sum_{k = 0}^n \frac{1}{k!}$ und sei
  $e = \underset{n \to \infty}\lim s_n$ die Eulersche Zahl.}
\begin{enumerate}[(a)]
\item Zeigen Sie $\abs\big{e - s_n} \leq \frac{1}{n \cdot n!}$ für alle
  $n \in \mathbb{N}_{> 0}$.

  \subparagraph{Lsg.} Es gilt $\qty\big(n + k)! > n! \cdot \qty\big(n + 1)^k$, da
  \[
    \qty\big(n + k)! =
    \underset{\geq \qty\big(n + 1)^k}{
      \underbrace{
        \qty\big(n + k) \cdot \qty\big(n + k - 1) \cdot
        \ldots \cdot \qty\big(n + 1)
      }
    } \cdot
    \underset{= n!}{\underbrace{n \cdot \qty\big(n - 1) \cdot \ldots \cdot 1}}
  \]
  \begin{flalign*}
    \abs\big{e - s_n}
    &= \abs{
      \sum_{k = 0}^{\infty} \frac{1}{k!} - \sum_{k = 0}^n \frac{1}{k!}
    }
    = \abs\Bigg{
      \sum_{k = n + 1}^{\infty} \frac{1}{k!} +
      \underset{= 0}{
        \underbrace{\sum_{k = 0}^n \frac{1}{k!} - \sum_{k = 0}^n \frac{1}{k!}}
     }
      } \\
    &= \abs{\sum_{k = n + 1}^{\infty} \frac{1}{k!}}
    = \abs{\sum_{k = 1}^{\infty} \frac{1}{(n + k)!}}
    = \sum_{k = 1}^{\infty} \frac{1}{(n + k)!} \\
    &\leq \sum_{k = 1}^{\infty} \frac{1}{n! \cdot \qty\big(n + 1)^k} \\
    &= \frac{1}{n!} \cdot \sum_{k = 1}^{\infty} \qty(\frac{1}{n + 1})^k \\
    &= \frac{1}{n!} \cdot \qty(\frac{1}{1 - \frac{1}{n + 1}} - 1) \\
    &= \frac{1}{n!} \cdot \qty(\frac{n + 1}{n} - \frac{n}{n}) \\
    &= \frac{1}{n \cdot n!}
  \end{flalign*}

\item Bestimmen Sie mit Hilfe von Teilaufgabe (a) ein Zahl $N \in \mathbb{N}$,
  für die $\abs\big{e - s_N} \leq 0.5 \cdot 10^{-4}$ gilt und geben Sie den Wert
  von $s_N$ an.

  \subparagraph{Lsg.} $0.5 \cdot 10^{-4} = \frac{1}{2} \cdot \frac{1}{10^4} =
  \frac{1}{20000}$.
  Nun ist
  \[
    \overset{= 5040}{
      \overbrace{
        \underset{= 720}{
          \underbrace{
            \overset{= 120}{
              \overbrace{
                \underset{= 24}{
                  \underbrace{
                    \overset{= 6}{
                      \overbrace{
                        1 \; \cdot \; 2 \; \cdot \; 3 \;
                      }
                    } \cdot \; 4 \;
                  }
                } \cdot \; 5 \;
              }
            } \cdot \; 6 \;
          }
        } \cdot \; 7
      }
    }
  \]
  Somit ist $0.5 \cdot 10^{-4} < \frac{1}{7 \cdot 7!} = \frac{1}{7 \cdot 5040}
  = \frac{1}{35280}$ und
  \[
    s_7 = \frac{1}{1} + \frac{1}{1} + \frac{1}{2} + \frac{1}{6} + \frac{1}{24}
    + \frac{1}{120} + \frac{1}{720} + \frac{1}{5040}
    = \frac{685}{252} \approx 2.7183
  \]
\end{enumerate}

\end{document}
