\documentclass{scrreprt}

\usepackage{aligned-overset}
\usepackage{amsmath}
\usepackage{amssymb}
\usepackage{bm}
\usepackage[shortlabels]{enumitem}
\usepackage{hyperref}
\usepackage[utf8]{inputenc}
\usepackage{multicol}
\usepackage{mathtools}
\usepackage{physics}
\usepackage{tabularx}
\usepackage{titling}
\usepackage{fancyhdr}
\usepackage{xfrac}
\usepackage[dvipsnames]{xcolor}
\usepackage{pgfplots}

\pgfplotsset{compat = newest}
\usetikzlibrary{intersections}
\usetikzlibrary{patterns}
\usepgfplotslibrary{fillbetween}

\author{Karsten Lehmann (Übungsgruppe 1)\\Mat. Nr 4935758}
\date{WiSe 2021/2022}
\title{Hausaufgaben Blatt 12\\Analysis - Grundlegende Konzepte}

\setlength{\headheight}{26pt}
\pagestyle{fancy}
\fancyhf{}
\lhead{\thetitle}
\rhead{\theauthor}
\lfoot{\thedate}
\rfoot{Seite \thepage}

\begin{document}
\paragraph{64. Sei $s_n \coloneqq \sum_{k = 0}^n \frac{1}{k!}$ und sei
  $e = \underset{n \to \infty}\lim s_n$ die Eulersche Zahl.}
\begin{enumerate}[(a)]
\item Zeigen Sie $\abs\big{e - s_n} \leq \frac{1}{n \cdot n!}$ für alle
  $n \in \mathbb{N}_{> 0}$.

  \subparagraph{Lsg.} Es gilt $\qty\big(n + k)! > n! \cdot \qty\big(n + 1)^k$, da
  \[
    \qty\big(n + k)! =
    \underset{\geq \qty\big(n + 1)^k}{
      \underbrace{
        \qty\big(n + k) \cdot \qty\big(n + k - 1) \cdot
        \ldots \cdot \qty\big(n + 1)
      }
    } \cdot
    \underset{= n!}{\underbrace{n \cdot \qty\big(n - 1) \cdot \ldots \cdot 1}}
  \]
  \begin{flalign*}
    \abs\big{e - s_n}
    &= \abs{
      \sum_{k = 0}^{\infty} \frac{1}{k!} - \sum_{k = 0}^n \frac{1}{k!}
    }
    = \abs\Bigg{
      \sum_{k = n + 1}^{\infty} \frac{1}{k!} +
      \underset{= 0}{
        \underbrace{\sum_{k = 0}^n \frac{1}{k!} - \sum_{k = 0}^n \frac{1}{k!}}
     }
      } \\
    &= \abs{\sum_{k = n + 1}^{\infty} \frac{1}{k!}}
    = \abs{\sum_{k = 1}^{\infty} \frac{1}{(n + k)!}}
    = \sum_{k = 1}^{\infty} \frac{1}{(n + k)!} \\
    &\leq \sum_{k = 1}^{\infty} \frac{1}{n! \cdot \qty\big(n + 1)^k} \\
    &= \frac{1}{n!} \cdot \sum_{k = 1}^{\infty} \qty(\frac{1}{n + 1})^k \\
    &= \frac{1}{n!} \cdot \qty(\frac{1}{1 - \frac{1}{n + 1}} - 1) \\
    &= \frac{1}{n!} \cdot \qty(\frac{n + 1}{n} - \frac{n}{n}) \\
    &= \frac{1}{n \cdot n!}
  \end{flalign*}

\item Bestimmen Sie mit Hilfe von Teilaufgabe (a) ein Zahl $N \in \mathbb{N}$,
  für die $\abs\big{e - s_N} \leq 0.5 \cdot 10^{-4}$ gilt und geben Sie den Wert
  von $s_N$ an.

  \subparagraph{Lsg.} $0.5 \cdot 10^{-4} = \frac{1}{2} \cdot \frac{1}{10^4} =
  \frac{1}{20000}$.
  Nun ist
  \[
    \overset{= 5040}{
      \overbrace{
        \underset{= 720}{
          \underbrace{
            \overset{= 120}{
              \overbrace{
                \underset{= 24}{
                  \underbrace{
                    \overset{= 6}{
                      \overbrace{
                        1 \; \cdot \; 2 \; \cdot \; 3 \;
                      }
                    } \cdot \; 4 \;
                  }
                } \cdot \; 5 \;
              }
            } \cdot \; 6 \;
          }
        } \cdot \; 7
      }
    }
  \]
  Somit ist $0.5 \cdot 10^{-4} < \frac{1}{7 \cdot 7!} = \frac{1}{7 \cdot 5040}
  = \frac{1}{35280}$ und
  \[
    s_7 = \frac{1}{1} + \frac{1}{1} + \frac{1}{2} + \frac{1}{6} + \frac{1}{24}
    + \frac{1}{120} + \frac{1}{720} + \frac{1}{5040}
    = \frac{685}{252} \approx 2.7183
  \]

\newpage
\item Zeigen Sie, dass die Eulersche Zahl $e$ irrational ist.

  \subparagraph{Lsg.} Alle Glieder der Reihe $\sum_{k = 0}^{\infty} \frac{1}{k!}$
  sind positiv und
  $\sum_{k = 0}^{1} \frac{1}{k!} = \frac{1}{1} + \frac{1}{1} = 2$.

  $\Rightarrow \sum_{k = 0}^{\infty} \frac{1}{k!} \geq 2$

  Weiterhin ist
  \[
    \sum_{k = 0}^{\infty} =
    \qty(\frac{1}{1} + \frac{1}{1} + \sum_{k = 2}^{\infty} \frac{1}{k!})
    \leq
    \qty(1 + \sum_{k = 1}^{\infty}\qty(\frac{1}{2})^k)
    = \qty(1 + \frac{1}{1 - \frac{1}{2}} - 1) = 3
  \]

  $\Rightarrow \sum_{k = 0}^{\infty} \frac{1}{k!} \leq 3$

  Angenommen $e$ wäre rational, dann existieren $p, q \in \mathbb{N}$, so dass
  $\frac{p}{q} = e$.
  Da $2 \leq e \leq 3$ und $e \ne 2$ als auch $e \ne 3$ folgt $p > 2, q > 1$.

  Da $s_q$ eine endliche Summe von positiven rationalen Zahlen ist und
  aufgrund der rekursiven Definition der Fakultät für $n \leq q$ der Ausdruch
  $n!$ ein Teiler von $q!$ ist, existiert ein $m \in \mathbb{N}$ mit
  $\frac{m}{q!} = s_q$.

  Weiterhin ist $s_q \leq e$, ansonsten Widerspruch zu
  $\lim_{n \to \infty} s_n = e$ als Summe positiver Glieder.

  Folglich ist
  \begin{flalign*}
    \abs\big{e - s_q}
    \overset{e > s_q}&= e - s_q \\
    &= \frac{p\qty\big(q - 1)!}{q\qty\big(q - 1)!} - s_q \\
    &= \frac{p\qty\big(q - 1)!}{q!} - \frac{m}{q!} \\
    &= \frac{p\qty\big(q - 1)! - m}{q!} \\
    &= \frac{q \cdot \qty\Big(p \cdot \qty\big(q - 1)! - m)}{qq!} \\
    \overset{\text{(a)}}&\leq \frac{1}{qq!}
  \end{flalign*}

  $\Rightarrow 1 \geq q \cdot \qty\Big(p\qty \big(q - 1)! - m)
  \overset{q > 1} > p\qty\big(q - 1)! - m
  \overset{m \in \mathbb{N}}> p \qty\big(q - 1)
  \overset{p > 2}> 2\cdot \qty\big(q - 1)
  \overset{q > 1}> 2$, ein Widerspruch.

  $\Rightarrow$ \underline{$e$ ist irrational}
\end{enumerate}

\newpage
\paragraph{66. Vereinfache den folgenden Ausdruck mittels Polynomdivison über
  $\mathbb{R}$ soweit wie möglich:}
\[
  \frac{x^4 - 3x^3 - 14x^2 + 48x - 32}
  {\qty\big(x - 1)\qty\big(x - 2)\qty\big(x - 3)}
\]

\subparagraph{Lsg.} Zuerst wird der Nenner ausmultipliziert:
$\qty\big(x - 1) \cdot \qty\big(x - 2) \cdot \qty\big(x - 3) =
x^3 - 6x^2 + 11x - 6$.

\[
  \begin{array}{cccccccc}
    \colorbox{Cyan!30}{$x^4$} & - 3x^3 & - 14x^2 & + 48x & - 32 &:&
      \colorbox{Cyan!30}{$x^3$} - 6x^2 + 11x - 6
      &= \colorbox{Cyan!30}{$x$} + \colorbox{Lavender!50}{$3$} +
      \frac{-7x^2 + 21x - 14}{\qty\big(x - 1)\qty\big(x - 2)\qty\big(x - 3)} \\
    - \big(x^4 & -6x^3 & + 11x^2 & - 6x \big) \\
    \cline{1 - 4}
    & \colorbox{Lavender!50}{$3x^3$} & -25x^2 & +54x & - 32  &:&
      \colorbox{Lavender!50}{$x^3$} - 6x^2 + 11x - 6 \\
    & -\big(3x^3 & -18x^2 & + 33x & -18 \big) \\
    \cline{2 - 5}
    & & -7x^2 & + 21x & -14
  \end{array}
\]
und
\begin{flalign*}
  x + 3 + \frac{-7x^2 + 21x - 14}{\qty\big(x - 1)\qty\big(x - 2)\qty\big(x - 3)}
  &= x + 3 - 7 \cdot \frac{
    x^2 - 3x + 2
  }{
    \qty\big(x - 1)\qty\big(x - 2)\qty\big(x - 3)
  } \\
  &= x + 3 - 7 \cdot \frac{
    \qty\big(x - 1) \cdot \qty\big(x - 2)
  }{
    \qty\big(x - 1)\qty\big(x - 2)\qty\big(x - 3)
  } \\
  &= x + 3 - \frac{7}{x - 3}
\end{flalign*}

\end{document}
