\documentclass{article}
\usepackage{aligned-overset}
\usepackage{amsmath}
\usepackage{amssymb}
\usepackage{amsthm}
\usepackage{bm}
\usepackage[shortlabels]{enumitem}
\usepackage{hyperref}
\usepackage[utf8]{inputenc}
\usepackage{mathtools}
\usepackage{physics}
\usepackage{pgfplots}
\usepackage{titling}
\usepackage{fancyhdr}
\usepackage{xfrac}

\newtheorem*{definition}{Definition}
\newtheorem*{satz}{Satz}

\author{Karsten Lehmann}
\date{WiSe 2020}
\title{Konvergenz}

\pagestyle{fancy}
\fancyhf{}
\lhead{\thetitle}
\rhead{\theauthor}
\lfoot{\thedate}
\rfoot{Seite \thepage}

\begin{document}
\section*{Konvergenz und Grenzwert}

\begin{definition}[Folge]
  Eine Folge ist eine Funktion $a \colon \mathbb{N} \to \mathbb{R}$. \\
  Dabei ist
  \begin{itemize}
  \item $a(n) \iff a_n$
  \item $a \iff \left(a_n\right)_{n\in\mathbb{N}}$ oder kurz $\left(a_n\right)$
  \end{itemize}
\end{definition}

\begin{definition}[Konvergente Folge]
  Eine Folge $\left(a_n\right)$ heißt konvergent, wenn es ein $a \in \mathbb{R}$ gibt, so dass

  \[
    \forall \epsilon > 0 \exists n_0 \in \mathbb{N} \forall n \in \mathbb{N}_{\geq n_0} \colon \abs{a_n - a} < \epsilon
  \]

  In Worten ausgedrückt: Die Folgenglieder nähern sich für ein zunehmendes $n$ immer weiter an $a$ an.
  $a$ heißt in diesem Fall Grenzwert.

  \begin{tikzpicture}

  \end{tikzpicture}
\end{definition}

\begin{definition}[Divergente Folge]
  Eine Folge die nicht konvergent ist.
\end{definition}

\begin{definition}[Nullfolge]
  Als Nullfolge bezeichnet man eine Folge mit dem Grenzwert $0$.
\end{definition}

\begin{definition}[Cauchyfolge]
  Eine Folge $\left(a_n\right)$ heißt Cauchyfolge, wenn
  \[
    \forall \epsilon > 0 \exists n_0 \in \mathbb{N} \forall n, m \exists \mathbb{N}_{\geq n_0} \colon \abs{a_n - a_m} < \epsilon
  \]

  In Worten ausgedrückt: Der Abstand der Folgenglieder wird für ein zunehmendes $n$ beliebig klein.
\end{definition}

\begin{definition}[Beschränkte Folge]
  Eine Folge $\left( a_n \right)$ heißt beschränkt, wenn es $C \geq 0$ gibt mit
  \[
    \forall n \in \mathbb{N} \colon \abs{a_n} \leq C
  \]
\end{definition}

\begin{definition}[Häufungswert]
  Die Zahl $a$ heißt Häufungswert der Folge $\left(a_n\right)$, wenn

  \[
    \forall \epsilon > 0 \forall N \in \mathbb{N} \exists n \in \mathbb{N}_{\geq N} \colon \abs{a_n - a} < \epsilon
  \]

  In Worten ausgedrückt: Ab jedem Punkt der Folge sind viele (aber nicht alle) Glieder der Folge beliebig nah an $a$.
\end{definition}

\begin{definition}[Teilfolge]
  Sei $\left(a_n\right)_n$ eine Folge und $\left( n_k \right)_k$ eine streng monoton wachsende Folge in $\mathbb{N}$ ($n_k < n_{k+1}$),
  dann heißt $\left(a_{n_k}\right)_k$ Teilfolge von $\left(a_n\right)_n$.
\end{definition}

\begin{satz}[Bolzano-Weierstraß]
  Jede beschränkte Folge besitzt (mindestens) eine konvergente Teilfolge und (mindestens) einen Häufungspunkt.
\end{satz}

\end{document}