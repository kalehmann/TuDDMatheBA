\documentclass{scrreprt}

\usepackage{aligned-overset}
\usepackage{amsmath}
\usepackage{amssymb}
\usepackage{bm}
\usepackage{chngcntr}
\usepackage[shortlabels]{enumitem}
\usepackage{hyperref}
\usepackage[utf8]{inputenc}
\usepackage{mathtools}
\usepackage{physics}
\usepackage{tabularx}
\usepackage{titling}
\usepackage{fancyhdr}
\usepackage{xfrac}
\usepackage[table]{xcolor}
\usepackage{pgfplots}

%% See https://tex.stackexchange.com/a/44954
\newcounter{myequation}
\makeatletter
\@addtoreset{equation}{myequation}
\makeatother

%% Fix equation numbering for scrreprt class.
\counterwithout{equation}{chapter}

\pgfplotsset{compat = newest}
\usepgfplotslibrary{patchplots}
\usetikzlibrary{intersections}
\usetikzlibrary{shapes}
\usetikzlibrary{shapes.geometric}
\usetikzlibrary{patterns}
\usepgfplotslibrary{fillbetween}

\author{Karsten Lehmann}
\date{SoSe 2021}
\title{Übung 13 Analysis - Weiterführende Konzepte}

\pagestyle{fancy}
\fancyhf{}
\lhead{\thetitle}
\rhead{\theauthor}
\lfoot{\thedate}
\rfoot{Seite \thepage}

\newcommand\skalprod[1]{\left\langle #1 \right\rangle}
\newcommand\nnorm[1]{\left\lvert\left\lvert\left\lvert #1 \right\rvert\right\rvert\right\rvert}

\begin{document}
\setcounter{chapter}{1}
\section*{Variation von Kurven}
\paragraph{Aufgabe 1:} (Graph einer Funktion als Kurve).
Gegeben sei eine stetige Funktion $f \colon [a, b] \to \mathbb{R}^N$.
Weiter sei $\gamma \colon [a, b] \to \mathbb{R}^{N + 1}$ definiert durch
\begin{flalign*}
  \gamma(t) &\coloneqq \qty\big(t, f(t)) &
\end{flalign*}
\begin{enumerate}[(i)]
\item Zeigen Sie, dass $\gamma$ eine Kurve ist.
  \subparagraph{Lösung:} Es ist zu zeigen, dass $\gamma$ stetig ist.

  Da $f$ stetig ist, ist auch die Abbildung
  $t \mapsto \qty\big(t, f(t)) = \gamma(t)$ stetig
  $\Rightarrow \gamma$ ist eine Kurve.

\item Beweisen Sie: Ist $\gamma$ stetig differenzierbar, so ist
  $\gamma$ von endlicher Variation.
\item Berechnen Sie die Länge von $\gamma$ im Fall
  $f \colon [0, 1] \to \mathbb{R}, f(x) = \alpha x + \beta$
\end{enumerate}

\end{document}