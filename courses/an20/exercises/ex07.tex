\documentclass{scrreprt}

\usepackage{aligned-overset}
\usepackage{amsmath}
\usepackage{amssymb}
\usepackage{bm}
\usepackage[shortlabels]{enumitem}
\usepackage{hyperref}
\usepackage[utf8]{inputenc}
\usepackage{mathtools}
\usepackage{physics}
\usepackage{tabularx}
\usepackage{titling}
\usepackage{fancyhdr}
\usepackage{xfrac}
\usepackage{pgfplots}

\definecolor{light-gray}{gray}{.9}

\pgfplotsset{compat = newest}
\usetikzlibrary{intersections}
\usetikzlibrary{patterns}
\usepgfplotslibrary{fillbetween}
\usepackage{titlesec}

\titleformat{\section}[runin]
  {\normalfont\Large\bfseries}{\thesection}{1em}{}
\titleformat{\subsection}[runin]
  {\normalfont\large\bfseries}{\thesubsection}{1em}{}

\author{Karsten Lehmann}
\date{SoSe 2021}
\title{Übung 07 Analysis - Weiterführende Konzepte}

\pagestyle{fancy}
\fancyhf{}
\lhead{\thetitle}
\rhead{\theauthor}
\lfoot{\thedate}
\rfoot{Seite \thepage}

\begin{document}
\fcolorbox{black}{light-gray}{\begin{minipage}{\textwidth}
    \textbf{Linearer Operator} (auch lineare Abbildung) ist eine
    strukturerhaltende Abbildung zwischen Vektorräumen über einen
    gemeinsamen Körper.

    Seien $X, Y$ Vektorräume und $T \colon X \to Y$ eine Abbildung, dann heißt
    $T$ linearer Operator, wenn
    \begin{enumerate}
    \item $\forall x, y \in X$ und $\lambda \in \mathbb{R} \colon T(\lambda x) = \lambda T(x)$
    \item $\forall x, y \in X \colon T(x + y) = T(x) + T(y)$
    \end{enumerate}
\end{minipage}}

Sei $X, Y$ normiert, ein Operator $T \colon X \to Y$ linear.
\begin{enumerate}[(i)]
\item $T$ ist stetig $\iff$
\item $T$ ist in $0$ stetig $\iff$
\item $T$ ist beschränkt, d.h. $\exists c > 0 \forall x \in X\colon \norm{T_x}_y \leq c \norm{x}$
\end{enumerate}
\[
  T \in \mathcal{L}\qty{X, Y} = \qty{T \colon X \to Y {\Big |} T \text{ ist linear stetig}} (*)
\]
\begin{flalign*}
  \norm{T} &= \inf\qty{c > 0 \, {\Big |} \text{ Es gilt } (*)} & \\
  &= \sup\qty{\frac{\norm{Tx}_y}{\norm{x}_x} \, {\Big |} \, x \ne 0} & \frac{\norm{T_x}_y}{\norm{x}_x} = \norm{T\qty(\frac{x}{\norm{x}_x})} \\
  &= \sup\qty{\norm{T_x}_y {\Big |} \norm{x}_x = 1} =  \sup\qty{\norm{T_x}_y {\Big |} \norm{x}_x \leq 1}
\end{flalign*}

$T \in \mathcal{L}\qty(\mathbb{R}^N, \mathbb{R}^M)$ und Basen $\phi$ auf
$\mathbb{R}^N$ und $\psi$ auf $\mathbb{R}^M$ seien gegeben.

$\Rightarrow$ Es gibt genau eine Matrix $A$, die die Koordinatenvektoren transformiert.

$x = \sum_{k=1}^N x_k \phi_k, \underline{x} = \qty{x_1, \ldots, x_n} T_x = \sum_{k=1}^N x_k T(\phi_k) = \sum_{j=1}^M y_i \psi_j, \underline{y} = \qty{y_1, \ldots, y_m}$

$\Rightarrow \underline{y} = A \cdot \underline{x}$

\section*{Aufgabe 1} Der Raum $\mathcal{L}\qty(\mathbb{R}^N, \mathbb{R}^M)$ kann mit
$R^{M \times N}$ identifiziert werden.
Sind Normen $\norm{\cdot}_{\mathbb{R}^N}$ und
$\norm{\cdot}_{\mathbb{R}^M}$ auf den Räumen $\mathbb{R}^N$ bzw.
$\mathbb{R}^M$ gegeben, so ist durch
\[
  \norm{A} \coloneqq \sup\qty{ \norm{A \cdot x}_{\mathbb{R}^M} {\Big |} \norm{x}_{\mathbb{R}^N} \leq 1}
\]
eine Operatornorm auf $R^{M \times N}$ definiert.
Die Räume $\mathbb{R}^N$ und $\mathbb{R}^M$ seien mit der $\norm{\cdot}_1$-Norm
versehen.
Beweisen Sie, dass sich als Operatornorm von $A \in R^{M \times N}$ die
sogenannte \textit{Spaltensummennorm} ergibt:
\[
  \norm{A} = \underset{k=1,\ldots,N}\sum_{j=1}^M \abs{a_{jk}} \text{ für alle } A \in \mathcal{L}\qty(\mathbb{R}^N, \mathbb{R}^M)
\]

\newpage
\textit{Lsg.} \\
Sei $T \in \mathcal{L}\qty(\mathbb{R}^N, \mathbb{R}^M)$ und $A$ die
Abbildungsmatrix bezüglich der Standardbasen.
\[
  \norm{x}_1 = \sum_{k = 1}^N \abs{x_k}, \quad \norm{y}_1 = \sum_{j = 1}^M \abs{y_j}
\]
Wir betrachten zunächst
\[
  A \cdot e^k =
  \qty(
    \begin{array}{ccccc}
      a_{11} & \ldots & a_{1k} & \ldots & a_{1N} \\ \\ \\
      \vdots & \vdots & \vdots & \vdots & \vdots \\ \\ \\
      a_{M1} & \ldots & a_{Mk} & \ldots & a_{MN}
    \end{array}
  )
  \cdot
  \qty(
    \begin{array}{c}
      0 \\
      \vdots \\
      0 \\
      1 \\
      0 \\
      \vdots \\
      0
    \end{array}
  )
  = \qty(a_{1k}, \ldots, a_{Mk}) \text{ (die $k$-te Spalte)}
\]
\[
  \Rightarrow \norm{A \cdot e_k}_1 = \norm{\qty(a_{1k}, \ldots, a_{Mk})}_1
  = \sum_{j = 1}^M \abs{a_{jk}}
\]
Wegen $\norm{e_k}_1 = 1$ folgt
$\frac{\norm{A \cdot e_k}_1}{\norm{e_k}_1} = \sum_{j = 1}^M \abs{a_{jk}}
\leq \sup\qty{\frac{\norm{A \cdot x}_1}{\norm{x}_1} {\Big |} x \in \mathbb{R^N}} = \norm{A}$
\begin{flalign*}
  \Rightarrow & \max_{k=1, \ldots, N} \sum_{j = 1}^M \abs{a_{jk}} \leq \norm{A} &
\end{flalign*}
Weiter gilt: Jedes $x \in \mathbb{R}^N$ besitzt eine eindeutige Darstellung der Form
$x = \sum_{k = 1}^N x_k e_k$
\begin{align*}
  \Rightarrow \norm{A \cdot x}_1 &= \norm{\sum_{k = 1}^N x_k \cdot A \cdot e_k}_1
  \leq \sum_{k = 1}^N \abs{x_k} \cdot \norm{A \cdot e_k}_1 \\
  &\leq \max_{k=1, \ldots, N}\norm{A \cdot e_k} \cdot \underset{= \norm{x}_1}{\underbrace{\sum_{k = 1}^N \abs{x_k}}} \\
  &= \max_{k=1, \ldots, N}\norm{A \cdot e_k} \cdot \norm{x}_1
\end{align*}
\begin{align*}
  \overset{x \ne 0}&\Rightarrow \frac{\norm{A \cdot x}_1}{\norm{x}_1} \leq \max_{k=1, \ldots, N}\sum_{j = 1}^M \abs{a_jk} \\
  \norm{A} &= \sup\qty{\frac{\norm{A \cdot x}_1}{\norm{x}_1} {\Big |}, \, x \ne 0 } \leq \max_{k=1, \ldots, N}\sum_{j = 1}^M \abs{a_jk} \\
\end{align*}
$\norm{A} = \max_{k = 1 \ldots N} \sum_{j = 1}^M \abs{a_{jk}}$ (die Spaltensummennorm)
\end{document}