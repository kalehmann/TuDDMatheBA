\documentclass{scrreprt}

\usepackage{aligned-overset}
\usepackage{amsmath}
\usepackage{amssymb}
\usepackage{bm}
\usepackage[shortlabels]{enumitem}
\usepackage{hyperref}
\usepackage[utf8]{inputenc}
\usepackage{mathtools}
\usepackage{physics}
\usepackage{tabularx}
\usepackage{titling}
\usepackage{fancyhdr}
\usepackage{xfrac}
\usepackage{pgfplots}

\definecolor{light-gray}{gray}{.9}

\pgfplotsset{compat = newest}
\usepgfplotslibrary{patchplots}

\usetikzlibrary{intersections}
\usetikzlibrary{patterns}
\usepgfplotslibrary{fillbetween}

\author{Karsten Lehmann}
\date{SoSe 2021}
\title{Übung 10 Analysis - Weiterführende Konzepte}

\pagestyle{fancy}
\fancyhf{}
\lhead{\thetitle}
\rhead{\theauthor}
\lfoot{\thedate}
\rfoot{Seite \thepage}

\newcommand\skalprod[1]{\left\langle #1 \right\rangle}
\newcommand\nnorm[1]{\left\lvert\left\lvert\left\lvert #1 \right\rvert\right\rvert\right\rvert}

\begin{document}

\paragraph{Aufgabe 1} Gegeben seien ein nichtleerer vollständiger metrischer
Raum $(M, d)$ und eine kontrahierende Abbildung
$\varphi \colon M \to M$ mit der Lipschtizkonstanten $q \in (0, 1)$.
Nach dem Banachschen Fixpunktsatz besitzt $\varphi$ genau einen Fixpunkt
$x^* \in M$ und für jedes $x_0 \in M$ konvergiert die Folge $\qty(x_n)$
mit $x_{n + 1} = \varphi\qty(x_n)$ gegen $x^*$.
Beweises Sie die folgenden Abschätzungen:
\begin{enumerate}[a)]
\item Für alle $x \in M$ gilt: $d(\varphi(x), x*) \leq \frac{q}{1 - q}d(\varphi(x), x)$
  \label{sec:1_a}
  \subparagraph{Lösung:} Für alle $x \in M$ gilt
  \begin{flalign*}
    d \qty(\varphi(x), x^*) &= d \qty(\varphi(x), \varphi(x^*)) & \\
    &\leq q \cdot d(x, x*) \\
    \overset{\makebox[20pt]{\small Dreiecksungleichung}}&\leq q \cdot d(x, \varphi(x)) + q \cdot d(\varphi(x), x^*) & {\Big |} -  q \cdot d(\varphi(x), x^*) \\
    \Rightarrow \quad (1 - q) \cdot d(\varphi(x), x^*) &\leq q \cdot d(x, \varphi(x)) \\
    d(\varphi(x), x^*) &\leq \frac{q}{1 - q} \cdot d(x, \varphi(x)) \\
    d(\varphi(x), x^*) \overset{\makebox[30pt]{\small Symmetrie}}&\leq \frac{q}{1 - q} \cdot d(\varphi(x), x) \\
  \end{flalign*}

\item Für jeden Startwert $x_0 \in M$ und $n \in \mathbb{N}$ gilt die
  \textit{a posteriori Abschätzung}
  \[
    d\qty(x_n, x^*) \leq \frac{q}{1 - q} d\qty(x_n, x_{n - 1})
  \]
  \label{sec:1_b}
  \subparagraph{Lösung:} Für $x = x_{n - 1}$ folgt aus \hyperref[sec:1_a]{a)}
  \begin{flalign*}
    d\qty(\varphi\qty(x_{n - 1}), x^*) \overset{\makebox[60pt]{\tiny Definition von $x_n$}}&= d\qty(x_n, x^*) & \\
    &\leq \frac{q}{1 - q} d\qty(x_{n - 1}, \varphi\qty(x_{n - 1})) \\
    &= \frac{q}{1 - q} d\qty(x_{n - 1}, x_n)
  \end{flalign*}

\newpage
\item Für jeden Startwert $x_0 \in M$ und $n \in \mathbb{N}$ gilt die
  \textit{a priori Abschätzung}
  \[
    d\qty(x_n, x^*) \leq \frac{q^n}{1 - q} d\qty(x_1, x_0)
  \]

  \subparagraph{Lösung:} Sei
  $P \coloneqq \qty{n \in \mathbb{N} \middle| d\qty(x_n, x^*) \leq \frac{q^n}{1 - q} d\qty(x_1, x_0)}$.

  Für $n = 1$ gilt nach \hyperref[sec:1_b]{b)}
  \begin{flalign*}
    d\qty(\varphi\qty(x_0), x^*) &= d(x_1, x^*) & \\
    &\leq \frac{q}{1 - q} d\qty(x_1, x_0) = \frac{q}{1 - q} d\qty(\varphi\qty(x_0), x_0) \\
    \Rightarrow n = 1 &\in P
  \end{flalign*}
  Sei $n \in P$. Für $n + 1$ folgt
  \begin{flalign*}
    d\qty(x_{n + 1}, x^*) &= d(\varphi\qty(x_n), \varphi\qty(x^*))
    \leq q \cdot d \qty(x_n, x^*) & \\
    \overset{n \in P}&\leq q \cdot \frac{q^n}{1 - q} d\qty(x_1, x_0) = \frac{q^{n + 1}}{1 - q} d\qty(x_1, x_0) \\
    \Rightarrow n + 1 &\in P \Rightarrow P \text{ ist induktiv, d.h. Formel gilt für alle } n \in \mathbb{N}
  \end{flalign*}
\end{enumerate}

\paragraph{Aufgabe 2:} Gegeben sei $a > \frac{1}{2}$.
Beweisen Sie, dass die rekursiv definierte Folge $\qty(x_n)_{n \in \mathbb{N}}$
mit $x_0 \coloneqq 1$ und
\begin{flalign*}
  x_{n + 1} &\coloneqq \frac{a\qty(1 + x_n)}{a + x_n} &
\end{flalign*}
konvergent ist.
Wie lautet der Grenzwert der Folge?
Für welche Werte von $a$ ist eine schnelle Konvergenz der Folge
$\qty(x_n)_{n \in \mathbb{N}}$ zu erwarten?

\subparagraph{Lösung:} Sei $\varphi(x) \coloneqq \frac{a(1 + x)}{a + x}$
für $x \in [0, +\infty) \eqqcolon M$.
$\Rightarrow \varphi \colon M \to \mathbb{R}$ mit $\varphi(M) \subseteq M$
ist eine Selbstabbildung von $M$.
Sei $d$ die vom Betrag induzierte Metrik auf $M$.
$(M, d)$ ist vollständig, da $M \subseteq \mathbb{R}$ bezüglich des Betrages
abgeschlossen ist.
Für $0 \leq x < y$ existiert ein $t \in (x, y)$ mit
\[
  d(\varphi(x) - \varphi(y)) = \abs{\varphi(x) - \varphi(y)}
  \underset{\text{Mittelwertsatz}}{\overset{\varphi(y) = \varphi(x) + r(y - x)}=}
  \abs{\varphi'(t)} \cdot \abs{y - x}
  = \abs{\varphi'(t)} d(x, y)
\]
Für $t \geq 0$ gilt
$\varphi'(t) = a \frac{(a + t) - (1 + t)}{(a + t)^2} = \frac{a(a - 1)}{(a + t)^2}
\overset{t = 0}\leq \frac{a\abs{a - 1}}{a^2} = \frac{\abs{a - 1}}{a} = q$.
\begin{itemize}
\item Für $a = 1$ gilt $q = 0$.
\item Für $a \in \qty(\frac{1}{2}, 1)$ gilt $q = \frac{1 - a}{a} = \frac{1}{a} - 1 \overset{a = \frac{1}{2}}\leq 1$
\item Für $a \in (1, +\infty)$ gilt $q = \frac{a - 1}{a} = 1 - \frac{1}{a} < 1$
\end{itemize}

Da $a$ fest gewählt ist, ist auch $q$ fest.
Für alle $a > \frac{1}{2}$ ist $\varphi$ eine kontrahierende Abbildung und
besitzt genau einen Fixpunkt $x^* \in M$, der sich aus
$x^* = \varphi\qty(x^*) = \frac{a\qty(1 + x^*)}{1 + x^*}$ ergibt.
\[
  x^* = \varphi\qty(x^*)
  \iff x^* \cdot (a + x^*) = a \cdot (1 + x^*)
  \iff \qty(x^*)^2 = a
  \overset{x_n > 0}\iff x^* = \sqrt{a}
\]
Wir haben hier ein Iterationsverfahren um Näherungsweise die Wurzel einer
positiven reellen Zahl zu bestimmen.
Eine schnelle Konvergenz ist für kleine Werte von $q$, also einem $a$ nahe bei
$1$ zu erwarten.

\paragraph{Aufgabe 3:} Gegeben sei die Nullstellenaufgabe
\[
  f(x) \coloneqq x^2 - \ln x  - 2 = 0, x > 0
\]
\begin{enumerate}[a)]
\item Bestimmen Sie Existenz und Lage der Nullstellen von $f$.
  \label{sec:3_a}
  \subparagraph{Lösung:} $\lim_{x \to +\infty} (x) =
  \lim_{x \to +\infty} x^2 - \ln x  - 2
  = \lim_{x \to +\infty} x^2 \cdot \qty(1 + \frac{\ln x}{x ^ 2} - \frac{2}{x^2}) = +\infty$

  $\lim_{x \downarrow 0} (x) =
  \lim_{x \downarrow 0} x^2 - \ln x  - 2 = +\infty$.
  Für $x = 1$ gilt $f(x) = 1$
  $\Rightarrow f$ besitzt in den Intervallen $(0, 1)$ und $(1, +\infty)$
  jeweils mindestens eine Nullstelle.
  Weiterhin gilt $f'(x) = 2x - \frac{1}{x} = \frac{1}{x} \qty\Big(2 x^2  - 1)$
  $\Rightarrow f$ ist für $x \in \qty(0, \frac{1}{\sqrt(2)})$ streng monoton
  fallend und für $x \in \qty(\frac{1}{\sqrt{2}}, +\infty)$ ist $f$ streng
  monoton wachsend mit
  $f\qty(\frac{1}{\sqrt{2}}) = \frac{1}{2} - \frac{1}{2} \ln \frac{1}{2} - 2 < 0$.
  $\Rightarrow f$ besitzt in jedem der Intervalle
  $\qty(0, \frac{1}{\sqrt{2}})$ und $\qty(\frac{1}{\sqrt{2}}, +\infty)$
  höchstens eine Nullstelle.
  $\Rightarrow f$ besitzt genau zwei Nullstellen.

\item Finden Sie ein geeignetes äquivalentes Fixpunktproblem, welches
  die Vorraussetzungen des Fixpunktsatzes von Banach in einer abgeschlossenen
  Umgebung der größeren unter \hyperref[sec:3_a]{a)} ermittelten Nullstelle
  erfüllt.
  Beschränken Sie sich auf das Intervall $[1, 2]$

  \subparagraph{Lösung:} Wir bestimmen die Nullstelle in $[1, 2] \eqqcolon M$.
  \label{sec:3_b}
  Da $f(2) = 4 - \underset{< 1}{\underbrace{\ln 2}} - 2 > 1$.
  $M$ ist mit der Betragsmetrik ein vollständiger metrischer Raum.

  Wir erhalten eine Abbildung $\varphi (x)$ durch Auflösen von $f(x) = 0$
  nach $x$ mit $f(x) = 0 \iff x = \varphi(x)$.
  \begin{flalign*}
    \varphi_1(x) &= x - f(x) & \\
    \varphi_2(x) &= \frac{2 + \ln x}{x} \\
    \varphi_3(x) &= \sqrt{2 + \ln x} \\
    \varphi_4(x) &= e^{x^2 - 2}
  \end{flalign*}
  \newpage
  Es gilt $\abs{\varphi_1'(x)} = \abs{1 - f'(x)} = \abs{1 - \qty(2x - \frac{1}{2})} = \abs{\frac{-2x^2 + x - 1}{x}}$
  \begin{align*}
    \varphi_1'(x) &\leq \frac{2x^2 + x + 1}{x} \leq \frac{2 \cdot 2^2 + 2 + 1}{1} = 11 \text{ (keine optimale Abschätzung)} \\
    \abs{\varphi_2'(x)} &= \abs{\frac{1 - (2 + \ln x)}{x^2}} = \abs{\frac{1 + \ln x}{x^2}} < \frac{1 + x}{x^2} = \frac{1}{x^2} + \frac{1}{x} \leq 2 \\
    \abs{\varphi_3'(x)} &= \colorbox{yellow}{$\abs{\frac{1}{2 \sqrt{2 + \ln x} \cdot x}} \overset{x = 1}\leq \frac{1}{2 \sqrt{2}} = q < 1$} \\
    \abs{\varphi_4'(x)} &= \abs{e^{x^2 - 2} \cdot 2x} \overset{x = 2}\leq \underset{> 1}{\underbrace{e^2 \cdot 4}}
  \end{align*}
  Es ist nun noch zu zeigen, dass $\varphi_3$ eine Selbstabbildung ist,
  also $\varphi_3(M) \subseteq M$.
  Wegen $\varphi_3'(x) > 0$ für $x > 0$ ist $\varphi_3$ monoton wachsend.

  $\Rightarrow \varphi_3(1) = \sqrt{2} \leq \varphi_3(x) \leq \varphi_3(2) = \sqrt{2 + \ln 2} < 2$

  $\Rightarrow \varphi_3(M) \subseteq M$.

  Somit ist der Fixpunktsatz von Banach anwendbar.

\item Bestimmen Sie für die Startwerte $x_0 = 1$ bzw. $x_0 = 2$ die ersten vier
  Iteriertien für das unter \hyperref[sec:3_b]{b)} bestimmte Fixpunktproblem.

  \subparagraph{Lösung:}
  \begin{tabular}{| c c c c c c |}
    \hline
    $x_0$ & $x_1$ & $x_2$ & $x_3$ & $x_4$ & $x_5$ \\
    \hline
    $1$ & $1.414$ & $1.532$ & $1.558$ & $1.563$ & $1.564$ \\
    $2$ & $1.641$ & $1.580$ & $1.568$ & $1.565$ & $1.565$ \\
    \hline
  \end{tabular}
\end{enumerate}

\paragraph{Aufgabe 4:} Beweisen Sie, dass es genau ein $x \in C([-1, 1])$ gibt
mit
\[
  x(t) = \exp\qty(t^2) + \frac{1}{5}\qty\Big(2 - \sin\qty\big(3x(t))) \: \text{für alle } t \in [-1, 1]
\]

\subparagraph{Lösung:} $M = C([-1, 1]), d_{\infty}(x, y) = \norm{x - y}_{\infty}$.
$(M, d)$ ist ein vollständiger metrischer Raum.

\begin{flalign*}
  \varphi\qty\big(x(t)) &\coloneqq e^{t^2} + \frac{1}{5} \qty\Big(2 - \sin\qty\big(3x(t))), t \in [-1, 1] & \\
  \varphi &\colon (x) \colon [-1, 1] \to \mathbb{R} \: \text{ist stetig} \\
  \varphi &\colon M \to M \: \text{Selbstabbildung (Abbildung von den stetigen Funktionen zu den stetigen Funktionen)}
\end{flalign*}
Seien $x, y \in M$ gegeben.
Dann gilt für $t \in [-1, 1]$
\begin{flalign*}
  \abs{\varphi(x)(t) - \varphi(y)(t)}
  &= \abs{\qty(e^{t^2} + \frac{1}{5}\qty\Big(2 - \sin(3x(t)))) - \qty(e^{t^2} + \frac{1}{5}\qty\Big(2 - \sin(3y(t))))} \\
  &= \frac{1}{5} \abs{\sin(3x(t)) - \sin(3y(t))}
\end{flalign*}
Sei $g(u) \coloneqq \sin u, u \in \mathbb{R}$.
Nach dem Mittelwertsatz existiert für $u < v$ ein $\xi \in (u, v)$ mit
$\abs{g(u) - g(v)} = \abs{g'(\xi)} \cdot \abs{u - v} = \underset{\leq 1}{\underbrace{\abs{\cos \xi}}} \abs{u - v} \leq \abs{u - v}$.

Mit $u = 3x(t), v = 3y(t)$ folgt für alle $t \in [-1, 1]$
\[
  \abs{\varphi\qty\big(x(t)) - \varphi\qty\big(y(t))}
  \leq \frac{3}{5}\abs{x(t) - y(t)}
  \leq \frac{3}{5} \norm{x - y}_{\infty}
  = \frac{3}{5}d_{\infty}(x, y)
\]
$\Rightarrow d_{\infty}(\varphi(x), \varphi(y))
= \sup_{t \in [-1, 1]} \abs{\varphi\qty\big(x(t)) - \varphi\qty\big(y(t))}
\leq \frac{3}{5} d_{\infty}(x, y)$

$\Rightarrow \varphi$ ist kontrahierend.
Nach dem Banachschen Fixpunktsatz gibt es genau einen Fixpunkt.

\end{document}