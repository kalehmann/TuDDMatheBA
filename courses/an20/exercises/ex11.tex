\documentclass{scrreprt}

\usepackage{aligned-overset}
\usepackage{amsmath}
\usepackage{amssymb}
\usepackage{bm}
\usepackage[shortlabels]{enumitem}
\usepackage{hyperref}
\usepackage[utf8]{inputenc}
\usepackage{mathtools}
\usepackage{physics}
\usepackage{tabularx}
\usepackage{titling}
\usepackage{fancyhdr}
\usepackage{xfrac}
\usepackage[table]{xcolor}
\usepackage{pgfplots}

\definecolor{light-gray}{gray}{.9}

\pgfplotsset{compat = newest}
\usepgfplotslibrary{patchplots}

\usetikzlibrary{intersections}
\usetikzlibrary{patterns}
\usepgfplotslibrary{fillbetween}

\author{Karsten Lehmann}
\date{SoSe 2021}
\title{Übung 10 Analysis - Weiterführende Konzepte}

\pagestyle{fancy}
\fancyhf{}
\lhead{\thetitle}
\rhead{\theauthor}
\lfoot{\thedate}
\rfoot{Seite \thepage}

\newcommand\skalprod[1]{\left\langle #1 \right\rangle}
\newcommand\nnorm[1]{\left\lvert\left\lvert\left\lvert #1 \right\rvert\right\rvert\right\rvert}

\begin{document}
 \fcolorbox{black}{light-gray}{
    \begin{minipage}{\textwidth}
      \textbf{Satz über die lokale Auflösbarkeit (endlichdimensionale Version)}

      $U \subseteq \mathbb{R}^N, V \subseteq \mathbb{R}^M$, offen.
      $F \colon U \times V \to \mathbb{R}^M$ sei eine $C^k$-Funktion
      ($k$-mal stetig differenzierbar), so dass
      $(x, y) \mapsto J_F^{(y)} (x, y)$ auf $U \times V$ invertierbar ist.

      $\Rightarrow$ Für alle $\qty(\overline{x}, \overline{y}) \in U \times V$
      existieren offene Umgebungen $\tilde{U} \subseteq \mathbb{R}^N$ von
      $\overline{x}$ und $\tilde{V} \subseteq \mathbb{R}^N$ von
      $\overline{y}$ und eine $C^k$-Funktion
      $f \colon \tilde{U} \times \tilde{V} \subseteq U \times V \to \tilde{V}$
      mit folgenden Eigenschaften:
      \begin{itemize}
      \item $\forall (x, y) \in \tilde{U} \times \tilde{V} \colon F(x, y) = F(\overline{x}, \overline{y}) \iff y = f(x)$
      \item $J_f(x) = - \qty\Big(J_F^{(y)}\qty(x, f(x)))^{-1} \cdot J_F^{(x)}\qty(x, f(x))$
      \end{itemize}
      Insofern gilt $\overline{y} = f(\overline{x})$ und
      $J_f(\overline{x}) = - \qty(J_f^{(y)}(\overline{x}, \overline{y}))^{-1} J_F^{(x)}(\overline{x}, \overline{y})$
    \end{minipage}
  }


\paragraph{Aufgabe 2} Gegeben sei eine Funktion
$f \colon \mathbb{R}^4 \to \mathbb{R}^2$ mit
\[
  F(x, y, u, v) = \begin{pmatrix}
    x^2 + y^2 - u^2 - v^2 \\
    x^2 + 2y^2 + 3u^2 + 4v^2 - 1
  \end{pmatrix}, \text{ für $(x, y, u, v) \in \mathbb{R}^4$}
\]
Betrachtet werde das nichtlineare Gleichungssystem
$F(x, y, u, v) = (0, 0)$ in der Umgebung des Punktes
$\qty\big(x_0, y_0, u_0, v_0) = \qty(\frac{1}{5}, \frac{2}{5}, \frac{2}{5}, \frac{1}{5})$

\begin{enumerate}[a)]
\item Überprüfen Sie, ob dieser Punkt das Gleichungssystem erfüllt.

  \subparagraph{Lösung:} Durch Einsetzen und ausrechnen.
  \begin{flalign*}
    F\qty(x_0, y_0, u_0, v_0) &= \begin{pmatrix}
      \frac{1}{25} + \frac{4}{25} - \frac{4}{25} - \frac{1}{25} \\
      \frac{1}{25} + \frac{8}{25} + \frac{12}{25} + \frac{4}{25} - \frac{25}{25}
    \end{pmatrix} & \\
    &= \begin{pmatrix}
      0 \\
      0
    \end{pmatrix}
  \end{flalign*}
  $\Rightarrow$ der Punkt erfüllt das Gleichungssytem.

\item Zeigen Sie, dass es eine Umgebung $\tilde{U}$ von $\qty(x_0, y_0)$
  und positive Funktionen $u, v \colon \tilde{U} \to \mathbb{R}$ gibt mit
  $F\qty\big(x, y, u(x, y), v(x, y)) = (0, 0)$

  \subparagraph{Lösung:} \textbf{Ist $F$ eine $C^1$-Funktion?}
  Es gilt
  \[
    J_F(x, y, u, v) = \left(
      \begin{array}{>{\color{red}}c >{\color{red}}c >{\color{black!40!green}}c >{\color{black!40!green}}c}
        2x & 2y & -2u & -2v \\
        2x & 4y & 6u & 8v \\
      \end{array}
    \right)
  \]
  Dabei teilt sich die Jakobimatrix auf in
  \colorbox{red!30}{$J_F^{{x, y}}(\ldots)$} und
  \colorbox{black!20!green!40}{$J_F^{(u, v)}(\ldots)$}.

  Die Ableitung $(x, y, u, v) \mapsto J_F(\ldots)$ ist stetig
  $\Rightarrow F$ ist eine $C^1$-Funktion (einmal stetig differenzierbar).

  \textbf{Ist die Jakobimatrix invertierbar?}
  \[
    \det J_F^{(u, v)} (x, y, u, v) = -4uv \ne 0, \text{ für } u, v \ne 0
  \]
  Insbesondere ist $\det J_F^{(u, v)} (x_0, y_0, u_0, v_0) = -\frac{8}{25} < 0$.

  $\Rightarrow \exists \qty\Big(J_F^{(u, v)}\qty(x_0, y_0, u_0, v_0))$

  Wie können wir darauf schließen, dass es eine Umgebung des Punktes
  $\qty(x_0, y_0, u_0, v_0)$ gibt, in der $J_F^{(u, v)}$ invertierbar ist?

  $(x, y, u, v) \mapsto J_F^{(u, v)} (x, y, u, v)$ ist stetig und
  $\det J_F^{(u, v)} \qty(x_0, y_0, u_0, v_0) < 0$

  $\Rightarrow$ Es existiert eine offene Umgebung $\overline{W}$ von
  $\qty(x_0, y_0, u_0, v_0)$ mit $\det J_F^{(u, v)} (x, y, u, v) < 0$
  für alle $x, y, u, v \in \overline{W}$

  $\Rightarrow J_F^{(u,v)}$ ist auf $\overline{W}$ invertierbar.

  $\Rightarrow$ alle Vorraussetzungen des Satzes zur lokalen Auflösbarkeit sind
  erfüllt.

  $\Rightarrow$ es existiert $\tilde{U} \subseteq \mathbb{R}^2$ von $\qty(x_0, y_0)$
  und eine $C^1$-Funktion $f \colon \tilde{U} \to \mathbb{R}^2$ mit
  \[
    F\qty\big(x, y, f(x, y)) = \begin{pmatrix}
      0 \\
      0
    \end{pmatrix}
    \text{ und }
    f(x, y) = \begin{pmatrix}
      u(x, y) \\
      v(x, y)
    \end{pmatrix}
  \]
  für alle $(x, y) \in \tilde{U}$.

  \textbf{Warum sind $u, v$ positiv?}

  Es gilt $u\qty(x_0, y_0) = u_0 > 0, v\qty(x_0, y_0) = v_0 > 0$
  und $u, v \colon \tilde{U} \to \mathbb{R}^2$ sind $C^1$-Funktionen,
  also auch stetig.
  Demnach kann $\tilde{U}$ so gewählt werden, dass $u, v$ positiv sind.

\item Berechnen Sie die partielle Ableitung der Funktionen $u$ und $v$ an
  der Stelle $\qty(x_0, y_0) = \qty(\frac{1}{5}, \frac{2}{5})$.

  \subparagraph{Lösung:} Für $(x, y) \in \tilde{U}$ gilt
  \begin{flalign*}
    J_f(x, y) &= - \qty(J_F^{(u,v)} \qty\big(x, y, u(x, y), v(x, y))) \cdot J_F^{(x, y)} \qty(\ldots) & \\
    &= - \begin{pmatrix}
      -2u & -2v \\
      6u & 8v \\
    \end{pmatrix}^{-1} \cdot \begin{pmatrix}
      2x & 2y \\
      2x & 4y \\
    \end{pmatrix} \\
    &= - \frac{1}{-4uv} \begin{pmatrix}
      8v & 2u \\
      -6u & -2u \\
    \end{pmatrix} \cdot \begin{pmatrix}
      2x & 2y \\
      2x & 4y
    \end{pmatrix}
    = \begin{pmatrix}
      5 \frac{x}{u} & 6 \frac{y}{u} \\
      - 4 \frac{x}{u} & - 5 \frac{y}{v}
    \end{pmatrix}
  \end{flalign*}
  Man kann $f(x, y)$ explizit angeben:
  \[
    f(x, y) = \begin{pmatrix}
      \sqrt{-1 + 5x^2 + 6y^2} \\
      \sqrt{1 - 4x^2 - 5y^2}
    \end{pmatrix}
  \]
  $D_f = \qty{(x, y) \in \mathbb{R}^2 {\big |} 5 x^2 + 6y^2 < 1 < 4x^2 + 5y^2  }$

\end{enumerate}

\end{document}