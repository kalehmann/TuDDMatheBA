\documentclass{scrreprt}

\usepackage{aligned-overset}
\usepackage{amsmath}
\usepackage{amssymb}
\usepackage{bm}
\usepackage[shortlabels]{enumitem}
\usepackage{hyperref}
\usepackage[utf8]{inputenc}
\usepackage{mathtools}
\usepackage{physics}
\usepackage{tabularx}
\usepackage{titling}
\usepackage{fancyhdr}
\usepackage{xfrac}
\usepackage{pgfplots}

\definecolor{light-gray}{gray}{.9}

\pgfplotsset{compat = newest}
\usepgfplotslibrary{patchplots}

\usetikzlibrary{intersections}
\usetikzlibrary{patterns}
\usepgfplotslibrary{fillbetween}

\author{Karsten Lehmann}
\date{SoSe 2021}
\title{Übung 09 Analysis - Weiterführende Konzepte}

\pagestyle{fancy}
\fancyhf{}
\lhead{\thetitle}
\rhead{\theauthor}
\lfoot{\thedate}
\rfoot{Seite \thepage}

\newcommand\skalprod[1]{\left\langle #1 \right\rangle}
\newcommand\nnorm[1]{\left\lvert\left\lvert\left\lvert #1 \right\rvert\right\rvert\right\rvert}

\begin{document}

\paragraph{Aufgabe 1} Seien $M \subseteq \mathbb{R}^N$ eine offene Menge und
$f \colon M \to \mathbb{R}$.
Die Menge
\begin{flalign*}
  N_C &\coloneqq \qty{x \in M {\Big |} f(x) = c} &
\end{flalign*}
heißt \textit{Niveaumenge} von $f$ zum Niveau $c \in \mathbb{R}$.
Sei $x_0 \in N_c$.
Zeigen Sie, dass $\nabla f\qty(x_0)$ auf $N_c$ in folgendem Sinne senkrecht
steht:

Für jede stetig differenzierbare Funktion
$\varphi \colon (-\epsilon, \epsilon) \to \mathbb{R}^N, \epsilon > 0$,
mit $\varphi(0) = x_0$ und $\varphi(t) \in N_c$ für $\abs{t} < \epsilon$
gilt
\begin{flalign*}
  \skalprod{\nabla f\qty(x_0), \varphi'(0)} &= 0 &
\end{flalign*}

\subparagraph{Lösung:} Die Funktion
$g \coloneqq f \circ \varphi \colon (-\epsilon, \epsilon) \to \mathbb{R}$
ist ebenfalls stetig differenzierbar.
Für $\abs{t} < \epsilon$ gilt $g(t) = f(\varphi(t)) = f\qty(x_0) = c$.
Somit ist $g'(t)$ konstant mit
\[
  0 = g'(t) = f'\qty(\varphi(t)) \cdot \varphi'(t)
  = \skalprod{\nabla f(\varphi(t)), \varphi'(t)}
  \quad \text{(Kettenregel)}
\]

\end{document}