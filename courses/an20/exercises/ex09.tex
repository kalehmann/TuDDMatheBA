\documentclass{scrreprt}

\usepackage{aligned-overset}
\usepackage{amsmath}
\usepackage{amssymb}
\usepackage{bm}
\usepackage[shortlabels]{enumitem}
\usepackage{hyperref}
\usepackage[utf8]{inputenc}
\usepackage{mathtools}
\usepackage{physics}
\usepackage{tabularx}
\usepackage{titling}
\usepackage{fancyhdr}
\usepackage{xfrac}
\usepackage{pgfplots}

\definecolor{light-gray}{gray}{.9}

\pgfplotsset{compat = newest}
\usepgfplotslibrary{patchplots}

\usetikzlibrary{intersections}
\usetikzlibrary{patterns}
\usepgfplotslibrary{fillbetween}

\author{Karsten Lehmann}
\date{SoSe 2021}
\title{Übung 09 Analysis - Weiterführende Konzepte}

\pagestyle{fancy}
\fancyhf{}
\lhead{\thetitle}
\rhead{\theauthor}
\lfoot{\thedate}
\rfoot{Seite \thepage}

\newcommand\skalprod[1]{\left\langle #1 \right\rangle}
\newcommand\nnorm[1]{\left\lvert\left\lvert\left\lvert #1 \right\rvert\right\rvert\right\rvert}

\begin{document}

\paragraph{Aufgabe 1} Seien $M \subseteq \mathbb{R}^N$ eine offene Menge und
$f \colon M \to \mathbb{R}$.
Die Menge
\begin{flalign*}
  N_C &\coloneqq \qty{x \in M {\Big |} f(x) = c} &
\end{flalign*}
heißt \textit{Niveaumenge} von $f$ zum Niveau $c \in \mathbb{R}$.
Sei $x_0 \in N_c$.
Zeigen Sie, dass $\nabla f\qty(x_0)$ auf $N_c$ in folgendem Sinne senkrecht
steht:

Für jede stetig differenzierbare Funktion
$\varphi \colon (-\epsilon, \epsilon) \to \mathbb{R}^N, \epsilon > 0$,
mit $\varphi(0) = x_0$ und $\varphi(t) \in N_c$ für $\abs{t} < \epsilon$
gilt
\begin{flalign*}
  \skalprod{\nabla f\qty(x_0), \varphi'(0)} &= 0 &
\end{flalign*}

\subparagraph{Lösung:} Die Funktion
$g \coloneqq f \circ \varphi \colon (-\epsilon, \epsilon) \to \mathbb{R}$
ist ebenfalls stetig differenzierbar.
Für $\abs{t} < \epsilon$ gilt $g(t) = f(\varphi(t)) = f\qty(x_0) = c$.
Somit ist $g'(t)$ konstant mit
\[
  0 = g'(t) = f'\qty(\varphi(t)) \cdot \varphi'(t)
  = \skalprod{\nabla f(\varphi(t)), \varphi'(t)}
  \quad \text{(Kettenregel)}
\]

\paragraph{Aufgabe 2} Gegeben sei die Funktion
$f \colon \mathbb{R}^2 \to \mathbb{R}$ mit
$f\qty(x_1, x_2) = e^{2x_1}\qty(x_1 + x_2^2)$.
\begin{enumerate}[a)]
\item Verwenden Sie die Multiindexschreibweise um die Ausdrücke
  \[
    \qty(h_1\frac{\partial}{\partial x_1} + h_2 \frac{\partial}{\partial x_2})^k f\qty(x_1, x_2) \text{ für } k = 0,1,2,3
  \]
  darzustellen und berechnen Sie diese für die gegebene Funktion $f$.

  \subparagraph{Lösung:} Die Funktion $f$ ist beliebig oft stetig partiell
  differenzierbar.
  Für $a = \qty(a_1, a_2) \in \mathbb{N}_0^2$ und $h = \qty(h_1, h_2) \in \mathbb{R}^2$ ist
  \[
    h^a \frac{\partial^a}{\partial x^a}f(x)
    = h^{\qty(a_1, a_2)} \frac{\partial^{\qty(a_1, a_2)}}{\partial x^{\qty(a_1, a_2)}}f(x)
    = h_1^{a_1} \cdot h_2^{a_2} \frac{\partial^{a_1}}{\partial x_1^{a_1}} \qty(\frac{\partial^{a_2}}{\partial x_2^{a_2}} f(x))
  \]
  \begin{flalign*}
    \qty(h_1\frac{\partial}{\partial x_1} + h_2 \frac{\partial}{\partial x_2})^0 f\qty(x_1, x_2)
    &= f\qty(x_1, x_2) & \\
    \qty(h_1\frac{\partial}{\partial x_1} + h_2 \frac{\partial}{\partial x_2})^1 f\qty(x_1, x_2)
    &= h^{(1, 0)}\frac{\partial^1}{\partial x_1^1} + h^{(0, 1)}\frac{\partial^1}{\partial x_2^1} \\
    &= e^{2x_1} \qty(2x_1 + 2x_2^2 + 1) h_1 + 2e^{2x_1} x_2 h_2 \\
    &= e^{2x_1} \qty\Big(\qty(2x_1 + 2x_2^2 + 1)h_1 + x_2h_2)
  \end{flalign*}
  \newpage
  \begin{flalign*}
    \qty(h_1\frac{\partial}{\partial x_1} + h_2 \frac{\partial}{\partial x_2})^2 f\qty(x_1, x_2)
    &= \qty(h_1^2\frac{\partial^2}{\partial x_1^2}
    + h_1h_2 \frac{\partial}{\partial x_1}\frac{\partial}{\partial x_2}
    + h_2h_1 \frac{\partial}{\partial x_2}\frac{\partial}{\partial x_1}
    + h_2^2\frac{\partial^2}{\partial x_2^2}) f(x) & \\
    &= \begin{pmatrix}
      h_1 \frac{\partial^2}{\partial x_1^2}
      + h_2 \frac{\partial}{\partial x_1}\frac{\partial}{\partial x_2} &
      h_1 \frac{\partial}{\partial x_2}\frac{\partial}{\partial x_1}
      + h_2 \frac{\partial^2}{\partial x_2^2})
    \end{pmatrix} \cdot
    \begin{pmatrix}
      h_1 \\
      h_2
    \end{pmatrix} f(x) \\
    &= \begin{pmatrix}
      h_1 & h_2
    \end{pmatrix}
    \cdot
    \begin{pmatrix}
      \frac{\partial^2}{\partial x_1^2} & \frac{\partial}{\partial x_2}\frac{\partial}{\partial x_1} \\
      \frac{\partial}{\partial x_1}\frac{\partial}{\partial x_2} & \frac{\partial^2}{\partial x_2^2}) \\
    \end{pmatrix} \cdot
    \begin{pmatrix}
      h_1 \\
      h_2
    \end{pmatrix} f(x) \\
     &= h^T \cdot
    \begin{pmatrix}
      \frac{\partial^2}{\partial x_1^2} & \frac{\partial}{\partial x_2}\frac{\partial}{\partial x_1} \\
      \frac{\partial}{\partial x_1}\frac{\partial}{\partial x_2} & \frac{\partial^2}{\partial x_2^2}) \\
    \end{pmatrix} \cdot h f(x) \\
    &= h^T H_f(x) h \quad \text{(Die Hessematrix von $f$ in Richtung $h$)} \\
    &= e^{2x_1}h^T \begin{pmatrix}
      4x_1 + 4x_2^2 + 4 & 4x_2 \\
      4x_2 & 2
    \end{pmatrix} h \\
    \qty(h_1\frac{\partial}{\partial x_1} + h_2 \frac{\partial}{\partial x_2})^3 f\qty(x_1, x_2)
    &= \qty(h^{(3, 0)} \frac{\partial^{(3, 0)}}{\partial x^{(3, 0)}}
    + h^{(2, 1)} \frac{\partial^{(2, 1)}}{\partial x^{(2, 1)}}
    + h^{(1, 2)} \frac{\partial^{(1, 2)}}{\partial x^{(1, 2)}}
    + h^{(0, 3)} \frac{\partial^{(0, 3)}}{\partial x^{(0, 3)}}) f(x) \\
    &= 4 h_1^3 e^{2x_1}\qty(x_1 + x_2^2 + 2) + 8 e^{2x_1}h_1^2h_2 + 4 e^{2x_1}h_1h_2^2 + 0 \\
    &= 4e^{2x_1}h_1 \qty\Big(h_1^2 \qty(x_1 + x_2^2 + 2) + 2h_1h_2 + h_2^2)
  \end{flalign*}
  (Die 3. Ableitung von $f$ in Richtung $h$).
  \newpage
  Vollständige Auflistung der verwendeten Ausdrücke:
  \begin{flalign*}
    h^{(0,0)} \frac{\partial^{(0, 0)}}{\partial x^{(0, 0)}} f(x)
    &= h_1^0 \cdot h_2^0 \cdot \frac{\partial^0}{\partial x_1^0} \qty(\frac{\partial^0 f(x)}{\partial x_2^0})
    = \frac{\partial^0}{\partial x_1^0} f(x) = f(x) & \\
    h^{(1,0)} \frac{\partial^{(1, 0)}}{\partial x^{(1, 0)}} f(x)
    &= h_1^1 \cdot h_2^0 \cdot \frac{\partial^1}{\partial x_1^1} \qty(\frac{\partial^0 f(x)}{\partial x_2^0})
    = h_1 \frac{\partial}{\partial x_1} f(x)
    = e^{2x_1} \qty(2x_1 + 2x_2^2 + 1) h_1 \\
    h^{(0,1)} \frac{\partial^{(0, 1)}}{\partial x^{(0, 1)}} f(x)
    &= h_1^0 \cdot h_2^1 \cdot \frac{\partial^0}{\partial x_1^0} \qty(\frac{\partial^1 f(x)}{\partial x_2^1})
    = h_2 \cdot \frac{\partial}{\partial x_2} f(x)
    = 2e^{2x_1} x_2 h_2 \\
    h^{(1,1)} \frac{\partial^{(1, 1)}}{\partial x^{(1, 1)}} f(x)
    &= h_1^1 \cdot h_2^1 \cdot \frac{\partial^1}{\partial x_1^1} \qty(\frac{\partial^1 f(x)}{\partial x_2^1})
    = 4 e^{2x_1}x_2h_1h_2 \\
    h^{(0,2)} \frac{\partial^{(0, 2)}}{\partial x^{(0, 2)}} f(x)
    &= h_2^2 \cdot\frac{\partial^2 f(x)}{\partial x_2^2}
    = 2h_2^2 e^{2x_1} \\
    h^{(2,0)} \frac{\partial^{(2, 0)}}{\partial x^{(2, 0)}} f(x)
    &= h_1^2 \cdot \frac{\partial^2 f(x)}{\partial x_1^2}
    = 4 h_1^2 e^{2x_1}\qty(x_1 + x_2^2 + 1) \\
    h^{(2,1)} \frac{\partial^{(2, 1)}}{\partial x^{(2, 1)}} f(x)
    &= h_1^2 \cdot h_2^1 \cdot \frac{\partial^2}{\partial x_1^2} \qty(\frac{\partial^1 f(x)}{\partial x_2^1})
    = 8 e^{2x_1}h_1^2h_2 \\
    h^{(1,2)} \frac{\partial^{(1, 2)}}{\partial x^{(1, 2)}} f(x)
    &= h_1^1 \cdot h_2^2 \cdot \frac{\partial^1}{\partial x_1^1} \qty(\frac{\partial^2 f(x)}{\partial x_2^2})
    = 4 e^{2x_1}h_1h_2^2 \\
    h^{(3,0)} \frac{\partial^{(3, 0)}}{\partial x^{(3, 0)}} f(x)
    &= h_1^3 \cdot \frac{\partial^3 f(x)}{\partial x_1^3}
    = 4 h_1^3 e^{2x_1}\qty(x_1 + x_2^2 + 2) \\
    h^{(0,3)} \frac{\partial^{(0, 3)}}{\partial x^{(0, 3)}} f(x)
    &= h_2^3 \cdot\frac{\partial^3 f(x)}{\partial x_2^3}
    = 2h_2^3 \partial^3_2 f(x) = 0
  \end{flalign*}

\end{enumerate}

\end{document}