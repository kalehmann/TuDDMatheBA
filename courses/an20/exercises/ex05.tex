\documentclass{scrreprt}

\usepackage{aligned-overset}
\usepackage{amsmath}
\usepackage{amssymb}
\usepackage{bm}
\usepackage[shortlabels]{enumitem}
\usepackage{hyperref}
\usepackage[utf8]{inputenc}
\usepackage{mathtools}
\usepackage{physics}
\usepackage{tabularx}
\usepackage{titling}
\usepackage{fancyhdr}
\usepackage{xfrac}
\usepackage{pgfplots}

\definecolor{light-gray}{gray}{.9}

\pgfplotsset{compat = newest}
\usetikzlibrary{intersections}
\usetikzlibrary{patterns}
\usepgfplotslibrary{fillbetween}

\author{Karsten Lehmann}
\date{SoSe 2021}
\title{Übung 05 Analysis - Weiterführende Konzepte}

\pagestyle{fancy}
\fancyhf{}
\lhead{\thetitle}
\rhead{\theauthor}
\lfoot{\thedate}
\rfoot{Seite \thepage}

\begin{document}

\section*{Stetigkeit in metrischen Räumen}

Sei $d$ die Betragsmetrik auf $\mathbb{R}$.
Die Menge $M \coloneqq [0, 1] \cup [2, 4)$ sei mit der induzierten Metrik
$d|_M$ versehen.
Untersuchen Sie mit Hilfe der \colorbox{green}{$\epsilon-\delta$-Definition}, in welchen Punkten
von $M$ die Abbildung $f \colon \qty(M, d|_M) \to \qty(\mathbb{R}, d)$
definiert durch
\[
  f(x) \coloneqq \begin{cases}
    \colorbox{blue!20}{1} & \text{für } x \in [0, 1] \\
    \colorbox{yellow!50}{0} & \text{für } x \in [2, 4] \\
  \end{cases}
\]
stetig ist. \\

\textit{Lsg.}

\colorbox{green}{$\epsilon-\delta$-Definition}: Seien $\qty(M_1, d_1)$
und $\qty(M_2, d_2)$ metrische Räume.
Eine Funktion $f \colon M_1 \to M_2$ heißt stetig in $x \in M_1$, wenn
\[
  \forall \epsilon > 0 \exists \delta > 0 \forall y \in M_1 \colon d_1(x, y) < \delta
  \iff d_2(f(x), f(y)) < \epsilon
\]

Es gilt für $x \in M$ und $\delta > 0$ per Definition der induzierten Metrik:
\colorbox{orange}{$B_{d|_M} (x, \delta) = B_d(x, \delta) \cap M$}

Teilen wir die Menge $M$ nun in die beiden Teilmengen
$M_1 = [0, 1]$ und $M_2 = [2, 4)$ auf.

Sei die Menge $A = (0, 1) \cup (2, 4) \subseteq M$ mit der induzierten Metrik
$d|_A$ und $A, d|_A)$ ein metrischer Raum.
\begin{itemize}
\item Die Menge $A$ ist offen $\Rightarrow$ für jedes $x \in A$ existiert ein
  $\delta > 0$ mit $B_{d|_A}(x, \delta \subseteq A)$.
  \[
    \colorbox{orange}{$B_{d|_M} (x, \delta) = B_d(x, \delta) \cap M$}
    = B_d(x, \delta) \cap A \subseteq M_1 \cup M_2
  \]

\item $x = 0 \colon$ Für $\delta \in (0, 1]$ ist
  $B_{d|_M}(x, \delta) = B_{\underset{d}{\underbrace{\abs{\cdot}}}}(x, \delta) \cap M
  = [0, \delta) \subseteq M_1$

\item $x = 2 \colon$ Für $\delta \in (0, 1]$ ist
  $B_{d|_M}(x, \delta) = B_{\underset{d}{\underbrace{\abs{\cdot}}}}(x, \delta) \cap M
  = [2, 2 + \delta) \subseteq M_2$

\item $x = 1 \colon$ Für $\delta \in (0, 1]$ ist
  $B_{d|_M}(x, \delta) = B_{\underset{d}{\underbrace{\abs{\cdot}}}}(x, \delta) \cap M
  = (1 - \delta, 1] \subseteq M_1$

\end{itemize}
\begin{flalign*}
  &\Rightarrow \text{ für jedes } x \in M_1 \text{ existiert ein } \delta > 0
  \text{ mit } B_{d_M}(x, \delta) \subseteq M_1 & \\
  &\Rightarrow y \in B_{d|_M}(x, \delta) \colon f(x) = f(y) = \colorbox{blue!20}{$1$} \\
  &\Rightarrow \colon \forall \epsilon > 0 \colon d(f(x), f(y)) = \abs{f(x) - f(y)} = 0 < \epsilon \\
  &\Rightarrow f \text{ ist auf $M_1$ stetig}
\end{flalign*}
Analog für $M_2$:
\begin{flalign*}
  &\Rightarrow \text{ für jedes } x \in M_2 \text{ existiert ein } \delta > 0
  \text{ mit } B_{d_M}(x, \delta) \subseteq M_2 & \\
  &\Rightarrow y \in B_{d|_M}(x, \delta) \colon f(x) = f(y) = \colorbox{yellow!50}{$0$} \\
  &\Rightarrow \colon \forall \epsilon > 0 \colon d(f(x), f(y)) = \abs{f(x) - f(y)} = 0 < \epsilon \\
  &\Rightarrow f \text{ ist auf $M_2$ stetig}
\end{flalign*}

\newpage
Gegeben sei die Funktion $f \colon \mathbb{R}^2 \to \mathbb{R}$ mit
\[
  f(x) \coloneqq \begin{cases}
    \frac{x_1 \cdot x_2}{x_1^2 + x_2^2} & \text{für } x = \qty(x_1, x_2) \ne (0, 0) \\
    0 & \text{für } x = (0, 0)
  \end{cases}
\]
\begin{enumerate}[a)]
\item Zeigen Sie, dass die Funktion $f$ partiell stetig auf $\mathbb{R}^2$ ist,
  dass heißt für jeden Punkt $x = (x_1, x_2) \in \mathbb{R}^2$ sind die
  Abbildungen $g_1(t) \coloneqq f(x_1 + t, x_2)$ und
  $g_2(t) \coloneqq f(x_1, x_2 + t)$ stetig in $t = 0$.
\item Zeigen Sie, dass die Funktion $f$ nicht stetig auf $\mathbb{R}^2$ ist.

  \textit{Lsg.} Die Funktion ist nicht stetig im Punkt $x = (0, 0)$.

  Falls die Funktion $f$ in $x$ stetig wäre, müsste gelten:
  \[
    \lim_{t \to 0} f(t, t) = f(0, 0)
  \]
  Nun ist jedoch $f(0, 0) = 0$ und
  $\lim_{t \to 0} f(t, t) = \lim_{t \to 0} \frac{t^2}{2t^2} = \frac{1}{2}$.
  Ein Widerspruch, damit ist $f$ auf $\mathbb{R}^2$ nicht stetig.

\end{enumerate}

Seien $\qty(M_1, d_1)$ und $\qty(M_2, d_2)$ metrische Räume.
\begin{enumerate}[(i)]
\item Beweisen Sie, dass eine Abbildung $f \colon M_1 \to M_2$ genau dann
  stetig ist, wenn das Urbild
  $f^{-1}(A) = \qty{x \in M_1 \middle| f(x) \in A} \subseteq M_1$
  abgeschlossener Mengen $A \subseteq M_2$ in $M_1$ abgeschlossen ist.

  \textit{Lsg.} Es wird bewiesen, dass
  \textit{Das Urbild offener Mengen ist offen}
  $\iff$ \textit{Das Urbild abgeschlossener Mengen ist abgeschlossen}

  $"\Rightarrow"$ Sei nun das Urbild offener Mengen offen, $A \subseteq M_2$ eine
  abgeschlossene Teilmenge und $B = M_2 \setminus A$ eine offene Teilmenge
  von $M_2$.
  \begin{flalign*}
    M_1 \setminus f^{-1}(B) &= \qty{x \in M_1 \middle| f(x) \notin B } & \\
                            &= \qty{x \in M_1 \middle| f(x) \in A} = f^{-1}(A)
  \end{flalign*}
  \begin{flalign*}
    &\Rightarrow f^{-1}(A) \text{ ist das Komplement der in $M_1$ offenen Menge $f^{-1}(B)$} & \\
    &\Rightarrow \text{ Das Urbild $f^{-1}(A)$ der abgeschlossenen Menge $A$ ist in $M_1$ abgeschlossen}
  \end{flalign*}
  $"\Leftarrow"$ Sei nun das Urbild abgeschlossener Mengen abgeschlossen,
  $B \subseteq M_2$ eine offene Teilmenge und $A = M_2 \setminus B$ eine
  abgeschlossene Teilmenge von $M_2$.
  \begin{flalign*}
    &\Rightarrow f^{-1}(A) \text{ ist abgeschlossen} & \\
    &\Rightarrow M_1 \setminus f^{-1}(A) = f^{-1}(B) \text{ ist offen}
  \end{flalign*}

  Nach Proposition 6.3.3 (iii) der Vorlesung -
  ``Für jede offene Menge $U \subseteq M_2$ ist das Urbild $f^{-1}(U)$ offen in $M_1$''
  - folgt die Behauptung.

\newpage
\item Sei $d_1$ die diskrete Metrik auf $M_1$.
  Welche Stetigkeitseigenschaften besitzt eine beliebige Abbildung
  $f \colon M_1 \to M_2$?

  \textit{Lsg.}
  \[
    d_1 (x, y) = \begin{cases}
      0 & x = y \\
      1 & x \ne y \\
    \end{cases}
  \]
  Für alle $x \in M_1$ gilt $B\qty(x, \sfrac{1}{2}) \subseteq \qty{x}$.
  Sei nun $A \in M_1$ eine beliebige Teilmenge, dann ist $A$ offen, da
  für alle $x \in A \colon B\qty(x, \sfrac{1}{2}) \subseteq A$.

  Sei $C = M_1 \setminus A$ die Komplementärmenge von $A$, dann ist $B$
  ebenfalls offen, da für alle
  $x \in C \colon B\qty(x, \sfrac{1}{2}) \subseteq A$.

  $\Rightarrow$ alle Teilmengen von $M_1$ sind gleichzeitig offen und
  abgeschlossen.

  $\Rightarrow$ Für eine beliebige Abbildung $f \colon M_1 \to M_2$
  und eine beliebige Teilmenge $A \in M_1$ ist sowohl $A$ offen und
  abgeschlossen, als auch $f^{-1}(A)$ offen und abgeschlossen.

  $\Rightarrow f$ ist stetig.

\end{enumerate}

\section*{Kompakte Mengen}

Gegeben seien metrische Räume $\qty(M_1, d_1)$, $\qty(M_2, d_2)$ und eine
stetige Funktion $f \colon M_1 \to M_2$.
\begin{enumerate}[a)]
\item Beweisen Sie folgende Aussage:
  Ist $K$ eine nichtleere kompakte Teilmenge von $M_1$, so ist $f$ auf $K$
  gleichmäßig stetig.

  \textit{Lsg.} Sei $f$ nicht gleichmäßig stetig auf $K$
  \[
    \exists \epsilon > 0 \forall \delta > 0 \exists x, y \in K \colon
    d_1(x, y) < \delta \Rightarrow d_2(f(x), f(y)) \geq \epsilon
  \]

  Sei $\delta_n \coloneqq \frac{1}{n}, n \in \mathbb{N}$.
  Dann existieren Folgen $\qty(x_n)$ und $\qty(y_n)$ in $K$ mit
  $d_1(x_n, y_n) < \frac{1}{n}$ und
  $d_2 \qty(f\qty(x_n), f\qty(y_n)) \geq \epsilon$.

  Nach Proposition 6.3.2 der Vorlesung -
  ``Die Funktion $f \colon M_1 \to M_2$ ist stetig in $x \in M_1 \iff $
  Für jede Folge $\qty(x_n)$ in $M_1$ gilt
  $\lim_{n \to \infty} x_n = x \Rightarrow \lim_{n \to \infty} f\qty(x_n) = f(x)$''
  - und der Kompaktheit von $K$ existiert eine Teilfolge $\qty(x_{n_k})$
  und ein $x \in K$ mit $x_n \overset{k \to \infty}\longrightarrow x$.
  \[
    \Rightarrow d_1\qty(x, y_{n_k}) \geq d_1\qty(x, x_{n_k}) + d_1\qty(x_{n_k}, y_{n_k})
    \overset{k \to \infty}\longrightarrow x
  \]
  $\Rightarrow \qty(y_{n_k})$ konvergiert gegen $x$.

  Aus der Stetigkeit von $f$ folgt
  $f\qty(x_{n_k}) \overset{k \to \infty}\longrightarrow f(x)$ und
  $f\qty(y_{n_k}) \overset{k \to \infty}\longrightarrow f(x)$.
  \[
    \Rightarrow d_2\qty(f\qty(x_{n_k}), f\qty(y_{n_k}))
    \geq d_2\qty(f\qty(x_{n_k}), f(x)) + d_2\qty(f(x), f\qty(y_{n_k}))
    \overset{k \to \infty}\longrightarrow 0
  \]
  Das ist ein Widerspruch zu $d_2\qty(f\qty(x_{n_k}), f\qty(y_{n_k})) \geq \epsilon$.

\item Folgt aus der Kompaktheit von $K \subseteq M_2$ auch die Kompaktheit
  des Urbildes $f^{-1}(K) \subseteq M_1$ von $K$?

  \textit{Lsg.} Sei $f(x) \coloneqq 0, x \in R$.
  Dann ist das Urbild $f^{-1}(\qty{0}) = \mathbb{R}$ und $\mathbb{R}$ ist nicht kompakt.

\end{enumerate}


\end{document}