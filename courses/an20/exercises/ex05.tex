\documentclass{scrreprt}

\usepackage{aligned-overset}
\usepackage{amsmath}
\usepackage{amssymb}
\usepackage{bm}
\usepackage[shortlabels]{enumitem}
\usepackage{hyperref}
\usepackage[utf8]{inputenc}
\usepackage{mathtools}
\usepackage{physics}
\usepackage{tabularx}
\usepackage{titling}
\usepackage{fancyhdr}
\usepackage{xfrac}
\usepackage{pgfplots}

\definecolor{light-gray}{gray}{.9}

\pgfplotsset{compat = newest}
\usetikzlibrary{intersections}
\usetikzlibrary{patterns}
\usepgfplotslibrary{fillbetween}

\author{Karsten Lehmann}
\date{SoSe 2021}
\title{Übung 05 Analysis - Weiterführende Konzepte}

\pagestyle{fancy}
\fancyhf{}
\lhead{\thetitle}
\rhead{\theauthor}
\lfoot{\thedate}
\rfoot{Seite \thepage}

\begin{document}

\section*{Stetigkeit in metrischen Räumen}

Sei $d$ die Betragsmetrik auf $\mathbb{R}$.
Die Menge $M \coloneqq [0, 1] \cup [2, 4)$ sei mit der induzierten Metrik
$d|_M$ versehen.
Untersuchen Sie mit Hilfe der \colorbox{green}{$\epsilon-\delta$-Definition}, in welchen Punkten
von $M$ die Abbildung $f \colon \qty(M, d|_M) \to \qty(\mathbb{R}, d)$
definiert durch
\[
  f(x) \coloneqq \begin{cases}
    \colorbox{blue!20}{1} & \text{für } x \in [0, 1] \\
    \colorbox{yellow!50}{0} & \text{für } x \in [2, 4] \\
  \end{cases}
\]
stetig ist. \\

\textit{Lsg.}

\colorbox{green}{$\epsilon-\delta$-Definition}: Seien $\qty(M_1, d_1)$
und $\qty(M_2, d_2)$ metrische Räume.
Eine Funktion $f \colon M_1 \to M_2$ heißt stetig in $x \in M_1$, wenn
\[
  \forall \epsilon > 0 \exists \delta > 0 \forall y \in M_1 \colon d_1(x, y) < \delta
  \iff d_2(f(x), f(y)) < \epsilon
\]

Es gilt für $x \in M$ und $\delta > 0$ per Definition der induzierten Metrik:
\colorbox{orange}{$B_{d|_M} (x, \delta) = B_d(x, \delta) \cap M$}

Teilen wir die Menge $M$ nun in die beiden Teilmengen
$M_1 = [0, 1]$ und $M_2 = [2, 4)$ auf.

Sei die Menge $A = (0, 1) \cup (2, 4) \subseteq M$ mit der induzierten Metrik
$d|_A$ und $A, d|_A)$ ein metrischer Raum.
\begin{itemize}
\item Die Menge $A$ ist offen $\Rightarrow$ für jedes $x \in A$ existiert ein
  $\delta > 0$ mit $B_{d|_A}(x, \delta \subseteq A)$.
  \[
    \colorbox{orange}{$B_{d|_M} (x, \delta) = B_d(x, \delta) \cap M$}
    = B_d(x, \delta) \cap A \subseteq M_1 \cup M_2
  \]

\item $x = 0 \colon$ Für $\delta \in (0, 1]$ ist
  $B_{d|_M}(x, \delta) = B_{\underset{d}{\underbrace{\abs{\cdot}}}}(x, \delta) \cap M
  = [0, \delta) \subseteq M_1$

\item $x = 2 \colon$ Für $\delta \in (0, 1]$ ist
  $B_{d|_M}(x, \delta) = B_{\underset{d}{\underbrace{\abs{\cdot}}}}(x, \delta) \cap M
  = [2, 2 + \delta) \subseteq M_2$

\item $x = 1 \colon$ Für $\delta \in (0, 1]$ ist
  $B_{d|_M}(x, \delta) = B_{\underset{d}{\underbrace{\abs{\cdot}}}}(x, \delta) \cap M
  = (1 - \delta, 1] \subseteq M_1$

\end{itemize}
\begin{flalign*}
  &\Rightarrow \text{ für jedes } x \in M_1 \text{ existiert ein } \delta > 0
  \text{ mit } B_{d_M}(x, \delta) \subseteq M_1 & \\
  &\Rightarrow y \in B_{d|_M}(x, \delta) \colon f(x) = f(y) = \colorbox{blue!20}{$1$} \\
  &\Rightarrow \colon \forall \epsilon > 0 \colon d(f(x), f(y)) = \abs{f(x) - f(y)} = 0 < \epsilon \\
  &\Rightarrow f \text{ ist auf $M_1$ stetig}
\end{flalign*}
Analog für $M_2$:
\begin{flalign*}
  &\Rightarrow \text{ für jedes } x \in M_2 \text{ existiert ein } \delta > 0
  \text{ mit } B_{d_M}(x, \delta) \subseteq M_2 & \\
  &\Rightarrow y \in B_{d|_M}(x, \delta) \colon f(x) = f(y) = \colorbox{yellow!50}{$0$} \\
  &\Rightarrow \colon \forall \epsilon > 0 \colon d(f(x), f(y)) = \abs{f(x) - f(y)} = 0 < \epsilon \\
  &\Rightarrow f \text{ ist auf $M_2$ stetig}
\end{flalign*}

\newpage
Gegeben sei die Funktion $f \colon \mathbb{R}^2 \to \mathbb{R}$ mit
\[
  f(x) \coloneqq \begin{cases}
    \frac{x_1 \cdot x_2}{x_1^2 + x_2^2} & \text{für } x = \qty(x_1, x_2) \ne (0, 0) \\
    0 & \text{für } x = (0, 0)
  \end{cases}
\]
\begin{enumerate}[a)]
\item Zeigen Sie, dass die Funktion $f$ partiell stetig auf $\mathbb{R}^2$ ist,
  dass heißt für jeden Punkt $x = (x_1, x_2) \in \mathbb{R}^2$ sind die
  Abbildungen $g_1(t) \coloneqq f(x_1 + t, x_2)$ und
  $g_2(t) \coloneqq f(x_1, x_2 + t)$ stetig in $t = 0$.
\item Zeigen Sie, dass die Funktion $f$ nicht stetig auf $\mathbb{R}^2$ ist.

  \textit{Lsg.} Die Funktion ist nicht stetig im Punkt $x = (0, 0)$.

  Falls die Funktion $f$ in $x$ stetig wäre, müsste gelten:
  \[
    \lim_{t \to 0} f(t, t) = f(0, 0)
  \]
  Nun ist jedoch $f(0, 0) = 0$ und
  $\lim_{t \to 0} f(t, t) = \lim_{t \to 0} \frac{t^2}{2t^2} = \frac{1}{2}$.
  Ein Widerspruch, damit ist $f$ auf $\mathbb{R}^2$ nicht stetig.

\end{enumerate}

\end{document}