\documentclass{scrreprt}

\usepackage{aligned-overset}
\usepackage{amsmath}
\usepackage{amssymb}
\usepackage[german]{babel}
\usepackage{bm}
\usepackage{chngcntr}
\usepackage[shortlabels]{enumitem}
\usepackage{hyperref}
\usepackage[utf8]{inputenc}
\usepackage{mathtools}
\usepackage{physics}
\usepackage{tabularx}
\usepackage{titling}
\usepackage{fancyhdr}
\usepackage{xfrac}
\usepackage[table]{xcolor}
\usepackage{pgfplots}

\definecolor{light-gray}{gray}{.9}

%% See https://tex.stackexchange.com/a/44954
\newcounter{myequation}
\makeatletter
\@addtoreset{equation}{myequation}
\makeatother

%% Fix equation numbering for scrreprt class.
\counterwithout{equation}{chapter}

\pgfplotsset{compat = newest}
\usepgfplotslibrary{patchplots}
\usetikzlibrary{intersections}
\usetikzlibrary{shapes}
\usetikzlibrary{shapes.geometric}
\usetikzlibrary{patterns}
\usepgfplotslibrary{fillbetween}

\author{Karsten Lehmann}
\date{SoSe 2021}
\title{Analysis - Lagrange Multiplikation}

\fancypagestyle{main}{
  \fancyhf{}
  \lhead{\thetitle}
  \rhead{\theauthor}
  \lfoot{\thedate}
  \rfoot{Seite \thepage}
}
\fancypagestyle{plain}{
  \fancyhf{}
  \lhead{\thetitle}
  \rhead{\theauthor}
  \lfoot{\thedate}
  \rfoot{Seite \thepage}
}
\pagestyle{main}

\newcommand\skalprod[1]{\left\langle #1 \right\rangle}
\newcommand\nnorm[1]{\left\lvert\left\lvert\left\lvert #1 \right\rvert\right\rvert\right\rvert}
\newcommand\rang{\text{rang}\,}

% Hide the chapter and section numbers
\renewcommand\thechapter{}
\renewcommand\thesection{}
% Custom title for table of contents
\renewcaptionname{german}{\contentsname}{Prüfungsrelevante Aufgaben}

\begin{document}

\chapter{Lagrange Multiplikation}

Der Formalismus der Lagrange-Multiplikation wird verwendet,
um bei einer Gleichung mit mehreren Veränderlichen unter der
Gültigkeit einer Nebenbedingung kritische Punkte
(eventuelle Minimas und Maximas) zu finden.

\section{Voraussetzungen}

\begin{enumerate}[1)]
\item \label{vor:1}
  Sei $U \subseteq \mathbb{R}^N$ eine
  \hyperref[sec:offene_menge]{offene Menge} und eine
  \hyperref[sec:c-k-funktion]{$C^1$-Funktion}
  $f \colon U \to \mathbb{R}$ gegeben.

\item \label{vor:2}
  Existenz einer Nebenbedingung, zum Beispiel $x_1 + x_2 = c$.
  Aus Umstellen der Nebenbedingung nach $0$ folgt eine
  \hyperref[sec:c-k-funktion]{$C^1$-Funktion}
  $h \colon U \to \mathbb{R}$.

  $\Rightarrow h(x_1, x_2) = x_1 + x_2 - c$

\item \label{vor:3}
  Für jedes Element $x \in \qty\big{ y \in U \, {\big |} \, h(y) = 0}$
  hat die Matrix $\hyperref[sec:jacobi]{J_h(x)}$
  \hyperref[sec:rang]{vollen Rang}.
\end{enumerate}

\section{Vorgehensweise}

\begin{enumerate}[1)]
\item \label{step:1}
  Bildung der Lagrange-Funktion $F \colon U \times \mathbb{R} \to \mathbb{R}$ mit
  \[
    F(x_1, \ldots, x_n, \lambda) = f\qty\big(x_1, \ldots, x_n) + \lambda \cdot h\qty\big(x_1, \ldots, x_n)
  \]

\item \label{step:2}
  Ermitteln der Richtungsableitungen
  $\frac{\partial F}{\partial x_1}, \ldots, \frac{\partial F}{\partial x_n},
  \frac{\partial F}{\partial \lambda}$

\item \label{step:3}
  Gleichsetzen der Richtungsableitungen mit $0$ und
  Lösen des Gleichungssystems.

  $\Rightarrow$ die Lösungen sind die kritischen Punkte.
\end{enumerate}

\chapter{Definitionen}

\section{Determinante}
\label{sec:determinante}

Die \textbf{Determinante} ist eine Zahl die einer quadratischen
Matrix zugeordnet ist.
\begin{flalign*}
  \det \begin{pmatrix}
    a & b \\
    c & d
  \end{pmatrix} &=
  a \cdot d - b \cdot c & \\
  \det \begin{pmatrix}
    a & b & c \\
    d & e & f \\
    g & h & i
  \end{pmatrix} &=
  a \cdot \det \begin{pmatrix}
    e & f \\
    h & i
  \end{pmatrix} - d \cdot \det \begin{pmatrix}
    b & c \\
    h & i
  \end{pmatrix} + g \cdot \det \begin{pmatrix}
    b & c \\
    e & f
  \end{pmatrix}
\end{flalign*}

\section{Jacobimatrix}
\label{sec:jacobi}

Sei $U \subseteq \mathbb{R}^N$ eine Teilmenge und
$f \colon U \to \mathbb{R}^M$
eine im Punkt \( x \in U \)
\hyperref[sec:partielle_ableitung]{partiell differenzierbare Funktion}.
Dann heißt
\[
  J_f(x) = \begin{pmatrix}
    \frac{\partial f_1}{\partial x_1} (x) & \ldots & \frac{\partial f_1}{\partial x_M} \\
    \vdots & \ddots & \vdots \\
    \frac{\partial f_N}{\partial x_1} (x) & \ldots & \frac{\partial f_N}{\partial x_M} \\
  \end{pmatrix}
\]
die \textbf{Jacobimatrix} von $f$ in Punkt $x$.
Falls $f$ im Punkt $x$ differenzierbar ist, gilt $f'(x) = J_f(x)$.

Um einen Teil der Jacobimatrix darzustellen werden die entsprechenden Variablen
über dem $J$ in Klammern geschrieben, zum Beispiel
\[
  J_f^{(x_m, x_n)}(x) = \begin{pmatrix}
    \frac{\partial f_1}{\partial x_m} (x) & \frac{\partial f_1}{\partial x_n} \\
    \vdots & \vdots \\
    \frac{\partial f_N}{\partial x_m} (x) & \frac{\partial f_N}{\partial x_n} \\
  \end{pmatrix}
\]

\section{$k$-mal stetig differenzierbare Funktion}
\label{sec:c-k-funktion}

Eine \textbf{$k$-mal stetig differenzierbare Funktion}, oder auch
\textbf{$C^k$-Funktion}, ist eine Funktion die $k$-mal differenzierbar ist
und deren $k$-te Ableitung \hyperref[sec:stetige_funktion]{stetig} ist.

\newpage
\section{Offene Kugel}
\label{sec:offene_kugel}

Der Ausdruck $B_M(x, r)$ bezeichnet die \textbf{offene Kugel} um den
Punkt $x$ mit dem Radius $r$ auf der Menge $M$.

\[
  B(x, r) = \qty\Big{y \in M {\Big |} \norm{x - y} < r}
\]

\section{Offenen Menge}
\label{sec:offene_menge}

Eine Teilmenge $U \subseteq M$ heißt \textbf{offen} (in $M$), falls es für
jedes Element $x \in U$ ein $r > 0$ gibt, so dass
$\hyperref[sec:offene_kugel]{B(x, r)} \subseteq U$.
(dass heißt, falls sie \hyperref[sec:umgebung]{Umgebung} jedes ihrer Elemente
ist).

\section{Partielle Ableitung}
\label{sec:partielle_ableitung}

Sei $U \subseteq \mathbb{R}^M$ und $f \colon U \to \mathbb{R}^N$.
Die Funktion $f$ heißt in $x \in U$ nach $x_k$ ($1 \leq k \leq M$)
partiell differenzierbar, falls sie in
\hyperref[sec:richtungsableitung]{Richtung} des $k$-ten Einheisvektors
($k$-te Koordinate gleich 1, der Rest 0) differenzierbar ist.
Dann heißt $\frac{\partial f}{\partial x_k} (x)$ die
\textbf{partielle Ableitung} von $f$ nach $x_k$ im Punkt $x$.

\section{Rang (einer Matrix)}
\label{sec:rang}

Der \textbf{Rang} einer Matrix ist die Zahl der linear
unabhängigen Zeilen- / Spalten-vektoren einer Matrix.
Sei $A$ eine $m \times n$ Matrix.
Man spricht vom \textbf{vollem Rang} der Matrix $A$, falls
$\rang(A) = \min \qty{m, n}$.

Eine quadratische Matrix hat vollen Rang, falls ihre
\hyperref[sec:determinante]{Determinante} von $0$
verschieden ist.

\section{Richtungsableitung}
\label{sec:richtungsableitung}

Die Funktion $f \colon U \subseteq \mathbb{R}^M \to \mathbb{R}^N$
heißt im Punkt $x \in U$ in Richtung $e \in \mathbb{R}^M$ differenzierbar,
falls die Abbildung

\begin{flalign*}
  ]-r, r [ &\to \mathbb{R}^N, & \\
  t &\mapsto f(x + t \cdot e )
\end{flalign*}
für $r > 0$ klein genug in $0$ differenzierbar ist.

Dann heißt
\[
  \frac{\partial f}{\partial e}(x) = \lim_{t \to 0} \frac{f(x + t \cdot e) - f(x)}{t}
\]
die \textbf{Richtungsableitung} von $f$ im Punkt $x$ in Richtung $e$.


\section{Stetige Funktion}
\label{sec:stetige_funktion}

Seien $\qty\big(M_1, d_1)$, $\qty\big(M_2, d_2)$ zwei metrische Räume und $f \colon M_1 \to M_2$ eine Abbildung.

Die Funktion $f$ ist \textbf{stetig}, falls
\[
  \forall \epsilon > 0 \exists \delta > 0
  \forall x, y \in M_1 \colon
  d_1(x, y) < \delta \Rightarrow d_2(f(x), f(y)) < \epsilon
\]

\section{Umgebung}
\label{sec:umgebung}

Eine Teilmenge $U \subseteq M$ heißt Umgebung eines Punktes $x \in M$,
falls es ein $r > 0$ gibt, so dass
$\hyperref[sec:offene_kugel]{B(x, r)} \subseteq U$.

\chapter{Beispiele}

\section{Aufgabe 1}

Sei $f \colon \mathbb{R}^2 \to \mathbb{R}$ mit $f(x, y) = 4xy - y^2$.
Ermitteln sie das Maximum von $f$ unter der Bedingung $x + y = 5$.

\paragraph{Lösung:} $\mathbb{R}^2$ ist offen $f$ ist stetig differenzierbar mit
$\nabla f(x, y) = \begin{pmatrix}
  4y \\
  4x - 2y
\end{pmatrix}$ (\hyperref[vor:1]{Voraussetzung 1}).
Nach dem Umstellen der Nebenbedingung nach $0$ erhält man
$h(x, y) = x + y - 5$.
Die Funktion $h$ ist mit $\nabla f(x, y) = \begin{pmatrix}
  1 \\
  1
\end{pmatrix}$ ebenfalls stetig differenzierbar
(\hyperref[vor:2]{Voraussetzung 2}).
Da $\rank \nabla h(x, y)^T \ne \begin{pmatrix}0 & 0\end{pmatrix}$
besitzt $J_h(x, y)$ vollen Rang (\hyperref[vor:3]{Voraussetzung 3}).

Die Lagrange-Funktion ist nun
$F(x, y, \lambda) = 4xy - y^2 + \lambda \cdot \qty\big(x + y - 5)$
(\hyperref[step:1]{Schritt 1}).
Weiterhin lauten die Richtungsableitungen der Funktion $F$ (\hyperref[step:2]{Schritt 2})
\begin{flalign*}
  \frac{\partial F}{\partial x}(x, y, z) &= 4y + \lambda = 0 & \\
  \frac{\partial F}{\partial y}(x, y, z) &= 4x - 2y + \lambda = 0 \\
  \frac{\partial F}{\partial \lambda}(x, y, z) &= x + y - 5 = 0
\end{flalign*}

Die Lösung des Gleichungssystems ist
$(x, y, \lambda) = (3, 2, -8)$ (\hyperref[step:3]{Schritt 3}).

$\Rightarrow$ Der kritische Punkt liegt bei $(x, y) = (3, 2)$.

\end{document}