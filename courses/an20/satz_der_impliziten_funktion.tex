\documentclass{scrreprt}

\usepackage{aligned-overset}
\usepackage{amsmath}
\usepackage{amssymb}
\usepackage[german]{babel}
\usepackage{bm}
\usepackage{chngcntr}
\usepackage[shortlabels]{enumitem}
\usepackage{hyperref}
\usepackage[utf8]{inputenc}
\usepackage{mathtools}
\usepackage{physics}
\usepackage{tabularx}
\usepackage{titling}
\usepackage{fancyhdr}
\usepackage{xfrac}
\usepackage[table]{xcolor}
\usepackage{pgfplots}

\definecolor{light-gray}{gray}{.9}

%% See https://tex.stackexchange.com/a/44954
\newcounter{myequation}
\makeatletter
\@addtoreset{equation}{myequation}
\makeatother

%% Fix equation numbering for scrreprt class.
\counterwithout{equation}{chapter}

\pgfplotsset{compat = newest}
\usepgfplotslibrary{patchplots}
\usetikzlibrary{intersections}
\usetikzlibrary{shapes}
\usetikzlibrary{shapes.geometric}
\usetikzlibrary{patterns}
\usepgfplotslibrary{fillbetween}

\author{Karsten Lehmann}
\date{SoSe 2021}
\title{Analysis - Satz über die implizite Funktion}

\fancypagestyle{main}{
  \fancyhf{}
  \lhead{\thetitle}
  \rhead{\theauthor}
  \lfoot{\thedate}
  \rfoot{Seite \thepage}
}
\fancypagestyle{plain}{
  \fancyhf{}
  \lhead{\thetitle}
  \rhead{\theauthor}
  \lfoot{\thedate}
  \rfoot{Seite \thepage}
}
\pagestyle{main}

\newcommand\skalprod[1]{\left\langle #1 \right\rangle}
\newcommand\nnorm[1]{\left\lvert\left\lvert\left\lvert #1 \right\rvert\right\rvert\right\rvert}

% Hide the chapter and section numbers
\renewcommand\thechapter{}
\renewcommand\thesection{}
% Custom title for table of contents
\renewcaptionname{german}{\contentsname}{Prüfungsrelevante Aufgaben}

\begin{document}

\chapter{Satz über die implizite Funktion}

\section{Voraussetzungen}

\begin{enumerate}[1)]
\item \label{vor:1}
  Seien $U \subseteq \mathbb{R}^N$ und $V \subseteq \mathbb{R}^M$ zwei
  \hyperref[sec:offene_menge]{offene} Mengen.

\item \label{vor:2}
  Weiterhin sei $F \colon U \times V \to \mathbb{R}^M$ eine
  \hyperref[sec:c-k-funktion]{$C^k$-Funktion}.
  ($k$ kann auch $1$ sein)

\item \label{vor:3}
  Die Abbildung
  $\qty\big(x, y) \mapsto \hyperref[sec:jacobi]{J_F^{(y)}\qty\big(x, y)}$
  ist auf $U \times V$ \hyperref[sec:inv_matrix]{invertierbar}.
\end{enumerate}

\section{Folgerungen}

Für alle $\qty\big(\overline{x}, \overline{y}) \in U \times V$
existieren \hyperref[sec:offene_menge]{offene}
\hyperref[sec:umgebung]{Umgebungen}
$\tilde{U} \subseteq U$ von $\overline{x}$ und
$\tilde{V} \subseteq V$ von $\overline{y}$, so wie eine
\hyperref[sec:c-k-funktion]{$C^k$-Funktion}
\[
  f \colon \tilde{U} \times \tilde{V} \subseteq U \times V \to \tilde{V}
\]
mit folgenden Eigenschaften
\begin{itemize}
\item \label{folg:1}
  $\forall \: \qty\big(x, y) \in \tilde{U} \times \tilde{V} \colon F(x, y)
  = F(\overline{x}, \overline{y}) \iff y = f(x)$

\item \label{folg:2}
  $J_f(x) = - \qty(\hyperref[sec:jacobi]{J_F^{(y)}} \qty\big(x, f(x)))^{\hyperref[sec:inv_matrix]{-1}}
  \cdot \hyperref[sec:jacobi]{J_F^{(x)}} \qty\big(x, f(x))$
\end{itemize}

\label{folg:3}
Es gilt $\overline{y} = f(\overline{x})$ und
$\hyperref[sec:jacobi]{J_f\qty\big(\overline{x})}
= - \qty(\hyperref[sec:jacobi]{J_F^{(y)}} \qty\big(\overline{x}, \overline{y}))^{\hyperref[sec:inv_matrix]{-1}}
\cdot \hyperref[sec:jacobi]{J_F^{(x)}} \qty\big(\overline{x}, \overline{y})$

\chapter{Definitionen}

\section{Determinante}
\label{sec:determinante}

Die \textbf{Determinante} ist eine Zahl die einer quadratischen
Matrix zugeordnet ist.
\begin{flalign*}
  \det \begin{pmatrix}
    a & b \\
    c & d
  \end{pmatrix} &=
  a \cdot d - b \cdot c & \\
  \det \begin{pmatrix}
    a & b & c \\
    d & e & f \\
    g & h & i
  \end{pmatrix} &=
  a \cdot \det \begin{pmatrix}
    e & f \\
    h & i
  \end{pmatrix} - d \cdot \det \begin{pmatrix}
    b & c \\
    h & i
  \end{pmatrix} + g \cdot \det \begin{pmatrix}
    b & c \\
    e & f
  \end{pmatrix}
\end{flalign*}

\section{Einheitsmatrix}
\label{sec:einheitsmatrix}

Die Einheitsmatrix ist die Matrix, deren Hauptdiagonale aus $1$-en
besteht (die restlichen Elemente entsprechen $0$).
\[
  I_n = \begin{pmatrix}
    1 & 0 & \cdots & 0 \\
    0 & 1 & \ddots & \vdots \\
    \vdots & \ddots & \ddots & 0 \\
    0 & \cdots  & 0 & 1
  \end{pmatrix}
\]
Dabei gilt $\hyperref[sec:determinante]{\det I_n} = 1$

\section{Invertierbare Matrix}
\label{sec:inv_matrix}

Eine quadratische Matrix $A$ heißt \textbf{invertierbar}, wenn eine weitere Matrix $B$
existiert, so dass $A \cdot B = \hyperref[sec:einheitsmatrix]{I}$.
Voraussetzung für die \textbf{Invertierbarkeit}, ist eine
\hyperref[sec:determinante]{Determinante} ungleich $0$.

\newpage
\section{Jacobimatrix}
\label{sec:jacobi}

Sei $U \subseteq \mathbb{R}^N$ eine Teilmenge und
$f \colon U \to \mathbb{R}^M$
eine im Punkt \( x \in U \)
\hyperref[sec:partielle_ableitung]{partiell differenzierbare Funktion}.
Dann heißt
\[
  J_f(x) = \begin{pmatrix}
    \frac{\partial f_1}{\partial x_1} (x) & \ldots & \frac{\partial f_1}{\partial x_M} \\
    \vdots & \ddots & \vdots \\
    \frac{\partial f_N}{\partial x_1} (x) & \ldots & \frac{\partial f_N}{\partial x_M} \\
  \end{pmatrix}
\]
die \textbf{Jacobimatrix} von $f$ in Punkt $x$.
Falls $f$ im Punkt $x$ differenzierbar ist, gilt $f'(x) = J_f(x)$.

Um einen Teil der Jacobimatrix darzustellen werden die entsprechenden Variablen
über dem $J$ in Klammern geschrieben, zum Beispiel
\[
  J_f^{(x_m, x_n)}(x) = \begin{pmatrix}
    \frac{\partial f_1}{\partial x_m} (x) & \frac{\partial f_1}{\partial x_n} \\
    \vdots & \vdots \\
    \frac{\partial f_N}{\partial x_m} (x) & \frac{\partial f_N}{\partial x_n} \\
  \end{pmatrix}
\]

\section{$k$-mal stetig differenzierbare Funktion}
\label{sec:c-k-funktion}

Eine \textbf{$k$-mal stetig differenzierbare Funktion}, oder auch
\textbf{$C^k$-Funktion}, ist eine Funktion die $k$-mal differenzierbar ist
und deren $k$-te Ableitung \hyperref[sec:stetige_funktion]{stetig} ist.

\section{Offene Kugel}
\label{sec:offene_kugel}

Der Ausdruck $B_M(x, r)$ bezeichnet die \textbf{offene Kugel} um den
Punkt $x$ mit dem Radius $r$ auf der Menge $M$.

\[
  B(x, r) = \qty\Big{y \in M {\Big |} \norm{x - y} < r}
\]

\section{Offenen Menge}
\label{sec:offene_menge}

Eine Teilmenge $U \subseteq M$ heißt \textbf{offen} (in $M$), falls es für
jedes Element $x \in U$ ein $r > 0$ gibt, so dass
$\hyperref[sec:offene_kugel]{B(x, r)} \subseteq U$.
(dass heißt, falls sie \hyperref[sec:umgebung]{Umgebung} jedes ihrer Elemente
ist).

\section{Partielle Ableitung}
\label{sec:partielle_ableitung}

Sei $U \subseteq \mathbb{R}^M$ und $f \colon U \to \mathbb{R}^N$.
Die Funktion $f$ heißt in $x \in U$ nach $x_k$ ($1 \leq k \leq M$)
partiell differenzierbar, falls sie in
\hyperref[sec:richtungsableitung]{Richtung} des $k$-ten Einheisvektors
($k$-te Koordinate gleich 1, der Rest 0) differenzierbar ist.
Dann heißt $\frac{\partial f}{\partial x_k} (x)$ die
\textbf{partielle Ableitung} von $f$ nach $x_k$ im Punkt $x$.

\section{Richtungsableitung}
\label{sec:richtungsableitung}

Die Funktion $f \colon U \subseteq \mathbb{R}^M \to \mathbb{R}^N$
heißt im Punkt $x \in U$ in Richtung $e \in \mathbb{R}^M$ differenzierbar,
falls die Abbildung

\begin{flalign*}
  ]-r, r [ &\to \mathbb{R}^N, & \\
  t &\mapsto f(x + t \cdot e )
\end{flalign*}
für $r > 0$ klein genug in $0$ differenzierbar ist.

Dann heißt
\[
  \frac{\partial f}{\partial e}(x) = \lim_{t \to 0} \frac{f(x + t \cdot e) - f(x)}{t}
\]
die \textbf{Richtungsableitung} von $f$ im Punkt $x$ in Richtung $e$.

\section{Stetige Funktion}
\label{sec:stetige_funktion}

Seien $\qty\big(M_1, d_1)$, $\qty\big(M_2, d_2)$ zwei metrische Räume und $f \colon M_1 \to M_2$ eine Abbildung.

Die Funktion $f$ ist \textbf{stetig}, falls
\[
  \forall \epsilon > 0 \exists \delta > 0
  \forall x, y \in M_1 \colon
  d_1(x, y) < \delta \Rightarrow d_2(f(x), f(y)) < \epsilon
\]

\section{Umgebung}
\label{sec:umgebung}

Eine Teilmenge $U \subseteq M$ heißt Umgebung eines Punktes $x \in M$,
falls es ein $r > 0$ gibt, so dass
$\hyperref[sec:offene_kugel]{B(x, r)} \subseteq U$.

\chapter{Beispiele}

\section{Aufgabe 1}

Sei $F \colon \mathbb{R}^3 \to \mathbb{R}$ definiert durch
$F(x, y, z) = x^4z - 2xy^3 + yz^3 - 8$.
Für den Punkt $\qty\big(\overline{x}, \overline{y}, \overline{z})
= (1, 1, 2)$ ist $F\qty\big(\overline{x}, \overline{y}, \overline{z}) = 0$.

\begin{enumerate}[1)]
\item Zeigen Sie, dass es eine Umgebung
  $\tilde{U} \times \tilde{V} \subseteq \mathbb{R}^2 \times \mathbb{R}$
  des Punktes $\qty\big(\overline{x}, \overline{y}, \overline{z})$
  und eine $C^1$-Funktion
  $f \colon \tilde{U} \subseteq \mathbb{R}^2 \to \tilde{V} \subseteq \mathbb{R}$
  gibt mit $F(x, y, f(x, y)) = 0$ für alle $(x, y) \in \tilde{U}$.

  \paragraph{Lösung:} $U = \mathbb{R}^2$ und $V = \mathbb{R}$ sind offenen
  Mengen (\hyperref[vor:1]{Voraussetzung 1}).
  \begin{flalign*}
    J_F &= \qty\Big(\begin{array}{>{\color{red}} c >{\color{red}} c >{\color{blue}} c}
      \frac{\partial f}{\partial x} & \frac{\partial f}{\partial y} & \frac{\partial f}{\partial z}
    \end{array}) & \\
    &= \qty\Big(\begin{array}{>{\color{red}} c >{\color{red}} c >{\color{blue}} c}
      4x^3z - 2y^3 & z^3 - 6xy^2 & x^4 + 3yz^2
    \end{array}) &
  \end{flalign*}

  Die Abbildung $(x, y, z) \mapsto J_F(x, y, z)$ ist als Komposition stetiger
  Funktionen ebenfalls stetig.

  $\Rightarrow F$ ist eine $C^1$-Funktion (\hyperref[vor:2]{Voraussetzung 2}).

  Wegen
  $\det \colorbox{blue!40}{$J_F^{(z)} (\ldots)$}(\overline{x}, \overline{y}, \overline{z}) =
  \overline{x}^4 + 3\overline{y} \cdot \overline{z}^2 = 13$ und
  $(x, y, z) \mapsto J_F^{(z)}(x, y, z)$ ist stetig, folgt die
  Existenz einer offenen Umgebung
  $\overline{W} = \tilde{U} \times \tilde{V}$ von
  $(\overline{x}, \overline{y}, \overline{z})$ mit
  $\det J_F^{(z)} > 0$ für alle $x, y, z \in \overline{W}$.

  $\Rightarrow J_F^{(z)}$ ist auf $\overline{W}$ invertierbar
  (\hyperref[vor:3]{Voraussetzung 3}).

  $\Rightarrow$ aus dem Satz über die implizite Funktion folgt
  die Existenz einer $C^1$-Funktion $f \colon \tilde{U} \to \tilde{V}$
  mit der \hyperref[folg:1]{Eigenschaft}
  \[
    \forall (x, y) \in \tilde{U} \colon F(x, y, f(x, y)) =
    F\qty\big(\overline{x}, \overline{y}, \overline{z})
    = 0
  \]
\newpage
\item Geben Sie eine Formel zur Berechnung von $J_f$ an.
  \paragraph{Lösung:} Aus der \hyperref[folg:2]{zweiten Eigenschaft von $f$}
  folgt:
  \begin{flalign*}
    J_f(x, y) &= - \qty\Big(J_F^{(z)}(x, y, f(x, y)))^{-1} \cdot J_F^{(x, y)}(x, y, f(x, y)) & \\
    &= - \qty\Big(x^4 + 3y\qty\big(f(x))^2)^{-1} \cdot \begin{pmatrix}
      4x^3f(x) - 2 y^3 & \qty\big(f(x))^3 - 6xy^2
    \end{pmatrix} \\
    &= - \frac{1}{x^4 + 3y\qty\big(f(x))^2} \cdot \begin{pmatrix}
      4x^3f(x) - 2 y^3 & \qty\big(f(x))^3 - 6xy^2
    \end{pmatrix} \\
    &= \begin{pmatrix}
      \frac{4x^3f(x) - 2 y^3}{x^4 + 3y\qty\big(f(x))^2} &
      \frac{\qty\big(f(x))^3 - 6xy^2}{x^4 + 3y\qty\big(f(x))^2}
    \end{pmatrix} \\
  \end{flalign*}

\item Berechnen Sie $\nabla f(1, 1)$.

  \paragraph{Lösung:} $\nabla f(1, 1) = \nabla f(\overline{x}, \overline{y})$.
  Aus der \hyperref[folg:3]{dritten Eigenschaft von $f$} folgt
  \[
    J_f\qty\big(\overline{x}, \overline{y} )
    = - \qty( \colorbox{blue!40}{$J_F^{(z)} \qty\big(\overline{x}, \overline{y}, \overline{z})$})^{-1}
    \cdot \colorbox{red!40}{$J_F^{(x,y)} \qty\big(\overline{x}, \overline{y}, \overline{z})$}
  \]
  \begin{flalign*}
    \nabla f(1, 1) &=  - \qty\big(13)^{-1} \cdot \begin{pmatrix}
      6 & 2
    \end{pmatrix} & \\
    &= \begin{pmatrix}
      -\frac{6}{13} & -\frac{2}{13}
    \end{pmatrix}
  \end{flalign*}
\end{enumerate}
\end{document}