\documentclass{article}
\usepackage{aligned-overset}
\usepackage{amsmath}
\usepackage{amssymb}
\usepackage{bm}
\usepackage[shortlabels]{enumitem}
\usepackage{hyperref}
\usepackage[utf8]{inputenc}
\usepackage{mathtools}
\usepackage{physics}
\usepackage{tabularx}
\usepackage{titling}
\usepackage{fancyhdr}
\usepackage{xfrac}
\usepackage{pgfplots}

\pgfplotsset{compat = newest}

\author{Albina Oscherowa \\ Lukas Kamratzki \\ Karsten Lehmann}
\date{SoSe 2021}
\title{Hausaufgabe 05 Analysis - Weiterführende Konzepte}

\pagestyle{fancy}
\fancyhf{}
\lhead{\thetitle}
\rhead{\theauthor}
\lfoot{\thedate}
\rfoot{Seite \thepage}

\begin{document}

\section*{Hausaufgabe 1}

Wir betrachten die metrischen Räume $\qty(\mathbb{R}^2, \norm{\cdot}_2)$ und
$\qty(\mathbb{R}^2, d)$, wobei $d$ die französische Eisenbahnmetrik sei.
Untersuchen Sie, welche Stetigkeitseigenschaften die folgenden Abbildungen
besitzen.
\begin{enumerate}[(i)]
\item $g \colon \qty(\mathbb{R}^2, d) \to \qty(\mathbb{R}^2, \norm{\cdot}_2)$
  mit $g(x) \coloneqq x$
\item $h \colon \qty(\mathbb{R}^2, \norm{\cdot}_2) \to \qty(\mathbb{R}^2, d)$
  mit $h(x) \coloneqq x$
\end{enumerate}

\section*{Hausaufgabe 2}

Untersuchen Sie die Funktionen $f \colon \mathbb{R}^2 \to \mathbb{R}$ definiert
durch
\[
  f(x, y) = \begin{cases}
    \frac{\sin(xy)}{x} & \text{für } x \ne 0 \\
    y & \text{für } x = 0 \\
  \end{cases}
\]
auf Stetigkeit.

\section*{Hausaufgabe 3}

Wir betrachten $\qty(\mathbb{R}^N, \norm{\cdot})$ mit einer beliebigen Norm
$\norm{\cdot}$.
Beweisen Sie folgende Aussagen: Sind $A, B \subseteq \mathbb{R}^N$ kompakte
Mengen, dann sind die Mengen
\[
  A \times B = \qty{(x, y) \middle| x \in A, y \in B} \subseteq \mathbb{R}^{2N}
  \text{ und }
  A + B = \qty{x + y \middle| x \in A, y \in B} \subseteq \mathbb{R}^N
\]
ebenfalls kompakt.

\end{document}
