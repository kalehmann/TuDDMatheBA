\documentclass{scrreprt}

\usepackage{aligned-overset}
\usepackage{amsmath}
\usepackage{amssymb}
\usepackage{bm}
\usepackage[shortlabels]{enumitem}
\usepackage{hyperref}
\usepackage[utf8]{inputenc}
\usepackage{mathtools}
\usepackage{physics}
\usepackage{tabularx}
\usepackage{titling}
\usepackage{fancyhdr}
\usepackage{xfrac}
\usepackage{pgfplots}

\pgfplotsset{compat = newest}

\author{Albina Oscherowa \\ Lukas Kamratzki \\ Karsten Lehmann}
\date{SoSe 2021}
\title{Hausaufgabe 05 \\Analysis - Weiterführende Konzepte}

\setlength{\headheight}{26pt}
\pagestyle{fancy}
\fancyhf{}
\lhead{\thetitle}
\rhead{\theauthor}
\lfoot{\thedate}
\rfoot{Seite \thepage}

\begin{document}

\section*{Hausaufgabe 1}

Wir betrachten die metrischen Räume $\qty(\mathbb{R}^2, \norm{\cdot}_2)$ und
$\qty(\mathbb{R}^2, d)$, wobei $d$ die französische Eisenbahnmetrik sei.
Untersuchen Sie, welche Stetigkeitseigenschaften die folgenden Abbildungen
besitzen. \\

\textit{Lsg.} Die französische Eisenbahnmetrik verdankt ihren Namen dem
zentralisiertem Eisenbahnnetz Frankreichs.
Wenn Paris nicht auf der Luftlinie zweier Orte $A$ und $B$ liegt, muss man
auf dem Weg von $A$ nach $B$ in Paris umsteigen.
Die zurückgelegte Strecke von $A$ nach $B$ setzt sich somit aus der Entfernung
von $A$ nach Paris und der Entfernung von Paris nach $B$ zusammen.

Es gilt nun zu beweisen, dass dies auch für den Sonderfall, dass Paris auf der
Luftlinie zwischen $A$ und $B$ liegt gilt.
Sei Paris im Punkt $\qty(0, 0)$ und zwei Punkte $x, y \in \mathbb{R}^2$ so,
dass sie auf einer Geraden mit Paris liegen, dass heißt
$xt + (1 - t)y = 0, t \in \mathbb{R}$.
\begin{align*}
  \qty(d(x, y))^2 &= \norm{x - y}_2^2 = \norm{\frac{1 - t}{1 - t} x + \frac{t}{1 - t} y}_2^2
                   = \norm{\frac{1}{1 - t} x}_2^2
                   = \sum_{k = 1}^2 \abs{\frac{1}{1 - t} x_k}^2 & \\
                  &= \frac{1}{\abs{1 - t}^2} \norm{x}_2^2
                   = \norm{x}_2^2 \qty(\frac{\abs{1 - t}^2}{\abs{1 - t}^2} + 2\frac{\abs{t}}{\abs{1 - t}^2} - \frac{\abs{t}^2}{\abs{1 - t}^2}) \\
                  &= \norm{x}_2^2 +
                   \underset{
                     2 \qty(\sum_{k = 1}^2 \abs{x_k}^2 \cdot \sum_{k = 1}^2 \abs{- \frac{t}{1 - t} x_k}^2)^{\sfrac{1}{2}} = 2 \norm{x}_2 \norm{y}_2
                   }{\underbrace{2\frac{\abs{t}}{\abs{1 - t}^2} \norm{x}_2^2}}
                   + \norm{-\frac{t}{1 - t} x}_2^2 \\
  \qty(d(x, y))^2 &= \norm{x}_2^2 + 2 \norm{x}_2 \norm{y}_2 + \norm{y}_2^2 \\
  d(x, y)         &= \norm{x} + \norm{y}
\end{align*}

\begin{enumerate}[(i)]
\item $g \colon \qty(\mathbb{R}^2, d) \to \qty(\mathbb{R}^2, \norm{\cdot}_2)$
  mit $g(x) \coloneqq x$

  \textit{Lsg.} Seien $d_1$ die französische Eisenbahnmetrik und $d_2$ die von
  der euklidischen Norm induzierte Metrik.

  \textbf{Annahme}: $g$ ist Lipschitzstetig.
  \begin{align*}
    \forall x, y \in \mathbb{R}^2 \colon d_2(g(x), g(y)) &\leq d_1(x, y) \\
    \norm{x - y}_2 &\leq \norm{x}_2 + \norm{y}_2 \\
  \end{align*}
  Diese Aussage ist offensichtlich war, damit ist $g$ Lipschitzstetig,
  gleichmäßig stetig und stetig.


\newpage
\item $h \colon \qty(\mathbb{R}^2, \norm{\cdot}_2) \to \qty(\mathbb{R}^2, d)$
  mit $h(x) \coloneqq x$

  \textit{Lsg.} Seien $d_1$ die von der euklidischen Norm induzierte Metrik und
  $d_2$ die französische Eisenbahnmetrik.

  \textbf{Annahme}: $h$ ist Lipschitzstetig.
  \begin{align*}
    \exists L \geq 0 \forall x, y \in \mathbb{R}^2 \colon d_2(h(x), h(y)) &\leq L d_1(x, y) \\
    \norm{x}_2 + \norm{y}_2 &\leq L \norm{x - y}_2 \\
    \text{Angenommen $x = y = (1, 0)$} \to \quad 1 + 1 &\leq L 0 \\
    \Rightarrow h \text{ ist nicht Lipschitzstetig.}
  \end{align*}

  \textbf{Annahme}: $h$ ist gleichmäßig stetig.
  \begin{align*}
    \forall \epsilon > 0 \exists \delta > 0 \forall x, y \in \mathbb{R}^2 \colon d_1(x, y) < \delta
    &\Rightarrow d_2(h(x), h(y)) \\
    \norm{x - y}_2 < \delta &\Rightarrow \norm{x}_2 + \norm{y}_2 < \epsilon
  \end{align*}
  Sei nun neben $x$ und $y$ auch $z = (0, 0)$ aus $\mathbb{R}^2$.
  Nach der Dreiecksungleichung ist
  $\norm{x - y}_2 \leq \norm{x - z}_2 + \norm{y - z}_2$.
  Da $\norm{x - z}_2 + \norm{y - z}_2 = \norm{x}_2 + \norm{y}_2$ ist die Annahme
  für $\delta = \frac{\epsilon}{2}$ wahr, somit ist $h$ gleichmäßig stetig.

\end{enumerate}

\newpage
\section*{Hausaufgabe 2}

Untersuchen Sie die Funktionen $f \colon \mathbb{R}^2 \to \mathbb{R}$ definiert
durch
\[
  f(x, y) = \begin{cases}
    \frac{\sin(xy)}{x} & \text{für } x \ne 0 \\
    y & \text{für } x = 0 \\
  \end{cases}
\]
auf Stetigkeit.

\newpage
\section*{Hausaufgabe 3}

Wir betrachten $\qty(\mathbb{R}^N, \norm{\cdot})$ mit einer beliebigen Norm
$\norm{\cdot}$.
Beweisen Sie folgende Aussagen: Sind $A, B \subseteq \mathbb{R}^N$ kompakte
Mengen, dann sind die Mengen
\[
  A \times B = \qty{(x, y) \middle| x \in A, y \in B} \subseteq \mathbb{R}^{2N}
  \text{ und }
  A + B = \qty{x + y \middle| x \in A, y \in B} \subseteq \mathbb{R}^N
\]
ebenfalls kompakt.

\end{document}
