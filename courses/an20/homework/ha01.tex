\documentclass{article}
\usepackage{aligned-overset}
\usepackage{amsmath}
\usepackage{amssymb}
\usepackage{bm}
\usepackage[shortlabels]{enumitem}
\usepackage{hyperref}
\usepackage[utf8]{inputenc}
\usepackage{mathtools}
\usepackage{physics}
\usepackage{tabularx}
\usepackage{titling}
\usepackage{fancyhdr}
\usepackage{xfrac}
\usepackage{pgfplots}

\pgfplotsset{compat = newest}

\author{}
\date{SoSe 2021}
\title{Hausaufgabe 01 Analysis - Weiterführende Konzepte}

\pagestyle{fancy}
\fancyhf{}
\lhead{\thetitle}
\rhead{\theauthor}
\lfoot{\thedate}
\rfoot{Seite \thepage}

\begin{document}
\section*{Hausaufgabe 1}

Es seien $f \colon [a, b] \to \mathbb{R}$ eine stetige Funktion und
$g \colon [c, d] \to \mathbb{R}$ eine Treppenfunktion mit
$f([a, b]) \subseteq [c, d]$ und $g([c, d]) \subseteq [a, b]$.

\begin{enumerate}[a)]
\item Zeigen Sie, dass $f \circ g \colon [c, d] \to \mathbb{R}$ eine
  Treppenfunktion ist

  $f \circ g$ ist eine Treppenfunktion auf dem Intervall $[c, d]$, wenn es
  eine Partition $\sigma \colon c = x_0 < x_1 < \ldots < x_n = d$ gibt, so dass
  $f \circ g$ auf jedem Intervall $\left[x_k, x_{x + 1}\right[$ mit
  $0 \leq k \leq n - 1$ konstant ist.
  Das heißt $(f \circ g)(p) = (f \circ g)(q) \forall p, q \in \left[ x_k, x_{k + 1}\right[$.

  Nun ist $g$ auf diesem Intervall bereits eine Treppenfunktion auf dem
  Intervall $[c, d]$.
  Somit existiert auch eine Partition
  $\sigma \colon c = x_0 < x_1 < \ldots < x_n = d$, so dass auf jedem Intervall
  $\left[x_k, x_{x + 1}\right[$ mit $0 \leq k \leq n - 1$ gilt:
  \[
    g(p) = g(q) \forall p, q \in \left[ x_k, x_{k + 1}\right[
  \]

  Anstatt der Annahme
  \[
    (f \circ g)(p) = (f \circ g)(q) \forall p, q \in \left[ x_k, x_{k + 1}\right[
  \]
  kann man auch
  \[
    f(g(p)) = f(g(q)) \forall p, q \in \left[ x_k, x_{k + 1}\right[
  \]
  schreiben. Wenn man zusätzlich noch $g(p)$ mit $g(q)$ substituiert, sieht man
  dass die Funktion
  $f \circ g$ auf jedem Intervall $\left[x_k, x_{x + 1}\right[$ mit
  $0 \leq k \leq n - 1$ konstant ist.

  Somit ist $f \circ g$ eine Treppenfunktion.
  
\item Geben Sie Funktionen $f, g$ an, für die $g \circ f$ keine
  Treppenfunktion ist.

  Sei $f(x) \coloneqq \frac{c + d}{2}$ und $g(x) = \begin{cases}
    a & \text{wenn $x \leq \frac{c + d}{2}$} \\
    b & \text{wenn $x > \frac{c + d}{2}$}
  \end{cases}$.
  Somit existiert keine Partition
  $\sigma \colon a = x_0 < x_1 < \ldots < x_n = b$
  so dass $g \circ f$ auf dem Intervall $[x_0, x_1[$ konstant ist.
\end{enumerate}

\section*{Hausaufgabe 2}

Sind die folgenden angegebenen Funktionen
$f_k \colon [0, 2] \to \mathbb{R}, k = 1, 2, 3, 4$ Riemann-integrierbar?
Begründen Sie Ihre Entscheidung.

\begin{enumerate}
\item $f_1(x) \coloneqq \left\lfloor x \right\rfloor$
  
  \begin{tikzpicture}
    \begin{axis}[
      axis lines=middle,
      xmax = 2.5,
      xmin = -0.5,
      ymax = 9,
      ymin = -1]
      \addplot [
      jump mark mid,
      domain=0:2,
      samples=100,
      very thick, red
      ] {floor(4*x)};
    \end{axis}
  \end{tikzpicture}

  Für die Funktion existiert die Partition
  $\sigma \coloneqq \left\{ 0, 0.25, 0.5, 0.75, 1, 1.25, 1.5, 1.75, 2 \right\}$,
  so dass $f_1$ auf jedem Intervall $[x_x, x_{k + 1}[ (0 \leq k \leq 8)$
  konstant ist.

  Damit ist $f_1$ eine Treppenfunktion und wie aus der Vorlesung bereits
  bekannt ist jede Treppenfunktion Riemann-integrierbar.
  
\item $f_2(x) \coloneqq e^{-x^2}$

  \begin{tikzpicture}
    \begin{axis}[
      axis lines=middle,
      xmax = 2.5,
      xmin = -0.5,
      ymax = 1.2,
      ymin = -0.2]
      \addplot [
      domain=0:2,
      samples=100,
      very thick, red
      ] {e^(-x^2)};
    \end{axis}
  \end{tikzpicture}

  Angenommen $g(x) = 1$ und $h(x) = x^2$ seien Funktionen,
  Dann entspricht $e^{-x^2} = \frac{g}{\exp \circ j}$.
  Aus den Beispielen und Rechenregeln für stetige Funktionen unter Punkt
  3.2 der Vorlesung folgt, dass $e^{-x^2}$ stetig ist.
  Nach Proposition 5.1.5 der Vorlesung ist jede stetige Funktion auf einem
  kompakten Intervall Riemann-integrierbar, somit ist $f_2$ Riemann-integrierbar.
  
\end{enumerate}

\section*{Hausaufgabe 4}

Ermitteln Sie größtmögliche Intervalle $I \in \mathbb{R}$ auf denen
die folgenden Ausdrücke Stammfunktionen besitzen.
Berechnen Sie Stammfunktionen auf diesen Intervallen.

\begin{enumerate}[a)]
\item $f(x) = 2^xe^x + 23 \sqrt{x^3\sqrt{x\sqrt{x}}}$

  $I = [0, \infty]$ sei $F(t) = F(x), x \in I$

  \begin{align*}
    \int_0^t \left( 2^te^t + 23 t^{\frac{15}{8}} \right)\,dt &=
    \overset{(i)}{\int_0^t \left( 2^te^t \right) \,dt} + 23 \cdot \overset{(ii)}{\int_0^t \left(t^{\frac{15}{8}} \right) \,dt} \\
  \end{align*}
  \begin{minipage}[t]{.45\textwidth}
    \textbf{$(i)$}
    \begin{align*}
      &\int_0^t \left( 2^te^t \right) \,dt \\
      = & \int_0^t \left( e^{(\ln(2) + 1)t} \right) \,dt \\
      = & \frac{e^{(\ln(2) + 1)t}}{\ln(2) + 1}
    \end{align*}
  \end{minipage}
  \hfill
  \begin{minipage}[t]{.45\textwidth}
    \textbf{$(ii)$}
    \begin{align*}
      &\int_0^t \left( t^{\frac{15}{8}} \right) \,dt \\
      = & \frac{8}{23} \cdot t^{\frac{23}{8}} \\
      = & 2 \cdot (t + 1)^{\frac{3}{2}} + t \cdot \arctan (t) + c
    \end{align*}
  \end{minipage}
  \[
    F(t) = \frac{e^{(\ln(2) + 1)t}}{\ln(2) + 1} + \frac{8}{23} \cdot t^{\frac{23}{8}} + c
  \]

\item $f(x) = \frac{3}{\sqrt{x} + \sqrt{x + 1}} + \sqrt{x^2}{x^2 + 1}$

  $I = (0, \infty]$ sei $F(t) = F(x), x \in I$
  \begin{align*}
    &\int_0^t \left( \frac{3}{\sqrt{t} + \sqrt{t + 1}} \right)\,dt + \int_0^t \left( \frac{t^2}{\sqrt{t^2 + 1}} \right)\,dt \\
    = &3 \cdot \int_0^t \left( \sqrt{t + 1} - \sqrt{t} \right)\,dt + \int_0^t \left( 1 - \frac{1}{t^2 - 1} \right)\,dt
  \end{align*}
\item $f(x) = \frac{\sqrt{1 + x} + \sqrt{1 - x^2}}{\sqrt{1 - x^4}}$

  $I = [0, 1)$ sei $F(t) = F(x), x \in I$

  \begin{align*}
    \int_0^t \left( \frac{\sqrt{t^2 + 1} + \sqrt{1 - t^2}}{\sqrt{1 - t^4}}\right) \,dt
    &= \int_0^t \left( \frac{\sqrt{t^2 + 1}}{\sqrt{1 - t^4}} + \frac{\sqrt{1 - t^2}}{\sqrt{1 - t^4}}\right) \,dt \\
    &= \overset{(i)}{\int_0^t \left( \frac{\sqrt{t^2 + 1}}{\sqrt{1 - t^4}}\right) \,dt} +
       \overset{(ii)}{\int_0^t \left(\frac{\sqrt{1 - t^2}}{\sqrt{1 - t^4}}\right) \,dt} \\
  \end{align*}
  \begin{minipage}[t]{.45\textwidth}
    \textbf{$(i)$}
    \begin{align*}
      &\int_0^t \left( \frac{\sqrt{t^2 + 1}}{\sqrt{1 - t^4}}\right) \,dt \\
      = & \int \frac{1}{\sqrt{1 - t^2}} \\
      = & \arcsin (x)
    \end{align*}
  \end{minipage}
  \hfill
  \begin{minipage}[t]{.45\textwidth}
    \textbf{$(ii)$}
    \begin{align*}
      &\int_0^t \left( \frac{\sqrt{1 - t^2}}{\sqrt{1 - t^4}}\right) \,dt \\
      = & \int_0^t \frac{1}{\sqrt{t^2 + 1}} \\
      = & \ln \left( \sqrt{t^2 + 1} + t\right)
    \end{align*}
  \end{minipage}
  \[
    F(t) = \arcsin (x) + \ln \left( \sqrt{t^2 + 1} + t\right) + c
  \]
\end{enumerate}

\end{document}