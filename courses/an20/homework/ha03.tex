\documentclass{article}
\usepackage{aligned-overset}
\usepackage{amsmath}
\usepackage{amssymb}
\usepackage{bm}
\usepackage[shortlabels]{enumitem}
\usepackage{hyperref}
\usepackage[utf8]{inputenc}
\usepackage{mathrsfs}
\usepackage{mathtools}
\usepackage{physics}
\usepackage{tabularx}
\usepackage{titling}
\usepackage{fancyhdr}
\usepackage{xfrac}

\author{Albina Oscherowa \\ Lukas Kamratzki \\ Karsten Lehmann}
\date{SoSe 2021}
\title{Hausaufgabe 03 Analysis - Weiterführende Konzepte}

\pagestyle{fancy}
\fancyhf{}
\lhead{\thetitle}
\rhead{\theauthor}
\lfoot{\thedate}
\rfoot{Seite \thepage}

\begin{document}

\section*{Hausaufgabe 1}

Für eine stetige Funktion $f \colon [0, \infty) \to \mathbb{R}$ und
$\lambda \in \mathbb{R}$ sei
\[
  \mathscr{L}f(\lambda) \coloneqq \int_0^{\infty} f(x)e^{-\lambda x}\,dx
\]
wenn das uneigentliche Integral existiert.
$\mathscr{L}f$ wird als Laplace-Transformation von $f$ bezeichnet.
Beweisen Sie die folgenden Aussagen:
\begin{enumerate}[(i)]
\item Ist $f$ beschränkt, so existiert $\mathscr{L}f$ für alle $\lambda > 0$.

  \textit{Lsg.:} Da $f$ beschränkt ist, existiert ein $C \in \mathbb{R}_{\geq 0}$
  mit $\abs{f(x)} \leq C \quad \forall \, x \in \mathbb{R}$.
  Sei $g(x) = \abs{f(x)} e^{-\lambda x}$ und $h(x) = C \cdot e^{-\lambda x}$.
  
  \begin{align*}
    &\Rightarrow \quad g \leq  h  \\
    \overset{s > r > 0}&\Rightarrow \quad \int_r^s g(x)\,dx \leq \int_r^s h(x)\,dx
  \end{align*}
  Aus der Dreiecksungleich folgt
  \begin{flalign*}
    \abs{\int_r^s f(x) e^{-\lambda x}\,dx} \leq \int_r^s \abs{f(x)} e^{-\lambda x} \,dx
    &\leq \int_r^s C \cdot e^{-\lambda x}\,dx \\
    &= \qty[-\frac{c}{\lambda} \cdot e^{-\lambda x}]_r^s \\
    &= \frac{C}{\lambda} \qty(e^{-\lambda r} - e^{-\lambda s}) \\
    &\leq \frac{C}{\lambda} e^{-\lambda r} = \epsilon \\
    \Rightarrow r = -\frac{\ln\qty(\frac{\lambda\epsilon}{C})}{\lambda}
  \end{flalign*}
  Somit besitzt $\mathscr{L}f(\lambda)$ für $\lambda > 0$ die Eigenschaft
  \[
    \forall \epsilon > 0 \exists R > 0 \forall s \geq r > R \abs{\mathscr{L}f(\lambda)} < \epsilon
  \]
  Aus der Aufgabe 2. (i) des Blattes 3 folgt die Existenz von $\mathscr{L}f$
  
\item Existiert $\mathscr{L}f$ für eine $\lambda_0 \in \mathbb{R}$, so existiert
  $\mathscr{L}f$ für alle $\lambda \geq \lambda_0$.
\end{enumerate}

\newpage
\section*{Hausaufgabe 2}

Seien $(X, \norm{\cdot})$ ein normierter Raum und
$d(x, y) \coloneqq \norm{x - y}, x, y \in X$.
Bekanntlich ist $(X, d)$ ein metrischer Raum.
\begin{enumerate}[a)]
\item Geben Sie für $x \in X$ und $\epsilon > 0$ die Mengen $B(x, \epsilon)$ an.
  Welche Beziehung besteht zwischen den Mengen $B(0, \epsilon)$ und
  $B(x, \epsilon)$?

  \textit{Lsg.:}
  \begin{align*}
    B(x, \epsilon) &= \qty{ y \in X \middle| \norm{x - y} < \epsilon} \\
    B(0, \epsilon) &= \qty{ y \in X \middle| \norm{-y} < \epsilon}
  \end{align*}
  
\item Zeigen Sie, dass die Kugeln $B(x, \epsilon)$ für $x \in X$ und
  $\epsilon > 0$ konvexe Mengen sind.

  \textbf{Hinweis:} Eine Teilmenge $M \subseteq$ eines Vektorraumes $X$ heißt
  konvex. wenn aus $x, y \in M$ und $\lambda \in (0, 1)$ die Beziehung
  $\lambda x + (1 - \lambda)y \in M$ folgt, d.h., die Verbindungsstrecke
  zweier Punkte aus $M$ ist in $M$ enthalten.
\end{enumerate}

\newpage
\section*{Hausaufgabe 3}

Sei $M$ eine nichleere Menge.
\begin{enumerate}[a)]
\item Zeigen Sie, dass durch $d(x, y) \colon \begin{cases}
    1 & \text{für $x \ne y$} \\
    0 & \text{für $x = y$} \\
  \end{cases}$ eine Metrik auf $M \times M$ definiert ist.

  \textit{Lsg.:} $d$ definiert eine Metrik auf $M \times M$, wenn
  \begin{itemize}
  \item $\forall x, y \in M \colon d(x,y) \geq 0$ und
    $\forall x,y \in M \colon d(x,y) = 0 \iff x = y$

    Diese Bedingung ist durch die Definition von $d$ erfüllt.

  \item $\forall x,y \in M \colon d(x, y) = d(x, y)$
    \begin{enumerate}[label={Fall \arabic*:},leftmargin=*]
    \item $x \ne y \colon d(x, y) = 1 = d(y, x)$
    \item $x = y \colon d(x, y) = 0 = d(y, x)$
    \end{enumerate}
    
  \item $\forall x,y,z \in M \colon d(x,y) \leq d(x,z) + d(y,z)$
    \begin{enumerate}[label={Fall \arabic*:},leftmargin=*]
    \item $x \ne y \land x \ne z \land y \ne z \colon d(x, y) = 1 \leq d(x, z) + d(y, z) = 2$
    \item $x = y \land x = z \land y = z \colon d(x, y) = 0 \leq d(x, z) + d(y, z) = 0$
    \item $x \ne y \land y = z \colon d(x, y) = 1 \leq d(x, z) + d(y, z) = 1$
    \item $x = y \land x \ne z \colon d(x, y) = 0 \leq d(x, z) + d(y, z) = 2$
    \end{enumerate}
    
  \end{itemize}
  
\item Geben Sie für ein festes $x \in M$ und $r > 0$ die Kugeln
  $B(x, r) = \qty{ y \in M \middle| d(x, y) < r}$ und
  $B[x, r] = \qty{ y \in M \middle| d(x, y) \leq r}$ an.

  \textit{Lsg.:}
  \begin{enumerate}[label={Fall \arabic*:},leftmargin=*]
  \item $0 < r < 1$

    $d(x, y) < r$ wenn $x = y$

    $\Rightarrow$ Kugel entspricht $\qty{x}$
  \item $r = 1$

    $B(x, r) = \qty{x}$ und $B[x,r] = M$
  \item $r > 1$

    $B(x, r) = B[x, r] = M$
  \end{enumerate}
\end{enumerate}

\textbf{Hinweis:} Die so definierte Metrik $d$ heißt diskrete Metrik von $M$.

\newpage
\section*{Hausaufgabe 4}

Untersuchen Sie, ob folgende Aussagen wahr oder falsch sind:
\begin{enumerate}[(i)]
\item Für alle $x, y \in \mathbb{R}^N$ und $p \in (1, \infty)$ gilt
  \[
    \qty(\sum_{k = 1}^N \abs{x_k y_k}^3)^{\sfrac{1}{3}} \leq \norm{x}_{3p} \norm{y}_{\frac{3p}{p - 1}}
  \]

  \textit{Lsg.:}
  \begin{align*}
    \qty(\sum_{k = 1}^N \abs{x_k y_k}^3)^{\sfrac{1}{3}}
    &\leq \qty(\sum_{k = 1}^N \qty(\abs{x_k}^3)^p )^{\frac{1}{3} \cdot \frac{1}{p}} \cdot
      \qty(\sum_{k = 1}^N \qty(\abs{y_k}^3)^p )^{\frac{1}{3} \cdot \frac{1}{p}} \\
    &= \qty(\sum_{k = 1}^N \abs{x_k}^{3p})^{\frac{1}{3p}} \cdot
      \qty(\sum_{k = 1}^N \abs{y_k}^{\frac{3p}{p - 1}} )^{\frac{p - 1}{3p}} \\
    &= \norm{x}_{3p} \cdot \norm{y}_{\frac{3p}{p - 1}}
  \end{align*}
\item Für alle $x, y \in \mathbb{R}^N$ und $p \in (1, \infty)$ gilt
  \[
    \qty(\sum_{k = 1}^N \abs{x_k y_k}^3)^{\sfrac{1}{3}} \leq \leq \norm{x}_{3p} \norm{y}_1
  \]
\item Seien $x, y \in \mathbb{R}^N$ und $p \in (1, \infty)$. Dann gilt
  \[
    \norm{x + y}_p \geq \norm{x}_p + \norm{y}_p \Rightarrow x = y
  \]
\end{enumerate}

\end{document}