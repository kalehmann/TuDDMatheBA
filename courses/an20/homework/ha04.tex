\documentclass{article}
\usepackage{aligned-overset}
\usepackage{amsmath}
\usepackage{amssymb}
\usepackage{bm}
\usepackage[shortlabels]{enumitem}
\usepackage{hyperref}
\usepackage[utf8]{inputenc}
\usepackage{mathrsfs}
\usepackage{mathtools}
\usepackage{physics}
\usepackage{tabularx}
\usepackage{titling}
\usepackage{fancyhdr}
\usepackage{xfrac}

\author{Albina Oscherowa \\ Lukas Kamratzki \\ Karsten Lehmann}
\date{SoSe 2021}
\title{Hausaufgabe 04 Analysis - Weiterführende Konzepte}

\pagestyle{fancy}
\fancyhf{}
\lhead{\thetitle}
\rhead{\theauthor}
\lfoot{\thedate}
\rfoot{Seite \thepage}

\begin{document}

\section*{Hausaufgabe 1}

Wir betrachten $D = \mathbb{Q}$ als Teilmenge von $\mathbb{R}$.
Zeigen Sie, dass $\mathring{\mathbb{Q}} = \emptyset$, $\overline{\mathbb{Q}} = \mathbb{R}$
und $\partial \mathbb{Q} = \mathbb{R}$ gelten. \\

\textit{Lsg.} Das Innere von $\mathbb{Q}$, die Menge $\mathring{\mathbb{Q}}$ ist definiert als die Vereinigung aller
offenen Teilmengen von $\mathbb{Q}$.
Eine Teilmenge $A \subseteq \mathbb{Q}$ heißt offen, wenn es ein Element $q \in \mathbb{Q}$
und ein $r > 0$ gibt, so dass $B(x, r) \subseteq A$.
Nun sind $q, q + r \in \mathbb{R}, q < q + r$.
Damit ist $(q, q + r) \setminus \mathbb{Q} \ne \emptyset$, denn zwischen zwei reellen Zahlen liegt
immer mindestens eine irrationale Zahl (Korollar 1.4.9 der Vorlesung).
Somit existiert keine offene Teilmenge $A \subseteq \mathbb{Q}, A \ne \emptyset$.

Damit ist $\mathring{\mathbb{Q}} = \emptyset$. \\

Der Abschluß von $\mathbb{Q}$, die Menge $\overline{\mathbb{Q}}$ ist definiert als
die Schnittmenge aller abgeschlossenen Teilmengen von $\mathbb{R}$, die $\mathbb{Q}$
enthalten.

Sei $A$ eine abgeschlossene, echte Teilmenge von $\mathbb{R}$, die $\mathbb{Q}$ enthält.
Damit $A$ abgeschlossen ist, muss das Komplement von $A$, die Menge $\mathbb{R} \setminus A$
offen sein, das heißt für jedes $x \in A$ existiert ein $r > 0$ mit $B(x, r) \subseteq A$.
Nun sind $x, x + r \in \mathbb{R}, x < x + r$.
Damit ist $(q, q + r) \cap \mathbb{Q} \ne \emptyset$, denn zwischen zwei reellen Zahlen liegt
immer mindestens eine rationale Zahl (Korollar 1.4.9 der Vorlesung).
Das ist ein Widerspruch dazu, dass $\mathbb{Q} \subseteq A$.
Somit existiert keine abgeschlossene, echte Teilmenge von $\mathbb{R}$, die $\mathbb{Q}$ enthält.
Damit ist die Schnittmenge aller abgeschlossenen Teilmengen von $\mathbb{R}$, die $\mathbb{Q}$
enthalten gleich $\mathbb{R} = \overline{\mathbb{Q}}$.

Der Rand von $\mathbb{Q}$, die Menge $\partial \mathbb{Q}$ ist definiert als
$\overline{\mathbb{Q}} \setminus \mathring{\mathbb{Q}} = \mathbb{R} \setminus \mathbb{R} = \emptyset$.

\end{document}