\documentclass{article}
\usepackage{aligned-overset}
\usepackage{amsmath}
\usepackage{amssymb}
\usepackage{bm}
\usepackage[shortlabels]{enumitem}
\usepackage{hyperref}
\usepackage[utf8]{inputenc}
\usepackage{mathrsfs}
\usepackage{mathtools}
\usepackage{physics}
\usepackage{tabularx}
\usepackage{titling}
\usepackage{fancyhdr}
\usepackage{xfrac}

\author{Albina Oscherowa \\ Lukas Kamratzki \\ Karsten Lehmann}
\date{SoSe 2021}
\title{Hausaufgabe 04 Analysis - Weiterführende Konzepte}

\pagestyle{fancy}
\fancyhf{}
\lhead{\thetitle}
\rhead{\theauthor}
\lfoot{\thedate}
\rfoot{Seite \thepage}

\begin{document}

\section*{Hausaufgabe 1}

Wir betrachten $D = \mathbb{Q}$ als Teilmenge von $\mathbb{R}$.
Zeigen Sie, dass $\mathring{\mathbb{Q}} = \emptyset$, $\overline{\mathbb{Q}} = \mathbb{R}$
und $\partial \mathbb{Q} = \mathbb{R}$ gelten. \\

\textit{Lsg.} Das Innere von $\mathbb{Q}$, die Menge $\mathring{\mathbb{Q}}$ ist definiert als die Vereinigung aller
offenen Teilmengen von $\mathbb{Q}$.
Eine Teilmenge $A \subseteq \mathbb{Q}$ heißt offen, wenn es ein Element $q \in \mathbb{Q}$
und ein $r > 0$ gibt, so dass $B(x, r) \subseteq A$.
Nun sind $q, q + r \in \mathbb{R}, q < q + r$.
Damit ist $(q, q + r) \setminus \mathbb{Q} \ne \emptyset$, denn zwischen zwei reellen Zahlen liegt
immer mindestens eine irrationale Zahl (Korollar 1.4.9 der Vorlesung).
Somit existiert keine offene Teilmenge $A \subseteq \mathbb{Q}, A \ne \emptyset$.

Damit ist $\mathring{\mathbb{Q}} = \emptyset$. \\

Der Abschluß von $\mathbb{Q}$, die Menge $\overline{\mathbb{Q}}$ ist definiert als
die Schnittmenge aller abgeschlossenen Teilmengen von $\mathbb{R}$, die $\mathbb{Q}$
enthalten.

Sei $A$ eine abgeschlossene, echte Teilmenge von $\mathbb{R}$, die $\mathbb{Q}$ enthält.
Damit $A$ abgeschlossen ist, muss das Komplement von $A$, die Menge $\mathbb{R} \setminus A$
offen sein, das heißt für jedes $x \in A$ existiert ein $r > 0$ mit $B(x, r) \subseteq A$.
Nun sind $x, x + r \in \mathbb{R}, x < x + r$.
Damit ist $(q, q + r) \cap \mathbb{Q} \ne \emptyset$, denn zwischen zwei reellen Zahlen liegt
immer mindestens eine rationale Zahl (Korollar 1.4.9 der Vorlesung).
Das ist ein Widerspruch dazu, dass $\mathbb{Q} \subseteq A$.
Somit existiert keine abgeschlossene, echte Teilmenge von $\mathbb{R}$, die $\mathbb{Q}$ enthält.
Damit ist die Schnittmenge aller abgeschlossenen Teilmengen von $\mathbb{R}$, die $\mathbb{Q}$
enthalten gleich $\mathbb{R} = \overline{\mathbb{Q}}$.

Der Rand von $\mathbb{Q}$, die Menge $\partial \mathbb{Q}$ ist definiert als
$\overline{\mathbb{Q}} \setminus \mathring{\mathbb{Q}} = \mathbb{R} \setminus \mathbb{R} = \emptyset$.

\section*{Hausaufgabe 2}

Sei $M$ eine nichtleere Menge und $d$ die diskrete Metrik auf $M$.

\[
  d(x, y) \coloneqq \begin{cases}
    0 & x = y \\
    1 & x \ne y \\
  \end{cases}
\]

\begin{enumerate}[a)]
\item Sei $x \in M$ gegeben. Beschreiben Sie die Menge aller Folgen $\qty(x_n)$ in $M$
  mit $\lim_{n \to \infty} x_n = x$.

  \textit{Lsg.} Für eine Folge $\qty(x_n)$ aus $M$ ist $\lim_{n \to \infty} x_n = x$,
  wenn
  \[
    \forall \epsilon > 0 \exists n_0 \in \mathbb{N} \forall n \in \mathbb{N}_{\geq n_0} \colon d(x_n, x) < \epsilon
  \]
  Das heißt, alle Glieder der Folge $\qty(x_n)$ sind ab einem $n_0$ konstant gleich $x$.
\newpage
\item Geben Sie die Systeme der offenen und abgeschlossen Mengen in $M$ an.

  \textit{Lsg.} Eine Teilmenge $A \subseteq M$ heißt offen, wenn es für jedes $a \in A$
  ein $r > 0$ gibt mit $B(a, r) = \qty{y \in M \colon d(a, y) < r} \subseteq A$.

  Für jedes $a \in A$ und $r \in (0, 1]$ gilt $B(a, r) = \qty{a} \subseteq{A}$.
  Somit ist jede Teilmenge von $M$ eine offene Menge.

  Gleichzeitig ist auch das Komplement jeder Teilmenge aus $M$ eine offene Menge.
  Damit sind alle Teilmengen aus $M$ offen und abgeschlossen.
\end{enumerate}

\section*{Hausaufgabe 3}

Sei $M \coloneqq (0, \infty)$ und $d \colon M \times M \to \mathbb{R}$ sei definiert durch
$d(x, y) = \abs{\ln x - \ln y}$.

\begin{enumerate}[(i)]
\item Beweisen Sie, dass $(M, d)$ ein metrischer Raum ist.

  \textit{Lsg.} $(M, d)$ ist ein metrischer Raum, falls die Abbildung $d$ folgende 3 Eigenschaften
  erfüllt:
  \begin{enumerate}[(i)]
  \item $\forall x, y \in M \colon d(x, y) \geq 0$ und $d(x, y) = 0 \iff x = f$

    Der erste Teil der Eigenschaft ist durch die Definition des Betrags erfüllt.
    Weiterhin  gilt $\abs{\ln x - \ln y} = 0 \iff \ln x = \ln y \iff x = y$,
    da $\ln$ monoton steigend und somit injektiv ist.

  \item \textbf{Symmetrie}. $\forall x, y \in M \colon d(x, y) = d(y, x)$

    Diese Eigenschaft ist erfüllt, da
    \[
      \abs{\ln x - \ln y} = \abs{-1 \cdot (\ln y - \ln x)} = \abs{\ln y - \ln x}
    \]

  \item \textbf{Dreiecksungleichung}. $\forall x, y, z \in M \colon d(x, y) \leq d(x, z) + d(z, y)$

    Diese Eigenschaft ist erfüllt, da
    \begin{align*}
      d(x, y) &= \abs{\ln x - \ln y} \\
              &= \abs{\ln x + 0 - \ln y} \\
              &= \abs{\ln x - \ln z + \ln z -  \ln y} \\
              &\leq \abs{\ln x - \ln z} + \abs{\ln z - \ln y} \\
              &= d(x, z) + d(z, y)
    \end{align*}
  \end{enumerate}

\newpage
\item Sei $d|_M$ die von $(\mathbb{R}, \abs{\cdot})$ induzierte Metrik auf $M$.
  Untersuchen Sie, ob die Räume $(M, d)$ und $\qty(M, d|_M)$ vollständig sind. \\

  \textit{Lsg.} $d|_M \colon M \times M \to \mathbb{R}, d|_m(x, y) = \abs{x - y}$.

  Sei $a_n \coloneqq \frac{1}{n} \in M$.
  Dann ist $\qty(a_n)$ in $(\mathbb{R}, \abs{\cdot})$ eine konvergente Folge mit
  $\lim_{n \to \infty} a_n = 0$.
  Damit ist $\qty(a_n)$ eine Cachy-Folge in $(\mathbb{R}, \abs{\cdot})$ und $(M, d|_M)$.

  Wegen $0 \notin M$ ist $\qty(a_n)$ nicht konvergent in $(M, d|_M)$ und somit ist $(M, d|_M)$
  nicht vollständig. \\

  Sei $\qty(b_n)$ eine Cauchy-Folge in $(M, d)$.
  Sei $e^r$ so, dass
  \[
    \forall \epsilon > 0 \exists n_0 \in \mathbb{N} \forall n \in \mathbb{N}_{\geq n_0} \colon
    \abs{\ln b_n - \ln e^r} = \abs{\ln b_n - r} < \epsilon
  \]
  Da $\qty(b_n)$ eine Cauchy-Folge ist, gilt $\ln b_n \in \mathbb{R}$ somit ist auch $r \in \mathbb{R}$
  und $e^r \in M$.
  Damit konvergieren alle Cauchy-Folgen in $\qty(M, d)$ gegen einen Wert in $M$ und $\qty(M, d)$
  ist vollständig.

\end{enumerate}

\section*{Hausaufgabe 4}

Untersuchen Sie, ob die folgenden Folgen $\qty(x_n)$ aus $\mathbb{R}^3$ beschränkt bzw.
konvergent bezüglich der Euklidischen Norm sind.
Geben Sie gegebenfalls den Grenzwert von $\qty(x_n)$ an.

\begin{enumerate}[(i)]
\item Für ein $q \in (0, 1)$ und $f(x) = 1 + qx$ sei
  $x_n = \qty(f^n(q), f^n(1), f^n(1 + q)) \in \mathbb{R}^3$, hierbei bezeichne
  $f^n = \underset{n-\text{mal}}{\underbrace{f \circ \ldots \circ f}}$ die $n$-fache
  Anwendung der Funktion $f$. \\

  \textit{Lsg.} Bestimmung einer expliziten Darstellung von $f^n(x)$

  \textit{Behauptung:} $P(n) \colon f^n (x) = \qty(\sum_{k = 0}^{n - 1} q^k) + q^nx$ \\
  \textit{Induktionsanfang}: $P(1) \colon f^1(x) = \qty(\sum_{k = 0}^{0} q^k) + q^1x = 1 + qx$ \\
  \textit{Induktionsschritt}: $P(n + 1) \colon f^{n + 1} = \qty(\sum_{k = 0}^{n} q^k) + q^{n + 1}x
  = 1 + q\qty(\qty(\sum_{k = 0}^{n - 1} q^k) + q^{n}x) = f(f^{n - 1}(x))$.
  \begin{align*}
    x_n = \qty(\frac{q^n - 1}{q - 1} + q^{n + 1}, \frac{q^{n + 1} - 1}{q - 1}, \frac{q^{n + 2} - 1}{q - 1})
  \end{align*}
  Die Folgen $\qty(\frac{q^n - 1}{q - 1} + q^{n + 1})_{n \in \mathbb{N}}$,
  $\qty(\frac{q^{n + 1} - 1}{q - 1})_{n \in \mathbb{N}}$
  und
  $\qty(\frac{q^{n + 2} - 1}{q - 1})_{n \in \mathbb{N}}$ sind konvergente Folgen aus
  $\mathbb{R}$ mit $\lim_{m \to \infty} (\ldots) = \frac{1}{1 - q}$.
  Folglich ist auch $(x_n)$ beschränkt mit
  $\lim_{n \to \infty} x_n = \qty( \frac{1}{1 - q},  \frac{1}{1 - q},  \frac{1}{1 - q}) \in \mathbb{R}^3$.
\newpage
\item
  Für eine Nullfolge $\qty(y_n)$ aus $\mathbb{R}$ sei $x_n = \qty(y_n, y_{2n}^2, y_{3n}^n) \in \mathbb{R}^3$.

  \textit{Lsg.} $\qty(y_n)$ ist eine Nullfolge aus $\mathbb{R}$.
  Somit sind auch $\qty(y_{2n})$ und $\qty(y_{3n})$ Nullfolgen.
  Aus Proposition 2.1.5 (b) der Vorlesung folgt, dass auch $y_{2n}^2$ und $y_{3n}^n$ Mullfolgen sind.

  Folglich ist auch $\qty(x_n)$ beschränkt mit $\lim_{n \to \infty} x_n = 0$

\end{enumerate}

\end{document}