\documentclass{scrreprt}

\usepackage{aligned-overset}
\usepackage{amsmath}
\usepackage{amssymb}
\usepackage{bm}
\usepackage[shortlabels]{enumitem}
\usepackage{hyperref}
\usepackage[utf8]{inputenc}
\usepackage{mathtools}
\usepackage{physics}
\usepackage{tabularx}
\usepackage{titling}
\usepackage{fancyhdr}
\usepackage{xfrac}
\usepackage{pgfplots}

\pgfplotsset{compat = newest}
\usepgfplotslibrary{fillbetween}

\author{Albina Oscherowa \\ Lukas Kamratzki \\ Karsten Lehmann}
\date{SoSe 2021}
\title{Hausaufgabe 09 \\Analysis - Weiterführende Konzepte}

\setlength{\headheight}{26pt}
\pagestyle{fancy}
\fancyhf{}
\lhead{\thetitle}
\rhead{\theauthor}
\lfoot{\thedate}
\rfoot{Seite \thepage}

\newcommand\skalprod[1]{\left\langle #1 \right\rangle}

\begin{document}
\paragraph{Hausaufgabe 1} Sei $M \subseteq \mathbb{R}^N$ eine nichtleere offene
Menge.
Der \textit{Laplace-Operator} $\Delta \colon C^2(M) \to C(M)$ ist definiert
durch
\[
  \Delta f(x) \coloneqq \qty(\frac{\partial^2f}{\partial x_1^2} + \ldots + \frac{\partial^2 f}{\partial x_N^2})(x)
  = \sum_{k = 1}^N \frac{\partial^2 f(x)}{\partial x_k^2} \: \text{für} \: f \in C^2(M)
\]
\begin{enumerate}[a)]
\item Berechnen Sie $\Delta f(x)$ für die Funktion
  $f(x) \coloneqq -\ln\qty(x_1^2 + \ldots + x_N^2), x \ne 0, N \in \mathbb{N}$.
  Für welches $N \in \mathbb{N}$ erfüllt $f$ die \textit{Laplace-Gleichung}
  $\Delta f(x) = 0$ auf der Menge $M = \mathbb{R}^N \setminus \qty{0}$?

  \subparagraph{Lösung:}
  $\frac{\partial^2 f}{\partial x_1^2} = 2 \cdot \frac{\qty(x_1^2 + \ldots + x_N^2) - 2 x_1}{x_1^2 + \ldots + x_N^2}$.
  \begin{flalign*}
    \Delta f(x) &= \sum_{k = 1}^N 2 \cdot \frac{\qty(x_1^2 + \ldots + x_N^2) - 4 \cdot x_k^2}{x_1^2 + \ldots + x_N^2} & \\
    &= 2N \cdot \frac{\qty(x_1^2 + \ldots + x_N^2)}{x_1^2 + \ldots + x_N^2} - 4 \sum_{k = 1}^N \frac{x_k^2}{x_1^2 + \ldots + x_N^2} \\
    &= 2N - 4\frac{x_1^2 + \ldots + x_N^2}{x_1^2 + \ldots + x_N^2} \\
    &= 2N - 4
  \end{flalign*}
  Die \textit{Laplace-Gleichung} ist erfüllt für $N = 2$.

\item Ermitteln Sie für die Funktion
  $f \colon \mathbb{R}^N \setminus \qty{0} \to \mathbb{R}, p \in \mathbb{Z}$,
  definiert durch
  \[
    f(x) = f(x_1, \ldots, x_N) \coloneqq \norm{x}_2^p = \qty(x_1^2 + \ldots + x_N^2)^{\frac{p}{2}}, p \in \mathbb{Z}
  \]
  die partiellen Ableitungen
  \colorbox{yellow}{$\frac{\partial}{\partial x_k}f(x)$} und
  \colorbox{orange!60}{$\frac{\partial^2}{\partial x_j\partial x_k} f(x)$}
  für $j, k = 1, \ldots, N$.
  Berechnen Sie $\Delta f(x)$.
  Was ergibt sich speziell für $p = 2 - N, N \geq 3$?
  \subparagraph{Lösung:}
  $g(x) = g\qty(x_1, \ldots, x_N) \coloneqq \qty(x_1^2 + \ldots + x_N^2)$,
  $h(x) \coloneqq x^{\frac{p}{2}}$,
  $f(x) = h(g(x))$
  \begin{flalign*}
    \colorbox{yellow}{$\frac{\partial}{\partial x_k}(x)$} &= h'(g(x)) \cdot g'(x) = h'(g(x)) \cdot g'(x) & \\
    &= \frac{1}{2}p\qty(x_1^2 + \ldots + x_N)^{\frac{p}{2} - 1} \cdot 2 x_k \\
    &= p x_k \qty(x_1^2 + \ldots + x_N)^{\frac{p}{2} - 1} = x_k \cdot p \cdot \norm{x}_2^{p - 1} \\
    \colorbox{orange!60}{$\frac{\partial}{\partial x_j \partial x_k}(x)$}
    &= (p - 2) \cdot p \cdot x_j \cdot x_k \cdot \qty(x_1^2 + \ldots x_N^2)^{\frac{p}{2} - 2} \\
    &= (p - 2) \cdot p \cdot x_j \cdot x_k \cdot \norm{x}_2^{p - 2} \\
    \frac{\partial^2}{\partial x_k}(x)
    &= p \cdot \qty(x_1^2 + \ldots x_N^2)^{\frac{p}{2} - 1} + p \cdot x_k^2
  \end{flalign*}
  \newpage
  \begin{flalign*}
    \Delta f(x) &= \sum_{k = 1}^N p \cdot \qty(x_1^2 + \ldots x_N^2)^{\frac{p}{2} - 1} + p \cdot x_k^2w & \\
    &= p \cdot \sum_{k = 1}^N \qty(x_1^2 + \ldots x_N^2)^{\frac{p}{2} - 1} + x_k^2 \\
    &= p \cdot \qty(x_1^2 + \ldots x_N^2)^{\frac{p}{2}}
  \end{flalign*}
  Sei nun $p = 2 - N, N \geq 3$.
  \begin{flalign*}
    \Delta f = - \frac{1}{N - 2 \cdot \qty(x_1^2 + \ldots x_N^2)^{\frac{N}{2} - 1}}
  \end{flalign*}
\end{enumerate}

\paragraph{Hausaufgabe 2} Gegeben sei die Funktion
$f \colon \mathbb{R}^2 \to \mathbb{R}$ mit
$f(x, y) = e^{x + y}(x - y)$
\begin{enumerate}[a)]
\item Ermitteln Sie das Taylor-Polynom 2. Ordnung von $f$ an der Stelle
  $(x_0, y_0) = (0, 0)$.

  \subparagraph{Lösung:} Die Funktion $f$ ist beliebig oft stetig partiell
  differenzierbar.
  Für $h = \qty(h_1, h_2) \in \mathbb{R}^2$ ist
  \begin{flalign*}
    \qty(h_1 \frac{\partial}{\partial x} + h_2 \frac{\partial}{\partial y})^0 f(x, y)
    &= f(x, y) & \\
    \qty(h_1 \frac{\partial}{\partial x} + h_2 \frac{\partial}{\partial y})^1 f(x, y)
    &= \skalprod{\nabla f(x, y), h} = e^{x + y}\qty\Big(h_1(x - y + 1) + h_2(x - y - 1)) \\
    \qty(h_1 \frac{\partial}{\partial x} + h_2 \frac{\partial}{\partial y})^2 f(x, y)
    &= e^{x + y} \cdot h^T \cdot \begin{pmatrix}
      x - y + 2 & x - y \\
      x - y & x - y - 2 \\
    \end{pmatrix} \cdot h \\
    \qty(h_1 \frac{\partial}{\partial x} + h_2 \frac{\partial}{\partial y})^3 f(x, y)
    &= \qty(h_1^3 \frac{\partial^3}{\partial x^3}
    + h_1^2h_2 \frac{\partial^2}{\partial x^2}\frac{\partial}{\partial y}
    + h_2^2h_1 \frac{\partial^2}{\partial y^2}\frac{\partial}{\partial x}
    + h_2^3 \frac{\partial^3}{\partial y^3}) f(x, y) \\
    = e^{x + y} \Big(
      h_1^3 (x - y + 3) +
      &h_1^2h_2 (x - y + 1) +
      h_1h_2^2 (x - y - 1) +
      h_2^3 (x - y - 3)
    \Big)
  \end{flalign*}
  \begin{flalign*}
    T_{x_0,2}^f(x)
    &= \sum_{k = 0}^N \qty(x_1 \frac{\partial}{\partial x_1} + x_2 \frac{\partial}{\partial x_2})^k f(0,0) & \\
    &= f(x_0, y_0) + f'(x_0, y_0)(x, y) + \frac{1}{2} \begin{pmatrix} x & y \end{pmatrix}H_f(x_0, y_0)\begin{pmatrix} x \\ y \end{pmatrix} \\
    &= 0 + x - y + \frac{2x^2 - 2y^2}{2} \\
    &= x^2 + x - (y^2 + y)
  \end{flalign*}

\newpage
\item Geben Sie eine Darstellung des zugehörigen Restgliedes an.
  \subparagraph{Lösung:} Lagrangesches Restglied:
  \begin{flalign*}
    R_{x_0, 2} &= \frac{f^{(3)}(\xi)}{6} (x - x_0)^3 & \\
    &= \frac{1}{6} e^{\xi_1 + \xi_2} \qty(
      x_1^3 (\xi_1 - \xi_2 + 3) +
      x_1^2h_2 (\xi_1 - \xi_2 + 1) +
      x_1h_2^2 (\xi_1 - \xi_2 - 1) +
      x_2^3 (\xi_1 - \xi_2  - 3)
    )
  \end{flalign*}
\end{enumerate}

\paragraph{Hausaufgabe 3} Untersuchen Sie die Funktion
$f(x, y) = e^{-(x^2 + y^2)}\qty(x^2 + 2y^2), x, y \in \mathbb{R}$ auf lokale
Extremal- und Sattelpunkte.

\subparagraph{Lösung:} $\nabla f(x, y) = \begin{pmatrix}
  -2x e^{-x^2 - y^2} (x^2 + 2 y^2  - 1) & -2y e^{-x^2 - y^2} (x^2 + 2y^2 - 2)
\end{pmatrix}$
Der Ausdruck $\nabla f(x, y) = 0$ gilt für $(x, y) = (0, 0), (0, \pm1), (\pm1, 0)$.
\begin{flalign*}
  H_f(x, y) &= \begin{pmatrix}
    \frac{\partial^2 f}{\partial x^2} f(x, y) & \frac{\partial^2 f}{\partial x \partial y} f(x, y) \\
    \frac{\partial^2 f}{\partial y \partial x} f(x, y) & \frac{\partial^2 f}{\partial y^2} f(x, y) \\
  \end{pmatrix} & \\
  &= e^{-x^2 - y^2} \cdot \begin{pmatrix}
    4 x^ 4 + 2 x^2 (4 y^2  - 5) - 4 y^2  + 2 & 4xy \qty\big(x^2 + 2y^2  - 3) \\
    4xy \qty\big(x^2 + 2y^2  - 3) & x^2 (4 y^2 - 2) + 8 y^4 - 20 y^2 + 4
  \end{pmatrix} \\
  H_f(0, 0) &= \begin{pmatrix}
    2 & 0 \\
    0 & 4
  \end{pmatrix}
  \quad 2 > 0 \land \det H_f(0, 0) > 0 \\
  & \Rightarrow H_f(0, 0) \text{ ist nach Aufgabe 3. (i) positiv definit}\\
  H_f(0, \pm1) &= \begin{pmatrix}
    -\frac{2}{e} & 0 \\
    0 & -\frac{8}{e}
  \end{pmatrix}
  \quad -\frac{2}{e} < 0 \land \det H_f(0, 0) > 0 \\
  &\Rightarrow H_f(0, \pm1) \text{ist nach Aufgabe 3. (ii) negativ definit} \\
  H_f(\pm1, 0) &= \begin{pmatrix}
    -\frac{4}{e} & 0 \\
    0 & \frac{2}{e}
  \end{pmatrix}
  \quad -\frac{4}{e} < 0 \land \det H_f(0, 0) < 0 \\
  &\Rightarrow H_f(\pm1, 0) \text{ist nach Aufgabe 3. (ii) indefinit}
\end{flalign*}

Somit besitzt $f$ an der Stelle $(0, 0)$ ein lokales Minimum, an den Stellen
$(0, \pm1)$ lokale Maxima und an den Stellen $(\pm1, 0)$ Sattelpunkte.

\end{document}