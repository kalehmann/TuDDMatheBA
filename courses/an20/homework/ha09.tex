\documentclass{scrreprt}

\usepackage{aligned-overset}
\usepackage{amsmath}
\usepackage{amssymb}
\usepackage{bm}
\usepackage[shortlabels]{enumitem}
\usepackage{hyperref}
\usepackage[utf8]{inputenc}
\usepackage{mathtools}
\usepackage{physics}
\usepackage{tabularx}
\usepackage{titling}
\usepackage{fancyhdr}
\usepackage{xfrac}
\usepackage{pgfplots}

\pgfplotsset{compat = newest}
\usepgfplotslibrary{fillbetween}

\author{Albina Oscherowa \\ Lukas Kamratzki \\ Karsten Lehmann}
\date{SoSe 2021}
\title{Hausaufgabe 09 \\Analysis - Weiterführende Konzepte}

\setlength{\headheight}{26pt}
\pagestyle{fancy}
\fancyhf{}
\lhead{\thetitle}
\rhead{\theauthor}
\lfoot{\thedate}
\rfoot{Seite \thepage}

\begin{document}
\paragraph{Hausaufgabe 1} Sei $M \subseteq \mathbb{R}^N$ eine nichtleere offene
Menge.
Der \textit{Laplace-Operator} $\Delta \colon C^2(M) \to C(M)$ ist definiert
durch
\[
  \Delta f(x) \coloneqq \qty(\frac{\partial^2f}{\partial x_1^2} + \ldots + \frac{\partial^2 f}{\partial x_N^2})(x)
  = \sum_{k = 1}^N \frac{\partial^2 f(x)}{\partial x_k^2} \: \text{für} \: f \in C^2(M)
\]
\begin{enumerate}[a)]
\item Berechnen Sie $\Delta f(x)$ für die Funktion
  $f(x) \coloneqq -\ln\qty(x_1^2 + \ldots + x_N^2), x \ne 0, N \in \mathbb{N}$.
  Für welches $N \in \mathbb{N}$ erfüllt $f$ die \textit{Laplace-Gleichung}
  $\Delta f(x) = 0$ auf der Menge $M = \mathbb{R}^N \setminus \qty{0}$?

  \subparagraph{Lösung:}
  $\frac{\partial^2 f}{\partial x_1^2} = 2 \cdot \frac{\qty(x_1^2 + \ldots + x_N^2) - 2 x_1}{x_1^2 + \ldots + x_N^2}$.
  \begin{flalign*}
    \Delta f(x) &= \sum_{k = 1}^N 2 \cdot \frac{\qty(x_1^2 + \ldots + x_N^2) - 4 \cdot x_k^2}{x_1^2 + \ldots + x_N^2} & \\
    &= 2N \cdot \frac{\qty(x_1^2 + \ldots + x_N^2)}{x_1^2 + \ldots + x_N^2} - 4 \sum_{k = 1}^N \frac{x_k^2}{x_1^2 + \ldots + x_N^2} \\
    &= 2N - 4\frac{x_1^2 + \ldots + x_N^2}{x_1^2 + \ldots + x_N^2} \\
    &= 2N - 4
  \end{flalign*}
  Die \textit{Laplace-Gleichung} ist erfüllt für $N = 2$.
\end{enumerate}

\end{document}