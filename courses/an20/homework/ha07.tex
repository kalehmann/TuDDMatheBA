\documentclass{scrreprt}

\usepackage{aligned-overset}
\usepackage{amsmath}
\usepackage{amssymb}
\usepackage{bm}
\usepackage[shortlabels]{enumitem}
\usepackage{hyperref}
\usepackage[utf8]{inputenc}
\usepackage{mathtools}
\usepackage{physics}
\usepackage{tabularx}
\usepackage{titling}
\usepackage{fancyhdr}
\usepackage{xfrac}
\usepackage{pgfplots}

\pgfplotsset{compat = newest}
\usepgfplotslibrary{fillbetween}

\author{}
\date{SoSe 2021}
\title{Hausaufgabe 07 \\Analysis - Weiterführende Konzepte}

\setlength{\headheight}{26pt}
\pagestyle{fancy}
\fancyhf{}
\lhead{\thetitle}
\rhead{\theauthor}
\lfoot{\thedate}
\rfoot{Seite \thepage}

\begin{document}

\paragraph{Hausaufgabe 2} Der Raum $\mathbb{R}^{N \times N}$ sei durch eine
Matrixnorm normiert.
Ermitteln Sie von $f \colon \mathbb{R}^{N \times N} \to \mathbb{R}^{N \times N}$
mit $f(X) \coloneqq X^2$ die Weierstraßsche Zerlegungsformel für ein festes
$X_0 \in \mathbb{R}^{N \times N}$.
Bestimmen Sie daraus die Ableitung $f'(X_0)$.

\subparagraph{Lsg.}

Es gilt
\begin{align*}
  f(x_0 + h) - f(x_0) &= (x_0 + h)^2 - x_0^2 = x_0^2 + 2x_0h + h^2 - x_0^2 \\
                      &= \underset{h \mapsto 2x_0h \text{ ist linear}}{\underbrace{2x_0h}} +
                        \underset{r(h)}{\underbrace{h^2}}
\end{align*}
$\Rightarrow f'(X_0)(h) = 2X_0h$

\end{document}
