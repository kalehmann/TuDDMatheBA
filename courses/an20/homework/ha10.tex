\documentclass{scrreprt}

\usepackage{aligned-overset}
\usepackage{amsmath}
\usepackage{amssymb}
\usepackage{bm}
\usepackage[shortlabels]{enumitem}
\usepackage{hyperref}
\usepackage[utf8]{inputenc}
\usepackage{multicol}
\usepackage{mathtools}
\usepackage{physics}
\usepackage{tabularx}
\usepackage{titling}
\usepackage{fancyhdr}
\usepackage{xfrac}
\usepackage{pgfplots}

\pgfplotsset{compat = newest}
\usepgfplotslibrary{fillbetween}

\author{Albina Oscherowa \\ Lukas Kamratzki \\ Karsten Lehmann}
\date{SoSe 2021}
\title{Hausaufgabe 10 \\Analysis - Weiterführende Konzepte}

\setlength{\headheight}{26pt}
\pagestyle{fancy}
\fancyhf{}
\lhead{\thetitle}
\rhead{\theauthor}
\lfoot{\thedate}
\rfoot{Seite \thepage}

\newcommand\skalprod[1]{\left\langle #1 \right\rangle}

\begin{document}
\paragraph{Hausaufgabe 1} Wir betrachten die folgenden Teilmengen
$M \subseteq \mathbb{R}$, die mit der von $\qty(\mathbb{R}, \abs{\cdot})$
induzierten Topologie versehen seien, und die Funktionen
$\varphi \colon M \to \mathbb{R}$ definiert durch
\begin{multicols}{2}
  \begin{enumerate}[(1)]
  \item $M = [1, \infty), \varphi(x) = x + \frac{1}{x}$
  \item $M = (0, 1], \varphi(x) = \frac{1}{3}\qty\big(x^2 + 2)$
  \item $M = [0, 2], \varphi(x) = \frac{1}{3}\qty\big(4 - x^2)$
  \item $M = [3, 5], \varphi(x) = \arctan x + 3$
  \end{enumerate}
\end{multicols}
Untersuchen Sie für jeden der Fälle, ob
\begin{enumerate}[a)]
\item die Funktion $\varphi$ kontrahierend ist

  \subparagraph{Lösung:}
  \begin{enumerate}[(1)]
  \label{sec:1_a_1}
  \item $\forall x, y \in [1, \infty) \colon \abs{x + \frac{1}{x} - y - \frac{1}{y}}
    = \abs{x - y + \frac{y - x}{x \cdot y}}
    = \abs{\frac{\qty\big(xy - 1)}{xy} \qty\big(x - y)}
    = \abs{\frac{\qty\big(xy - 1)}{xy}} \abs{x - y}$
    $\lim_{x \to \infty}\frac{xy - 1}{xy} = 1 \Rightarrow$ die Abbildung ist nicht kontrahierend.

  \item $\forall x, y \in (0, 1] \colon \abs{\varphi(x) - \varphi(y)}
    = \abs{\frac{1}{3}\qty\big(x^2 + 2) - \frac{1}{3}\qty\big(y^2 + 2)}
    = \frac{1}{3}\abs{x^2 - y^2}
    \leq \frac{\abs{x + y}}{3}\abs{x - y}$
    $\overset{x = 1, y = 1} \leq \frac{2}{3}\abs{x - y}
    \Rightarrow$ die Abbildung ist kontrahierend mit
    $q = \frac{2}{3}$

  \label{sec:1_a_3}
  \item $\forall x, y \in (0, 1] \colon \abs{\varphi(x) - \varphi(y)}
    = \abs{\frac{1}{3}\qty\big(4 - x^2) - \frac{1}{3}\qty\big(4 - y^2)}
    = \frac{1}{3}\abs{y^2 - x^2}
    \leq \frac{\abs{x + y}}{3}\abs{x - y}$
    $\overset{x = 1, y = 1} \leq \frac{2}{3}\abs{x - y}
    \Rightarrow$ die Abbildung ist kontrahierend mit
    $q = \frac{2}{3}$

  \label{sec:1_a_4}
  \item $\forall x, y \in (0, 1] \colon \abs{\varphi(x) - \varphi(y)}
    = \abs{\arctan x + 3 - \arctan y + 3} = \abs{\arctan x - \arctan y}$
    $\Rightarrow$ die Abbildung ist nicht kontrahierend.
  \end{enumerate}

\item $\varphi$ eine Selbstabbildung von $M$ ist, d.h. $\varphi(M) \subseteq M$ gilt

  \subparagraph{Lösung:}
  \begin{enumerate}[(1)]
  \item Sei $x \in M$, dann ist
    $\varphi(x) = \underset{\in M}{\underbrace{x}} + \underset{> 0}{\underbrace{\frac{1}{x}}}$
    ebenfalls in $M$ \\
    $\Rightarrow \varphi$ ist eine Selbstabbildung.

  \item Wegen $\varphi'(x) = \frac{2}{3}x > 0$ für $x > 0$ ist $\varphi$ monoton wachsend.
    $\varphi(x) \overset{x \to 0}= \frac{2}{3} \in M$ und $\varphi(1) = 1 \in M$
    $\Rightarrow \varphi$ ist eine Selbstabbildung.

  \label{sec:1_b_3}
  \item Wegen $\varphi'(x) = -\frac{2}{3}x < 0$ für $x > 0$ ist $\varphi$ monoton fallend.
    $\varphi(0) = \frac{4}{3} \in M$ und $\varphi(2) = 0 \in M$
    $\Rightarrow \varphi$ ist eine Selbstabbildung.

  \item $0 \leq \arctan M \leq \frac{\pi}{2} \Rightarrow 3 \leq \varphi(M) \leq \frac{6 + \pi}{2}$
    $\Rightarrow \varphi(M) \subseteq M$
  \end{enumerate}

\newpage
\item alle Voraussetzungen des Banachschen Fixpunktsatzes erfüllt sind

  \subparagraph{Lösung:} Für den Fixpunktsatz wird vorausgesetzt, dass
  \begin{itemize}
  \item $\qty\big(M, d)$ ein vollständiger, metrischer, nichtleerer Raum ist
  \item $\varphi$ eine Selbstabbildung ist
  \item $\varphi$ kontrahierend ist
  \end{itemize}
  \begin{enumerate}[(1)]
  \item $\varphi$ kontrahiert wie in \hyperref[sec:1_a_1]{1 a) (1)} gezeigt nicht,
    somit sind die Voraussetzungen für den Banchschen Fixpunktsatz nicht erfüllt.

  \item $M$ ist nicht vollständig, da für $\qty\big(a_n) \coloneqq \frac{1}{n}$
    gilt $\lim_{n \to \infty} a_n = 0 \notin M$

    $\Rightarrow$ die Voraussetzungen für den Banachschen Fixpunktsatz sind nicht erfüllt.

  \item $\qty\big(M, d)$ ist als abgeschlossene Teilmenge des vollständigen Raumes
    $\qty\big(\mathbb{R}, d)$ ebenfalls abgeschlossen.z
    Weiterhin ist $\varphi$ nach \hyperref[sec:1_a_3]{a)} kontrahierend und nach
    \hyperref[sec:1_b_3]{b)} eine Selbstabbildung.

    $\Rightarrow$ die Voraussetzungen für den Banachschen Fixpunktsatz sind erfüllt.

  \item $\varphi$ ist nach \hyperref[sec:1_a_4]{1 a) (4)} nicht kontrahierend,
    somit sind die Voraussetzungen für den Banchschen Fixpunktsatz nicht erfüllt.
  \end{enumerate}

\item $\varphi$ einen Fixpunkt in $M$ besitzt.
  Berechnen Sie diesen gegebenenfalls.

  \subparagraph{Lösung:}

  \begin{multicols}{2}
    \begin{enumerate}[(1)]
    \item Lösung durch Gleichsetzen: $x = \varphi(x)$
      \begin{flalign*}
        x &= x + \frac{1}{x} & \\
        x^2 &= x^2 + 1 \\
        0 &= 1
      \end{flalign*}
      $\Rightarrow \varphi$ besitzt keinen Fixpunkt.

    \item Lösung durch Gleichsetzen: $x = \varphi(x)$
      \begin{flalign*}
        x &=  \frac{1}{3}\qty\big(x^2 + 2) & \\
        3x &= x^2 + 2 \\
        x_1 &= 2 \notin M, \: x_2 = 1 \in M
      \end{flalign*}
      $\Rightarrow \varphi$ besitzt mit $x = 1$ einen Fixpunkt in $M$

    \item Lösung durch Gleichsetzen: $x = \varphi(x)$
      \begin{flalign*}
        x &= \frac{1}{3}\qty\big(4 - x^2) & \\
        3x &= 4 - x^2 \\
        x_1 &= 1 \in M, \; x_2 = -4 \notin M
      \end{flalign*}
      $\Rightarrow \varphi$ besitzt mit $x = 1$ einen Fixpunkt in $M$

    \item Lösung durch Gleichsetzen: $x = \varphi(x)$
      \\\\\\\\\\\\\\\\
    \end{enumerate}
  \end{multicols}
\end{enumerate}

\newpage
\paragraph{Aufgabe 2} Gegeben seien ein vollständiger metrischer Raum
$\qty\big(M, d)$ und eine Abbildung $\varphi \colon M \to M$, so dass
$\varphi^n$ für ein $n \in \mathbb{N}$ kontrahierend ist.
Beweisen Sie, dass $\varphi$ genau einen Fixpunkt $x^* \in M$ besitzt.

\subparagraph{Lösung:} Sei $n \in \mathbb{N}$ so gewählt, dass $\varphi^i$
kontrahiert.
Unter der Voraussetzung, dass $M$ nichtleer ist, existiert nach dem Banachschen
Fixpunktsatz genau ein $x^* \in M$ mit $\varphi^n\qty\big(x^*) = x^*$.
\begin{flalign*}
  \varphi\qty\big(x^*) \overset{x^* \varphi^n\qty\big(x^*)}&= \varphi\qty\big(\varphi^n\qty\big(x^*)) & \\
  & = \varphi^n\qty\big(\varphi\qty\big(x^*))
\end{flalign*}
$\Rightarrow \varphi\qty\big(x^*)$ ist ein Fixpunkt von $\varphi^n$.
Da $\varphi^n$ eine nach Voraussetzung eine Kontraktion ist und $M$
ein nichtleerer ($x^* \in M$) vollständiger metrischer Raum ist, besitzt
$\varphi^n$ \textbf{genau einen} Fixpunkt.
\begin{flalign*}
  &\Rightarrow x^* = \varphi\qty\big(x^*) & \\
  &\Rightarrow x^* \text{ ist ein Fixpunkt von } \varphi
\end{flalign*}
Sei nun $x_2 \ne x^*$ ein weiterer Fixpunkt von $\varphi$. Dann ist
\[
  x_2 = \varphi\qty\big(x_2) = \varphi\qty\big(\varphi\qty\big(x_2))
  = \ldots = \varphi^n\qty\big(x^2),
\]
ein Widerspruch zum Banachschen Fixpunktsatz.
$\Rightarrow \varphi$ besitzt genau einen Fixpunkt.
\end{document}