\documentclass{scrreprt}

\usepackage{aligned-overset}
\usepackage{amsmath}
\usepackage{amssymb}
\usepackage{bm}
\usepackage[shortlabels]{enumitem}
\usepackage{hyperref}
\usepackage[utf8]{inputenc}
\usepackage{multicol}
\usepackage{mathtools}
\usepackage{physics}
\usepackage{tabularx}
\usepackage{titling}
\usepackage{fancyhdr}
\usepackage{xfrac}
\usepackage{pgfplots}

\pgfplotsset{compat = newest}
\usepgfplotslibrary{fillbetween}

\author{Albina Oscherowa \\ Lukas Kamratzki \\ Karsten Lehmann}
\date{SoSe 2021}
\title{Hausaufgabe 10 \\Analysis - Weiterführende Konzepte}

\setlength{\headheight}{26pt}
\pagestyle{fancy}
\fancyhf{}
\lhead{\thetitle}
\rhead{\theauthor}
\lfoot{\thedate}
\rfoot{Seite \thepage}

\newcommand\skalprod[1]{\left\langle #1 \right\rangle}

\begin{document}
\paragraph{Hausaufgabe 1} Wir betrachten die folgenden Teilmengen
$M \subseteq \mathbb{R}$, die mit der von $\qty(\mathbb{R}, \abs{\cdot})$
induzierten Topologie versehen seien, und die Funktionen
$\varphi \colon M \to \mathbb{R}$ definiert durch
\begin{multicols}{2}
  \begin{enumerate}[(1)]
  \item $M = [1, \infty), \varphi(x) = x + \frac{1}{x}$
  \item $M = (0, 1], \varphi(x) = \frac{1}{3}\qty\big(x^2 + 2)$
  \item $M = [0, 2], \varphi(x) = \frac{1}{3}\qty\big(4 - x^2)$
  \item $M = [3, 5], \varphi(x) = \arctan x + 3$
  \end{enumerate}
\end{multicols}
Untersuchen Sie für jeden der Fälle, ob
\begin{enumerate}[a)]
\item die Funktion $\varphi$ kontrahierend ist

  \subparagraph{Lösung:}
  \begin{enumerate}[(1)]
  \item $\forall x, y \in [1, \infty) \colon \abs{x + \frac{1}{x} - y - \frac{1}{y}}
    = \abs{x - y + \frac{y - x}{x \cdot y}}
    = \abs{\frac{\qty\big(xy - 1)}{xy} \qty\big(x - y)}
    = \abs{\frac{\qty\big(xy - 1)}{xy}} \abs{x - y}$
    $\lim_{x \to \infty}\frac{xy - 1}{xy} = 1 \Rightarrow$ die Abbildung ist nicht kontrahierend.
  \item $\forall x, y \in (0, 1] \colon \abs{\varphi(x) - \varphi(y)}
    = \abs{\frac{1}{3}\qty\big(x^2 + 2) - \frac{1}{3}\qty\big(y^2 + 2)}
    = \frac{1}{3}\abs{x^2 - y^2}
    \leq \frac{\abs{x + y}}{3}\abs{x - y}$
    $\overset{x = 1, y = 1} \leq \frac{2}{3}\abs{x - y}
    \Rightarrow$ die Abbildung ist kontrahierend mit
    $q = \frac{2}{3}$
  \item $\forall x, y \in (0, 1] \colon \abs{\varphi(x) - \varphi(y)}
    = \abs{\frac{1}{3}\qty\big(4 - x^2) - \frac{1}{3}\qty\big(4 - y^2)}
    = \frac{1}{3}\abs{y^2 - x^2}
    \leq \frac{\abs{x + y}}{3}\abs{x - y}$
    $\overset{x = 1, y = 1} \leq \frac{2}{3}\abs{x - y}
    \Rightarrow$ die Abbildung ist kontrahierend mit
    $q = \frac{2}{3}$
  \item $\forall x, y \in (0, 1] \colon \abs{\varphi(x) - \varphi(y)}
    = \abs{\arctan x + 3 - \arctan y + 3} = \abs{\arctan x - \arctan y}$
    $\Rightarrow$ die Abbildung ist nicht kontrahierend.
  \end{enumerate}
\end{enumerate}
\end{document}