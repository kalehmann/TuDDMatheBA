\documentclass{scrreprt}

\usepackage{aligned-overset}
\usepackage{amsmath}
\usepackage{amssymb}
\usepackage{bm}
\usepackage[shortlabels]{enumitem}
\usepackage{hyperref}
\usepackage[utf8]{inputenc}
\usepackage{multicol}
\usepackage{mathtools}
\usepackage{physics}
\usepackage{tabularx}
\usepackage{titling}
\usepackage{fancyhdr}
\usepackage{xfrac}
\usepackage{pgfplots}

\pgfplotsset{compat = newest}
\usepgfplotslibrary{fillbetween}

\author{Albina Oscherowa \\ Lukas Kamratzki \\ Karsten Lehmann}
\date{SoSe 2021}
\title{Hausaufgabe 11 \\Analysis - Weiterführende Konzepte}

\setlength{\headheight}{26pt}
\pagestyle{fancy}
\fancyhf{}
\lhead{\thetitle}
\rhead{\theauthor}
\lfoot{\thedate}
\rfoot{Seite \thepage}

\begin{document}
\paragraph{Hausaufgabe 1} Sei $F \colon \mathbb{R}^3 \to \mathbb{R}$
definiert durch $F(x, y, z) = x^4z - 2xy^3 + yz^3 - 8$.
\begin{enumerate}[a)]
\item Zeigen Sie, dass es eine Umgebung
  $\tilde{U} \times \tilde{V} \subseteq \mathbb{R}^2 \times \mathbb{R}$
  des Punktes $(\overline{x}, \overline{y}, \overline{z}) = (1, 1, 2)$
  und eine $C^1$-Funktion
  $f \colon \tilde{U} \subseteq \mathbb{R}^2 \to \tilde{V} \subseteq \mathbb{R}$
  gibt mit $F(x, y, f(x, y)) = 0$ für alle $(x, y) \in \tilde{U}$.

  \subparagraph{Lösung:}
  \begin{flalign*}
    J_F &= \qty\Big(\begin{array}{>{\color{red}} c >{\color{red}} c >{\color{blue}} c}
      4x^3z - 2y^3 & 6y^2 + z^3 & x^4 + 3yz^2
    \end{array}) &
  \end{flalign*}
  Dabei teilt sich die Jakobimatrix auf in
  \colorbox{red!40}{$J_F^{(x,y)} (\ldots)$} und
  \colorbox{blue!40}{$J_F^{(z)} (\ldots)$}.
  Weiterhin ist die Ableitung $(x, y, z) \mapsto J_F(x, y, z)$ stetig.

  $\Rightarrow F$ ist eine $C^1$-Funktion.

  Wegen
  $\det J_F^{(z)}(\overline{x}, \overline{y}, \overline{z}) =
  \overline{x}^4 + 3\overline{y} \cdot \overline{z}^2 = 13$ und
  $(x, y, z) \mapsto J_F^{(z)}(x, y, z)$ ist stetig, folgt die
  Existenz einer offenen Umgebung $\overline{W}$ von
  $(\overline{x}, \overline{y}, \overline{z})$ mit
  $\det J_F^{(z)} > 0$ für alle $x, y, z \in \overline{W}$.

  $\Rightarrow J_F^{(z)}$ ist auf $\overline{W}$ invertierbar.

  $\Rightarrow$ nach dem Satz über die lokale Auflösbarkeit existieren
  offene Umgebungen $\tilde{U} \in \mathbb{R}^2$ und $\tilde{V} \in \mathbb{R}$,
  sowie eine $C^1$-Funktion $f \colon \tilde{U} \to \tilde{V}$
  mit der Eigenschaft
  \[
    \forall (x, y, z) \in \tilde{U} \times \tilde{V} \colon
    F(x, y, z) = F(\overline{x}, \overline{y}, \overline{z}) = 0
    \iff z = f(x, y)
  \]

\item Geben Sie eine Formel zur Berechnung $f'$ bzw. $J_f$ bzw.
  $\nabla f(x, y)$ auf $\tilde{U}$ an.

  \subparagraph{Lösung:} Nach dem Satz über die lokale Auflösbarkeit ist
  \begin{flalign*}
    J_f(x, y) &= - \qty\Big(J_F^{(z)}(x, y, f(x, y)))^{-1} \cdot J_F^{(x, y)}(x, y, f(x, y)) & \\
    &= - \qty\Big(x^4 + 3y\qty\big(f(x))^2)^{-1} \cdot \begin{pmatrix}
      4x^3f(x) - 2 y^3 & 6y^2 + \qty\big(f(x))^3
    \end{pmatrix} \\
    &= - \frac{1}{x^4 + 3y\qty\big(f(x))^2} \cdot \begin{pmatrix}
      4x^3f(x) - 2 y^3 & 6y^2 + \qty\big(f(x))^3
    \end{pmatrix} \\
    &= \begin{pmatrix}
      \frac{4x^3f(x) - 2 y^3}{x^4 + 3y\qty\big(f(x))^2} &
      \frac{6y^2 + \qty\big(f(x))^3}{x^4 + 3y\qty\big(f(x))^2}
    \end{pmatrix} \\
  \end{flalign*}

\item Berechnen Sie $\nabla f(1, 1)$.
\end{enumerate}

\end{document}