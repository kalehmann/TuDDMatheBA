\documentclass{scrreprt}

\usepackage{aligned-overset}
\usepackage{amsmath}
\usepackage{amssymb}
\usepackage{bm}
\usepackage[shortlabels]{enumitem}
\usepackage{hyperref}
\usepackage[utf8]{inputenc}
\usepackage{multicol}
\usepackage{mathtools}
\usepackage{physics}
\usepackage{tabularx}
\usepackage{titling}
\usepackage{fancyhdr}
\usepackage{xfrac}
\usepackage{pgfplots}

\pgfplotsset{compat = newest}
\usepgfplotslibrary{fillbetween}

\author{Albina Oscherowa \\ Lukas Kamratzki \\ Karsten Lehmann}
\date{SoSe 2021}
\title{Hausaufgabe 11 \\Analysis - Weiterführende Konzepte}

\setlength{\headheight}{26pt}
\pagestyle{fancy}
\fancyhf{}
\lhead{\thetitle}
\rhead{\theauthor}
\lfoot{\thedate}
\rfoot{Seite \thepage}

\begin{document}
\paragraph{Hausaufgabe 1} Sei $F \colon \mathbb{R}^3 \to \mathbb{R}$
definiert durch $F(x, y, z) = x^4z - 2xy^3 + yz^3 - 8$.
\begin{enumerate}[a)]
\item Zeigen Sie, dass es eine Umgebung
  $\tilde{U} \times \tilde{V} \subseteq \mathbb{R}^2 \times \mathbb{R}$
  des Punktes $(\overline{x}, \overline{y}, \overline{z}) = (1, 1, 2)$
  und eine $C^1$-Funktion
  $f \colon \tilde{U} \subseteq \mathbb{R}^2 \to \tilde{V} \subseteq \mathbb{R}$
  gibt mit $F(x, y, f(x, y)) = 0$ für alle $(x, y) \in \tilde{U}$.

  \subparagraph{Lösung:}
  \begin{flalign*}
    J_F &= \qty\Big(\begin{array}{>{\color{red}} c >{\color{red}} c >{\color{blue}} c}
      4x^3z - 2y^3 & 6y^2 + z^3 & x^4 + 3yz^2
    \end{array}) &
  \end{flalign*}
  Dabei teilt sich die Jakobimatrix auf in
  \colorbox{red!40}{$J_F^{(x,y)} (\ldots)$} und
  \colorbox{blue!40}{$J_F^{(z)} (\ldots)$}.
  Weiterhin ist die Ableitung $(x, y, z) \mapsto J_F(x, y, z)$ stetig.

  $\Rightarrow F$ ist eine $C^1$-Funktion.

  Wegen
  $\det J_F^{(z)}(\overline{x}, \overline{y}, \overline{z}) =
  \overline{x}^4 + 3\overline{y} \cdot \overline{z}^2 = 13$ und
  $(x, y, z) \mapsto J_F^{(z)}(x, y, z)$ ist stetig, folgt die
  Existenz einer offenen Umgebung $\overline{W}$ von
  $(\overline{x}, \overline{y}, \overline{z})$ mit
  $\det J_F^{(z)} > 0$ für alle $x, y, z \in \overline{W}$.

  $\Rightarrow J_F^{(z)}$ ist auf $\overline{W}$ invertierbar.

  $\Rightarrow$ nach dem Satz über die lokale Auflösbarkeit existieren
  offene Umgebungen $\tilde{U} \in \mathbb{R}^2$ und $\tilde{V} \in \mathbb{R}$,
  sowie eine $C^1$-Funktion $f \colon \tilde{U} \to \tilde{V}$
  mit der Eigenschaft
  \[
    \forall (x, y, z) \in \tilde{U} \times \tilde{V} \colon
    F(x, y, z) = F(\overline{x}, \overline{y}, \overline{z}) = 0
    \iff z = f(x, y)
  \]

\item Geben Sie eine Formel zur Berechnung $f'$ bzw. $J_f$ bzw.
  $\nabla f(x, y)$ auf $\tilde{U}$ an.

  \subparagraph{Lösung:} Nach dem Satz über die lokale Auflösbarkeit ist
  \begin{flalign*}
    J_f(x, y) &= - \qty\Big(J_F^{(z)}(x, y, f(x, y)))^{-1} \cdot J_F^{(x, y)}(x, y, f(x, y)) & \\
    &= - \qty\Big(x^4 + 3y\qty\big(f(x))^2)^{-1} \cdot \begin{pmatrix}
      4x^3f(x) - 2 y^3 & 6y^2 + \qty\big(f(x))^3
    \end{pmatrix} \\
    &= - \frac{1}{x^4 + 3y\qty\big(f(x))^2} \cdot \begin{pmatrix}
      4x^3f(x) - 2 y^3 & 6y^2 + \qty\big(f(x))^3
    \end{pmatrix} \\
    &= \begin{pmatrix}
      \frac{4x^3f(x) - 2 y^3}{x^4 + 3y\qty\big(f(x))^2} &
      \frac{6y^2 + \qty\big(f(x))^3}{x^4 + 3y\qty\big(f(x))^2}
    \end{pmatrix} \\
  \end{flalign*}

\newpage
\item Berechnen Sie $\nabla f(1, 1)$.

  \subparagraph{Lösung:} Aus $f(\overline{x}, \overline{y}) = \overline{z}$ folgt
  \begin{flalign*}
    \nabla f(1, 1) &= - \qty\Big(J_F^{(z)}(1, 1, 2))^{-1} \cdot J_F^{(x, y)}(1, 1, 2) & \\
    &= - \qty\big(13)^{-1} \cdot \begin{pmatrix}
      6 & 14
    \end{pmatrix} \\
    &= \begin{pmatrix}
      -\frac{6}{13} & -\frac{14}{13}
    \end{pmatrix}
  \end{flalign*}
\end{enumerate}

\paragraph{Hausaufgabe 2:} Gegeben sei ein Vektor
$a = \qty(a_0, a_1, \ldots, a_N) \in \mathbb{R}^{N + 1}$ mit $a_N \ne 0$.
$F \colon \mathbb{R}^{N + 1} \times \mathbb{R} \to \mathbb{R}$ sei definiert durch
\[
  F(a, x) = F(a_0, a_1, \ldots, a_N, x) \coloneqq
  a_Nx^N + \ldots + a_1x + a_0 = \sum_{k = 0}^N a_kx^k, x \in \mathbb{R}
\]
Für ein $\overset{a} \in \mathbb{R}^{N + 1}$ sei $\overline{x} \in \mathbb{R}$
eine einfache Nullstelle des Polynoms $F(\overline{a}, x)$.
Beweisen Sie: Es existiert eine Umgebung $U \subseteq \mathbb{R}^{N + 1}$ von
$\overline{a}$ und eine stetig differenzierbare Funktion
$g \colon U \to \mathbb{R}$ mit $g(\overline{a}) = \overline{x}$ und
$F(a, g(a)) = 0$ für alle $a \in U$.

\subparagraph{Lösung:}
\begin{flalign*}
  J_F &= \begin{pmatrix}
    {\color{red} 1} &
    {\color{red} x} &
    {\color{red} \ldots} &
    {\color{red} x^N} &
    {\color{blue} \sum_{k = 1}^N k \cdot a_kx^{k - 1}}
  \end{pmatrix} &
\end{flalign*}
Die Jacobimatrix teilt sich auf in
\colorbox{red!40}{$J_F^{(a)} (a, x)$} und
\colorbox{blue!40}{$J_F^{(x)} (a, x)$}
(eine endliche Summe stetiger Funktionen),
die Ableitung $(a, x) \mapsto J_F(a, x)$ ist stetig und
$F$ ist somit eine $C^1$-Funktion.

\noindent
Weiterhin ist $\det J_F^{(x)}(\overline{a}, \overline{x}) = \sum_{k = 1}^N k \cdot \overline{a}_k\overline{x}^{k - 1}
\overset{\overline{x} \text{ ist einfache Nullstelle}}\ne 0$.

\noindent
$\Rightarrow \exists \qty(J_F^{(x)}(\overline{a}, \overline{x}))^{-1}$

\noindent
$\Rightarrow$ es existiert einen offene Umgebung von
$(\overline{a}, \overline{x})$ mit $\det J_F^{(x)} (a, x) \ne 0$
für jedes $a, x$ in dieser Umgebung.

\noindent
$\Rightarrow$ nach dem Satz über die lokale Auflösbarkeit existieren
offene Umgebungen $U$ von $\overline{a}$, $V$ von $\overline{x}$ und eine
$C^1$-Funktion $g \colon U \to V$ mit den Eigenschaften
\begin{itemize}
\item $g(\overline{a}) = \overline{x}$
\item $\forall (a, x) \in U \times V \colon F(a, f(a)) = F(\overline{a}, \overline{x}) = 0$
\end{itemize}

\newpage
\paragraph{Hausaufgabe 3:} Seien
$U \coloneqq B(0, 1) = \qty{ x \in \mathbb{R}^N {\Big |} \norm{x} < 1 } \subseteq \mathbb{R}^N$
und $V \coloneqq \mathbb{R}^N$.
Welche Beziehung besteht zwischen $f$ und $g$?
Zeigen Sie, dass die Abbildungen $f \colon U \to V$ und $g \colon V \to U$ mit
\[
  f(x) \coloneqq \frac{x}{\sqrt{1 - \norm{x}^2}}
  \text{ und }
  g(y) \coloneqq \frac{y}{\sqrt{1 + \norm{y}^2}}
\]
$C^1$-Diffeomorphismen sind.
Geben Sie die zugehörigen Jacobi-Matrizen an.

\subparagraph{Lösung:} Sei $\norm{\cdot}$ die euklidische Norm.
$\frac{\partial}{\partial x}\norm{x} = \frac{x}{\norm{x}}$,
$\frac{\partial}{\partial x}\norm{x}^2 = 2\norm{x} \cdot \frac{x}{\norm{x}} = 2x$.
\[
  \frac{\partial}{\partial x} \frac{1}{\sqrt{1 - \norm{x}^2}}
  = -\frac{1}{2 \qty(1 - \norm{x}^2)^{\frac{3}{2}}} \cdot (-2x)
  = \frac{x}{\qty(1 - \norm{x}^2)^{\frac{3}{2}}}
\]

\begin{minipage}[t]{.4\textwidth}
  \begin{flalign*}
    J_f(x) &= \qty\Big( \frac{1}{\sqrt{1 - \norm{x}^2}} + \frac{x}{\qty(1 - \norm{x}^2)^{\frac{3}{2}}} \cdot x )& \\
    &= \frac{1 - \norm{x}^2 + x^2}{\qty(1 - \norm{x}^2)^{\frac{3}{2}}}
  \end{flalign*}
\end{minipage}
\begin{minipage}[t]{.4\textwidth}
  \begin{flalign*}
    J_g(y) &= \qty\Big( \frac{1}{\sqrt{1 + \norm{y}^2}} - \frac{y}{\qty(1 + \norm{y}^2)^{\frac{3}{2}}} \cdot y )& \\
    &= \frac{1 + \norm{y}^2 - y^2}{\qty(1 + \norm{y}^2)^{\frac{3}{2}}}
  \end{flalign*}
\end{minipage}
Die Ableitungen $(x) \mapsto J_f(x), y \mapsto J_g(y)$
sind für $x \in U, y \in V$ stetig.

\noindent
$\Rightarrow f, g$ sind $C^1$-Funktionen.

\begin{minipage}[t]{.4\textwidth}
  \begin{flalign*}
    g(f(x)) &= \frac{\frac{x}{\sqrt{1 - \norm{x}^2}}}{\sqrt{1 + \norm{\frac{x}{\sqrt{1 - \norm{x}^2}}}^2}} & \\
    &= \frac{\frac{x}{\sqrt{1 - \norm{x}^2}}}{\sqrt{1 + \frac{\norm{x}^2}{1 - \norm{x}^2}}} \\
    &= \frac{\frac{x}{\sqrt{1 - \norm{x}^2}}}{\sqrt{\frac{1 - \norm{x}^2 + \norm{x}^2}{1 - \norm{x}^2}}} \\
    &= x
  \end{flalign*}
\end{minipage}
\begin{minipage}[t]{.4\textwidth}
  \begin{flalign*}
    f(g(y)) &= \frac{\frac{y}{\sqrt{1 + \norm{y}^2}}}{\sqrt{1 - \norm{\frac{y}{\sqrt{1 + \norm{y}^2}}}^2}} & \\
    &= \frac{\frac{y}{\sqrt{1 + \norm{y}^2}}}{\sqrt{1 - \frac{\norm{y}^2}{1 + \norm{y}^2}}} \\
    &= \frac{\frac{x}{\sqrt{1 + \norm{y}^2}}}{\sqrt{\frac{1 + \norm{y}^2 - \norm{y}^2}{1 + \norm{y}^2}}} \\
    &= y
  \end{flalign*}
\end{minipage}

\noindent
$\Rightarrow f^{-1} = g \land g^{-1} = f$

\noindent
Für jedes $x \in U$ existiert genau ein $y \in V$ mit $f(x) = y$ und $g(y) = x$.

\noindent
$\Rightarrow f$ und $g$ sind bijektiv.

\noindent
$\Rightarrow f, g$ sind $C^1$-Diffeomorphismen.

\end{document}