\documentclass{scrreprt}

\usepackage{aligned-overset}
\usepackage{amsmath}
\usepackage{amsthm}
\usepackage{amssymb}
\usepackage{bm}
\usepackage[shortlabels]{enumitem}
\usepackage{hyperref}
\usepackage[utf8]{inputenc}
\usepackage{listings}
\usepackage{mathtools}
\usepackage{multicol}
\usepackage{physics}
\usepackage{titling}
\usepackage{fancyhdr}

\author{Karsten Lehmann}
\date{SoSe 2022}
\title{Übungsblatt 11\\Programmierung}

\setlength{\headheight}{26pt}
\pagestyle{fancy}
\fancyhf{}
\lhead{\thetitle}
\rhead{\theauthor}
\lfoot{\thedate}
\rfoot{Seite \thepage}

\begin{document}
\paragraph{Aufgabe 1}
\begin{enumerate}[(a)]
\item Gegeben sei folgender $\text{AM}_1$-Code:
  \begin{footnotesize}
  \begin{lstlisting}[numbers=left, multicols=3]
INIT  1;
CALL  12;
JMP   0;
INIT  2;
LOADI (-2);
STORE (global, 1);
LIT   3;
STORE (lokal, 1);
LOAD  (global, 1);
STORE (lokal, 2);
RET   1;
INIT  1;
READ  (lokal, 1);
LOADA (lokal, 1);
PUSH;
CALL  4;
WRITE (global, 1);
RET   0;
  \end{lstlisting}
  \end{footnotesize}
  Dokumentieren Sie 12 Schritte der $\text{AM}_1$ mit der Startkonfiguration
  \[
    \sigma = \qty\big(
      13, \epsilon, 0 \colon 3 \colon 0 \colon 0 \colon 0, 3, 8, \epsilon
    )
  \]

  \subparagraph{Lsg.} In der folgenden Tabelle steht \textbf{BZ} für
  Befehlszähler, \textbf{DK} für Datenkeller, \textbf{LK} für Laufzeitkeller,
  \textbf{REF} für Referenzzeiger, \textbf{In} für Eingabeband und \textbf{Out}
  für das Ausgabeband. \\

  \begin{tabular}{|c|c|c|c|c|c|}
    \hline
    BZ & DK & LK & REF & In & Out \\
    \hline
    12 & $\epsilon$ & $(0 \colon 0) \colon (3 \colon 0 \colon 0 \colon 8)$ & 4 & 6 & $\epsilon$ \\
    \hline
    13 & $\epsilon$ & $(0 \colon 0) \colon (3 \colon 0 \colon 6 \colon 8)$ & 4 & $\epsilon$ & $\epsilon$ \\
    \hline
    14 & $8$ & $(0 \colon 0) \colon (3 \colon 0 \colon 6 \colon 8)$ & 4 & $\epsilon$ & $\epsilon$ \\
    \hline
    15 & $\epsilon$ & $(0 \colon 0) \colon (3 \colon 0 \colon 6 \colon 8) \colon (8$ & 4 & $\epsilon$ & $\epsilon$ \\
    \hline
    16 & $5$ & $(0 \colon 0) \colon (3 \colon 0 \colon 6 \colon 8) \colon (8$ & 4 & $\epsilon$ & $\epsilon$ \\
    \hline
    16 & $\epsilon$ & $(0 \colon 0) \colon (3 \colon 0 \colon 6 \colon 8) \colon (8 \colon 5$ & 4 & $\epsilon$ & $\epsilon$ \\
    \hline
  \end{tabular}

\end{enumerate}
\end{document}
