\documentclass{scrreprt}

\usepackage{aligned-overset}
\usepackage{amsmath}
\usepackage{amsthm}
\usepackage{amssymb}
\usepackage{bm}
\usepackage[shortlabels]{enumitem}
\usepackage{hyperref}
\usepackage[utf8]{inputenc}
\usepackage{listings}
\usepackage{mathtools}
\usepackage{multirow}
\usepackage{physics}
\usepackage{titling}
\usepackage{fancyhdr}

\author{Karsten Lehmann}
\date{SoSe 2022}
\title{Übungsblatt 6\\Programmierung}

\setlength{\headheight}{26pt}
\pagestyle{fancy}
\fancyhf{}
\lhead{\thetitle}
\rhead{\theauthor}
\lfoot{\thedate}
\rfoot{Seite \thepage}

\begin{document}
\paragraph{Aufgabe 1} Folgende Definitionen seien gegeben:
\begin{lstlisting}[language=Haskell, numbers=left]
data Tree a = Node a (Tree a) (Tree a) | Leaf a

mirror :: Tree a -> Tree a
mirror (Node x t1 t2) = Node x (mirror t2) (mirror t1)
mirror (Leaf x) = Leaf x

yield :: Tree a -> [a]
yield (Node _ t1 t2) = yield t1 ++ yield t2
yield (Leaf x) = [x]
\end{lstlisting}
Dabei gelten folgende Eigenschaften für alle Typen
\lstinline[language=Haskell]{a}, Werte
\lstinline[language=Haskell]{x :: a} und List
\lstinline[language=Haskell]{xs, ys :: [a]}.

\begin{enumerate}[label={(E\arabic*)}]
\item \lstinline[language=Haskell]{reverse [x] = [x]}
\item \lstinline[language=Haskell]{reverse (xs ++ ys) = reverse ys ++ reverse xs}
\end{enumerate}

Beweisen Sie mittels struktureller Induktion, dass für jeden Typ
\lstinline[language=Haskell]{a} und jeden Baum
\lstinline[language=Haskell]{t :: Tree a} gilt
\begin{lstlisting}[language=Haskell]
  reverse (yield t) = yield (mirror t)
\end{lstlisting}
Zeigen Sie den Induktionsanfang und den Induktionsschritt.
Geben Sie bei jeder Umformung die benutzten Definitionen, Eigenschaft oder
Induktionsvoraussetzung an.
Quantifizieren Sie alle Variablen.

\subparagraph{Lsg.} \textbf{Induktionsanfang}: Sei ein Typ
\lstinline[language=Haskell]{a} beliebig und
\lstinline[language=Haskell]{t = Leaf a}.
Dann ist\\
\lstinline[language=Haskell]{reverse (yield t)} \\
$\overset{\text{Def. t}}=$
\lstinline[language=Haskell]{reverse (yield (Leaf a))} \\
$\overset{(9)}=$
\lstinline[language=Haskell]{reverse [x]} \\
$\overset{(E1)}=$
\lstinline[language=Haskell]{[x]} \\
$\overset{(9}=$
\lstinline[language=Haskell]{yield (Leaf x)} \\
$\overset{(5}=$
\lstinline[language=Haskell]{yield (mirror (Leaf x))} \\

\noindent
\textbf{Induktionsschritt}: Sei ein Typ
\lstinline[language=Haskell]{a} beliebig und
\lstinline[language=Haskell]{t1, t2 :: Tree a}, so dass
\begin{lstlisting}[firstnumber=10, language=Haskell, numbers=left]
  reverse (yield t1) = yield (mirror t1)
  reverse (yield t2) = yield (mirror t2)
\end{lstlisting}
gilt.
Es folgt für ein beliebiges
\lstinline[language=Haskell]{x :: a} und
\lstinline[language=Haskell]{t = (Node x t1 t2)}, dass \\
\lstinline[language=Haskell]{reverse (yield t)} \\
$\overset{\text{Def. t}}=$
\lstinline[language=Haskell]{reverse (yield (Node x t1 t2))} \\
$\overset{(8)}=$
\lstinline[language=Haskell]{reverse (yield t1 ++ yield t2)} \\
$\overset{(E2)}=$
\lstinline[language=Haskell]{reverse (yield t2) ++ reverse (yield t1)} \\

und außerdem ist \\
\lstinline[language=Haskell]{yield (mirror t)} \\
$\overset{\text{Def. t}}=$
\lstinline[language=Haskell]{yield (mirror (Node x t1 t2))} \\
$\overset{(4)}=$
\lstinline[language=Haskell]{yield (Node x (mirror t2) (mirror t1))} \\
$\overset{(8)}=$
\lstinline[language=Haskell]{yield (mirror t2) ++ yield (mirror t1)} \\
$\overset{\text{Ind. Vor.}}=$
\lstinline[language=Haskell]{reverse (yield t2) ++ reverse (yield t1)} \\

Somit gilt die Behauptung für alle \lstinline[language=Haskell]{t :: Tree}

\paragraph{Aufgabe 2}
\begin{enumerate}[(a)]
\item Bestimmen Sie für jeden der folgenden $\lambda$-Terme $t$ die Mengen
  $FV(t)$ und $GV(t)$:
  \begin{itemize}
  \item $\qty\big(\lambda x.x \: y)\qty\big(\lambda y.y)$

    \subparagraph{Lsg.} $FV(t) = \qty\big{y}$, da $y$ im rechten Ausdruck zwar
    gebunden, jedoch im linken Ausdruck frei ist.
    Außerdem ist $GV(t) = \qty\big{x, y}$

  \item $\qty\big(\lambda x.\qty\big(\lambda y.z \qty\big(\lambda z.z \qty\big(\lambda x.y))))$

    \subparagraph{Lsg.} $FV(t) = \qty\big{z}$ und $GV(t) = \qty\big{x, y, z}$

  \item $\qty\big(\lambda x.\qty\big(\lambda y.x \: z \qty\big(y \: z))) \qty\big(\lambda x.y \qty\big(\lambda y.y))$

    \subparagraph{Lsg.} $FV(t) = \qty\big{y, z}$ und $GV(t) = \qty\big{x, y}$
  \end{itemize}

\newpage
\item Reduzieren Sie die folgenden $\lambda$-Terme zu Normalformen.
  Schreiben Sie vor jedem Ableitungsschritt die Menge der freien und
  gebundenen Variablen auf.

  \begin{itemize}
  \item $\qty\big(\lambda x.\qty\big(\lambda y.x \: z \qty\big(y \: z))) \qty\big(\lambda x.y \qty\big(\lambda y.y))$

    \subparagraph{Lsg.} \:\\

    \begin{tabular}{cc}
      $\underset{t1}{\underbrace{\qty\big(\lambda x.\qty\big(\lambda y.x \: z \qty\big(y \: z)))}}$ & $\underset{t_2}{\underbrace{\qty\big(\lambda x.y \qty\big(\lambda y.y))}}$ \\
      $GV(t_1) = \qty\big{x, y}$ und $FV(t_1) = \qty\big(z)$ & $GV(t_2) = \qty\big{x, y}$ und $FV(t_2) = \qty\big(y)$ \\
      \multicolumn{2}{c}{Die Mengen $GV\qty\big(t_1)$ und $FV\qty\big(t_2)$ überschneiden sich. $\Rightarrow \alpha$-Konversation} \\
      \hline
      $\underset{t1}{\underbrace{\qty\big(\lambda x.\qty\big(\lambda y'.x \: z \qty\big(y' \: z)))}}$ & $\underset{t_2}{\underbrace{\qty\big(\lambda x.y \qty\big(\lambda y.y))}}$ \\
      $GV(t_1) = \qty\big{x, y'}$ und $FV(t_1) = \qty\big(z)$ & $GV(t_2) = \qty\big{x, y}$ und $FV(t_2) = \qty\big(y)$ \\
      \hline
      $\underset{t_3}{\underbrace{\qty\big(\lambda y'.\qty\big(\lambda x.y \qty\big(\lambda y.y)) z \qty\big(y' \: z))}}$ \\
      $GV(t_3) = \qty\big{y', x, y}$ und $FV(t_3) = \qty\big(z)$  \\
      \hline
      $\underline{\underline{\qty\big(\lambda y'.y \qty\big(\lambda y.y) \qty\big(y' \: z))}}$ \\
    \end{tabular}

  \item $\qty\big(\lambda x.\qty\big(\lambda y.\qty\big(\lambda z.z))) x \qty\big(+ \: y \: 1)$

    \subparagraph{Lsg.} \:\\

    \begin{tabular}{ccc}
      $\qty\big(\lambda x.\qty\big(\lambda y.\qty\big(\lambda z.z)))$ & $x$ & $\qty\big(+ \: y \: 1)$ \\
      $GV = \qty\big(x, y, z), FV = \emptyset$ & $GV = \emptyset, FV = \qty\big{x}$ & $GV = \emptyset, FV = \qty\big{y}$ \\
      \hline
      \multicolumn{3}{c}{Ohne Überschneidungen kann der 2. Term direkt in den ersten Term eingesetzt werden} \\
      \hline
      $\qty\big(\lambda y.\qty\big(\lambda z.z))$ & & $\qty\big(+ \: y \: 1)$ \\
      $GV = \qty\big(y, z), FV = \emptyset$ & & $GV = \emptyset, FV = \qty\big{y}$ \\
      \hline
      $\underline{\underline{\qty\big(\lambda z.z)}}$
    \end{tabular}

  \newpage
  \item $\qty\big(\lambda x.\qty\big(\lambda y.x \qty\big(\lambda z.y \: z)))
    \qty\big(
      \qty\big(\qty\big(\lambda x. \qty\big(\lambda y.y)) 8)
      \qty\big(\lambda x.\qty\big(\lambda y.y) x)
    )$
    
    \subparagraph{Lsg.}
    $\underset{GV = \qty\big{x, y, z}, FV = \emptyset}{\underbrace{
      \qty\big(\lambda x.\qty\big(\lambda y.x \qty\big(\lambda z.y \: z)))
    }}
    \underset{GV = \qty\big{x, y}, FV = \emptyset}{\underbrace{
      \qty\big(
        \qty\big(\qty\big(\lambda x. \qty\big(\lambda y.y)) 8)
        \qty\big(\lambda x.\qty\big(\lambda y.y) x)
      )
    }} \underset{\beta}\Rightarrow$ \\
    $\qty\big(\lambda y.\qty\big(
        \underset{GV = \qty\big{x, y}, FV = \emptyset}{\underbrace{
          \qty\big(\qty\big(\lambda x. \qty\big(\lambda y.y)) 8)
        }}
        \qty\big(\lambda x.\qty\big(\lambda y.y) x)
    ) \qty\big(\lambda z.y \: z))  \underset{\beta}\Rightarrow$ \\
    $\qty\big(\lambda y.\qty\big(
      \underset{GV = \qty\big{y}, FV = \emptyset}{
        \underbrace{\qty\big(\lambda y.y)
      }}
      \underset{GV = \qty\big{x, y}, FV = \emptyset}{\underbrace{
        \qty\big(\lambda x.\qty\big(\lambda y.y) x)
      }}
    ) \qty\big(\lambda z.y \: z))  \underset{\beta}\Rightarrow$ \\
    $\qty\big(\lambda y.\qty\big(
      \lambda x.\underset{GV = \qty\big{y}}{\underbrace{\qty\big(\lambda y.y)}}
      \underset{FV=\qty\big{x}}{\underbrace{x}}
    ) \qty\big(\lambda z.y \: z))  \underset{\beta}\Rightarrow$ \\
    $\qty\big(\lambda y.\underset{GV = \qty\big{x}}{\underbrace{\qty\big(
      \lambda x.x
    )}} \underset{GV={z},FV={y}}{\underbrace{\qty\big(\lambda z.y \: z)}})  \underset{\beta}\Rightarrow$ \\
    $\qty\big(\lambda y.\qty\big(\lambda z.y \: z))$

  \item $(\lambda h.(\lambda x.h (x \: x)) (\lambda x.h (x \: x)))
    ((\lambda x.x) (+ \: 1 \: 5))$

    \subparagraph{Lsg.}
    \begin{flalign*}
      & (\lambda h.(\lambda x.h (x \: x)) (\lambda x.h (x \: x)))
      (\underset{GV=\qty{x}, FV=\emptyset}{
        \underbrace{
          (\lambda x.x)
        }
      } \underset{GV=\emptyset, FV=\emptyset}{
        \underbrace{
          (+ \: 1 \: 5)
        }
      })\\
      \overset{\beta}&\Rightarrow
      (\lambda h.
        (\lambda x.\underset{GV=\emptyset, FV=\qty{h,x}}{
          \underbrace{
            h (x \: x)
          }
        })
        \underset{GV=\qty{x}, FV=\qty{h}}{
          \underbrace{
            (\lambda x.h (x \: x))
          }
        }
      )
      (+ \: 1 \: 5) \\
      \overset{\beta}&\Rightarrow
      (\lambda h.h((\lambda x.h (x \: x))(\lambda x.h (x \: x)))
      )
      (+ \: 1 \: 5) \\
      &\Rightarrow \text{ Endlose Rekursion!}
    \end{flalign*}

  \item $(\lambda f.(\lambda a.(\lambda b.f \: a \: b)))
    (\lambda x.(\lambda y.x))$

    \subparagraph{Lsg.}
    \begin{flalign*}
      &(\lambda f.
        \underset{GV=\qty{ab}, FV=\qty{f}}{
          \underbrace{
            (\lambda a.(\lambda b.f \: a \: b))
          }
        }
      )
      \underset{GV=\qty{x,y}, FV=\emptyset}{
        \underbrace{
          (\lambda x.(\lambda y.x))
        }
      } \\
      \overset{\beta}&\Rightarrow
      (\lambda a.(\lambda b.
        (\lambda x.\underset{GV=\qty{x, y}, FV=\emptyset}{
          \underbrace{
            (\lambda y.x)
          }
        })
        \: \underset{FV=\qty{a}}{\underbrace{{a}}} \: b)
      ) \\
      \overset{\beta}&\Rightarrow
      (\lambda a.(\lambda b.
        (\lambda y.\underset{FV=\qty{a}}{\underbrace{a}})
        \underset{FV=\qty{b}}{\underbrace{b}})
      ) \\
      \overset{\beta}&\Rightarrow (\lambda a.(\lambda b.a))  
    \end{flalign*}
  \end{itemize}
\end{enumerate}

\paragraph{Aufgabe 3}
\begin{enumerate}[(a)]
\item Geben Sie einen Kombinator $A$ an, so dass
  $A t s u \Rightarrow^* s$ für alle $\lambda$-Terme $t, s$ und $u$

  \subparagraph{Lsg.} $A = (\lambda x.(\lambda y.(\lambda z.y)))$

\item Geben Sie einen Kombinator $B$ an, so dass
  $B t s \Rightarrow^* s t$ für alle $\lambda$-Terme $t$ und $s$

  \subparagraph{Lsg.} $B = (\lambda x.(\lambda y.y \: x))$

\item Geben Sie einen Kombinator $C$ an, so dass $C C \Rightarrow_{\beta} C C$

  \subparagraph{Lsg.} $C = (\lambda x.x \: x)$

\item Geben Sie einen Kombinator $D$ an, so dass $D \Rightarrow_{\beta} D$

  \subparagraph{Lsg.} $D = (CC)$
\end{enumerate}
\end{document}
