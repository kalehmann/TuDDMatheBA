\documentclass{scrreprt}

\usepackage{aligned-overset}
\usepackage{amsmath}
\usepackage{amsthm}
\usepackage{amssymb}
\usepackage{bm}
\usepackage[shortlabels]{enumitem}
\usepackage{hyperref}
\usepackage[utf8]{inputenc}
\usepackage{listings}
\usepackage{mathtools}
\usepackage{multirow}
\usepackage{physics}
\usepackage{titling}
\usepackage{fancyhdr}

\author{Karsten Lehmann}
\date{SoSe 2022}
\title{Übungsblatt 7\\Programmierung}

\setlength{\headheight}{26pt}
\pagestyle{fancy}
\fancyhf{}
\lhead{\thetitle}
\rhead{\theauthor}
\lfoot{\thedate}
\rfoot{Seite \thepage}

\begin{document}
\paragraph{Aufgabe 1}
\begin{enumerate}[(a)]
\item Berechnen Sie die Normalform des $\lambda$-Terms
  $(\lambda f \: x.f \: f \: x) (\lambda y.x) z$, indem Sie ihn schrittweise
  Reduzieren.
  Geben Sie bei jedem Schritt für relevante Teilmengen jeweils die freien und
  gebundenen Variablen an.

  \subparagraph{Lsg.}
  \begin{flalign*}
    &(\lambda f \: x.
    \underset{GV=\emptyset, FV=\qty{f, x}}{
      \underbrace{
        f \: f \: x
      }
    }) \underset{GV=\qty{y}, FV=\qty{x}}{
      \underbrace{
        (\lambda y.x)
      }
    } \underset{FV=\qty{z}}{\underbrace{z}} \\
    &\Rightarrow^{\alpha} (\lambda f \: x.
    \underset{GV=\emptyset, FV=\qty{f, x}}{
      \underbrace{
        f \: f \: x
      }
    }) \underset{GV=\qty{y}, FV=\qty{x'}}{
      \underbrace{
        (\lambda y.x')
      }
    }  \underset{FV=\qty{z}}{\underbrace{z}} \\
    &\Rightarrow_{\beta}
    (
      (\lambda y.\underset{FV=\qty{x'}}{\underbrace{x'}})
      \underset{GV=\qty{y}, FV=\qty{x'}}{\underbrace{(\lambda y.x')}}
      z
    ) \\
    &\Rightarrow_{\beta} x'z
  \end{flalign*}
\end{enumerate}
\end{document}
