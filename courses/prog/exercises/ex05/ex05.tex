\documentclass{scrreprt}

\usepackage{aligned-overset}
\usepackage{amsmath}
\usepackage{amsthm}
\usepackage{amssymb}
\usepackage{bm}
\usepackage[shortlabels]{enumitem}
\usepackage{hyperref}
\usepackage[utf8]{inputenc}
\usepackage{listings}
\usepackage{mathtools}
\usepackage{physics}
\usepackage{titling}
\usepackage{fancyhdr}

\author{Karsten Lehmann}
\date{SoSe 2022}
\title{Übungsblatt 5\\Programmierung}

\setlength{\headheight}{26pt}
\pagestyle{fancy}
\fancyhf{}
\lhead{\thetitle}
\rhead{\theauthor}
\lfoot{\thedate}
\rfoot{Seite \thepage}

\begin{document}
\paragraph{Aufgabe 1} Gegeben seien folgende Terme über dem Rangalphabet
$\Sigma = \qty\big{\sigma^{(2)}, \gamma^{(1)}, \alpha^{(0)}}$:
\[
  t_1 = \sigma\qty\big(\sigma(x_1, \alpha), \sigma(\gamma(x_3), x_3))
  \text{ und }
  t_2 = \sigma\qty\big(\sigma(\gamma(x_2), \alpha), \sigma(x_2, x_3))
\]
\begin{enumerate}[(a)]
\item Wenden Sie den Unifikationsalgorithmus auf die Terme $t_1$ und $t_2$ an
  und geben Sie anschließend den von Ihnen bestimmten allgemeinsten Unifikator
  an.

  \subparagraph{Lsg.} \textbf{Der Unifikationsalgorithmus}

  Setze $M \coloneqq \qty{\begin{pmatrix}t_1 \\ t_2\end{pmatrix}}$.
  Wende nun solange wie möglich die folgenden Regeln an:
  \begin{enumerate}[(I)]
  \label{uni:1}
  \item \textbf{Dekomposition} ersetze ein Paar $\begin{pmatrix}
      \delta \qty\big(a_1, \ldots, a_n) \\
      \delta \qty\big(b_1, \ldots, b_n)
    \end{pmatrix}$, wobei $a_1, \ldots, a_n, b_1, \ldots, b_n$ Terme über
    Konstruktoren und Variablen sind durch Paare
    $\begin{pmatrix}a_1\\b_1\end{pmatrix},
    \ldots,
    \begin{pmatrix}a_n\\b_n\end{pmatrix}$.

  \label{uni:2}
  \item \textbf{Elimination trivialer Gleichungen} Entferne alle Paare
    $\begin{pmatrix}x\\x\end{pmatrix}$ aus $M$, wobei $x$ eine Variable ist.

  \label{uni:3}
  \item \textbf{Vertauschung} Ersetze ein Paar $\begin{pmatrix}t\\x\end{pmatrix}$
    - wobei $t$ keine Variable ist - durch das Paar
    $\begin{pmatrix}x\\t\end{pmatrix}$.

  \label{uni:4}
  \item \textbf{Substitution von Variablen} Wenn in $M$ ein Paar
    $\begin{pmatrix}x\\t\end{pmatrix}$ vorkommt (occur check), wobei $x$ eine Variable ist und
    $x$ nicht in $t$ vorkommt, dann ersetze alle Vorkommen von $x$ durch $t$
  \end{enumerate}

  Wenn danach $M$ die Form $\qty{
    \begin{pmatrix}x_1\\t_1\end{pmatrix},
    \ldots,
    \begin{pmatrix}x_n\\t_n\end{pmatrix}
  }$ hat, $x_1, \ldots, x_n$ paarweise verschiedene Variablen sind und
  $t_1, \ldots, t_n$ Terme über Variablen und Konstruktoren sind in denen
  die $x_i$'s nicht mehr vorkommen ist der allgemeinste Unifikator gefunden.
  
  \begin{flalign*}
    M &= \qty{\begin{pmatrix}
        \sigma\qty\big(\sigma(x_1, \alpha), \sigma(\gamma(x_3), x_3)) \\
        \sigma\qty\big(\sigma(\gamma(x_2), \alpha), \sigma(x_2, x_3))
      \end{pmatrix}} & \hyperref[uni:1](I) \\
    &= \qty{
      \begin{pmatrix}
        \sigma(x_1, \alpha) \\
        \sigma(\gamma(x_2), \alpha)
      \end{pmatrix}, \begin{pmatrix}
        \sigma(\gamma(x_3), x_3) \\
        \sigma(x_2, x_3)
      \end{pmatrix}} & \hyperref[uni:1](I) \\
    &= \qty{
      \begin{pmatrix}
        x_1 \\
        \gamma(x_2)
      \end{pmatrix}, \begin{pmatrix}
        \alpha \\
        \alpha
      \end{pmatrix}, \begin{pmatrix}
        \gamma(x_3) \\
        x_2
      \end{pmatrix}, \begin{pmatrix}
        x_3 \\
        x_3 \\
      \end{pmatrix}} & \hyperref[uni:1](I) \\
    &= \qty{
      \begin{pmatrix}
        x_1 \\
        \gamma(x_2)
      \end{pmatrix}, \begin{pmatrix}
        \gamma(x_3) \\
        x_2
      \end{pmatrix}, \begin{pmatrix}
        x_3 \\
        x_3 \\
      \end{pmatrix}} & \hyperref[uni:2](II) \\
    &= \qty{
      \begin{pmatrix}
        x_1 \\
        \gamma(x_2)
      \end{pmatrix}, \begin{pmatrix}
        \gamma(x_3) \\
        x_2
      \end{pmatrix}} & \hyperref[uni:3](III) \\
    &= \qty{
      \begin{pmatrix}
        x_1 \\
        \gamma(x_2)
      \end{pmatrix}, \begin{pmatrix}
        x_2 \\
        \gamma(x_3)
      \end{pmatrix}} & \hyperref[uni:4](IV) \\
     &= \qty{
      \begin{pmatrix}
        x_1 \\
        \gamma\qty\big(\gamma(x_3))
      \end{pmatrix}, \begin{pmatrix}
        x_2 \\
        \gamma(x_3)
      \end{pmatrix}}
  \end{flalign*}
  Nun ist $\varphi\qty\big(x_1) =  \gamma\qty\big(\gamma(x_3)),
  \varphi\qty\big(x_2) =  \gamma\qty\big(x_3),
  \varphi\qty\big(x_3) =  x_3$
  der allgemeinste Unifikator.

\item Geben Sie zwei weitere Unifikatoren an.

  \subparagraph{Lsg.}
  \[
    \varphi\qty\big(x_1) =  \gamma\qty\big(\gamma(\alpha)),
    \varphi\qty\big(x_2) =  \gamma\qty\big(\alpha),
    \varphi\qty\big(x_3) =  \alpha
  \]
  und
  \[
    \varphi\qty\big(x_1) =  \gamma\qty\big(\gamma(\gamma(\alpha))),
    \varphi\qty\big(x_2) =  \gamma\qty\big(\gamma(\gamma(\alpha))),
    \varphi\qty\big(x_3) =  \gamma\qty\big(\alpha)
  \]

\item Geben Sie zwei Terme $t_1$ und $t_2$ über dem Alphabet $\Sigma$ an, so
  dass in der Anwendung des Unifikationsalgorithmus der Occur Check fehlschlägt.

  \subparagraph{Lsg.}
  \[
  t_1 = x_1
  \text{ und }
  t_2 = \gamma\qty\big(x_1)
\]

\newpage
\item Gegeben seien die Haskell-Typterme
  \[
    t_1 = \qty\big(\texttt{a}, \qty[\texttt{a}]),
    t_2 = \qty\big(\texttt{Int}, \qty[\texttt{Double}])
    \text{ und }
    t_3 = \qty\big(\texttt{b}, \texttt{c})
  \]
  Welche Paare dieser Terme sind unifizierbar?
  Geben Sie für die unifizierbaren Paare den allgemeinsten
  Unifikator an.

  \subparagraph{Lsg.} Die Paare $t_3, t_1$ und $t_3, t_2$ sind unifizierbar.
  \begin{flalign*}
    M &= \qty{\begin{pmatrix}t_3\\t_1\end{pmatrix}}
    = \qty{\begin{pmatrix}
        \qty\big(\texttt{b}, \texttt{c}) \\
        \qty\big(\texttt{a}, \qty[\texttt{a}])
      \end{pmatrix}} & \hyperref[uni:1]{(I)} \\
    &= \qty{\begin{pmatrix}
        \texttt{b} \\
        \texttt{a}
      \end{pmatrix}, \begin{pmatrix}
        \texttt{c} \\
        \qty[\texttt{a}]
      \end{pmatrix}} \\
    &\Rightarrow \varphi\qty\big(\texttt{a}) = \texttt{a},
    \varphi\qty\big(\texttt{b}) = \texttt{a},
    \varphi\qty\big(\texttt{c}) = \qty[\texttt{a}]
  \end{flalign*}
  \begin{flalign*}
    M &= \qty{\begin{pmatrix}t_3\\t_2\end{pmatrix}}
    = \qty{\begin{pmatrix}
        \qty\big(\texttt{b}, \texttt{c}) \\
        \qty\big(\texttt{Int}, \qty[\texttt{Double}])
      \end{pmatrix}} & \hyperref[uni:1]{(I)} \\
    &= \qty{\begin{pmatrix}
        \texttt{b} \\
        \texttt{Int}
      \end{pmatrix}, \begin{pmatrix}
        \texttt{c} \\
        \qty[\texttt{Double}]
      \end{pmatrix}} \\
    &\Rightarrow \varphi\qty\big(\texttt{Int}) = \texttt{Int},
    \varphi\qty\big(\texttt{Double}) = \texttt{Double},
    \varphi\qty\big(\texttt{b}) = \texttt{Int},
    \varphi\qty\big(\texttt{c}) = \qty[\texttt{Double}]
  \end{flalign*}
  Für das Paar $t_1, t_2$ hingegen
  \begin{flalign*}
    M &= \qty{\begin{pmatrix}t_1\\t_2\end{pmatrix}}
    = \qty{\begin{pmatrix}
        \qty\big(\texttt{a}, \qty[\texttt{a}]) \\
        \qty\big(\texttt{Int}, \qty[\texttt{Double}])
      \end{pmatrix}} & \hyperref[uni:1]{(I)} \\
    &= \qty{\begin{pmatrix}
        \texttt{a} \\
        \texttt{Int}
      \end{pmatrix}, \begin{pmatrix}
        \qty[\texttt{a}] \\
        \qty[\texttt{Double}]
      \end{pmatrix}} \\
    &\Rightarrow \text{nicht unifizierbar!}
  \end{flalign*}
\end{enumerate}

\newpage
\paragraph{Aufgabe 2} Zeigen Sie unter Verwendung der folgenden Definitionen
durch strukturelle Induktion die Gültigkeit der Gleichung \\
\noindent
\lstinline[language=Haskell]{sum (foo xs) = 2 * sum xs - length xs} für jedes
\lstinline[language=Haskell]{xs :: [Int]}

\begin{lstlisting}[language=Haskell, numbers=left]
foo :: [Int] -> [Int]
foo [] = []
foo (x:xs) = x : x : (-1) : foo xs

sum :: [Int] -> Int
sum [] = 0
sum (x:xs) = x + sum xs

length :: [Int] -> Int
length [] = 0
length (x:xs) = 1 + length xs
\end{lstlisting}

Zeigen Sie dazu den Induktionsanfang und den Induktionsschritt.
Geben Sie bei dem Induktionsschritt die Induktionsvoraussetzung an.
Geben Sie bei allen Umformungen die benutzten Definitionen, Eigenschaften
bzw. Induktionsvoraussetzung an.
Quantifizieren Sie alle Variablen.

\subparagraph{Lsg.} \textbf{Induktionsanfang:} Sei \lstinline[language=Haskell]{xs = []}.
Es gilt \\
\lstinline[language=Haskell]{sum (foo xs) = 2 * sum xs - length xs} \\
$\overset{\text{Def. }\lstinline[language=Haskell]{xs}}=$
\lstinline[language=Haskell]{sum (foo []) = 2 * sum [] - length []} \\
$\overset{(10)}=$
\lstinline[language=Haskell]{sum (foo []) = 2 * sum [] - 0} \\
$\overset{(6)}=$
\lstinline[language=Haskell]{sum (foo []) = 2 * 0 - 0 = 0} \\
$\overset{(2)}=$
\lstinline[language=Haskell]{sum [] =  0} \\
$\overset{(6)}=$
\lstinline[language=Haskell]{sum [] =  0} \\
\noindent
\textbf{Induktionsschritt:} Sei nun \lstinline[language=Haskell]{xs :: [Int]},
sodass
\begin{lstlisting}[language=Haskell]
  sum (foo xs) = 2 * sum xs - length xs
\end{lstlisting}
gilt.
Dann folgt für jedes \lstinline[language=Haskell]{x :: Int} \\
\lstinline[language=Haskell]{sum (foo (x:xs)) = 2 * sum (x:xs) - length (x:xs)} \\
$\overset{(11)}=$
\lstinline[language=Haskell]{sum (foo (x:xs)) = 2 * sum (x:xs) - (1 + length xs)} \\
$\overset{(7)}=$
\lstinline[language=Haskell]{sum (foo (x:xs)) = 2 * (x + sum xs) - (1 + length xs)} \\
$\overset{\text{Distributivgesetz}}=$
\lstinline[language=Haskell]{sum (foo (x:xs)) = 2 * x + 2 * sum xs - 1 - length xs} \\
$\overset{\text{Kommutativgesetz}}=$
\lstinline[language=Haskell]{sum (foo (x:xs)) = 2 * x - 1 + 2 * sum xs - length xs} \\
$\overset{(3)}=$
\lstinline[language=Haskell]{sum (x:x:(-1):foo xs) = 2 * x - 1 + 2 * sum xs - length xs} \\
$\overset{(7)}=$
\lstinline[language=Haskell]{x + sum (x:(-1):foo xs) = 2 * x - 1 + 2 * sum xs - length xs} \\
$\overset{(7)}=$
\lstinline[language=Haskell]{x + x + sum ((-1):foo xs) = 2 * x - 1 + 2 * sum xs - length xs} \\
$\overset{(7)}=$
\lstinline[language=Haskell]{x + x - 1 +  sum (foo xs) = 2 * x - 1 + 2 * sum xs - length xs} \\
$\overset{\text{Ind. Vor.}}=$
\lstinline[language=Haskell]{x + x - 1 = 2 * x - 1} \\
$\overset{x + x = 2 * x}=$
\lstinline[language=Haskell]{2 * x - 1 = 2 * x - 1} \\

Oder alternativ \textbf{Induktionsschritt:} Sei nun \lstinline[language=Haskell]{xs :: [Int]},
sodass
\begin{lstlisting}[language=Haskell]
  sum (foo xs) = 2 * sum xs - length xs
\end{lstlisting}
gilt.
Dann folgt für jedes \lstinline[language=Haskell]{x :: Int} \\
\lstinline[language=Haskell]{sum (foo (x:xs))} \\
$\overset{(3)}=$ \lstinline[language=Haskell]{sum (x:x:(-1):foo xs)} \\
$\overset{(7)}=$ \lstinline[language=Haskell]{x + sum (x:(-1):foo xs)} \\
$\overset{(7)}=$ \lstinline[language=Haskell]{x + x + sum ((-1):foo xs)} \\
$\overset{(7)}=$ \lstinline[language=Haskell]{x + x -1 +  sum (foo xs)} \\
$\overset{\text{Ind. Vor.}}=$
\lstinline[language=Haskell]{x + x -1 +  2 * sum xs - length xs} \\
$\overset{\text{Kommutativgesetz}}=$
\lstinline[language=Haskell]{x + x +  2 * sum xs - 1 - length xs} \\
$\overset{x + x = 2 * x}=$
\lstinline[language=Haskell]{2 * x +  2 * sum xs - 1 - length xs} \\
$\overset{\text{Distributivgesetz}}=$
\lstinline[language=Haskell]{2 * (x + sum xs) - (1 + length xs)} \\
$\overset{(11)}=$
\lstinline[language=Haskell]{2 * (x + sum xs) - length (x:xs)} \\
$\overset{(7)}=$
\lstinline[language=Haskell]{2 * sum (x:xs) - length (x:xs)} \\

Somit gilt die Behauptung für alle \lstinline[language=Haskell]{xs :: Int}

\end{document}
