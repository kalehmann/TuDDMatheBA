\documentclass{article}

\usepackage{aligned-overset}
\usepackage{amsmath}
\usepackage{amssymb}
\usepackage{bbm}
\usepackage[shortlabels]{enumitem}
\usepackage{hyperref}
\usepackage[utf8]{inputenc}
\usepackage{interval}
\intervalconfig{
  soft open fences
}
\usepackage{mathtools}
\usepackage{pgfplots}
\usepackage{physics}
\usepackage{tikz}
\usetikzlibrary{positioning}
\usepackage{xcolor}
\definecolor{light-gray}{gray}{.9}

\author{Albina Oscherowa \\ Karsten Lehmann}
\date{02.12.2020}
\title{Hausaufgabe 05 Analysis - Grundlegende Konzepte}

\begin{document}


\maketitle
\newpage

\section*{Hausaufgabe 1}

\begin{enumerate}[a)]
\item
  Geben Sie für $n, k \in \mathbb{N}_0$ eine explizite Formel zur Berechnung von $\binom{n - 1}{k}$ an
  und beweisen Sie diese.

  \[
    \binom{n - 1}{k} = \prod\limits_{i=1}^k \frac{n - i}{i}
  \]

  \textbf{Beweis durch vollständige Induktion} \\
  \emph{Behauptung}:

  \[
    P_1(k) \colon \binom{n - 1}{k} = \prod\limits_{i=1}^k \frac{n - i}{i}
  \]

  \emph{Induktionsanfang}:

  \begin{align*}
    P_1(0) \colon \binom{n - 1}{0} \overset{\text{Definition endliche Produkte}}&= \prod\limits_{i=1}^0 \frac{n - i}{i} \\
    \binom{n - 1}{0} \overset{\text{Definition Binominalkoeffizient}}&= 1 \\
    1 &= 1 \\
    P_1(1) \colon \binom{n - 1}{1} \overset{\text{Definition endliche Produkte}}&= \prod\limits_{i=1}^1 \frac{n - i}{i} \\
    \binom{n - 1}{1} &= \frac{n - 1}{1} \\
    \binom{n - 1}{1} \overset{\text{Definition Binominalkoeffizient}}&= n - 1 \\
    n - 1 &= n - 1 \\
  \end{align*}

  Die Aussage $P_1$ ist für $0$ und $1$ wahr.

  \newpage
  \emph{Induktionsschritt}: Die Behauptung $P_1(k)$ sei wahr für ein $k \in \mathbb{N}$, dann gilt: 

  \begin{align*}
    P_1(k + 1) \colon \binom{n - 1}{k + 1} \overset{\text{Def. Binominalkoeffizient u. Summe}}&= \prod\limits_{i=1}^{k + 1} \frac{n - i}{i} \\
    \frac{n  - 1 - k}{k + 1} \binom{n - 1}{k} &= \prod\limits_{i=1}^{k} \frac{n - i}{i} \cdot \frac{n - (k + 1)}{k + 1}  \\
    \frac{n  - 1 - k}{k + 1} \binom{n - 1}{k} &= \prod\limits_{i=1}^{k} \frac{n - i}{i} \cdot \frac{n - k - 1)}{k + 1}  \\
    \frac{n  - 1 - k}{k + 1} \binom{n - 1}{k} &= \prod\limits_{i=1}^{k} \frac{n - i}{i} \cdot \frac{n - 1 - k)}{k + 1}  && | :\frac{n - 1 - k)}{k + 1} \\
    \binom{n - 1}{k} &= \prod\limits_{i=1}^{k} \frac{n - i}{i} \\
  \end{align*}
  Und das entspricht $P_1(k)$, somit ist die Behauptung wahr für $k + 1$. Aus dem Satz über die vollständige Induktion folgt die Behauptung. 

\item
  Beweisen Sie die folgenden Beziehungen
  \begin{enumerate}[1)]
  \item
    \[
      \sum\limits_{k=0}^{n}(-1)^k \cdot 2^k \binom{n}{k} = (-1)^n
    \]
  \item
    \[
      \sum\limits_{k=0}^{n} \frac{(-1)^k}{k + 1} \binom{n}{k} = \frac{1}{n + 1}
    \]
    \begin{align*}
      \sum\limits_{k=0}^{n} \frac{(-1)^k}{k + 1} \binom{n}{k} &= \sum\limits_{k=0}^{n} \frac{(-1)^k}{k + 1} \frac{k + 1}{n + 1} \binom{n + 1}{k + 1} \\
                                                              &= \frac{1}{n + 1} \sum\limits_{k=0}^{n} (-1)^k \binom{n + 1}{k + 1} \\
                                                              &= \frac{1}{n + 1} \sum\limits_{k=0}^{n} (-1)^k \left(\binom{n}{k + 1} +\binom{n}{k}\right) \\
                                                              &= \frac{1}{n + 1} \left(\sum\limits_{k=0}^{n} (-1)^k \binom{n}{k + 1} + \sum\limits_{k=0}^{n} (-1)^k \binom{n}{k}\right) \\
                                                              &= \frac{1}{n + 1} \left(\sum\limits_{k=0}^{n} (-1)^k \binom{n}{k} + \sum\limits_{k=1}^{n+1} (-1)^{k-1} \binom{n}{k}\right) \\
                                                              &= \frac{1}{n + 1} \left(\sum\limits_{k=0}^{n} (-1)^k \binom{n}{k} + \sum\limits_{k=1}^{n} (-1)^{k-1} \binom{n}{k}\right) \\
                                                              &= \frac{1}{n + 1} \left(\sum\limits_{k=0}^{n} (-1)^k \binom{n}{k} + \sum\limits_{k=0}^{n} (-1)^{k-1} \binom{n}{k}  -(-1 \cdot 1) \right) \\
                                                              &= \frac{1}{n + 1} \left(\sum\limits_{k=0}^{n} (-1)^k \binom{n}{k} + (-1) \sum\limits_{k=0}^{n} (-1)^{k-1} \binom{n}{k} + 1 \right) \\
                                                              &= \frac{1}{n + 1} \cdot 1 \\
                                                              &= \frac{1}{n + 1} \\
    \end{align*}
  \end{enumerate}
\end{enumerate}

\section*{Hausaufgabe 2}

Gegeben seien Zahlen $a, b \in \mathbb{R}$ mit $a , b$. Beweisen Sie, dass folgende Mengen gleichmächtig sind, indem
Sie geeignete bijektive Abbildung zwischen diesen Mengen angeben:

\begin{enumerate}[a)]
\item
  Sei $x_n \coloneqq \frac{1}{n+1}, n \in \mathbb{N}$
  \[
    f \colon \interval[open]{0}{1} \to \interval{0}{1}, \begin{cases}
      x \ne x_n & f(x) = x \\
      x = 0 & f(0) = \frac{1}{1 + 1} \\
      x = 1 & f(1) = \frac{1}{1 + 2} \\
      x = x_n & f(x_n) = x_{n+2}
    \end{cases}
  \]
\item
  \[
    f \colon \interval[open]{0}{1} \to \mathbb{R}, x \mapsto \frac{x-\frac{1}{2}}{x(x - 1)}
  \]
\end{enumerate}


\section*{Hausaufgabe 3}

Beweisen Sie, dass eine Menge $M$ nicht gleichmächtig zu ihrer Potenzmenge $2^M = P(M)$, definiert durch
$2^M \coloneqq \{ S | S \subseteq M \}$, ist. Zeigen Sie, dass es keine surjektive Abbildung
$f \colon M \to 2^M = P(M)$ gibt. \\

Sei $N$ eine Menge. Ich nehme an, dass die Mächtigkeit der Potenzmenge von $N$ gleich $2^{\abs{N}}$ entspricht.

\textbf{Beweis durch vollständige Induktion}:

\emph{Behauptung}:

\[
  Q(n) \colon \abs{P(\{m_1, \ldots, m_n\})} = 2^{\abs{\{m_1, \ldots, m_n\}}}
\]

\emph{Induktionsanfang}:

\begin{align*}
  Q(0) \colon \abs{P(\emptyset)} &= 2^{\abs{\emptyset}} \\
                                 &= 2^0 \\
\end{align*}

 Die Behauptung ist wahr, da die leere Menge genau eine Teilmenge, nämlich sich selbst besitzt. Somit ist die Aussage $Q(1)$ wahr.

\emph{Induktionsschritt}: Sei $Q(n)$ wahr für ein $n \in \mathbb{N}$ und ein Element $c \notin \{m_1, \ldots, m_n\}$. Dann gilt

\begin{align*}
  Q(n + 1) \colon \abs{P(\{m_1, \ldots, m_n \} \cup \{ c \})} &= 2^{\abs{\{m_1, \ldots, m_n \} \cup \{ c \}}} \\
                                                              &= 2^{(n + 1)} \\
                                                              &=2^n * 2 \\
\end{align*}

Die Potenzmenge von $\{m_1, \ldots, m_n\} \cup \{ c \}$ umfasst alle Teilmengen von $\{m_1, \ldots, m_n\}$ (also $P(\{m_1, \ldots, m_n\})$)
und zusätzlich alle Teilmengen von $\{m_1, \ldots, m_n\}$ vereinigt mit $\{ c \}$. Daraus folgt die Mächtigkeit von
$P(\{m_1, \ldots, m_n\} \cup \{ c \})$ mit $2 * P(\{m_1, \ldots, m_n\})$. Somit ist die Behauptung $Q(n + 1)$ wahr und
aus dem Satz über die Vollständige Induktion folgt die Behauptung.

\[
  \forall n \in \mathbb{N} \colon 2^n > n
\]


Die Potenzmenge von $M$ enthält jede Teilmenge von $M$. Sei nun $f \colon M \to P(M)$
eine injektive Abbildung und $N \coloneqq \{ m \in M | m \notin f(m)\}$ eine weitere Menge.
Da $N$ per Definition nur Elemente aus $M$ enthalten kann, ist $N \subseteq M$.

Wenn $f$ nun auch surjektiv ist, gibt es ein $m \in M$ mit $f(m) = N$.
Angenommen $m \subseteq N$, dann ist $m$ nach der Definition von $N$ nicht in $f(m)$ (also in $N$) enthalten.
Dies ist ein Widerspruch, somit muss $m \notin N$. Damit wäre $m$ nach der Definition von $N$ in $N$ enthalten.
Das ist auch ein Widerspruch, somit kann es kein solches $m$ geben und $f$ ist nicht surjektiv.


\end{document}
