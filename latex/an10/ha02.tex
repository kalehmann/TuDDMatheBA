\documentclass{article}

\usepackage{aligned-overset}
\usepackage{amsmath}
\usepackage{amssymb}
\usepackage{bbm}
\usepackage[shortlabels]{enumitem}
\usepackage{genealogytree}
\usepackage{hyperref}
\usepackage[utf8]{inputenc}
\usepackage{interval}
\intervalconfig{
  soft open fences
}

\usepackage{mathtools}
\usepackage{physics}
\usepackage{tikz}
\usetikzlibrary{positioning}
\usepackage{xcolor}
\definecolor{light-gray}{gray}{.9}

\author{Lisa Würzebesser (4981474) HA-02 \\ Lukas Kamratzki (3752330) HA-04 \\ Karsten Lehmann (4935758) HA-01}
\date{07.11.2020}
\title{Hausaufgabe 01 Analysis - Grundlegende Konzepte}

\begin{document}


\maketitle
\newpage

\section*{Hausaufgabe 1}

\begin{enumerate}[a)]
%% a)
\item Für alle $a, b, c \in \mathbb{R}$ gilt $a < b \land c \leq d \Rightarrow a + c < b + d$
  \begin{align*}
    a < b &\coloneqq b - a \in P \\
    c \leq d &\coloneqq d - c \in P \lor d - c = 0 \\
  \end{align*} 
  \begin{minipage}[t]{.45\textwidth}
    \textbf{Fall 1}: $d - c \in P$ (oder auch $c < d$): \\
    
    In diesem Falle gilt wie in der Vorlesung bereits bewiesen
    $a + c < b + d$
  \end{minipage}
  \hfill
  \vrule
  \hfill
  \begin{minipage}[t]{.45\textwidth}
    \textbf{Fall 2}: $d - c = 0 \Rightarrow c = d$
    \begin{align*}
      b &- a \in P \\
      \overset{\text{(A2)}}&{\Rightarrow} (b - a) + 0 \in P \\
      &\Rightarrow (b - a) + (d - c) \in P \\
      \overset{\text{(A1)}}&{\Rightarrow} (b + d) + (-a -c) \in P\\
      &\Rightarrow (b + d) - (a + c) \in P \\
      &\Rightarrow a + c < b + d \\
    \end{align*}
  \end{minipage}

%% b)
\item Für alle $a, b, c \in \mathbb{R}$ gilt $a \leq b \land c \leq d \Rightarrow a + c \leq b + d$
  \begin{align*}
    a \leq b &\coloneqq b - a \in P \lor b - a = 0 \\
    c \leq d &\coloneqq d - c \in P \lor d - c = 0 \\
    a + c \leq b + d &\coloneqq (b + d) - (a + c) \in P \lor (b + d) - (a + c) = 0 \\ 
  \end{align*}
  
  \begin{minipage}[t]{.45\textwidth}
    \textbf{Fall 1}: $d - c \in P \land b - a \in P$: \\
    
    In diesem Falle gilt wie in der Vorlesung bereits bewiesen
    $a + c < b + d$
  \end{minipage}
  \hfill
  \vrule
  \hfill
  \begin{minipage}[t]{.45\textwidth}
    \textbf{Fall 2}: $d - c = 0 \land b - a \in P$ \\
    
    Dieser Fall ist identisch zum Fall 2 der Teilaufgabe a).
  \end{minipage}
  \\ \\ \\ %% 3 Linebreaks are needed for a visible gap between the minipages 
  \begin{minipage}[t]{.45\textwidth}
    \textbf{Fall 3}: $d - c \in P \land b - a = 0$: \\
    
    Für diesen Fall gilt analog zum zweiten Fall der Teilaufgabe a)
    $a + c < b + d$
  \end{minipage}
  \hfill
  \vrule
  \hfill
  \begin{minipage}[t]{.45\textwidth}
    \textbf{Fall 4}: $d - c = 0 \land b - a = 0$ \\

    Angenommen $(b + d) - (a + c) = 0$, dann
    \begin{align*}
      (b + d) + ((-a) + (-c)) \overset{\text{(A9)}}&{=} 0 \\
      (b + (-a)) + (d + (-c)) \overset{\text{(A1)}}&{=} 0 \\
      (b - a) + 0 &= 0 \\
      (b - a) \overset{\text{(A2)}}&{=} 0 \\
    \end{align*}
    Und $b - a = 0$ ist für diesen Fall eine Wahre Aussage, somit ist $b + d = a + c$
  \end{minipage}
  Damit gilt für die Fälle 1 bis 3 $a + c < b + d$ und für den Fall 4 $a + c = b + d$.
  Somit gilt die Aussage $a + c \leq b + d$.

%% c)
\item Für alle $a, b, c \in \mathbb{R}$ gilt $a < b \land c < 0 \Rightarrow ac > bc$
  \begin{align*}
    &\Rightarrow b - a \in P \text{ und } -c \in P \\
    &\overset{\text{(A12)}}\Rightarrow (b - a) * (-c) \\
    &\overset{\text{(A9)}}= b * (-c)  + (-a) * (-c) \\
    &= -(b * c)  + a * c \\
    &= ac - bc \in P \\
    &\Rightarrow bc < ac \\
    &\Rightarrow ac > bc \\
  \end{align*}

  %% d)
\item Für alle $a, b, c \in \mathbb{R}$ gilt $a \leq b \land c < 0 \Rightarrow ac \geq bc$
  \[
    a \leq b \coloneqq b - a \in P \lor b - a = 0
  \]

  \begin{minipage}[t]{.45\textwidth}
    \textbf{Fall 1}: $b - a \in P$: \\
    
    In diesem Falle gilt analog der Teilaufgabe c) 
    $ac > bc$
  \end{minipage}
  \hfill
  \vrule
  \hfill
  \begin{minipage}[t]{.45\textwidth}
    \textbf{Fall 2}: $b - a = 0$ \\
    
    Angenommen es gilt $ a * c - b * c = 0$
    \begin{align*}
      (-a + b) * (-c) \overset{\text{(A9)}}&{=} 0 \\
      (b - a) * (-c) \overset{\text{(A4)}}&{=} 0 \\
    \end{align*}
    Wie in der Vorlesung bewiesen gilt $a * b = 0$ nur, wenn $a = 0$ oder $b = 0$.
    Da $b - a$ in diesem Fall $0$ ist, ist $(b - a) * (-c) = 0$ wahr.
    Somit ist $ac = bc$.
  \end{minipage}

  Aus diesen beiden Fällen ergibt sich, dass $ac \geq bc \forall a \leq b \land c < 0$ 
\end{enumerate}



\end{document}