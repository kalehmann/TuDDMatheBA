\documentclass{article}

\usepackage{aligned-overset}
\usepackage{amsmath}
\usepackage{amssymb}
\usepackage{bbm}
\usepackage[shortlabels]{enumitem}
\usepackage{genealogytree}
\usepackage{hyperref}
\usepackage[utf8]{inputenc}
\usepackage{interval}
\intervalconfig{
  soft open fences
}
\usepackage{mathtools}
\usepackage{pgfplots}
\usepackage{physics}
\usepackage{tikz}
\usetikzlibrary{positioning}
\usepackage{xcolor}
\definecolor{light-gray}{gray}{.9}

\author{Karsten Lehmann}
\date{21.11.2020}
\title{Hausaufgabe 04 Analysis - Grundlegende Konzepte}

\begin{document}


\maketitle
\newpage

\section*{Hausaufgabe 1}

\begin{enumerate}[a)]
\item Beweisen Sie, dass für alle $n \in \mathbb{N}$ gilt:
  $\sum\limits_{k = 1}^n k^2 = \frac{1}{6} n (n + 1) (2n + 1)$.

  \textbf{Beweis durch vollständige Induktion}: \\
  \emph{Behauptung}:
  \[
    P(n) \colon \sum\limits_{k = 1}^n k^2 = \frac{1}{6} n (n + 1) (2n + 1)
  \]
  \emph{Induktionsanfang}: 
  \begin{align*}
    P(1) &\colon \sum\limits_{k = 1}^n k^2 = \frac{1}{6} * 1 * (1 + 1) (2 * 1 + 1) \\
         &\colon 1^2 = \frac{1}{6} * 1 * 2 * 3 \\
         &\colon 1 = \frac{1 * 2 * 3}{6} = \frac{6}{6} = 1 \\
  \end{align*}
  \emph{Induktionsschritt}: Die Behauptung $P(n)$ sei wahr für ein $n \in \mathbb{N}$,
  dann gilt \\
  \begin{align*}
    P(n + 1) \colon \sum\limits_{k = 1}^n k^2 &= \frac{1}{6} * n * (n + 1) (2n + 1) + (n + 1)^2 \\
                                              &= \frac{1}{6} * (n^2 + 1) (2n + 1) + \frac{6n^2 + 12n + 6}{6} \\
                                              &= \frac{(n^2 + n) (2n + 1) + 6n^2 + 12n + 6}{6} \\
                                              &= \frac{2n^3  + 3n^2 + n + 6n^2 + 12n + 6}{6} \\
                                              &= \frac{2n^3 + 9n^2 + 13n + 6}{6} \\
                                              &= \frac{(n + 1)(2n^2 + 7n + 6)}{6} \\
                                              &= \frac{(n + 1)(n + 2)(2n + 3)}{6} \\
                                              &= \frac{1}{6} * (n + 1) (n + 2) (2n + 3)\\
  \end{align*}

  Somit ist die Behauptung wahr für $n + 1$.
  Aus dem Satz über die vollständige Induktion folgt die Behauptung.
  
\item
  \begin{align*}
    2^1 &\nleq 1! \\
    2^2 &\nleq 2! \\
    2^3 &\nleq 3! \\
    2^4 &\leq 4! \\
    2^5 &\leq 5! \\
    \ldots \\
  \end{align*}
  $2^n \leq n!$ gilt für alle $n \geq 4 \in \mathbb{N}$. \\
  \emph{Behauptung}:
  \[
    P(n) \colon 2^n \leq n! 
  \]
  \emph{Induktionsanfang}
  \begin{align*}
    P(4) \colon 2^4 &\leq 4! \\
                 16 &\leq 24 \\
  \end{align*}
  \emph{Induktionsschritt}: Die Behauptung $P(n)$ sei wahr für ein $n \in \mathbb{N}$,
  dann gilt \\
  
  \begin{align*}
    P(n + 1) \colon 2^{n + 1} &\leq (n + 1)! \\
                      2^n * 2 &\leq n! * (n + 1) \\
  \end{align*}
  Für alle $n \geq 4 \in \mathbb{N}$ gilt $2 < (n + 1)$. \\
  
  \begin{minipage}[t]{.45\textwidth}
    \textbf{Fall 1}: $n! - 2^n \in P$: \\
    
    Es folgt
    \[
      2^n * 2 < n! * 2 < n! * (n + 1)
    \]
  \end{minipage}
  \hfill
  \vrule
  \hfill
  \begin{minipage}[t]{.45\textwidth}
    \textbf{Fall 2}: $n! - 2^n = 0$ \\

    Also ist $n! = 2^n$. Man substituiere $n!$ mit $c$
    
    Aus den Folgerungen aus den Anordnungsaxiomen (1.2.3 (e) in der Vorlesung) folgt nun
    
    $2 * c < (n + 1) * c$.
  \end{minipage} \\
  Somit ist die Behauptung wahr für $n + 1$.
  Aus dem Satz über die vollständige Induktion folgt die Behauptung.

\newpage
\item Für $a \in \mathbb{R}$ gilt $s_n = \sum\limits_{k = 1}^n a^{2k - 1} = \frac{a(a^{2n} - 1)}{a^2 - 1}, n \in \mathbb{N}$ \\
  \emph{Behauptung}:
  \[
    P(n) \colon \sum\limits_{k = 1}^n a^{2k - 1} = \frac{a(a^{2n} - 1)}{a^2 - 1} 
  \]
  \emph{Induktionsanfang}:
  \begin{align*}
    P(1) \colon \sum\limits_{k = 1}^1 a^{2k - 1} &= \frac{a(a^{2 * 1} - 1)}{a^2 - 1}  \\
                                   a^{2 * 1 - 1} &= \frac{a(a^2 - 1)}{a^2 - 1}  \\
                                               a &= a * 1 = a \\
  \end{align*}

  \emph{Induktionsschritt}:  Die Behauptung $P(n)$ sei wahr für ein $n \in \mathbb{N}$,
  \begin{align*}
    P(n + 1) \colon \sum\limits_{k = 1}^{n + 1} a^{2k - 1} &= \frac{a(a^{2(n + 1)} - 1)}{a^2 - 1}  \\
                                                           &= \frac{a(a^{2n + 2} - 1)}{a^2 - 1}  \\
                                                           &= \frac{(a^{2n + 3} - a)}{a^2 - 1}  \\   
                                                           &= \frac{a^{2n + 3} - a^{2n + 1} + a^{2n + 1} - a}{a^2 - 1} \\
                                                           &= \frac{a^{2n + 1} - a}{a^2 - 1} + \frac{a^{2n + 3} - a^{2n + 1}}{a^2 - 1}  \\
                                                           &= \frac{a(a^{2n} - 1)}{a^2 - 1} + \frac{a^{2n + 1} * (a^2 - 1)}{a^2 - 1}  \\
                                                           &= \frac{a(a^{2n} - 1)}{a^2 - 1} + a^{2n + 1}  \\
  \end{align*}
  Somit ist die Behauptung wahr für $n + 1$. Aus dem Satz über die vollständige Induktion folgt die Behauptung.
\end{enumerate}

\newpage
\section*{Hausaufgabe 2}

\begin{enumerate}[a)]
\item $\forall n \in \mathbb{N} \colon n - 1 \in \mathbb{N} \cup \{ 0 \} \coloneqq \mathbb{N}_0$ \\
  \label{sec:2_a}
  \emph{Behauptung}: \\
  \[
    P(n) \colon n - 1 \in \mathbb{N} \cup \{ 0 \} 
  \]
  \emph{Induktionsanfang}: \\
  \begin{align*}
    P(1) \colon 1 - 1 &\in \mathbb{N} \cup \{ 0 \} \\
                    0 &\in \mathbb{N} \cup \{ 0 \} \\
  \end{align*}
  $0$ ist in $\{ 0 \}$ enthalten, also ist $P(1)$ wahr. \\
  \emph{Induktionsschritt}: Die Behauptung $P(n)$ sei wahr für ein $n \in \mathbb{N}$,
  dann gilt \\
  \begin{align*}
    (n + 1) - 1    &\in \mathbb{N} \cup \{ 0 \} && (\text{A1}) \\
    n + (1 + (-1)) &\in \mathbb{N} \cup \{ 0 \} && (\text{A3}) \\
    n + 0          &\in \mathbb{N} \cup \{ 0 \} && (\text{A2}) \\
    n              &\in \mathbb{N} \cup \{ 0 \} \\
  \end{align*}
  $n$ ist nach Definition ein Element aus $\mathbb{N}$, also ist die Behauptung wahr für $n + 1$.
  Aus dem Satz über die vollständige Induktion folgt die Behauptung.
  
\item $\forall n, m \in \mathbb{N} \text{ mit } m > n \colon m - n - 1 \in \mathbb{N}_0$
  \emph{Behauptung}: \\
  \[
    Q(n) \colon m - n - 1 \in \mathbb{N}_0 
  \]
  \emph{Induktionsanfang}: \\
  \[
    Q(1) \colon m - 1 - 1 \in \mathbb{N}_0
  \]
  $m - 1$ ist wie in \hyperref[sec:2_a]{2. a)} gezeigt in $\mathbb{N}_0$ enthalten. Aus $m > n$ folgt
  $m - 1 \in P$. Per Definition ist jedes Element von $P$ größer als $0$, es folgt $m - 1 \in \mathbb{N}$.
  Substituiert man nun $m - 1$ in $Q(1)$ mit $m'$ erhält man $Q(1) \colon m' - 1 \in \mathbb{N}_0$, eine wahre Aussage.

  \emph{Induktionsschritt}: Die Behauptung $Q(n)$ sei wahr für ein $n \in \mathbb{N}$,
  dann gilt \\
  \begin{align*}
    m - (n + 1) - 1    &\in \mathbb{N}_0 && (\text{A1}) \\
    m - n + (1 + (-1)) &\in \mathbb{N}_0 && (\text{A3}) \\
    m - n              &\in \mathbb{N}_0 \\
  \end{align*}
  $m$ und $n$ sind ganze Zahlen. Aus der Schule ist bekannt, dass das Ergebnis der Addition oder Subtraktion zweier
  ganzer Zahlen ebenfalls eine ganze Zahl ist, also ist $m - n \in \mathbb{Z}$. Zusätzlich ist $m - n \in P$, also ist
  $m - n > 0$. Die ganzen Zahlen größer $0$ sind definiert als die natürlichen Zahlen, somit ist die Behauptung wahr für
  $n + 1$.
  Aus dem Satz über die vollständige Induktion folgt die Behauptung.

\end{enumerate}

\section*{Hausaufgabe 3}

\emph{Behauptung}: \\
\[
  P(n) \colon \prod\limits_{k = 1}^n (1 + a_k) \geq 1 + \sum\limits_{k = 1}^n a_k
\]
\emph{Induktionsanfang}: \\
\begin{align*}
  P(1) \colon \prod\limits_{k = 1}^1 (1 + a_k) &\geq 1 + \sum\limits_{k = 1}^1 a_k \\
                                     1 + a_k  &\geq 1 + a_k \\
\end{align*}
Die Behauptung $P(1)$ ist offensichtlich wahr.  \\

\noindent
\emph{Induktionsschritt}:   Die Behauptung $P(n)$ sei wahr für ein $n \in \mathbb{N}$. Dann gilt \\

\begin{enumerate}[label={Fall \arabic*}]
\item $\forall k \in \{1, \ldots, n \} \colon a_k \in \interval[open right]{-1}{0}$ \\
  \[
    \prod\limits_{k = 1}^{n+1} (1 + a_k) \geq 1 + \sum\limits_{k = 1}^{n + 1} a_k
  \]

  Annahme: $\prod\limits_{k = 1}^{n+1} (1 + a_k) = 0$: \\
    
  Ein Produkt ist genau dann $0$, wenn einer der Faktoren $0$ ist (Folgerungen aus den Körperaxiomen 1.1.4 (f)).
  $(1 + a_k) = 0$ genau dann, wenn $a_k = -1$ (A3). Somit
  \[
    \exists k \in \{1, \ldots, n \} \colon a_k = -1
  \]
  
  Die linke Seite der Ungleichung ist nun $0$ unabhängig von allen weiteren $a_1, \ldots, a_n$. \\
  
  Betrachtet man die rechte Seite der Gleichung, ist bereits ein $a_1, \ldots, a_n$ mit $-1$ bekannt.
  Da per Intervall alle $a_1, \ldots, a_n$ kleiner als $0$ sind, ist die rechte Seite der Ungleichung
  kleiner oder gleich $0$

  Im folgenden wird nun angenommen $\prod\limits_{k = 1}^{n+1} (1 + a_k) \ne 0$. Dann gilt
  $\forall k \ in \{1, \ldots, n \} \colon (1 - a_k) \in \interval[open]{0}{1} \subseteq P$.

  \begin{align*}
    1                                                                   &\geq \prod\limits_{k = 1}^n (1 + a_k)                     && (\text{Vorlesung 1.2.3 (h)}) \\
    1^{-1}                                                              &\leq (\prod\limits_{k = 1}^n (1 + a_k))^{-1}              && (1 * 1 = 1, 1 = 1^{-1}) \\
    1                                                                   &\leq(\prod\limits_{k = 1}^n (1 + a_k))^{-1}               && | *(a_{n + 1} < 0) \\
    a_{n + 1}                                                           &\geq \frac{a_{n + 1}}{\prod\limits_{k = 1}^n (1 + a_k))}  && | -(\frac{a_{n + 1}}{\prod\limits_{k = 1}^n (1 + a_k))}) \\
    a_{n + 1} - \frac{a_{n + 1}}{\prod\limits_{k = 1}^n (1 + a_k))}     &\geq 0                                                    && | +1 \\
    1 + a_{n + 1} - \frac{a_{n + 1}}{\prod\limits_{k = 1}^n (1 + a_k))} &\geq 1                                                    && | *:(\prod\limits_{k = 1}^n (1 + a_k)) \\
    \prod\limits_{k = 1}^n (1 + a_k) * (1 + a_{n + 1}) - a_{n + 1}      &\geq \prod\limits_{k = 1}^n (1 + a_k)                     && | (P(n) \\
    \prod\limits_{k = 1}^n (1 + a_k) * (1 + a_{n + 1}) - a_{n + 1}      &\geq 1 + \sum\limits_{k = 1}^n a_k                        && | + (a_{n + 1}) \\
    \prod\limits_{k = 1}^n (1 + a_k) * (1 + a_{n + 1})                  &\geq 1 + \sum\limits_{k = 1}^n a_k + a_{n + 1}            && | + (a_{n + 1}) \\
    \prod\limits_{k = 1}^{n + 1} (1 + a_k)                              &\geq 1 + \sum\limits_{k = 1}^{n + 1} a_k \\
  \end{align*}

\newpage
\item $\forall k \in \{1, \ldots, n \} \colon a_k \geq 0$ \\  
  \begin{align*}
    \prod\limits_{k = 1}^{n + 1} (1 + a_k)                                            &\geq 1 + \sum\limits_{k = 1}^{n + 1} a_k       && (\text{Def. Summe und Def. Produkt})\\
    \prod\limits_{k = 1}^n (1 + a_k) * (1 + a_{n + 1})                                &\geq 1 + \sum\limits_{k = 1}^n a_k = a_{n + 1} && (P(n)) \\
    \left(1 + \sum\limits_{k = 1}^n a_k\right) * (1 + a_{n + 1})                      &\geq 1 + \sum\limits_{k = 1}^n a_k + a_{n + 1} \\
    1 + a_{n + 1} + \sum\limits_{k = 1}^n a_k + a_{n + 1} * \sum\limits_{k = 1}^n a_k &\geq 1 + \sum\limits_{k = 1}^n a_k + a_{n + 1} &&|  - \left(1 + a_{n + 1} + \sum\limits_{k = 1}^n a_k\right)\\
    a_{n + 1} * \sum\limits_{k = 1}^n a_k                                             &\geq 0 \\  
  \end{align*}

  \begin{minipage}[t]{.45\textwidth}
    \textbf{Fall 1}: $a_{n + 1} = 0$: \\
    
    Es folgt per den Folgerungen aus den Körperaxiomen (f)
    \[
      a_{n + 1} * \sum\limits_{k = 1}^n a_k = 0
    \]
    Somit ist die Behauptung in diesem Fall wahr.
  \end{minipage}
  \hfill
  \vrule
  \hfill
  \begin{minipage}[t]{.45\textwidth}
    \textbf{Fall 2}: $a_{n + 1} > 0$ \\
    
    Das Axiom (A11) besagt, dass die Summe aus Summanden größer $0$ ebenfalls größer als $0$ ist.
    Da in diesem Fall alle $a_k > 0$, folgt $\sum\limits_{k = 1}^n a_k > 0$
    
    Aus der Definition von $a_k$ und (A12) folgt nun $a_{n + 1} * \sum\limits_{k = 1}^n a_k > 0$ und
    die Behauptung ist wahr.
  \end{minipage} \\
\end{enumerate}

\noindent
Somit ist die Behauptung wahr für $n + 1$ und aus dem Satz über die vollständige Induktion folgt die Behauptung.



\end{document}