\documentclass{article}

\usepackage{aligned-overset}
\usepackage{amsmath}
\usepackage{amssymb}
\usepackage[shortlabels]{enumitem}
\usepackage{genealogytree}
\usepackage{hyperref}
\usepackage[utf8]{inputenc}
\usepackage{interval}
\usepackage{mathtools}
\usepackage{physics}
\usepackage{tikz}
\usetikzlibrary{positioning}
\usepackage{xcolor}
\definecolor{light-gray}{gray}{.9}

\author{Karsten Lehmann}
\date{WiSe 2020}
\title{Mitschriften Analysis - Grundlegende Konzepte (AN10)}

\begin{document}


\maketitle

\vfill
\begin{center}
  Dozent: Ralph Chill \\
  \href{mailto:ralph.chill@tu-dresden.de}{ralph.chill@tu-dresden.de}
\end{center}

\newpage

\section{Die reellen Zahlen}

\subsection{Körperaxiome}

Auf der Menge der reellen Zahlen $\mathbb{R}$ sind die folgenden beiden Verknüpfungen definiert: \\

\begin{minipage}{.45\textwidth}
  \textbf{Addition}
  \begin{align*}
    \mathbb{R} \times \mathbb{R} &\to \mathbb{R} \\
    (a,b) &\mapsto a + b \\
  \end{align*}
\end{minipage}
\hfill
\vrule
\hfill
\begin{minipage}{.45\textwidth}
  \textbf{Multiplikation}
  \begin{align*}
    \mathbb{R} \times \mathbb{R} &\to \mathbb{R} \\
    (a,b) &\mapsto a * b \\
  \end{align*}
\end{minipage}

Diese beiden Verknüpfungen sind so definiert, dass folgende Eigenschaften gelten:

\begin{enumerate}[label=(A\arabic*)]
\item \textbf{Assoziativgesetz}
  \[
    \forall a,b,c \in \mathbb{R} \colon (a + b) + c = a + (b + c)
  \]
\item \textbf{Existenz eines neutralen Elements}
  \[
    \exists 0 \in \mathbb{R} \forall a \in \mathbb{R} \colon a + 0 = a
  \]
  \textbf{Achtung!!} hier ist nicht unbedingt die Zahl 0 gemeint.
\item \textbf{Existenz eines Inversen}
  \[
    \forall a \in \mathbb{R} \exists -a \in \mathbb{R} \colon a + (-a) = 0
  \]
\item \textbf{Kommutativgesetz}
  \[
    \forall a,b \in \mathbb{R} \colon a + b = b + a
  \]
\item \textbf{Assoziativgesetz}
  \[
    \forall a,b,c \in \mathbb{R} \colon (a * b) * c = a * (b * c)
  \]
\item \textbf{Existenz eines neutralen Elements}
  \[
    \exists 1 \in \mathbb{R} \forall a \in \mathbb{R} \colon a * 1 = a
  \]
  \textbf{Achtung!!} hier ist nicht unbedingt die Zahl 1 gemeint.
\item \textbf{Existenz eines Inversen}
  \[
    \forall a \in \mathbb{R}, a \ne 0 \exists a^{-1} \in \mathbb{R} \colon a * a^{-1} = 1
  \]
\item \textbf{Kommutativgesetz}
  \[
    \forall a,b \in \mathbb{R} \colon a * b = b * a
  \]
\item \textbf{Distributivgesetz}
  \[
    \forall a,b,c \in \mathbb{R} \colon (a + b) * c = a * c + b * c
  \]
\end{enumerate}

\subsubsection{Definition}

Ein Tripel $(K,+,*)$ bestehend aus einer Menge $K$ und den Verknüpfungen $+$ und $*$, so dass
die Eigenschaften (A1) bis (A9) gelten, heißt Körper. \\

Also ist $(\mathbb{R},+,*)$ (oder kurz $\mathbb{R}$) ein Körper.
Es gibt auch andere Körper, zum Beispiel den Körper der rationalen Zahlen $\mathbb{Q}$ und den Körper
der komplexen Zahlen $\mathbb{C}$.
Zusätzlich gibt es sogar endliche Körper, zum Beispiel der kleinste mögliche Körper

\[
  F_2 = \{ 0, 1 \}
\]

Für diesen Körper sind Addition und Multiplikation folgendermaßen definiert: \\

\begin{minipage}{.45\textwidth}
  \textbf{Addition} \\
  
  \begin{tabular}{ c | c c }
    + & 0 & 1 \\
    \hline
    0 & 0 & 1 \\
    1 & 1 & 0 \\
  \end{tabular}
\end{minipage}
\hfill
\vrule
\hfill
\begin{minipage}{.45\textwidth}
  \textbf{Multiplikation} \\
  
  \begin{tabular}{ c | c c }
    * & 0 & 1 \\
    \hline
    0 & 0 & 0 \\
    1 & 0 & 1 \\
  \end{tabular}
\end{minipage}
\\

Dieser Körper macht unter anderem dann Sinn, wenn man für die $0$ alle geraden Zahlen und für die $1$
alle ungeraden Zahlen einsetzt. \\

Auch $F_3 = \{ 0, 1, 2 \}$ ist ein Körper mit \\

\begin{minipage}{.45\textwidth}
  \textbf{Addition} \\
  
  \begin{tabular}{ c | c c c }
    + & 0 & 1 & 2 \\
    \hline
    0 & 0 & 1 & 2 \\
    1 & 1 & 2 & 0 \\
    2 & 2 & 0 & 1 \\
  \end{tabular}
\end{minipage}
\hfill
\vrule
\hfill
\begin{minipage}{.45\textwidth}
  \textbf{Multiplikation} \\
  
  \begin{tabular}{ c | c c c}
    * & 0 & 1 & 2 \\
    \hline
    0 & 0 & 0 & 0 \\
    1 & 0 & 1 & 2 \\
    2 & 0 & 2 & 1 \\
  \end{tabular}
\end{minipage}
\\

Dieser Körper macht unter anderem dann Sinn, wenn man für die $0$ allen Zahlen einsetzt, die durch $3$
teilbar sind, für die $1$ alle Zahlen, für die gilt $x \in \mathbb{R}, x \mod 3 = 1$ und für die $2$
alle Zahlen, für die gilt $x \in \mathbb{R}, x \mod 3 = 2$.

Es existieren noch weitere endliche Körper, zum Beispiel mit 4, 5, 7, 8 und 9 Elementen.
Ein Körper mit 6 Elementen existiert nicht.

Die Menge der ganzen Zahlen $\mathbb{Z}$ ist kein Körper, da dass 7. Axiom (\textbf{Existenz eines Inversen})
nicht erfüllt ist.

%% Ich habe eine Section aus dem Skript ausgelassen, will aber die Nummerierung behalten
\addtocounter{subsubsection}{1}

\subsubsection{Konventionen und Schreibweisen}

``$\coloneqq$'' heißt ``ist per Definition gleich''
\begin{align*}
  ab &\coloneqq a * b \\
  2a &\coloneqq a + a \\
  a^2 &\coloneqq a * a \\
  a - b &\coloneqq a + (-b) \\
  \frac{a}{b} &\coloneqq a * (b^{-1}) \\
  a + b + c &\coloneqq (a + b) + c &\text{beachte (A1)} \\
  abc &\coloneqq a * b * c &\text{beachte (A5)} \\
\end{align*}

\subsubsection{Lemma (Folgerungen aus den Körperaxiomen)}

Die Folgerungen aus den Körperaxiomen gelten in jedem Körper.

\begin{enumerate}[(a)]
\item Die neutralen Elemente $0$ und $1$ sind eindeutig bestimmt.
\item Die inversen Elemente sind eindeutig bestimmt.
\item Für jedes $a,b \in \mathbb{R}$ gibt es genau ein $x \in \mathbb{R}$, so dass $a + x = b$
\item Für jedes $a,b \in \mathbb{R}, a \ne 0$ gibt es genau ein $x \in \mathbb{R}$, so dass
  $a * x = b$
\item $-(-a) = a$ und falls $a \ne 0$, dann gilt auch $(a^{-1})^{-1} = a$
\item $a * b = 0$ genau dann, wenn $a = 0$ oder $b = 0$
\item $-(a + b) = (-a) + (-b)$ und falls $a \ne 0, b \ne 0$ gilt $(a * b)^{-1} = a^{-1} * b^{-1}$
\item $-(a*b) = (-a) * b = a * (-b)$ und falls $a \ne 0$, dann $(-a)^{-1} = -a^{-1}$
\item Binomische Formeln
  \[
    (a + b)^2 = a^2 + 2ab + b^2
  \]
  \[
    (a - b)^2 = a^2 - 2ab + b^2
  \]
  \[
    (a + b) * (a - b) = a^2 - b^2
  \]
\item Bruchrechenregeln:
  \begin{align*}
    \frac{a + b}{c} &= \frac{a}{c} + \frac{b}{c} \\
    \frac{a}{b} + \frac{c}{d} &= \frac{a * d + b * c}{b * d} \\
    \frac{a}{b} * \frac{c}{d} &= \frac{a * c}{b * d} \\
    \frac{\frac{a}{b}}{\frac{c}{d}} &= \frac{a * d}{b * c} \\
  \end{align*}
\end{enumerate}

\textbf{Beweis} unter Verwendung der Axiome 1 bis 9.

\begin{enumerate}[(a)]
\item Sein $0$ und $0^{\prime}$ zwei neutrale Elemente der Addition, dann gilt
  \[
    0 \underset{\substack{0^{\prime} \text{ ist ein} \\ \text{neutrales Element}}}=
    0 + 0^{\prime} \overset{\text{(A4)}}=
    0^{\prime} + 0 \underset{\substack{0 \text{ ist ein} \\ \text{neutrales Element}}}= 0^{\prime}
  \]
  \\
  Daraus folgt die Eindeutigkeit des neutralen Elementes der Addition.
  \[
    0 = 0^{\prime}
  \]
  \\
  Für das neutrale Element der Multiplikation geht der Beweis analog.
\item Sei $a \in \mathbb{R}$. Seien $-a \in \mathbb{R}$ und $b \in \mathbb{R}$ zwei inverse Elemente.
  Dann gilt
  \begin{align*}
    -a &\underset{\substack{0 \text{ ist ein} \\ \text{neutrales Element}}}=-a + 0 \\
     &\underset{
       \substack{
         \text{Die } 0 \text{ lässt sich auch als Summe aus} \\
         a \text{ und dem inversen Elemen } b \text{ schreiben}
       }
    }= (-a) + (a + b) \\
    &\overset{\text{(A1)}}= ((-a) + a) + b \\
    &\overset{\text{(A4)}}= (a + (-a)) + b \\
    &\underset{
      \substack{
        -a \text{ ist} \\
        \text{inverses} \\
        \text{Element}
      }
    }= 0 + b \\
    &\overset{\text{(A4)}}= b + 0 \\
    -a &\underset{
      \substack{
        0 \text{ ist} \\
        \text{neutrales} \\
        \text{Element}
      }
    }= b \\   
  \end{align*}
  \\
  Daraus folgt, es gibt genau ein inverses Element der Addition.
\item Existenz eines $x \in \mathbb{R}$ für welches gilt $a + x = b$
  Sei $x = b - a$ (oder $x = b + (-a)$)
  Dann gilt
  \begin{align*}
    a + x &= a + (b + (-a)) \\
          &\overset{\text{(A4)}}= a + ((-a) + b) \\
          &\overset{\text{(A1)}}= (a + (-a)) + b \\
          &= 0 + b \\
          &\overset{\text{(A4)}}= b + 0 \\
    a + x &= b 
  \end{align*}

  Daraus folgt, dass \textbf{mindestens} ein $x \in \mathbb{R}$ existiert, für welches gilt $a + x = b$.
  Nun soll gezeigt werden, dass auch nur \textbf{genau ein} $x$ existiert, für welches diese Behauptung
  gilt. \\

  Sei $y$ ein weiteres Element, so dass $a + y = b$ und $x = b - a$.
  Dann gilt

  \begin{align*}
    x &= b + (-a) \\
      &= (a + y) + (-a) \\
      &\overset{\text{(A4)}}= (y + a) + (-a) \\
      &\overset{\text{(A1)}}= y + (a + (-a)) \\
      &= y + 0 \\
    x &= y \\
  \end{align*}
\item analog zu (c)
\item Es gilt
  \[
    (-a) + a = a + (-a) = 0
  \]
  Also ist $a$ ein inverses Element von $-a$ und aus (b) folgt, dass $a$ das einzige inverse Element von $-a$
  ist, somit folgt
  \[
    a = -(-a)
  \]
\item \emph{``$\Leftarrow$''} Zuerst wird bewiesen, dass eine Zahl multipliziert mit Null auch Null ergibt.
  Für alle $a \in \mathbb{R}$ gilt
  \begin{align*}
    0 &= a * 0 - a * 0 \\
      &= a * (0 + 0) - a * 0 \\
      &\overset{\text{(A9)}}= (a * 0 + a * 0) - a * 0 \\
      &\overset{\text{(A1)}}= a * 0 + (a * 0 - a * 0) \\
      &= a * 0 + 0 \\
    0 &= a * 0 \\
  \end{align*} \\
  \emph{``$\Rightarrow$''} \\
  \begin{minipage}[t]{.4\textwidth}
    \textbf{1. Fall} $a = 0$ \\
    Dann ist die Aussage \emph{``$a = 0$ oder $b = 0$''}natürlich wahr.
  \end{minipage}
  \hfill
  \vrule
  \hfill
  \begin{minipage}[t]{.4\textwidth}
    \textbf{2. Fall} $a \ne 0$. Wenn $a$ nicht Null ist, muss bewiesen werden, dass zwangsläufig $b = 0$ ist.
    \begin{align*}
      b &= a * 1 \\
        &= b * (a * a^{-1}) \\
        &\overset{\text{(A5)}}= (b * a) * a^{-1} \\
        &= 0 * a^{-1} \\
        &\overset{\text{(A8)}}= a^{-1} * 0 \\
      b &= 0 \\
        &\Rightarrow \text{\emph{"a = 0 oder b = 0"}}
    \end{align*}
  \end{minipage}
\item
  \begin{align*}
    (a * b) * (a^{-1} * b^{-1}) &\overset{\text{(A8)}}= (a * b) * (b^{-1} * a^{-1}) \\
                                &\overset{\text{(A5)}}= a * (b * b^{-1}) * a^{-1} \\
                                &= a * 1 * a^{-1} \\
                                &= a * a^{-1} \\
    (a * b) * (a^{-1} * b^{-1})  &= 1\\
  \end{align*}
  Also ist $(a^{-1} * b^{-1})$ ein multiplikatives inverses Element von $a * b$.
  Aus (b) (\emph{``Es gibt nur genau ein multiplikatives inverses}'') folgt:
  \[
    a^{-1} b^{-1} = (a * b)^{-1}
  \]
\item
  \begin{align*}
    a * b + a * (-b) &\overset{\text{(A9)}}= a * (b + (-b)) \\
                     &= a * 0 \\
                     &\overset{\text{(f)}}= 0 \\
  \end{align*}
  Also ist $a * (-b)$ ein additives Inverses von $a * b$.
  Aus (b) folgt
  \[
    -(a * b) = a * (-b)
  \]
\item
  \begin{align*}
    (a + b)^2 &\underset{\overset{\Uparrow}{\text{Per Konvention}}}= (a + b) * (a + b) \\
              &\overset{\text{(A9)}}= a * (a + b) + b * (a + b) \\
              &\overset{\text{(A9)}}= a * a + a * b + b * a + b * b \\
              &\overset{\text{(A8)}}= a * a + a * b + a * b + b * b \\
    (a + b)^2 &\underset{\overset{\Uparrow}{\text{Per Konvention}}}= a^2 + 2ab + b^2 \\    
  \end{align*}
  Für die anderen binomischen Formeln sieht das ganze analog aus.
\item Bitte nachrechnen
\end{enumerate}

\subsection{Die Anordnungsaxiome}

In der Menge $\mathbb{R}$ der reellen Zahlen gibt es eine ausgezeichnete Teilmenge $P \subseteq R$ mit den folgenden
Eigenschaften:

\begin{enumerate}[label=(A\arabic*)]
%% Fortführung der Nummerierung der Axiome
\setcounter{enumi}{9}
\item \textbf{Trichotomiegesetz}: Für jede reelle Zahl $a$ gilt genau eine der folgenden Aussagen:
  \[
    a = 0 \text{ oder } a \in P \text{ oder } -a \in P
  \]
\item $\forall a,b \in P \colon a + b \in P$
\item $\forall a,b \in P \colon a * b \in P$
\end{enumerate}

\subsubsection{Definition}

Ein Körper $(K, +, -)$, in dem es eine Teilmenge $P$ gibt, so dass die Anordnungsaxiome (A10) - (A12) gelten
heißt \underline{angeordneter Körper}. \\

Also ist $\mathbb{R}$ ein angeordneter Körper, aber auch der Körper $\mathbb{Q}$ ist angeordnet. \\

Der Körper $F_2$ kann nicht angeordnet werden, denn $1 = -1$ und somit kann das Trichotomiegesetz (A10) nicht
gelten. Genau so ist auch $F_3$ kein angeordneter Körper.

\subsubsection{Konventionen und Schreibweisen}

\begin{align*}
  \text{Anstelle von $a \in P$ schreibt man $a > 0$} \\
  a < b &\coloneqq b - a \in P \\
  a > b &\coloneqq a - b \in P \\
  a < b < c &\coloneqq a < b \land b < c \\
  a \leq b &\coloneqq a < b \lor a = b \\
  a \geq b &\coloneqq a = b \lor a > b \\
\end{align*}

\subsubsection{Lemma Folgerungen aus den Anordnungsaxiomen}

Diese gelten in jedem angeordneten Körper

\begin{enumerate}[(a)]
\item \textbf{Trichotomiegesetz}: Für zwei reelle Zahlen $a, b \in \mathbb{R}$ gilt genau eine der beiden folgenden
  Aussagen:

  \[
    a = b \text{ oder } a > b \text{ oder } a < b
  \]
\item \textbf{Transitivität}: Wenn $a < b$ und $b < c$, dann gilt
  \[
    a < c
  \]
\item Für $a < b$ und $c \in \mathbb{R}$ gilt
  \[
    a + c < b + c
  \]
\item Für $a < b$ und $c < d$ gilt
  \[
    a + c < b + d
  \]
\item Für $a < b$ und $c > 0$ gilt:
  \[
    a * c < b * c
  \]
\item Für $a < b$ und $c < 0$ gilt:
  \[
    a * c > b * c
  \]
\item Wenn $a \ne 0$, dann ist $a^2 > 0$. Insbesondere ist $1 > 0$.
\item Wenn $a > 0$, dann ist auch $a^{-1} > 0$. Genau so gilt:
  \[
    0 < a < b \Rightarrow 0 < b^{-1} < a^{-1}
  \]
\item $a < b \Rightarrow a < \frac{a + b}{2} < b (2 \coloneqq 1 + 1)$ 
\end{enumerate}

\textbf{Beweis}
\begin{enumerate}[(a)]
\item Wende das Trichotomiegesetz (A10) auf $b - a$ an, dass heißt es gilt
  \[
    b - a = 0 \text{ oder } b - a \in P \text{ oder } -(b - a) \in P
  \]
\item Es gelte $a < b$ und $b < c$, dass heißt $b - a \in P$ und $c - b \in P$
  \begin{align*}
    &\overset{\text{(A11)}}\Rightarrow c - b + b - a \in P \\
    &= c - a \in P \\
    &\Rightarrow c > a \lor a < c \\
  \end{align*}
\item Es gelte $a < b$, dass heißt $b - a \in P$. Dann gilt für $c \in \mathbb{R}$
  \begin{align*}
    P &\ni b + 0 - a \\
      &\ni b + (c + (-c)) - a \\
      &\ni (b + c) - c - a \\
      &\ni (b + c) - (a + c) \\
      &\Rightarrow a + c < b + c
  \end{align*}
\item Es gelte $a < b$ und $c < d$, dann gilt
  \begin{align*}
    a + c &< b + c &\text{ (wegen (c))} \\
    \text{und } c + b &< d + b &\text{ (wegen (c))} \\
    \overset{\text{(b)}}\Rightarrow a + c &< b + d \\ 
  \end{align*}
\item Es gelte $a < b$ und sei $c > 0$, dann gilt $b - a \in P$ und $c \in P$.
  \begin{align*}
    \overset{\text{(A12)}}\Rightarrow P &\ni (b -a) * c \\
                                        &\overset{\text{(A9)}}\ni b * c - a * c \\
                                        &\Rightarrow a * c < b * c \\
  \end{align*}
\item $a < b$ und $c < 0$
  \begin{align*}
    &\Rightarrow b - a \in P \text{ und } -c \in P \\
    &\overset{\text{(A12)}}\Rightarrow (b - a) * (-c) \\
    &\overset{\text{(A9)}}= b * (-c)  + (-a) * (-c) \\
    &= -(b * c)  + a * c \\
    &= ac - bc \in P \\
    &\Rightarrow bc < ac \\
    &\Rightarrow ac > bc \\
  \end{align*}
\item
  \begin{minipage}[t]{.4\textwidth}
    \textbf{1. Fall}: $a > 0$ \\

    \[
      a^2 = a * a \overset{\text{(A12)}}> 0
    \]
  \end{minipage}
  \hfill
  \vrule
  \hfill
  \begin{minipage}[t]{.4\textwidth}
    \textbf{2. Fall}: $a < 0$ \\

    Dann ist nach Aussage (f)

    \[
      a * a = a^2 > 0 * a = 0
    \]
  \end{minipage}  \\
  Insebsondere ist $1 = 1^2 > 0$
\item Übung
\item Es gilt $a < b$ Weiterhin ist $1 > 0$ wegen Aussage (g) und damit (siehe (A11)) $2 = 1 + 1 > 0$ \\

  Dann gilt
  \begin{align*}
    a &= 1 * a = \frac{(1 + 1) * a}{2} \\
      &overset{\text{(A9)}}= \frac{a + a}{2} < \frac{a+b}{2} \\
      &< \frac{b+b}{2}  = b \\
    a &< \frac{a + b}{2} < b \\ 
  \end{align*}
\end{enumerate}

\subsubsection{Korollar}

\fcolorbox{black}{light-gray}{
  \begin{minipage}{\textwidth}
    Korollar: Substantiv, Neutrum \\

    \emph{Satz, der aus einem bewiesenen Satz folgt, von ihm abgeleitet wird}
  \end{minipage}
}

Zu je zwei reellen Zahlen $a,b \in \mathbb{R}$ gibt es ein $c \in \mathbb{R}$ mit $a < c < b$

\textbf{Beweis}:

\[
  c = \frac{a + b}{2}
\]

Daraus folgt, dass ein angeordneter Körper unendlich viele Elemente enthält.
Somit kann ein endlicher Körper kein angeordneter Körper sein.

\subsubsection{Definition}

Für jede reelle Zahl $a \in \mathbb{R}$ definieren wir das \emph{Signum} (bzw. Vorzeichen)

\[
  \text{sgn}(a) \coloneqq \begin{cases}
    1  & \text{falls } a > 0 \\
    0  & \text{falls } a = 0 \\
    -1 & \text{falls } a < 0 \\
  \end{cases}
\]

und den \emph{Absolutbetrag}

\[
  \text{abs}(a) \coloneqq \abs{a} \coloneqq \begin{cases}
    a  & \text{falls } a > 0 \\
    0  & \text{falls } a = 0 \\
    -a & \text{falls } a < 0 \\
  \end{cases}
\]

Es gilt:

\[
  \abs{a} = \text{sgn}(a) * a
\]

\subsubsection{Satz}

Es gilt:

\begin{enumerate}[(a)]
\item $\abs{a} = \abs{-a}$
\item $\abs{a} \geq 0$ und ($\abs{a} = 0 \iff a = 0$)
\item $\abs{a * b} = \abs{a} * \abs{b}$
\item ($a \leq b$ und $-a \leq b$) wenn $\abs{a} \leq b$
\item $a = 0$ genau dann wenn $\forall \epsilon > 0 \colon \abs{a} < \epsilon$
\end{enumerate}

\textbf{Beweis}:

\begin{enumerate}[(a)]
\item Bemerke, dass $\text{sgn}(-a) = -\text{sgn}(a)$ und somit gilt
  \begin{align*}
    \abs{-a} = \text{sng}(-a) * (-a) &= (-\text{sgn}(a)) * (-a) \\
                                     &= \text{sgn}(a) * a \\
                                     &= \abs{a}
  \end{align*}
  
\item $\abs{a} \geq 0$ (Man betrachte die Fälle $a > 0$, $a = 0$ und $a < 0$). Sodann gilt:
  \[
    a = 0 \overset{text{Def}}\Rightarrow \abs{a} = 0
  \]
  Umgekehrt gilt:
  \begin{align*}
    \abs{a} = 0 &\Rightarrow \text{sgn}(a) * a = 0 \\
    \overset{\text{Folgerung aus den Körperaxiomen (f)}}&{\Rightarrow} \text{sgn}(a) = 0 \text{ oder } a = 0\\
                &\Rightarrow a = 0
  \end{align*}
    
\item $\abs{a * b} = \abs{a} * \abs{b}$ \\
  \begin{tabular}{ c | c | c | c | c | c | c }
    $a$ & $b$ & $a * b$ & $\text{sgn}(a)$ & $\text{sgn}(b)$ & $\text{sng}(a) * \text{sgn}(b)$ & $\text{sgn}(a * b)$ \\
    \hline
    -1 & -1 & 1  & -1 &  -1 & 1  & 1  \\
    -1 & 0  & 0  & -1 &  0  & 0  & 0  \\
    -1 & 1  & -1 & -1 &  1  & -1 & -1 \\
    0  & -1 & 0  & 0  &  -1 & 0  & 0  \\
    0  & 0  & 0  & 0  &  0  & 0  & 0  \\
    0  & 1  & 0  & 0  &  1  & 0  & 0  \\
    1  & -1 & -1 & 1  &  -1 & -1 & -1 \\
    1  & 0  & 0  & 1  &  0  & 0  & 0  \\
    1  & 1  & 1  & 1  &  1  & 1  & 1  \\
  \end{tabular}
  
  \begin{align*}
    \abs{a * b} &= \text{sgn}(a * b) * (a * b) \\
                &= \text{sgn}(a) * \text{sgn}(b) * a * b \\
                &= \text{sgn}(a) * a * \text{sgn}(b) * b \\
                &= \abs{a} * \abs{b} \\
  \end{align*}
\item ($a \leq b$ und $-a \leq b$) wenn $\abs{a} \leq b$ \\
  folgt aus $\abs{a} = a$ (falls $a \geq 0$) oder $\abs{a} = -a$ (falls $a < 0$)
  
\item $a = 0$ genau dann wenn $\forall \epsilon > 0 \colon \abs{a} < \epsilon$ \\
  ``$\Rightarrow$" $a = 0 \Rightarrow \forall \epsilon > 0 \colon \epsilon > 0 = \abs{a}$ \\
  ``$\Leftarrow$" Es gelte $\forall \epsilon > 0 \colon \abs{a} < \epsilon$ Angenommen $a \ne 0$, dann
  ist $\abs{a} > 0$ (wegen Aussage (b)). Aber dann gilt für $\epsilon \coloneqq \abs{a}$ $(>0)$ nicht $\abs{a} < \epsilon$.
  Somit kann $a \ne 0$ nicht gelten, d.h. $a = 0$.
\end{enumerate}

\subsubsection{Satz (Dreiecksungleichung)}

Für zwei reelen Zahlen $a,b \in \mathbb{R}$ gilt
\begin{enumerate}[(a)]
\item $\abs{a + b} \leq \abs{a} + \abs{b}$
\item $\abs{\abs{a} - \abs{b}}  \leq \abs{a - b}$
\end{enumerate}

\noindent
Warum sind dies die Dreiecksungleichungen? Im Dreieck ist die Länge einer Seite immer kleiner als die Summe der Längen
der beiden anderen Seiten.

\textbf{Beweis}:
\begin{enumerate}[(a)]
\item
  \begin{align*}
    \abs{a + b} &= \text{sgn}(a + b) * (a + b) \\
                &= \text{sgn}(a + b) * a + \text{sgn}(a + b) * b \\
                &\text{Man kann abschätzen, dass die Summe der Beträge größer oder gleich dem Betrag $a + b$ ist} \\
                &\leq \abs{\text{sgn}(a + b)} * \abs{a} + \abs{\text{sgn}(a + b)} * \abs{b} \\
                &\text{Man kann weiterhin den Betrag des Vorzeichens von $a + b$ auf $1$ abschätzen} \\
                &\leq 1 * \abs{a} + 1 * \abs{b} \\
                &\leq \abs{a} + \abs{b} \\
  \end{align*}
\item $\abs{\abs{a} - \abs{b}}  \leq \abs{a - b}$
  \begin{align*}
    \abs{a} &= \abs{a + 0} \\
            &= \abs{a + ((-b) + b)} \\
            &= \abs{(a - b) + b} \\
    \overset{\text{wegen (a)}}&{\leq} \abs{a - b} + \abs{b} \\
  \end{align*}
  bzw.
  \[
    \abs{a} - \abs{b} \leq \abs{a - b}
  \]
  Vertausche die Rollen von $a$ und $b$
  \begin{align*}
    &\Rightarrow \underbrace{\abs{b} - \abs{a}}_{-(\abs{a} - \abs{b})} \leq \abs{b - a} = \abs{a - b} \\
    \overset{\text{1.2.6 (d)}}&{\Rightarrow} \abs{\abs{a} * \abs{b}} \leq \abs{a * b} \\
  \end{align*}
\end{enumerate}

\subsubsection{Definition (Intervalle)}

Für zwei reelle Zahlen $a,b \in \mathbb{R}$ heißt
\begin{align*}
  (a,b) \coloneqq &\interval[open]{a}{b}       &\coloneqq \{ x \in \mathbb{R} | a < x < b \}       &\text{ offenes Intervall} \\
                  &\interval{a}{b}             &\coloneqq \{ x \in \mathbb{R} | a \leq x \leq b \} &\text{ abgeschlossenes Intervall} \\
  [a,b) \coloneqq &\interval[open right]{a}{b} &\coloneqq \{ x \in \mathbb{R} | a \leq x < b \}    &\text{ halboffenes Intervall} \\
  (a,b] \coloneqq &\interval[open left]{a}{b}  &\coloneqq \{ x \in \mathbb{R} | a < x \le b \}     &\text{ halboffenes Intervall} \\
\end{align*}

Diese vier genannten Intervalle sind alle beschränkte Intervalle. Wir definieren außerdem:

\begin{align*}
\interval[open]{a}{+\infty}       &= \{ x \in \mathbb{R} | x > a \} \\
\interval[open right]{a}{+\infty} &= \{ x \in \mathbb{R} | x  \geq a \} \\
\interval[open]{-\infty}{a}       &= \{ x \in \mathbb{R} | x < a \} \\
\interval[open left]{-\infty}{a}  &= \{ x \in \mathbb{R} | x \leq a \} \\
\interval[open]{a}{\infty}        &= \mathbb{R} \\
\end{align*}

Diese Intervalle sind unbeschränkte Intervalle.

\end{document}