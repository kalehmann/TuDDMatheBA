\documentclass{article}

\usepackage{amsmath}
\usepackage{amssymb}
\usepackage[shortlabels]{enumitem}
\usepackage{genealogytree}
\usepackage{hyperref}
\usepackage[utf8]{inputenc}
\usepackage{mathtools}
\usepackage{tikz}
\usetikzlibrary{positioning}
\usepackage{xcolor}
\definecolor{light-gray}{gray}{.9}

\author{Karsten Lehmann}
\date{WiSe 2020}
\title{Mitschriften Analysis - Grundlegende Konzepte (AN10)}

\begin{document}


\maketitle

\vfill
\begin{center}
  Dozent: Ralph Chill \\
  \href{mailto:ralph.chill@tu-dresden.de}{ralph.chill@tu-dresden.de}
\end{center}

\newpage

\section{Die reellen Zahlen}

\subsection{Körperaxiome}

Auf der Menge der reellen Zahlen $\mathbb{R}$ sind die folgenden beiden Verknüpfungen definiert: \\

\begin{minipage}{.45\textwidth}
  \textbf{Addition}
  \begin{align*}
    \mathbb{R} \times \mathbb{R} &\to \mathbb{R} \\
    (a,b) &\mapsto a + b \\
  \end{align*}
\end{minipage}
\hfill
\vrule
\hfill
\begin{minipage}{.45\textwidth}
  \textbf{Multiplikation}
  \begin{align*}
    \mathbb{R} \times \mathbb{R} &\to \mathbb{R} \\
    (a,b) &\mapsto a * b \\
  \end{align*}
\end{minipage}

Diese beiden Verknüpfungen sind so definiert, dass folgende Eigenschaften gelten:

\begin{enumerate}[label=(A\arabic*)]
\item \textbf{Assoziativgesetz}
  \[
    \forall a,b,c \in \mathbb{R} \colon (a + b) + c = a + (b + c)
  \]
\item \textbf{Existenz eines neutralen Elements}
  \[
    \exists 0 \in \mathbb{R} \forall a \in \mathbb{R} \colon a + 0 = a
  \]
  \textbf{Achtung!!} hier ist nicht unbedingt die Zahl 0 gemeint.
\item \textbf{Existenz eines Inversen}
  \[
    \forall a \in \mathbb{R} \exists -a \in \mathbb{R} \colon a + (-a) = 0
  \]
\item \textbf{Kommutativgesetz}
  \[
    \forall a,b \in \mathbb{R} \colon a + b = b + a
  \]
\item \textbf{Assoziativgesetz}
  \[
    \forall a,b,c \in \mathbb{R} \colon (a * b) * c = a * (b * c)
  \]
\item \textbf{Existenz eines neutralen Elements}
  \[
    \exists 1 \in \mathbb{R} \forall a \in \mathbb{R} \colon a * 1 = a
  \]
  \textbf{Achtung!!} hier ist nicht unbedingt die Zahl 1 gemeint.
\item \textbf{Existenz eines Inversen}
  \[
    \forall a \in \mathbb{R}, a \ne 0 \exists a^{-1} \in \mathbb{R} \colon a * a^{-1} = 1
  \]
\item \textbf{Kommutativgesetz}
  \[
    \forall a,b \in \mathbb{R} \colon a * b = b * a
  \]
\item \textbf{Distributivgesetz}
  \[
    \forall a,b,c \in \mathbb{R} \colon (a + b) * c = a * c + b * c
  \]
\end{enumerate}

\subsubsection{Definition}

Ein Tripel $(K,+,*)$ bestehend aus einer Menge $K$ und den Verknüpfungen $+$ und $*$, so dass
die Eigenschaften (A1) bis (A9) gelten, heißt Körper. \\

Also ist $(\mathbb{R},+,*)$ (oder kurz $\mathbb{R}$) ein Körper.
Es gibt auch andere Körper, zum Beispiel den Körper der rationalen Zahlen $\mathbb{Q}$ und den Körper
der komplexen Zahlen $\mathbb{C}$.
Zusätzlich gibt es sogar endliche Körper, zum Beispiel der kleinste mögliche Körper

\[
  F_2 = \{ 0, 1 \}
\]

Für diesen Körper sind Addition und Multiplikation folgendermaßen definiert: \\

\begin{minipage}{.45\textwidth}
  \textbf{Addition} \\
  
  \begin{tabular}{ c | c c }
    + & 0 & 1 \\
    \hline
    0 & 0 & 1 \\
    1 & 1 & 0 \\
  \end{tabular}
\end{minipage}
\hfill
\vrule
\hfill
\begin{minipage}{.45\textwidth}
  \textbf{Multiplikation} \\
  
  \begin{tabular}{ c | c c }
    * & 0 & 1 \\
    \hline
    0 & 0 & 0 \\
    1 & 0 & 1 \\
  \end{tabular}
\end{minipage}
\\

Dieser Körper macht unter anderem dann Sinn, wenn man für die $0$ alle geraden Zahlen und für die $1$
alle ungeraden Zahlen einsetzt. \\

Auch $F_3 = \{ 0, 1, 2 \}$ ist ein Körper mit \\

\begin{minipage}{.45\textwidth}
  \textbf{Addition} \\
  
  \begin{tabular}{ c | c c c }
    + & 0 & 1 & 2 \\
    \hline
    0 & 0 & 1 & 2 \\
    1 & 1 & 2 & 0 \\
    2 & 2 & 0 & 1 \\
  \end{tabular}
\end{minipage}
\hfill
\vrule
\hfill
\begin{minipage}{.45\textwidth}
  \textbf{Multiplikation} \\
  
  \begin{tabular}{ c | c c c}
    * & 0 & 1 & 2 \\
    \hline
    0 & 0 & 0 & 0 \\
    1 & 0 & 1 & 2 \\
    2 & 0 & 2 & 1 \\
  \end{tabular}
\end{minipage}
\\

Dieser Körper macht unter anderem dann Sinn, wenn man für die $0$ allen Zahlen einsetzt, die durch $3$
teilbar sind, für die $1$ alle Zahlen, für die gilt $x \in \mathbb{R}, x \mod 3 = 1$ und für die $2$
alle Zahlen, für die gilt $x \in \mathbb{R}, x \mod 3 = 2$.

Es existieren noch weitere endliche Körper, zum Beispiel mit 4, 5, 7, 8 und 9 Elementen.
Ein Körper mit 6 Elementen existiert nicht.

Die Menge der ganzen Zahlen $\mathbb{Z}$ ist kein Körper, da dass 7. Axiom (\textbf{Existenz eines Inversen})
nicht erfüllt ist.

%% Ich habe eine Section aus dem Skript ausgelassen, will aber die Nummerierung behalten
\addtocounter{subsubsection}{1}

\subsubsection{Konventionen und Schreibweisen}

``$\coloneqq$'' heißt ``ist per Definition gleich''
\begin{align*}
  ab &\coloneqq a * b \\
  2a &\coloneqq a + a \\
  a^2 &\coloneqq a * a \\
  a - b &\coloneqq a + (-b) \\
  \frac{a}{b} &\coloneqq a * (b^{-1}) \\
  a + b + c &\coloneqq (a + b) + c &\text{beachte (A1)} \\
  abc &\coloneqq a * b * c &\text{beachte (A5)} \\
\end{align*}

\end{document}