\documentclass{article}
\usepackage{aligned-overset}
\usepackage{amsmath}
\usepackage{amssymb}
\usepackage{bm}
\usepackage[shortlabels]{enumitem}
\usepackage{hyperref}
\usepackage[utf8]{inputenc}
\usepackage{mathtools}
\usepackage{physics}
\usepackage{titling}
\usepackage{fancyhdr}
\usepackage{xfrac}

\author{Karsten Lehmann}
\date{WiSe 2020}
\title{Hausaufgabe 07 Analysis - Grundlegende Konzepte}

\pagestyle{fancy}
\fancyhf{}
\lhead{\thetitle}
\rhead{\theauthor}
\lfoot{\thedate}
\rfoot{Seite \thepage}

\begin{document}
\section*{Hausaufgabe 1}

Beweisen Sie die folgenden Grenzwertaussagen:

\begin{enumerate}[(i)]
\item Sei $(a_n)_{n \in \mathbb{N}}$ eine konvergente Folge nichtnegativer Zahlen. Dann gilt
  \[
    \lim_{n \to \infty} \sqrt{a_n} = \sqrt{\lim_{n \to \infty} a_n}
  \]
  Sei $a = \lim_{n \to \infty} a_n$, dann gilt
  \begin{align*}
    \forall \epsilon > 0 \exists n_0 \in \mathbb{N} \forall n \in \mathbb{N}_{\geq  n_0} \colon \abs{a_n - a} &< \epsilon  \\
    \abs{(\sqrt{a_n} - \sqrt{a})(\sqrt{a_n} + \sqrt{a})} &< \epsilon \\
    \abs{\sqrt{a_n} - \sqrt{a}} \cdot \abs{\sqrt{a_n} + \sqrt{a}} &< \epsilon && | : (\abs{\sqrt{a_n} + \sqrt{a}})\\
    \abs{\sqrt{a_n} - \sqrt{a}} &< \frac{\epsilon}{\abs{\sqrt{a_n} + \sqrt{a}}}
  \end{align*}
  \begin{minipage}[t]{.4\textwidth}
    \textbf{Fall 1}: $\abs{\sqrt{a_n} + \sqrt{a}} \geq 1$
    \[
        \abs{\sqrt{a_n} - \sqrt{a}} < \epsilon 
    \]
  \end{minipage}
  \hfill
  \vrule
  \hfill
  \begin{minipage}[t]{.4\textwidth}
    \textbf{Fall 2}: $\abs{\sqrt{a_n} + \sqrt{a}} < 1$
    \[
        \abs{\sqrt{a_n} - \sqrt{a}} < \frac{\epsilon}{\sqrt{a_n} + \sqrt{a}} < \frac{\epsilon}{\sqrt{a}} 
    \]
  \end{minipage}

\item Es gilt $\lim_{n \to \infty} n^{\frac{1}{n}} = 1$
  \label{sec:1.2}
  \begin{align*}
    \abs{a_n - 1} &< \epsilon \\
    \abs{n^{\frac{1}{n}} - 1} &< \epsilon && | n^{\frac{1}{n}} \geq 1 \\
    n^{\frac{1}{n}} - 1 &< \epsilon && | + 1 \\
    n^{\frac{1}{n}} &< \epsilon + 1 && | (\ldots)^n \\
    n &< (\epsilon + 1)^n && \text{Binomische Formel (1.4.14 (g))} \\
    n &< \sum\limits_{m = 0}^n \begin{pmatrix}n \\ m \end{pmatrix} \epsilon^m \cdot  1^{n - m} \\
    n &< \sum\limits_{m = 0}^n \begin{pmatrix}n \\ m \end{pmatrix} \epsilon^m \\
    n &< \begin{pmatrix}n \\ 0 \end{pmatrix} \cdot \epsilon^0 + \sum\limits_{m = 1}^n \begin{pmatrix}n \\ m \end{pmatrix} \epsilon^m \\
    n &< 1 + \sum\limits_{m = 1}^n \begin{pmatrix}n \\ m \end{pmatrix} \epsilon^m \\
    n &< 1 + \begin{pmatrix}n \\ 1 \end{pmatrix} \epsilon^1 + \sum\limits_{m = 2}^n \begin{pmatrix}n \\ m \end{pmatrix} \epsilon^m \\
    n &< 1 + n\epsilon + \sum\limits_{m = 2}^n \begin{pmatrix}n \\ m \end{pmatrix} \epsilon^m \\
    n &< 1 + n\epsilon + \begin{pmatrix}n \\ 2 \end{pmatrix} \epsilon^2 + \sum\limits_{m = 3}^n \begin{pmatrix}n \\ m \end{pmatrix} \epsilon^m \\
    n &< 1 + n\epsilon + \frac{n(n - 1)}{2} \epsilon^2 + \sum\limits_{m = 3}^n \begin{pmatrix}n \\ m \end{pmatrix} \epsilon^m \\
                  & \text{Angenommen $n^k$ ist kleiner als ein Teil der rechten Seite} \\
    n &< 1 + \frac{n(n - 1)}{2} \epsilon^2 && | - 1 \\
    n - 1  &< \frac{n(n - 1)}{2} \epsilon^2 && | : (n - 1) \\
    1  &< \frac{n}{2} \epsilon^2 && | : \epsilon^2 \\
    \frac{1}{\epsilon^2}  &< \frac{n}{2} && | \cdot 2 \\
    \frac{2}{\epsilon^2}  &< n \\
  \end{align*}

  für alle $n \in \mathbb{N}_{\geq n_0}$, wenn $n_0 \in \mathbb{N}$ so gewählt ist, dass $n_0 < \frac{2}{}$.
\item Es gilt $\lim_{n \to \infty} \frac{n!}{n^n} = 0$
  \begin{align*}
    \abs{\frac{n!}{n^n} - 0} &< \epsilon \\
    \abs{\frac{n!}{n^n}} &< \epsilon && | \text{Def. $n!$, Def. $n^n$} \\
    \frac{n!}{n^n} &< \epsilon \\
    \frac{n}{n} \cdot \frac{n - 1}{n} \cdot \ldots \cdot \frac{1}{n} < \epsilon \\
    \prod\limits_{m=2}^{n}\frac{m}{n} \cdot \frac{1}{n} < \epsilon && | \prod\limits_{m=2}^{n}\frac{m}{n} \leq 1 \\
    \frac{1}{n} < \epsilon \\
  \end{align*}
  für alle $n \in \mathbb{N}_{\geq n_0}$, wenn $n_0 \in \mathbb{N}$ so gewählt ist, dass $\frac{1}{n_0} < \epsilon$ (Korollar 1.4.8 der Vorlesung).
\item Für $k \in \mathbb{N}_0$ gilt $\lim_{n \to \infty} n^{\frac{k}{n}} = 1$ \\
  Angenommen $k = 0$, dann ist $\lim_{n \to \infty} n^{\frac{0}{n}} = \lim_{n \to \infty} n^0 = \lim_{n \to \infty} 1 = 1$.

  \begin{align*}
    \lim_{n \to \infty} n^{\frac{k}{n}} &= \lim_{n \to \infty} \left(n^{\frac{1}{n}}\right)^k \\
    \overset{\text{Vorlesung 2.1.5 (f)}}&= \left( \lim_{n \to \infty} n^{\frac{1}{n}} \right)^k \\
    \overset{\hyperref[sec:1.2]{(ii)}}&= 1^k \\
                                        &= 1\\
  \end{align*}
  
\end{enumerate}

\newpage
\section*{Hausaufgabe 2}

Die reelle Folge $(a_n)$ sei definiert durch $a_1 \coloneqq 1$ und
$a_{n + 1} \coloneqq \sqrt{1 + a_n}$ für $n \in \mathbb{N}$. Untersuchen Sie diese Folge auf Konvergenz und
berechnen Sie gegebenenfalls ihren Grenzwert.

\begin{tabular}{cccc c}
  $a_1$ & $a_2$ & $a_3$ & $a_5$ & $a_5$ \\
  \hline
  $1$ & $\approx 1.4142$ & $\approx 1.5538$ & $\approx 1.5981$ & $\approx 1.6119$
\end{tabular}

Durch kurze Betrachtung der ersten Folgenglieder entsteht der Eindruck einer monoton wachsenden Folge. \\
\emph{Annahme}: $P(n) \colon a_{n+1} > a_n$ \\
\emph{Induktionsanfang} $P(1) \colon \sqrt{1 + 1} > 1$
Die Aussage $P(1)$ ist offensichtlich wahr. \\
\emph{Induktionsschritt}: Sei nun $P(n)$ für ein beliebiges $n \in \mathbb{N}$ wahr.

\begin{minipage}[t]{.4\textwidth}
  \begin{align*}
    P(n) \colon a_{n+1} &> a_n \\
    \sqrt{1 + a_n} &> a_n  && |(\ldots)^2 \\
    1 + a_n &> (a_n)^2     && | -(1 + a_n)\\
    0 &> (a_n)^2 - a_n - 1
  \end{align*}
\end{minipage}
\hfill
\vrule
\hfill
\begin{minipage}[t]{.4\textwidth}
  \label{defn:liman}
  \begin{align*}
    (a_n)_{1|2} &= \frac{1}{2} \pm \sqrt{\left(\frac{1}{2}\right)^2 + 1} \\
    (a_n)_1 &= \frac{1 + \sqrt{5}}{2} \\
    (a_n)_2 &= \frac{\sqrt{5} - 1}{2}
  \end{align*}
\end{minipage}

$(a_n)_2$ entfällt aufgrund des Induktionsanfangs und der Annahme.
Somit ist die Behauptung unter der Einschränkung $a_n < \frac{\sqrt{5} + 2}{2}$ wahr und aus
dem Satz über die vollständige Induktion folgt die Behauptung (unter der Einschränkung).

Nun soll bewiesen werden, dass $\forall a_n < \frac{\sqrt{5} + 1}{2} \colon a_{n+1} < \frac{\sqrt{5} + 1}{2}$:
\begin{align*}
  a_{n+1} &= \sqrt{1 + a_n} && \text{Abschätzung nach oben durch $a_n = \frac{\sqrt{5} + 1}{2}$} \\
  a_{n+1} &< \sqrt{1 + \frac{\sqrt{5} + 1}{2}} =  \frac{\sqrt{5} + 1}{2}
\end{align*}
Da $a_1 = 1$ gilt für alle weiteren Folgenglieder $a_{n+1} > a_n$ und $a_n < \frac{\sqrt{5} + 1}{2}$.
Aus Proposition 2.2.4 (a) der Vorlesung folgt nun die Beschränktheit von $(a_n)$.
Sei nun $a = \lim_{n \to \infty} (a_n)$, dann gilt auch $a = \lim_{n \to \infty} (a_{n+1})$.
\begin{align*}
  \lim_{n\to\infty} a_{n+1} \overset{\text{2.1.5 (f)}}&= \sqrt{\lim_{n\to\infty}1+a_n} \\
  \lim_{n\to\infty} a_{n+1} \overset{\text{2.1.5 (b)}}&= \sqrt{(\lim_{n\to\infty}1)+(lim_{n\to\infty}a_n)}
                                                      && | \underset{n\to\infty}\lim a_{n+1} = \underset{n\to\infty}\lim a_n  \\
  \lim_{n\to\infty} a_n \overset{\text{2.1.5 (b)}}&= \sqrt{(\lim_{n\to\infty}1)+(lim_{n\to\infty}a_n)} \\
\end{align*}

Die Lösung dafür wurde \hyperref[defn:liman]{oben} bereits berechnet, $\lim_{n\to\infty} a_n = \frac{\sqrt{5} + 1}{2}$

\newpage
\section*{Hausaufgabe 3}

Sein $(a_n)$ eine beschränkte Folge in $\mathbb{R}$. Beweise Sie:

\[
  \lim_{n \to \infty} a_n = 0 \iff \limsup_{n \to \infty} \abs{a_n} = 0
\]

\begin{itemize}
\item[$\Rightarrow$]
  \begin{align*}
    \forall \epsilon > 0 \exists n_0 \in \mathbb{N} \forall n \in \mathbb{N}_{\geq n_0} &\colon \abs{a_n - 0} < \epsilon \\
                                                                                        &= \abs{a_n} < \epsilon \\
                                                                                        &= \abs{\abs{a_n}} < \epsilon \\
                                                                                        &= \abs{\abs{a_n} - 0} < \epsilon \\
  \end{align*}
  Somit folgt aus $\lim_{n \to \infty} a_n = 0$, dass $\lim_{n \to \infty} \abs{a_n} = 0$.
  Weiterhin folgt aus der Definition des Grenzwertes, dass $\limsup_{n \to \infty} \abs{a_n} = 0$.
\item[$\Leftarrow$]
  Aus $\limsup_{n \to \infty} \abs{a_n} = 0$ folgt:
  \begin{align*}
    \forall \epsilon > 0 \forall N \in \mathbb{N} \exists n \in \mathbb{N}_{\geq N} &\colon \abs{\abs{a_n} - 0} < \epsilon \\
                                                                                    &= \abs{a_n} < \epsilon \\
  \end{align*}
  Angenommen die Folge $(\abs{a_n})$ hat einen weiteren Häufungswert $b \ne \limsup_{n \to \infty} \abs{a_n}$,
  dann muss gelten:
  \[
    \forall \epsilon > 0 \forall N \in \mathbb{N} \exists n \in \mathbb{N}_{\geq N} \colon \abs{\abs{a_n} - b} < \epsilon
  \]

  Per Definition des Betrages ist $\abs{a_n} \geq 0$ und per Definition ist der Limes superior der
  größte Häufungswert, somit folgt $b < 0$. Also ist $\abs{a_n} - b \geq \abs{b}$.
  Somit ist
  \[
    \forall \epsilon > 0 \forall N \in \mathbb{N} \exists n \in \mathbb{N}_{\geq N} \colon \abs{\abs{a_n} - b} < \epsilon
  \]
  für alle $\epsilon < \abs{b}$ nicht erfüllt.

  Damit existiert kein kleinerer Häufungswert als $0$, es folgt \\
  $\limsup_{n \to \infty} \abs{a_n} = \liminf_{n \to \infty} \abs{a_n} = 0$.
  
  Aus 2.2.8 der Vorlesung folgt, dass $\lim_{n \to \infty} \abs{a_n} = 0$. 
  
\end{itemize}

\end{document}