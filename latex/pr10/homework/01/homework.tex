\documentclass{article}

\usepackage[shortlabels]{enumitem}
\usepackage{hyperref}
\usepackage{listings}
\usepackage{rotating}
\usepackage{tabularx}
\usepackage{xcolor}
\definecolor{light-gray}{gray}{.9}

\author{
  Albina Oscherowa (4694823) \\
  Karsten Lehmann (4935758)
}
\date{WiSe 2020}
\title{Hausaufgaben Programmieren - Grundlegende Konzepte (PR10)}

\begin{document}
\maketitle

\newpage

\section*{Aufgabe 1}

Das tatsächliche Ergebnis des Ausdruck $x^4 - 4y^4 - 4y^2$ für $x = 665857$ und $y = 470832$ lautet $1$.
In der folgenden Tabelle sind die Ergebnisse für die angegeben \emph{Precision} P und \emph{Exponent Range} R
der Datentypen von $x$ und $y$ dargestellt.

{
  \footnotesize
  \begin{tabularx}{\textwidth}{X | r | r | r}
    Ausdruck & \begin{sideways}$P=6, R=37$\end{sideways} & \begin{sideways}$P=15, R=307$\end{sideways} & \begin{sideways}$P=18, R=4931$\end{sideways} \\
    \hline
    $x^4 - 4*y^4 - 4*y^2$                       & $-8.86731112E+11$ & $11885568$ & $7168$ \\
    $x*x*x*x - 4*y*y*y*y - 4*y*y$               & $-8.86731112E+11$ & $11885568$ & $7168$ \\
    $(x^2)^2 - (2*y^2)^2 - (2*y)^2$             & $-8.86731112E+11$ & $11885568$ & $7168$ \\
    $(x^2)^2 - (2*y)^2*(y^2 + 1)$               & $0$               & $0$        & $0$    \\
    $(x*x-2*y*(y+1)) * (x*x+2*y*(y+1)) + 8*y^3$ & $2.14225410E+16$  & $0$        & $1$    \\
    $(x^2-2*y^2) * (x^2+2*y^2) - 4*y^2$         & $-8.86731112E+11$ & $1$        & $1$    \\
  \end{tabularx}
}

\noindent
Wie zu erkennen, sind die korrekten Ergebnisse nur in der unteren rechten Ecke der Tabelle enthalten.
Warum ist das so? \\

\noindent
Die Variablen $x$ und $y$, sowie die Ergebnisse der Berechnungen werden als Gleitkommazahlen gespeichert. \\

\noindent
\fcolorbox{black}{light-gray}{\begin{minipage}{\textwidth}
    \textbf{Gleitkommazahlen} beinhalten Zahlen mit Stellen vor und nach dem Komma.
    Da bei Kommazahlen festgelegte Stellen für die Vor- und Nachkommastellen eher hinderlich wären,
    hat man bei der Gleitkommadarstellung dafür gesorgt, dass das Komma jede beliebige Stelle einnehmen kann.
    Man spricht davon, dass das Komma "gleitet". \cite{gleitkomma}
\end{minipage}}
\\

Diesen Gleitkommazahlen sind in der Genauigkeit Grenzen gesetzt. Dadurch kommt es zu Rundungsfehlern und
Auslöschungen.
Insgesamt kann man sehen, dass sich die Lösungen mit gesteigerter Genauigkeit den korrekten Werten annähern.
Weiterhin hängt die Qualität des Ergebnisses auch von den Zwischenschritten der Berechnung ab. 
Je geringer der Betrag der Ergebnisse der einzelnen Zwischenschritte der Rechnung ist, desto
genauer ist das Ergebnis.

\pagebreak
\section*{Aufgabe 2}

\begin{enumerate}[a)]
\item
  Dieses Programm gibt eine bestimmte Anzahl an Zweierpotenzen aus.
  Ab einer Eingabe von $31$ werden die letzten Potenzen Fehlerhaft
  ausgegeben. Für $2^{31}$ wird anstatt $2147483648$ der Wert
  $-2147483648$ ausgegeben. Für alle weiteren Potenzen wird $0$
  ausgegeben.
  \begin{lstlisting}[language=Fortran, showstringspaces=false]
PROGRAM Potenz_a
  IMPLICIT NONE
  INTEGER :: i, imax

  WRITE(*,*) "Anzahl Potenzen"
  READ(*,*) imax

  DO i = 1, imax
     WRITE(*,*) 2**i
  END DO
END PROGRAM Potenz_a
  \end{lstlisting}

\item
  Das Programm aus a) erweitert, um anstatt der fehlerhaften Potenzen auszugeben
  mit einer Fehlermeldung abzubrechen.
  \begin{lstlisting}[language=Fortran, showstringspaces=false]
PROGRAM Potenz_b
  IMPLICIT NONE
  INTEGER :: i, imax

  WRITE(*,*) "Anzahl Potenzen"
  READ(*,*) imax

  DO i = 1, imax
     IF (i > 30) THEN
        WRITE(*,*) "Groessere Potenzen koennen &
             &mit dem Datentyp nicht mehr abgebildet werden"
        EXIT
     END IF
     WRITE(*,*) 2**i
  END DO
END PROGRAM Potenz_b
  \end{lstlisting}

\pagebreak
\item
  Anstatt die Zweierpotenzen direkt zu berechnen wird hier jedes Mal der vorherige Wert
  mit $2$ multipliziert.
  \begin{lstlisting}[language=Fortran, showstringspaces=false]
PROGRAM Potenz_c
  IMPLICIT NONE
  INTEGER :: i, imax, pot

  WRITE(*,*) "Anzahl Potenzen"
  READ(*,*) imax

  pot = 1
  DO i = 1, imax
     IF (i > 30) THEN
        WRITE(*,*) "Groessere Potenzen koennen &
             &mit dem Datentyp nicht mehr abgebildet werden"
        EXIT
     END IF
     pot = pot * 2
     WRITE(*,*) pot
  END DO
END PROGRAM Potenz_c
  \end{lstlisting}

\item
  In diesem Beispiel wurde das Programm aus c) um einen ganzzahligen Datentyp mit 64 Bit erweitert.
  Somit können die Zweierpotenzen bis $2**62$ ausgegeben werden.
  \begin{lstlisting}[language=Fortran, showstringspaces=false]
PROGRAM Potenz_d
  IMPLICIT NONE
  INTEGER(kind=8) :: i, imax, pot

  WRITE(*,*) "Anzahl Potenzen"
  READ(*,*) imax

  pot = 1
  DO i = 1, imax
     IF (i > 62) THEN
        WRITE(*,*) "Groessere Potenzen koennen &
             &mit dem Datentyp nicht mehr abgebildet werden"
        EXIT
     END IF
     pot = pot * 2
     WRITE(*,*) pot
  END DO
END PROGRAM Potenz_d
  \end{lstlisting}

\pagebreak
\item
  In diesem Programm wird anstatt Zweierpotenzen die Falkultät einer Zahl berechnet.
  Diese Programm leidet unter dem selben Problem wie das Programm a), die Fakultät wird
  mit $-2147483648$ falsch ausgegeben.
  \begin{lstlisting}[language=Fortran, showstringspaces=false]
PROGRAM fakultaet
  IMPLICIT NONE
  INTEGER :: i, n, res

  WRITE(*,*) "Fakultaet von"
  READ(*,*) n

  res = 1
  DO i = 1, n
     res = res * i
  END DO
  WRITE(*,*) res
END PROGRAM fakultaet
\end{lstlisting}
\end{enumerate}

\begin{thebibliography}{9}

\bibitem{gleitkomma}
  \emph{``Gleitkommadarstellung / Gleitkommazahlen''},
  \href{https://www.elektronik-kompendium.de/sites/dig/1807231.htm}{https://www.elektronik-kompendium.de/sites/dig/1807231.htm},
  abgerufen am 06.11.2020 um 21:00
\end{thebibliography}
\end{document}