\documentclass{article}

\usepackage[shortlabels]{enumitem}
\usepackage{hyperref}
\usepackage{listings}
\usepackage{rotating}
\usepackage{struktex}
\usepackage{tabularx}
\usepackage{xcolor}
\definecolor{light-gray}{gray}{.9}

\author{
  Albina Oscherowa (4694823) \\
  Karsten Lehmann (4935758)
}
\date{WiSe 2020}
\title{Hausaufgaben Programmieren - Grundlegende Konzepte (PR10)}

\begin{document}
\maketitle

\newpage

\section*{Aufgabe 3}

\begin{struktogramm}(100, 100)
  \assign{Eingabe : Kredithöhe, Zinssatz}
  \assign{Zinsen := Kredithöhe * Zinssatz / 100}
  \until{Wiederhole bis Rate > Zinsen}
    \assign{Eingabe: rate}
  \untilend
  \assign{Restschuld := Kredithöhe}  
  \while[1]{Solange Restschuld > 0}  
    \assign{Laufzeit := Laufzeit + 1}  
    \assign{Zinsen := Restschuld * Zinssatz / 100}
    \assign{Restschuld := Restschuld + Zinsen - Rate}
    \assign{Zinssumme := Zinssumme + Zinsen}
  \whileend
  \ifthenelse{3}{3} {Restschuld < 0} {Ja}{Nein}
    \assign{Rate := Rate + Restschuld}
    \assign{Ausgabe: Rate}
    \change
  \ifend
  \assign{Ausgabe: Laufzeit, Zinssumme}  
\end{struktogramm}

\section*{Aufgabe 4}

\begin{struktogramm}(100, 100)
  \until{Wiederhole bis n > 0}
    \assign{Eingabe: n}
  \untilend
  
  \assign{Maximum := 0}
  \assign{Minimum := MAX\_INT}
  \assign{Summe := 0}
  \assign{Mittelwert := 0}
  \assign{Maximum := 0}

  \while[1]{Solange i < n}
    \assign{i = i + 1}
    \assign{Eingabe: x}
    \ifthenelse{3}{3} {x $>$ Maximum} {Ja} {Nein}
      \assign{Maximum = x}
      \change
    \ifend
    \ifthenelse{3}{3} {x $<$ Minimum} {Ja} {Nein}
      \assign{Minimum = x}
      \change
    \ifend
    \assign{Summe := Summe + x}
    \assign{Mittelwert := Summe / i}
  \whileend
  \assign{Ausgabe der Ergebnisse}
\end{struktogramm}

\section*{Aufgabe 5}

\begin{struktogramm}(150, 100)
  \assign{weiterspielen := wahr}
  \while[1]{solange weiterspielen}
    \assign{Ausgabe der Spielanleitung}
    \until{Solange bis l $<$ r}
      \assign{Eingabe: l, r}
    \untilend
    \assign{i := 0}
    \assign{erraten := falsch}
    \while[1]{solange nicht erraten}
      \assign{z := (l + r) / 2}
      \assign{Ausgabe: z}
      \assign{i := i + 1}
      \until{Solange bis c = "$<$" oder c = "$>$" oder c = "="}
        \assign{Eingabe: c}
      \untilend
      \ifthenelse{2}{4} {c = "="} {Ja} {Nein}
        \assign{erraten := wahr}
        \assign{Ausgabe: Zahl der Versuche i}
        \assign{Ausgabe: Frage nach weiterspielen}
        \until{Solange bis c = "y" oder c = "n"}
          \assign{Eingabe: c}
        \untilend
        \ifthenelse{3}{3} {c = "n"} {Ja} {Nein}
          \assign{Weiterspielen := Falsch}
        \change
        \ifend
        \change
        \ifthenelse {2} {5} {r - l $<$ 2} {Ja} {Nein}
          \assign{Ausgabe Logikfehler}
          \assign{Programm beenden}
        \change
          \ifthenelse {3}{3} {(r - l) = 2 und c = "$>$"} {Ja} {Nein}
            \assign{r = l}
          \change
            \ifthenelse {1}{1} {c = "$>$"} {ja} {nein}
              \assign{l = z}
            \change
              \assign{r = z}
            \ifend
          \ifend
        \ifend
      \ifend
    \whileend
  \whileend
\end{struktogramm}

\end{document}