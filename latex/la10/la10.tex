\documentclass{article}

\usepackage{amsmath}
\usepackage{amssymb}
\usepackage{genealogytree}
\usepackage{hyperref}
\usepackage[utf8]{inputenc}
\usepackage{xcolor}
\definecolor{light-gray}{gray}{.9}

\setlength\parindent{0pt}

\author{Karsten Lehmann}
\date{WiSe 2020}
\title{Mitschriften Lineare Algebra - Grundlegende Konzepte (LA10)}

\begin{document}


\maketitle

\vfill
\begin{center}
  Dozent: Vadim Alekseev \\
  \href{mailto:vadim.alekseev@tu-dresden.de}{vadim.alekseev@tu-dresden.de}
\end{center}

\newpage

\section*{Einführung}

\textbf{Was ist Mathematik?}

Mathematik beantwortet Fragen über Rechenstrukturen und geometrische Sachverhalte allgemeingültig.

\textbf{Theorem} (Pythagoras):
Wenn $a \leq b \leq c$ Längen von Seiten eines rechtwinkligen Dreiecks sind, so gilt:

\begin{center}
$a^2 + b^2 = c^2$
\end{center}

Hier haben wir Voraussetzungen \ldots
\begin{itemize}
\item $a \leq b \leq c$
\item Dreieck
\item Rechtwinklig
\end{itemize}

\ldots und eine Behauptung \emph{"dann gilt"} $a^2 + b^2 = c^2$ \\


Damit dieses Theorem wirklich allgemeingültig ist müssen die Voraussetzungen präzise definiert sein.
Zum Beispiel muss \emph{``Rechtwinkliges Dreieck''} genau definiert sein.

Man hat am Anfang des 20. Jahrhunderts erkannt, dass diese genaue Definition eine spezielle Sprache
benötigt - den Formalismus der Logik und Mengenlehre.

Da diese Sprache sehr schwer intuitiv zu erfassen ist, wird hier mit "naiver" Logik und Mengenlehre
angefangen, um später weitere Behauptungen in dieser Sprache formulieren zu können.

\section{Mengenlehre}

Die Mengenlehre befasst sich mit \emph{Mengen}.

\fcolorbox{black}{light-gray}{
  \begin{minipage}{\textwidth}
    Georg Cantor \gtrsymBorn~19. Februar/3.März 1854 \gtrsymDied 6. Januar 1918

    Deutscher Mathematiker und Begründer der Mengenlehre. \\

    Definition der Menge (1895):
    \emph{Unter einer Menge verstehen wir jede Zusammenfassung M von bestimmten wohlunterschiedenen Objekten
    m unserer Anschauung oder unseres Denkens (welche die Elemente von M genannt werden) zu einem Ganzen}
  \end{minipage}
}


\emph{Beispiel}: $Z = \{ \text{Zoomteilnehmer der Vorlesung} \}$ \\
$Y = \{ \text{Teilnehmer der Vorlesung auf YouTube} \}$, \\
$A = \{ 0, 1, 2, 3 \}$

\textbf{Notation}: Wenn $a$ ein Element der Menge $A$ ist, dann schreibt man $a \in A$.

\textbf{Standardnotation für bestimmte Mengen in der Mathematik}:

\begin{itemize}
\item Symbol $\mathbb{N}$: Natürliche Zahlen $\{0, 1, 2, 3, \ldots \}$
\item Symbol $\mathbb{Z}$: Ganze Zahlen $\{ \ldots, -1, 0, 1, \ldots \}$
\item Symbol $\mathbb{Q}$: Rationale Zahlen $\{ \frac{p}{q} | p \in \mathbb{Z}, q \in \mathbb{N}, q \ne 0 \} \subset \mathbb{R}$ 
\item Symbol $\mathbb{R}$: Reelle Zahlen
\end{itemize}

\textbf{Notation}: Um eine Menge durch Eigenschaften ihrer Elemente zu beschreiben, verwendet man die Notation

$A = \{a | a \text{ hat eine gewisse Eigenschaft} \}$ \\

\textbf{Teilmenge}: $A$ und $B$ sind zwei Mengen. $A$ ist eine Teilmenge von $B$ wenn jedes Element von $A$ auch
in $B$ enthalten ist. Man schreibt $A \subseteq B$. \\

Wenn $A \subseteq B$ und $B \subseteq A$, dann gilt $A = B$. \\

\textbf{Notation}: wenn $A \subseteq B, A \ne B$, dann schreibt man $A \subsetneq B$.

Konstruktion von neuen Mengen aus bekannten Mengen: \\

\textbf{Definition Vereinigung}: Wenn $A$ und $B$ Mengen sind, dann ist die Vereinigung von $A$ und $B$ die
Menge $A \cup B = \{ x | x \in A \text{ oder } x \in B \}$. Ein \emph{"oder"} meint in der Mathematik immer
\emph{"Nicht exklusives oder"}, das heißt $x$ kann in $A$, $B$ oder in beiden enthalten sein. \\

\textbf{Definition Durchschnitt}: Wenn $A$ und $B$ Mengen sind, dann ist der Durchschnitt von $A$ und $B$ die
Menge an Elementen, die in $A$ und $B$ liegen. Den Durchschnitt schreibt man

$A \cap B = \{ x | x \in A \text{ und } x \in B\}$ \\

Wenn $A$ und $B$ keine gemeinsamen Elemente haben, dann ist $A \cap B = \emptyset$.
Die \textbf{leere Menge} $\emptyset$ ist eine Menge, die keine Elemente enthält. \\

Beispiel: \\
$\{e \in \mathbb{R} | x^2 < 0\} = \emptyset$ \\

Eigenschaften, für jede Menge $A$ gilt:
\begin{itemize}
\item $A \cup \emptyset = A$
\item $A \cap \emptyset = \emptyset$
\end{itemize}

Frage: Angenommen $A$ ist eine Menge, gilt dann $\emptyset \subseteq A$?
Formal gesprochen ist diese Aussage \emph{"jedes Element der leeren Menge ist auch ein Element von A"}.
Jedoch hat die leere Menge keine Elemente.

Intuitiv lässt sich diese Frage schwer beantworten, deswegen wurde eine Konvention geschaffen: \\

Für jede Menge $A$ gilt $\emptyset \subseteq A$

\end{document}