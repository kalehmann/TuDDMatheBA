\documentclass{article}

\usepackage[dvipsnames]{xcolor}
\definecolor{light-gray}{gray}{.9}
\usepackage{aligned-overset}
\usepackage{amsmath}
\usepackage{amssymb}
\usepackage[shortlabels]{enumitem}
\usepackage{genealogytree}
\usepackage{hyperref}
\usepackage[utf8]{inputenc}
\usepackage{mathtools}
\usepackage{physics}
\usepackage{tikz}
\usetikzlibrary{positioning}
\setlength\parindent{0pt}

\author{Karsten Lehmann}
\date{WiSe 2020}
\title{Mitschriften Lineare Algebra - Grundlegende Konzepte (LA10)}

\begin{document}


\maketitle

\vfill
\begin{center}
  Dozent: Vadim Alekseev \\
  \href{mailto:vadim.alekseev@tu-dresden.de}{vadim.alekseev@tu-dresden.de}
\end{center}

\newpage

\section*{Einführung}

\textbf{Was ist Mathematik?}

Mathematik beantwortet Fragen über Rechenstrukturen und geometrische Sachverhalte allgemeingültig.

\textbf{Theorem} (Pythagoras):
Wenn $a \leq b \leq c$ Längen von Seiten eines rechtwinkligen Dreiecks sind, so gilt:

\begin{center}
$a^2 + b^2 = c^2$
\end{center}

Hier haben wir Voraussetzungen \ldots
\begin{itemize}
\item $a \leq b \leq c$
\item Dreieck
\item Rechtwinklig
\end{itemize}

\ldots und eine Behauptung \emph{"dann gilt"} $a^2 + b^2 = c^2$ \\


Damit dieses Theorem wirklich allgemeingültig ist müssen die Voraussetzungen präzise definiert sein.
Zum Beispiel muss \emph{``Rechtwinkliges Dreieck''} genau definiert sein.

Man hat am Anfang des 20. Jahrhunderts erkannt, dass diese genaue Definition eine spezielle Sprache
benötigt - den Formalismus der Logik und Mengenlehre.

Da diese Sprache sehr schwer intuitiv zu erfassen ist, wird hier mit "naiver" Logik und Mengenlehre
angefangen, um später weitere Behauptungen in dieser Sprache formulieren zu können.

\section{Mengenlehre}

Die Mengenlehre befasst sich mit \emph{Mengen}.

\fcolorbox{black}{light-gray}{
  \begin{minipage}{\textwidth}
    Georg Cantor \gtrsymBorn~19. Februar/3.März 1854 \gtrsymDied 6. Januar 1918

    Deutscher Mathematiker und Begründer der Mengenlehre. \\

    Definition der Menge (1895):
    \emph{Unter einer Menge verstehen wir jede Zusammenfassung M von bestimmten wohlunterschiedenen Objekten
    m unserer Anschauung oder unseres Denkens (welche die Elemente von M genannt werden) zu einem Ganzen}
  \end{minipage}
}


\emph{Beispiel}: $Z = \{ \text{Zoomteilnehmer der Vorlesung} \}$ \\
$Y = \{ \text{Teilnehmer der Vorlesung auf YouTube} \}$, \\
$A = \{ 0, 1, 2, 3 \}$

\textbf{Notation}: Wenn $a$ ein Element der Menge $A$ ist, dann schreibt man $a \in A$.

\textbf{Standardnotation für bestimmte Mengen in der Mathematik}:

\begin{itemize}
\item Symbol $\mathbb{N}$: Natürliche Zahlen $\{0, 1, 2, 3, \ldots \}$
\item Symbol $\mathbb{Z}$: Ganze Zahlen $\{ \ldots, -1, 0, 1, \ldots \}$
\item Symbol $\mathbb{Q}$: Rationale Zahlen $\{ \frac{p}{q} | p \in \mathbb{Z}, q \in \mathbb{N}, q \ne 0 \} \subset \mathbb{R}$ 
\item Symbol $\mathbb{R}$: Reelle Zahlen
\end{itemize}

\textbf{Notation}: Um eine Menge durch Eigenschaften ihrer Elemente zu beschreiben, verwendet man die Notation

$A = \{a | a \text{ hat eine gewisse Eigenschaft} \}$ \\

\textbf{Teilmenge}: $A$ und $B$ sind zwei Mengen. $A$ ist eine Teilmenge von $B$ wenn jedes Element von $A$ auch
in $B$ enthalten ist. Man schreibt $A \subseteq B$. \\

Wenn $A \subseteq B$ und $B \subseteq A$, dann gilt $A = B$. \\

\textbf{Notation}: wenn $A \subseteq B, A \ne B$, dann schreibt man $A \subsetneq B$.

Konstruktion von neuen Mengen aus bekannten Mengen: \\

\textbf{Definition Vereinigung}: Wenn $A$ und $B$ Mengen sind, dann ist die Vereinigung von $A$ und $B$ die
Menge $A \cup B = \{ x | x \in A \text{ oder } x \in B \}$. Ein \emph{"oder"} meint in der Mathematik immer
\emph{"Nicht exklusives oder"}, das heißt $x$ kann in $A$, $B$ oder in beiden enthalten sein. \\

\textbf{Definition Durchschnitt}: Wenn $A$ und $B$ Mengen sind, dann ist der Durchschnitt von $A$ und $B$ die
Menge an Elementen, die in $A$ und $B$ liegen. Den Durchschnitt schreibt man

$A \cap B = \{ x | x \in A \text{ und } x \in B\}$ \\

\textbf{Definition Differenz}: Wenn $A$ und $B$ Mengen sind, dann ist die Differenz von $A$ und $B$ die
Menge der Elemente, die in $A$ liegen, aber nicht in $B$. Die Differenz schreibt man

$A \setminus B = \{ x | x \in A \text{ und } x \notin B\}$ \\

Wenn $A$ und $B$ keine gemeinsamen Elemente haben, dann ist $A \cap B = \emptyset$.
Die \textbf{leere Menge} $\emptyset$ ist eine Menge, die keine Elemente enthält. \\

Beispiel: \\
$\{e \in \mathbb{R} | x^2 < 0\} = \emptyset$ \\

Eigenschaften, für jede Menge $A$ gilt:
o\begin{itemize}
\item $A \cup \emptyset = A$
\item $A \cap \emptyset = \emptyset$
\end{itemize}

Frage: Angenommen $A$ ist eine Menge, gilt dann $\emptyset \subseteq A$?
Formal gesprochen ist diese Aussage \emph{"jedes Element der leeren Menge ist auch ein Element von A"}.
Jedoch hat die leere Menge keine Elemente.

Intuitiv lässt sich diese Frage schwer beantworten, deswegen wurde eine Konvention geschaffen: \\

Für jede Menge $A$ gilt $\emptyset \subseteq A$

\subsection*{Ausblick: Aussagen}

Bei den gelisteten Definition sieht man folgende Konstrukte:

\begin{enumerate}
\item $x \notin B$
\item $x \in A$ und $x \in B$
\item $x \in A$ oder $x \in B$
\item \emph{"jedes Element von $A$ ist ein Element von $B$"}, oder anders gesagt aus $x \in A$ folgt $x \in B$
\end{enumerate}

Diese Konstrukte sind Beispiele für \textbf{Aussagen}. Eine Aussage ist eine Äußerung, von der man fragen kann
ob sie wahr oder falsch ist.

Die oben genannten Konstrukte entsprechen folgenden Operationen für Aussagen:

$P = (x \in A)$ und $Q = (x \in B)$:
\begin{enumerate}
\item "nicht": die Aussage $\neg Q = "Q$ ist nicht wahr" heißt Negation von Q
\item "und": die Aussage "$P$ und $Q$ sind beide wahr" heißt Konjunktion $P \wedge Q$ von $P$ und $Q$
\item "oder": die Aussage "mindestens eine der Aussagen $P$ und $Q$ ist wahr" heißt Disjunktion $P \vee Q$ von
  $P$ und $Q$
\item "folgt": die Aussage "wenn $P$, dann $Q$" heißt Implikation und wird durch $P \implies Q$ dargestellt.
\end{enumerate}

Wann ist eigentlich $P \implies Q$ wahr? Per Konvention ist diese Aussage nur dann falsch, wenn $P$ wahr und
$Q$ falsch ist, in allen anderen Fällen ist $P \implies Q$ wahr (Dies entspricht dem Prinzip "aus einer falschen Aussage folgt eine beliebige")!

Bei Beweisen muss man sich somit nur um den Fall, dass $P$ wahr ist kümmern.

\subsection*{Übung Wahrheitstabelle}

\begin{displaymath}
  \begin{array}{c | c }  
    Q & \neg Q \\
    \hline
    w & f \\
    f & w \\
  \end{array}
\end{displaymath}

\begin{displaymath}
  \begin{array}{ c | c | c | c }
    P & Q & P \land Q & P \lor Q \\
    \hline
    0 & 0 & 0 & 0 \\
    0 & 1 & 0 & 1 \\
    1 & 0 & 0 & 1 \\
    1 & 1 & 1 & 1 \\
  \end{array}
\end{displaymath}

\begin{displaymath}
  \begin{array}{ c | c | c }
    P & Q & P \implies Q \\
    \hline
    0 & 0 & 1 \\
    0 & 1 & 1 \\
    1 & 0 & 0 \\
    1 & 1 & 1 \\
  \end{array}
\end{displaymath}

Von diesem Standpunkt aus kann nun die Aussage $\emptyset \subseteq A$ erneut betrachtet werden,
sie ist dann gleich $(x \in \emptyset) \implies (x \in A)$.
Da $x \in \emptyset$ jedoch stets falsch ist, ist die Implikation nach dem obigen Prinzip wahr.

Somit gilt $\emptyset \subseteq A$. \\

Die Aussage "aus $P$ folgt $Q$ und aus $Q$ folgt $P$ $((P \implies Q) \land (Q \implies P))$ kürzt man
als $P \iff Q$ ab ($P$ gilt genau dann, wenn $Q$ gilt), dies ist die Aussage, dass $P$ und $Q$ \emph{äquivalent}
oder \emph{gleichbedeutend} sind. \\

\textbf{Übung} Überprüfen Sie, dass $P \iff Q$ nur dann wahr ist, wenn $P$ und $Q$ gleichzeitig wahr oder
gleichzeitig falsch sind.

\textbf{Übung} Überprüfen Sie, dass $P \implies Q$ äquivalent zu $\neg Q \implies \neg P$ ist.

\begin{displaymath}
  \begin{array}{ c | c | c | c | c | c}
    P & Q & P \implies Q & Q \implies P & \neg Q \implies \neg P & P \iff Q \\
    \hline
    0 & 0 & 1 & 1 & 1 & 1 \\
    0 & 1 & 1 & 0 & 1 & 1 \\
    1 & 0 & 0 & 1 & 0 & 0 \\
    1 & 1 & 1 & 1 & 1 & 1 \\
  \end{array}
\end{displaymath}

\textbf{Definition Abbildung}: Angenommen $X$ und $Y$ sind Mengen. Eine Abbildung (oder auch Funktion) $f$
von $X$ nach $Y$ (Notation: $f: X \to Y$) ist eine Vorschrift, welche jedem Element $x \in X$ genau ein
Element $f(x) \in Y$ zuordnet (Notation: $x \mapsto f(x)$). \\

\emph{Beispiel} $f: \mathbb{R} \to \mathbb{R}, x \mapsto x^{3}$ ist die Notation zur Einführung der Funktion
$f(x) = x^{3}$. \\

\emph{Beispiel (Identität)}. Angenommen $X$ ist eine Menge. Die Abbildung $\text{id}_{x}: X \to X, x \mapsto x$
heißt Identitätsbildung oder kurz Identität. \\

\textbf{Definition kartesisches Produkt}: Das kartesische Produkt zweier Mengen $A$ und $B$ ist die Menge der
Paare

$A \times B \coloneqq \{(a,b) | a \in A, b \in B\}$

\begin{tikzpicture}
  %% Überschrift
  \node at (3,5.5) {$A = [a_{0},b_{0}], B = [a_{1}, b_{1}]$};
  %% Achsen
  \draw (-1,0) -> (6,0);
  \draw (0,-1) -> (0,5);
  %% Intervall an der vertikalen Achse
  \node[rotate=90] at (0, 1) (b0) {[};
  \node[left = .3cm of b0] {$b_{0}$};
  \node[rotate=90] at (0, 4) (b1) {]};
  \node[left = .3cm of b1] {$b_{1}$};
  %% Intervall an der horizontalen Achse
  \node at (1, 0) (a0) {[};
  \node[below = .3cm of a0] {$a_{0}$};
  \node at (5, 0) (a1) {]};
  \node[below = .3cm of a1] {$a_{1}$};

  %% Rechteck innerhalb des Diagramms mit dashed Linien
  \draw (1,1) rectangle ++(4,3);
  \draw[dashed] (1,0) -- (1,5);
  \draw[dashed] (5,0) -- (5,5);
  \draw[dashed] (0,1) -- (6,1);
  \draw[dashed] (0,4) -- (6,4);

  %% Einzelner Punkt im Rechteck
  \coordinate (dot) at (4,3);
  \draw[fill] (dot) circle (.07);
  \node[right = 0cm of dot] {$(a,b)$};

  %% Rechteck mit B und A als Beschriftung der Seiten, sowie dem Punkt (a,b)
  \draw (1,-4.5) rectangle ++(4,3);
  \node at (0.5, -3) {B};
  \node at (3,-5) {A};
  \coordinate (dot2) at (4,-2.5);
  \draw[fill] (dot2) circle (.07);
  \node[right = 0cm of dot2] {$(a,b)$};  
\end{tikzpicture}

\emph{Beispiel (Projektionsabbildung)}: Angenommen $A$ und $B$ sind Mengen. Die Projektionsabbildung vom
kartesischen Produkt auf die erste Komponente ist so definiert:

\begin{align*}
\pi_{A}: A \times B &\to A, \\
(a,b) &\mapsto a\
\end{align*}

\textbf{Definition}: Eine Abbildung $f:X \to Y$ heißt

\begin{enumerate}[(i)]
\item injektiv, falls für alle $x_1 \ne x_2 \in X$ gilt: $f(x_1) \ne f(x_2)$
\item surjektiv, falls für jedes $y \in Y$ ein $x \in X$ existiert mit $f(x) = y$
\item bijektiv, falls sie injektiv und surjektiv ist
\end{enumerate}

\textbf{Definition}: Seien $X, Y$ Mengen, $A \subseteq X, B \subseteq Y, f \colon X \to Y$ eine Abbildung.
Das Bild (die Bildmenge) von $A$ unter $f$ wird definiert als

\[
  f(A) \coloneqq \{ f(a) | a \in A\}
\]

Das Urbild (die Urbildmenge) von $B$ unter $f$ wird definiert als

\[
  f^{-1}(B) \coloneqq \{ x \in X | f(x) \in B \}
\]

Wenn $B = \{ b \}$ nur aus einem Element besteht, schreibt man auch $f^{-1}(b)$. \\

Mann kann dann Injektivität und Surjektivität so umformulieren: \\
\begin{itemize}
\item $f \colon X \to Y$ ist surjektiv genau dann, wenn für jedes $y \in Y$ die Urbildmenge $f^{-1}(y)$ nicht leer ist. \\ 
\item $f \colon X \to Y$ ist injektiv genau dann, wenn jedes $y \in Y$ höchstens ein Urbild hat (die Urbildmenge $f^{-1}(y)$ ist leer
  oder besteht aus einem Element) \\
\end{itemize}

Daher gilt auch: $f \colon X \to Y$ ist bijektiv genau dann, wenn jedes Element $y \in Y$ genau ein Urbild hat.

\subsection*{Quantoren}

\fcolorbox{black}{light-gray}{
  \begin{minipage}{\textwidth}
    \textbf{Allquantor} \\

    Notation: $\forall$ = ``für alle''
  \end{minipage}
}
\vspace{3mm} \\
\fcolorbox{black}{light-gray}{
  \begin{minipage}{\textwidth}
    \textbf{Existenzquantor} \\

    Notation: $\exists$ = ``es existiert mindestens ein''
  \end{minipage}
}
\vspace{3mm} \\
Die Bedingung für Surjektivität schreibt man dann zum Beispiel so:

\[
  \forall y \in Y \exists x \in X \colon f(x) = y
\]

Es gibt folgende Negationsregeln für Aussagen mit Quantoren:

\[
  \neg (\forall x \in X P(x)) \iff \exists x \in X \colon \neg P(x)
\]

Oder aussgeschrieben: ``$P(x)$ gilt nicht für alle $x$ genau dann, wenn ein $x \in X$ existiert, für welches
$P(x)$ nicht zutrifft''

\[
  \neg (\exists x \in X P(x)) \iff \forall x \in X \colon \neg P(x)
\]

Oder ausgeschrieben: ``es existiert kein $x$ für welches $P(x)$ gilt genau dann, wenn für alle $x \in X P(x)$
falsch ist''. \\

Die Quantoren sollen nur die benutzt werden, wo sie die Lesbarkeit durch Abkürzung erhöhen.
Im Fließtext ist deren Verwendung kontraindiziert. \\

Die Nicht-Surjektivität von einer Abbildung $f \colon X \to Y$ können wir jetzt schnell aufschreiben:

\[
  \exists y \in Y \colon \forall f(x) \ne y
\]

\textbf{Defintition}: Seien $f \colon X \to Y$ und $g \colon Y \to Z$ zwei Abbildungen. Die Verknüpfung
$g \circ f \colon X \to Z$ ist die Abbildung, die durch die Vorschrift $x \mapsto g(f(x))$ definiert ist. \\

\emph{Beispiel}:

\begin{align*}
  f \colon \mathbb{R} \to \mathbb{R} \\
  x \mapsto x + 1 \\
  g \colon \mathbb{R} \to \mathbb{R} \\
  x \mapsto x^2 \\
  g \circ f = \colon \mathbb{R} \to \mathbb{R}, x \mapsto (x + 1)^2 \\
  f \circ g = \colon \mathbb{R} \to \mathbb{R}, x \mapsto x^2 + 1 \\
\end{align*}

Man sieht, im Allgemeinen $g \circ f \ne f \circ g$.

\[
  X \overset{f}\to Y \overset{g}\to Z \overset{h}\to W
\]

\textbf{Lemma} Seien $f \colon X \to Y, g \colon Y \to Z, h \colon Z \to W$ Abbildungen. Dann gilt
$(h \circ g) \circ f = h \circ (g \circ f)$ (die Verknüpfung ist assoziativ) \\

\emph{Beweis}: Für jedes $x \in X$ gilt

\[
  ((h \circ g) \circ f)(x) = (h \circ g)(f(x)) = h(g(f(x))) = h((g \circ f)(x)) = (h \circ (g \circ f))(x)
\]

\textbf{Lemma} Eine Abbildung $f \colon X \to Y$ ist genau dann bijektiv, wenn eine Abbildung $g \colon Y \to X$
existiert mit $g \circ f = \text{id}_x, f \circ g = \text{id}_y$

\emph{Vorbereitung}: umschreiben als \\
(Eine Abbildung $f \colon X \to Y$ ist bijektiv) $\iff$ (es existiert eine Abbildung $g \colon Y \to X$
mit $g \circ f = \text{id}_x, f \circ g = \text{id}_y$) \\

\emph{Beweis} ( $\Rightarrow$ ): Sei $f \colon X \to Y$ bijektiv. Das heißt, dass nach obiger Überlegung über Urbilder,
dass jedes Element $y \in Y$ genau ein Urbild in $X$ unter $f$ hat, welches wir durch $x_y$ bezeichnen.
Wir definieren dann die Abbildung $g \colon Y \to X$ durch $y \mapsto x_y$.

\[
  (g \circ f)(x) = g(f(x)) = x_{f(x)} = x
\]
(nach Definition von $x_y$ gilt $x_{f(x)} = x$). Ferner gilt

\[
  (f \circ g)(y) = f(g(y)) = f(x_y) = y
\]
(nach Definition von $x_y$ gilt $f(x_y) = y$). Also gilt $\Rightarrow$. \\

\emph{Beweis} ( $\Leftarrow$ ): Angenommen, es existiert $g: Y \to X$ mit $g \circ f = id_x, f \circ g = id_y$.
Wir müssen zeigen, dass $f$ injektiv und surjektiv ist.\\

Zur Surjektivität: sei $y \in Y$ beliebig; setze $x \coloneqq g(y)$.
Nun gilt $f(x) = f(g(y)) = (f \circ g)(y) = id_y(y) = y$. Somit ist $f$ surjektiv. \\

Zur Injektivität: es muss gezeigt werden, dass aus $f(x_1) = f(x_2)$ folgt: $x_1 = x_2$ (hier haben wir
eine Kontraposition benutzt, um eine äquivalente ``leichter zu beweisende'' Aussage zu bekommen).
Sei $x_1, x_2 \in X$ beliebig mit $f(x_1) = f(x_2)$.

Wende $g$ and: $g(f(x_1)) = g(f(x_2))$. Es gilt aber $g \circ f = id_x$, also folgt $x_1 = x_2$. Somit ist $f$ injektiv. \\

\textbf{Alternativer Beweis}: Widerspruchsbeweis \\

Zur Injektivität: wir müssen zeigen, dass aus $x_1 \ne x_2$ folgt: $f(x_1) \ne f(x_2)$. Nehmen wir an, das wäre falsch:
dann existieren zwei Elemente $x_1, x_2 \in X$ mit $x_1 \ne x_2$, aber $f(x_1) = f(x_2)$.
Es gilt aber $g \circ f = id_x$, also folgt $x_1 = x_2$. Dies widerspricht aber der Annahme, dass $x_1 \ne x_2$.
Somit ist die Annahme ``$f$ ist nicht injektiv'' falsch und die Injektivität von $f$ folgt. \\

\textbf{Definition}: Sei $f \colon X \to Y$ eine bijektive Abbildung. Die Abbildung $g$ aus dem obigen Lemma
heißt \emph{inverse Abbildung zu $f$} und wird durch $f^{-1}$ bezeichnet. \\

\[
  \underbrace{
    \overbrace{
      X
        \underset{\underset{f^-1}\longleftarrow}{\overset{f}\to}
      Y
        \underset{\underset{g^-1}\longleftarrow}{\overset{g}\to}
      Z
    }^{\overset{f \circ g}\longrightarrow}
  }_{\underset{f^-1 \circ g^-1}\longleftarrow}
\]

\textbf{Korollar}: Die Verknüpfung zweier bijektiver Abbildungen $f \colon X \to Y$ und $g \colon Y \to Y$
ist bijektiv und es gilt

\[
  (g \circ f)^{-1} = f^{-1} \circ g^{-1}
\]

\emph{Beweis}: Seien $f, g$ bijektiv; nach dem Lemma existieren inverse Abbildungen $f^{-1}, g^{-1}$.
Nun gilt:

\[
  f^{-1} \circ g^{-1} \circ (g \circ f) = f^{-1} \circ \text{id}_y \circ f = f^{-1} \circ f = \text{id}_x
\]

\[
  (g \circ f) \circ f^{-1} \circ g^{-1} = g \circ \text{id}_y \circ g^{-1} = g \circ g^{-1} = \text{id}_y
\]

Also erfüllt $f^{-1} \circ g^{-1}$ die Aussagen des Lemmas und daher ist $(g \circ f)$ bijektiv.

\subsubsection*{Kardinalität}

\textbf{Definition}: Die Anzahl der Element einer Endlichen Menge $X$ heißt Kardinalität oder Mächtigkeit
von $X$. Bezeichnung $\abs{X} \in \mathbb{N}$.

Die spannende Frage ist nun, wie man die Kardinalität von unendlichen Mengen definiert.
Georg Cantor hat hier bereits erkannt: Die Menge der reellen Zahlen $\mathbb{R}$ hat mehr Elemente als
die Menge der natürlichen Zahlen $\mathbb{N}$. \\

\textbf{Definition}: Eine Menge $X$ heißt abzählbar unendlich, wenn es eine Bijektion
$f \colon \mathbb{N} \to X$ gibt.  \\

\textbf{Lemma} Jede unendliche Menge besitzt eine abzählbar unendliche Teilmenge. \\

\emph{Beweis}:  Wenn $x_1, x_2, \ldots \in X$ paarweise verschiedene Elemente sind und
$X' = \{ x_n | n \in \mathbb{N} \}$ so ist die Abbildung
$f \colon \mathbb{N} \to X' \subseteq{X}, n \mapsto x_n$ eine Bijektion: sie ist
offensichtlich surjektiv, aber da die Elemente paarweise verschieden sind, ist sie auch injektiv. \\

\textbf{Definition}: Eine unendliche Menge, die nicht abzählbar ist, heißt überabzählbar. \\

\textbf{Satz} (Cantor): Die Menge der reellen Zahlen ist überabzählbar: es gibt keine Bijektion
$f \colon \mathbb{N} \to \mathbb{R}$

\fcolorbox{black}{light-gray}{
  \begin{minipage}{\textwidth}
    \textbf{Homomorphismus} von altgriechisch ``gleiche Form'' oder ``ähnliche Form'' \\

    Ein Homomorphismus bildet Elemente aus einer Menge in eine andere so ab, so dass sich ihre
    Bilder hinsichtlich der Struktur wie die Urbilder verhalten.
  \end{minipage}
}
\vspace{1mm} \\

Seien nun $X, Y$ zwei Mengen und

\[
  \text{Hom}(X, Y) \coloneqq \{ f \colon X \to Y \}
\]

die Menge der Abbildungen von $X$ nach $Y$.

Sei nun $X$ endlich, $X = \{ x_1, \ldots, x_n\}$. Wir ordnen jeder Abbildung $f \colon X \to Y$ das
$n$-Tupel ihrer Werte zu:

\[
  w(f) = (f(x_1), \ldots, f(x_n)) \in \underbrace{(Y \times \ldots \times Y)}_n \eqqcolon Y^n
\]

Da jede Abbildung $f \colon X \to Y$ eindeutig durch die Folge ihrer Werte definiert ist, ist die Abbildung
$w \colon \text{Hom}(X,y) \to Y^n, f \mapsto w(f)$ bijektiv. \\

\fcolorbox{black}{light-gray}{
  \begin{minipage}{\textwidth}
    \textbf{Symmetrie} oder ``symmetrische Relation'' einer Menge \\

    Eine Relation ist symmetrisch genau wenn, aus $x$ steht in Relation zu $y$ auch
    $y$ steht in Relation zu $x$ folgt.
  \end{minipage}
}
\vspace{1mm} \\

\textbf{Definition}: Sei $X$ eine Menge. Die Elemente der Menge

\[
  \text{Sym}(X) \coloneqq \{ f \colon X \to X | f \text{ bijektiv} \}
\]

heißen Permutation von $X$. \\

\textbf{Lemma} (Anzahl der Permutationen): Sei $n > 0$ eine natürliche Zahle und $X, Y$
endliche Mengen mit $\abs{X} = \abs{Y} = n$. Dann isst die Anzahl $s_n$ von Bijektionen $f \colon X \to Y$
gleich

\begin{align*}
  n! = 1 * 2 * \ldots * n && \text{"($n$ Fakultät)"} \\ 
\end{align*}

\textbf{Korollar}: Wenn $X$ nichtleer ist und $\abs{X} = n$, gilt $\abs{\text{Sym}(X)} = n!$ \\

``Naiver Beweisversuch'': In der obigen Konstruktion entsprechen Bijektionen den Folgen, wo jedes Element
von $Y$ genau einmal vorkommt. Für das erste Element so einer Folge haben wir $n$ Möglichkeiten,
danach haben wir für das zweite Element $n - 1$ Möglichkeiten, für das dritte $n - 2$ Möglichkeiten,
$\ldots$, und für das letzte Element nur eine Möglichkeit.
Insgesamt hat man also $n * (n - 1) * \ldots * 1 = n!$ Möglichkeiten. \\

\emph{Beweis}: Sei $X = \{ x_1, \ldots, x_n\}, Y = \{ y_1, \ldots, y_n \}$ und $s_n$ die gesuchte Anzahl
der Bijektionen (diese hängt nur von $n$ ab, weil jede bijektive Abbildung $X \to Y$ eindeutig durch eine
Permutation von $\{ 1, \ldots, n \}$ bestimmt ist.)

Jede Bijektion $f \colon X \to Y$ ist eindeutig durch ihren Wert $f(x_1)$ an $x_1$ und die Einschränkung
von $f$ auf $X' \coloneqq X \setminus \{ x_1 \}$ beschrieben.
Letztere ist eine Bijektion $f|_{x'} \colon X' \to Y \setminus \{ y_i \}$.
Die Menge der Bijektionen von $X$ nach $Y$ zerlegt sich in $n$ Teilmengen

\[
  F_i \coloneqq \{ f \colon X \to Y | f \text{ bijektiv und } f(x_1) = y_i \},  i = 1, \ldots, n
\]

und jedes Element $f \in F_i$ ist eindeutig durch die Bijektion $f|_{x'} \colon X' \to Y \setminus {y_i}$
beschrieben.
Das heißt, jedes $F_i$ hat Kardinalität gleich der Anzahl der Bijektionen $X' \to Y \setminus \{ y_i \}$.
Somit gilt: $s_n = s_{n - 1} * n$ und die Behauptung folgt durch den Abstieg zu $s_1 = 1$

\begin{align*}
  F_i \cap F_j                    &= \emptyset, i \ne j, \\
  \{ \text{Bijektion } X \to Y \} &= F_1 \cup \ldots \cup F_n && \text{Die Vereinigung von } F_1, \ldots, F_n \text{ ist disjunkt.}\\
\end{align*}

\[
  \abs{\{ \text{Bijektionen } X \to Y \}} = \abs{F_1} + \abs{F_2} + \ldots + \abs{F_n}
\]

\[
  s_n = s_{n - 1} * n = s_{n - 2} * (n - 1) * n = \ldots = 1 * 2 * \ldots * (n - 1) * n = n!
\]

\subsubsection*{Induktion}

Im letzten Beweis gibt es im letzten Schritt eine Stelle, die formal eine Frage aufwirft:
inwiefern ist dieser ``Abstieg zu $s_1$'' gerechtfertigt?
Bei dieser (recht einfachen) Aussage scheint es nicht so problematisch, aber in komplizierten Situationen
kann man bei solchen naiven Abstiegsargumenten relativ leicht einen Fehler machen.
Um solche Abstiegsargumente in ``normale'' Folgerungsaussagen zu zerlegen, die man leichter nachprüfen
kann, benutzt man folgendes \emph{Induktionsprinzip}: \\

\textbf{Induktionsprinzip}: Sei für jedes $n \in \mathbb{N}$ eine Aussage $P(n)$ gegeben.
Wenn die Aussage $P(0)$ (Induktionsanfang) wahr ist und für jedes $n$ aus der Gültigkeit aller
Aussagen $P(k)$ für $k < n$ (Induktionsvoraussetzung) die Aussage $P(n$) folgt (Induktionsschritt),
dann gilt $P(n)$ für alle $n \in \mathbb{N}$. \\

\emph{Beispiel}:

\[
  P(n) = (\text{Anzahl von Bijektionen zwischen zwei $n$-elementigen Mengen ist $n!$})
\]

Statt mit einem Abstieg wird hier mit einem Aufstieg von $P(0)$ an gearbeitet.
Manchmal will man die Aussage eher für alle $n \in \mathbb{N}^* = \{ n \in \mathbb{N} | n > 0 \}$ zeigen,
dann ist der Induktionsanfang die Aussage $P(1)$.

Mit Hilfe der Induktion könnte man im obigen Beweis den letzten Schritt so durchführen: \\

\emph{Beweis durch Induktion}: Wir benutzen Induktion über $n = \abs{X} = \abs{Y}$.
Für $n = 1$ ist die Aussage wahr, weil es nur eine Abbildung von $\{ x_1^\}$ nach $\{ y_1 \}$ gibt,
nämlich die, die $x_1$ auf $y_1$ schickt und diese ist bijektiv.

Sei $X = \{ x_1, \ldots, x_n\}, Y = \{ y_1, \ldots, y_n \}$ und $s$ die gesuchte Anzahl der Bijektionen.
Jede Bijektion $f \colon X \to Y$ lässt sich folgendermaßen zerlegen: wenn $f(x_1) = y_i$, ist
die Einschränkung von $f$ auf $X' \coloneqq X \setminus \{ x_1 \}$ eine Bijektion
$f|x' \colon X' \to Y \setminus \{ y_i \}$. Das heißt, die Menge der Bijektionen von $X$ nach $Y$ zerlegt sich in $n$
disjunkte Teilmengen.

\[
  F_i \coloneqq \{ f \colon X \to Y | f \text{ bijektiv und } f(x_1) = y_i \},  i = 1, \ldots, n
\]

und jede davon hat Kardinalität gleich der Anzahl der Bijektionen $X' \to Y \setminus \{ y_i \}$.
Nach Induktionsvoraussetzung ist diese gleich $(n - 1)!$ und es folgt $s_n = n * (n - 1)! = n!$.

\subsubsection*{Abbildungen als Graphen}

Der Sonderfall $X = Y = \emptyset$ kann man im obigen Lemma auch abarbeiten: man setzt $0! \coloneqq 1$,
weil $\abs{\text{Sym}(\emptyset)} = 1$. Das Letztere ist mit der Definition durch ``Vorschrift'' nicht
ganz offensichtlich, ergibt aber schon Sinn, wenn man eine Abbildung $f \colon X \to Y$ mit den
\emph{Graphen der Abbildung}

\[
  \Gamma_f = \{ (x, f(x)) | x \in X \} \subseteq X \times Y 
\]

identifiziert. Wenn man tatsächlich rein durch Formalismus der Mengenlehre definieren möchte, was eine
Abbildung ist, kann man es so machen: \\

\textbf{Definition}: Eine Abbildung $f \colon X \to Y$ ist eine Teilmenge $\Gamma_f \subseteq X \times Y$,
welche die Eigenschaft

\[
  \forall x \in X \hspace{2pt} \exists! y \in Y \colon (x, y) \in \Gamma_f
\]

hat. ($\exists!$ heißt ``existiert ein eindeutig bestimmtes'') \\

Dann ist für $X = \emptyset$ wegen $X \times Y = \emptyset$ klar, dass nur eine mögliche Abbildung
$f_\emptyset \colon \emptyset \to Y$ existiert, und zwar die mit $\Gamma_f = \emptyset$. Andererseits
gibt es keine Abbildung $X \to \emptyset$, wenn $X$ nichtleer ist, weil die Existenzaussage nie
erfüllt ist.

\subsection*{Gruppen}

\fcolorbox{black}{light-gray}{
  \begin{minipage}{\textwidth}
    Nils Henrik Abel \gtrsymBorn~5. August 1802 \gtrsymDied 6. April 1829

    Norwegischer Mathematiker und wichtiger Mitbegründer der Gruppentheorie. \\
    Ihm zu Ehren nennt man kommutative Gruppen auch \emph{Abelsche Gruppen}. 
  \end{minipage}
}

Übung: konstruieren Sie zwei Permutationen $f, g$ von $\{ 1, 2, 3 \}$,
die nicht kommutieren: $g \circ f \ne  f \circ g$
\[
  f \colon 1 \mapsto 2, 2 \mapsto 3, 3 \mapsto 1
\]
\[
  g \colon 1 \mapsto 1, 2 \mapsto 3, 3 \mapsto 2
\]
\[
  g \circ f \colon 1 \mapsto 3, 2 \mapsto 1, 3 \mapsto 1
\]
\[
  f \circ g \colon 1 \mapsto 2, 2 \mapsto 1, 3 \mapsto 3
\]
\[
  g \circ f \ne f \circ g
\]

Übung: $(\mathbb{Z} \setminus \{ 0 \}, *)$ ist keine abelsche Gruppe, warum?

\begin{enumerate}[label=(G\arabic*)]
\item $\forall a,b,c \in \mathbb{Z} \colon a * (b * c) = (a * b) * c$
\item $\exists e \in \mathbb{Z} \colon \forall a \in \mathbb{Z} \colon a * e = a$, mit $e = 1$
\item $\forall a \in \mathbb{Z} \colon \exists a^{-1} \in \mathbb{Z} \colon a * a^{-1} = e$ mit $e = 1$. \\
  Diese Aussage gilt nicht. Es gibt kein $a^{-1} \in \mathbb{Z}$ mit $2 * a^{-1} = 1$
\end{enumerate}

\setcounter{section}{2}
\section{Lineare Gleichungssystem und Matrizen}

\setcounter{subsection}{1}
\subsection{Matrizen}

\begin{align*}
  \begin{pmatrix}
    \textcolor{red}{1} & \textcolor{red}{2} & \textcolor{red}{3} \\
    4 & 5 & 6 \\
  \end{pmatrix}
  \cdot
  \begin{pmatrix}
    1  & 1  & 1 \\
    -1 & 0  & 0 \\
    2  & -2 & 0 \\
  \end{pmatrix}
  &=
  \begin{small}      
    \begin{pmatrix}
      \textcolor{red}{1} * 1 + \textcolor{red}{2} * (-1) + \textcolor{red}{3} * 2 &
      \textcolor{red}{1} * 1 + \textcolor{red}{2} * 0 + \textcolor{red}{3} * (-2) &
      \textcolor{red}{1} * 1 + \textcolor{red}{2} * 0 + \textcolor{red}{3} * 0 \\
      1 * 4 + 5 * (-1) + 6 * 2 & 4 * 1 + 5 * 0 + 6 * (-2) & 4 * 1 + 5 * 0 + 6 * 0 \\
    \end{pmatrix}
  \end{small}\\
  &=
  \begin{pmatrix}
    5  & -5 & 1 \\
    11 & -8 & 4 \\
  \end{pmatrix}
\end{align*}

\textbf{Vorsicht!! Selbst wenn $A \cdot B$ und $B \cdot A$ definiert sind, muss $A \cdot B = B \cdot A$
nicht gelten!}

\begin{align*}
  \begin{pmatrix}
    1 & 0 \\
    0 & 2 \\
  \end{pmatrix}
  \cdot
  \begin{pmatrix}
    1 & 2 \\
    3 & 4 \\
  \end{pmatrix}
  &=
  \begin{pmatrix}
    1 & 2 \\
    6 & 8 \\
  \end{pmatrix} \\
  \begin{pmatrix}
    1 & 2 \\
    3 & 4 \\
  \end{pmatrix}
  \cdot
  \begin{pmatrix}
    1 & 0 \\
    0 & 2 \\
  \end{pmatrix}
  &=
  \begin{pmatrix}
    1 & 4 \\
    3 & 8 \\
  \end{pmatrix}
\end{align*}

\section{Vektorräume}

\subsection{Definition und erste Beispiele}

Formale Summen sind Koeffizientenfolgen.

\[
  = \{ a \colon \mathbb{N} \to \mathbb{K} | a(k) \ne 0 \text{ für endlich viele } k \in \mathbb{N} \}
\]

\emph{Übung}. In jedem Vektorraum $V$ über einem Körper $\mathbb{K}$ gilt für beliebige

$\lambda, \mu \in \mathbb{K}, \upsilon, \omega \in V$

\begin{enumerate}[1)]
\item
  \[
    \lambda \cdot {\color{ForestGreen}0} = {\color{ForestGreen}0}
  \]
\item
  \[
    \lambda \cdot (-\upsilon) = - \lambda \cdot \upsilon
  \]
\item
  \[
    \lambda \cdot (\upsilon - \omega) = \lambda \cdot \upsilon - \lambda \cdot \omega)
  \]
\item
  \[
    {\color{BurntOrange}0} \cdot \upsilon = {\color{ForestGreen}0}
  \]
\item
  \[
    -1 \cdot \upsilon = - \upsilon
  \]
\item
  \[
    (\lambda - \mu) \cdot \upsilon = \lambda \cdot \upsilon - \mu \cdot \upsilon
  \]
\end{enumerate}

Welche Bedeutung hat $0$ in jedem der obigen Fälle?

{\color{ForestGreen}Nullvektor}, {\color{BurntOrange}Element aus Körper} \\

\emph{Beweis}

\begin{enumerate}[1)]
\item
  Sei $\upsilon \coloneqq \lambda \cdot 0$

  \begin{align*}
    \upsilon + \upsilon &= \lambda \cdot 0 + \lambda \cdot 0 \\
                        &= \lambda \cdot (0 + 0) \\
                        &= \lambda \cdot 0 = \upsilon \\
    \Rightarrow \ upsilon + \upsilon = \upsilon &\Rightarrow \upsilon = 0 \\
  \end{align*}
\item
  \begin{align*}
    \lambda \cdot (- \upsilon) &= -\lambda \cdot \upsilon \\
    \lambda \cdot (- \upsilon) + \lambda \cdot \upsilon &= \lambda \cdot (- \upsilon + \upsilon) \\
                               &= \lambda \cdot 0 \\
                               &= 0\\
  \end{align*}
\item
\item
\item
  \begin{align*}
    -1 \cdot \upsilon &= -\upsilon \\
    \iff -1 \cdot \upsilon + \upsilon &= 0 \\
    -1 \cdot \upsilon + \upsilon &= -1 \cdot \upsilon + 1 \cdot \upsilon \\
                      &= (-1 + 1) \cdot \upsilon \\
                      &= 0 \cdot \upsilon \\
                      &= 0 \\
  \end{align*}
\item
  \begin{align*}
    (\lambda - \mu) \cdot \upsilon                           &= \lambda \cdot \upsilon - \mu \cdot \upsilon \\
    \iff (\lambda - \mu) \cdot \upsilon + \mu \cdot \upsilon &= \lambda \cdot \upsilon \\
    \overset{(V2)}&{=} (\lambda - \mu + \mu) \cdot \upsilon \\
                                                             &= \lambda \cdot \upsilon \\
  \end{align*}
\end{enumerate}

\end{document}