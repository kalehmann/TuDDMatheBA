\documentclass{article}

\usepackage{amsmath}
\usepackage{amssymb}
\usepackage{genealogytree}
\usepackage{hyperref}
\usepackage[utf8]{inputenc}
\usepackage{mathtools}
\usepackage{tikz}
\usetikzlibrary{positioning}
\usepackage{xcolor}
\definecolor{light-gray}{gray}{.9}

\setlength\parindent{0pt}

\author{Karsten Lehmann}
\date{WiSe 2020}
\title{Mitschriften Lineare Algebra - Grundlegende Konzepte (LA10)}

\begin{document}


\maketitle

\vfill
\begin{center}
  Dozent: Vadim Alekseev \\
  \href{mailto:vadim.alekseev@tu-dresden.de}{vadim.alekseev@tu-dresden.de}
\end{center}

\newpage

\section*{Einführung}

\textbf{Was ist Mathematik?}

Mathematik beantwortet Fragen über Rechenstrukturen und geometrische Sachverhalte allgemeingültig.

\textbf{Theorem} (Pythagoras):
Wenn $a \leq b \leq c$ Längen von Seiten eines rechtwinkligen Dreiecks sind, so gilt:

\begin{center}
$a^2 + b^2 = c^2$
\end{center}

Hier haben wir Voraussetzungen \ldots
\begin{itemize}
\item $a \leq b \leq c$
\item Dreieck
\item Rechtwinklig
\end{itemize}

\ldots und eine Behauptung \emph{"dann gilt"} $a^2 + b^2 = c^2$ \\


Damit dieses Theorem wirklich allgemeingültig ist müssen die Voraussetzungen präzise definiert sein.
Zum Beispiel muss \emph{``Rechtwinkliges Dreieck''} genau definiert sein.

Man hat am Anfang des 20. Jahrhunderts erkannt, dass diese genaue Definition eine spezielle Sprache
benötigt - den Formalismus der Logik und Mengenlehre.

Da diese Sprache sehr schwer intuitiv zu erfassen ist, wird hier mit "naiver" Logik und Mengenlehre
angefangen, um später weitere Behauptungen in dieser Sprache formulieren zu können.

\section{Mengenlehre}

Die Mengenlehre befasst sich mit \emph{Mengen}.

\fcolorbox{black}{light-gray}{
  \begin{minipage}{\textwidth}
    Georg Cantor \gtrsymBorn~19. Februar/3.März 1854 \gtrsymDied 6. Januar 1918

    Deutscher Mathematiker und Begründer der Mengenlehre. \\

    Definition der Menge (1895):
    \emph{Unter einer Menge verstehen wir jede Zusammenfassung M von bestimmten wohlunterschiedenen Objekten
    m unserer Anschauung oder unseres Denkens (welche die Elemente von M genannt werden) zu einem Ganzen}
  \end{minipage}
}


\emph{Beispiel}: $Z = \{ \text{Zoomteilnehmer der Vorlesung} \}$ \\
$Y = \{ \text{Teilnehmer der Vorlesung auf YouTube} \}$, \\
$A = \{ 0, 1, 2, 3 \}$

\textbf{Notation}: Wenn $a$ ein Element der Menge $A$ ist, dann schreibt man $a \in A$.

\textbf{Standardnotation für bestimmte Mengen in der Mathematik}:

\begin{itemize}
\item Symbol $\mathbb{N}$: Natürliche Zahlen $\{0, 1, 2, 3, \ldots \}$
\item Symbol $\mathbb{Z}$: Ganze Zahlen $\{ \ldots, -1, 0, 1, \ldots \}$
\item Symbol $\mathbb{Q}$: Rationale Zahlen $\{ \frac{p}{q} | p \in \mathbb{Z}, q \in \mathbb{N}, q \ne 0 \} \subset \mathbb{R}$ 
\item Symbol $\mathbb{R}$: Reelle Zahlen
\end{itemize}

\textbf{Notation}: Um eine Menge durch Eigenschaften ihrer Elemente zu beschreiben, verwendet man die Notation

$A = \{a | a \text{ hat eine gewisse Eigenschaft} \}$ \\

\textbf{Teilmenge}: $A$ und $B$ sind zwei Mengen. $A$ ist eine Teilmenge von $B$ wenn jedes Element von $A$ auch
in $B$ enthalten ist. Man schreibt $A \subseteq B$. \\

Wenn $A \subseteq B$ und $B \subseteq A$, dann gilt $A = B$. \\

\textbf{Notation}: wenn $A \subseteq B, A \ne B$, dann schreibt man $A \subsetneq B$.

Konstruktion von neuen Mengen aus bekannten Mengen: \\

\textbf{Definition Vereinigung}: Wenn $A$ und $B$ Mengen sind, dann ist die Vereinigung von $A$ und $B$ die
Menge $A \cup B = \{ x | x \in A \text{ oder } x \in B \}$. Ein \emph{"oder"} meint in der Mathematik immer
\emph{"Nicht exklusives oder"}, das heißt $x$ kann in $A$, $B$ oder in beiden enthalten sein. \\

\textbf{Definition Durchschnitt}: Wenn $A$ und $B$ Mengen sind, dann ist der Durchschnitt von $A$ und $B$ die
Menge an Elementen, die in $A$ und $B$ liegen. Den Durchschnitt schreibt man

$A \cap B = \{ x | x \in A \text{ und } x \in B\}$ \\

\textbf{Definition Differenz}: Wenn $A$ und $B$ Mengen sind, dann ist die Differenz von $A$ und $B$ die
Menge der Elemente, die in $A$ liegen, aber nicht in $B$. Die Differenz schreibt man

$A \setminus B = \{ x | x \in A \text{ und } x \notin B\}$ \\

Wenn $A$ und $B$ keine gemeinsamen Elemente haben, dann ist $A \cap B = \emptyset$.
Die \textbf{leere Menge} $\emptyset$ ist eine Menge, die keine Elemente enthält. \\

Beispiel: \\
$\{e \in \mathbb{R} | x^2 < 0\} = \emptyset$ \\

Eigenschaften, für jede Menge $A$ gilt:
o\begin{itemize}
\item $A \cup \emptyset = A$
\item $A \cap \emptyset = \emptyset$
\end{itemize}

Frage: Angenommen $A$ ist eine Menge, gilt dann $\emptyset \subseteq A$?
Formal gesprochen ist diese Aussage \emph{"jedes Element der leeren Menge ist auch ein Element von A"}.
Jedoch hat die leere Menge keine Elemente.

Intuitiv lässt sich diese Frage schwer beantworten, deswegen wurde eine Konvention geschaffen: \\

Für jede Menge $A$ gilt $\emptyset \subseteq A$

\subsection*{Ausblick: Aussagen}

Bei den gelisteten Definition sieht man folgende Konstrukte:

\begin{enumerate}
\item $x \notin B$
\item $x \in A$ und $x \in B$
\item $x \in A$ oder $x \in B$
\item \emph{"jedes Element von $A$ ist ein Element von $B$"}, oder anders gesagt aus $x \in A$ folgt $x \in B$
\end{enumerate}

Diese Konstrukte sind Beispiele für \textbf{Aussagen}. Eine Aussage ist eine Äußerung, von der man fragen kann
ob sie wahr oder falsch ist.

Die oben genannten Konstrukte entsprechen folgenden Operationen für Aussagen:

$P = (x \in A)$ und $Q = (x \in B)$:
\begin{enumerate}
\item "nicht": die Aussage $\neg Q = "Q$ ist nicht wahr" heißt Negation von Q
\item "und": die Aussage "$P$ und $Q$ sind beide wahr" heißt Konjunktion $P \wedge Q$ von $P$ und $Q$
\item "oder": die Aussage "mindestens eine der Aussagen $P$ und $Q$ ist wahr" heißt Disjunktion $P \vee Q$ von
  $P$ und $Q$
\item "folgt": die Aussage "wenn $P$, dann $Q$" heißt Implikation und wird durch $P \implies Q$ dargestellt.
\end{enumerate}

Wann ist eigentlich $P \implies Q$ wahr? Per Konvention ist diese Aussage nur dann falsch, wenn $P$ wahr und
$Q$ falsch ist, in allen anderen Fällen ist $P \implies Q$ wahr (Dies entspricht dem Prinzip "aus einer falschen Aussage folgt eine beliebige")!

Bei Beweisen muss man sich somit nur um den Fall, dass $P$ wahr ist kümmern.

\subsection*{Übung Wahrheitstabelle}

\begin{displaymath}
  \begin{array}{c | c }  
    Q & \neg Q \\
    \hline
    w & f \\
    f & w \\
  \end{array}
\end{displaymath}

\begin{displaymath}
  \begin{array}{ c | c | c | c }
    P & Q & P \land Q & P \lor Q \\
    \hline
    0 & 0 & 0 & 0 \\
    0 & 1 & 0 & 1 \\
    1 & 0 & 0 & 1 \\
    1 & 1 & 1 & 1 \\
  \end{array}
\end{displaymath}

\begin{displaymath}
  \begin{array}{ c | c | c }
    P & Q & P \implies Q \\
    \hline
    0 & 0 & 1 \\
    0 & 1 & 1 \\
    1 & 0 & 0 \\
    1 & 1 & 1 \\
  \end{array}
\end{displaymath}

Von diesem Standpunkt aus kann nun die Aussage $\emptyset \subseteq A$ erneut betrachtet werden,
sie ist dann gleich $(x \in \emptyset) \implies (x \in A)$.
Da $x \in \emptyset$ jedoch stets falsch ist, ist die Implikation nach dem obigen Prinzip wahr.

Somit gilt $\emptyset \subseteq A$. \\

Die Aussage "aus $P$ folgt $Q$ und aus $Q$ folgt $P$ $((P \implies Q) \land (Q \implies P))$ kürzt man
als $P \iff Q$ ab ($P$ gilt genau dann, wenn $Q$ gilt), dies ist die Aussage, dass $P$ und $Q$ \emph{äquivalent}
oder \emph{gleichbedeutend} sind. \\

\textbf{Übung} Überprüfen Sie, dass $P \iff Q$ nur dann wahr ist, wenn $P$ und $Q$ gleichzeitig wahr oder
gleichzeitig falsch sind.

\textbf{Übung} Überprüfen Sie, dass $P \implies Q$ äquivalent zu $\neg Q \implies \neg P$ ist.

\begin{displaymath}
  \begin{array}{ c | c | c | c | c | c}
    P & Q & P \implies Q & Q \implies P & \neg Q \implies \neg P & P \iff Q \\
    \hline
    0 & 0 & 1 & 1 & 1 & 1 \\
    0 & 1 & 1 & 0 & 1 & 1 \\
    1 & 0 & 0 & 1 & 0 & 0 \\
    1 & 1 & 1 & 1 & 1 & 1 \\
  \end{array}
\end{displaymath}

\textbf{Definition Abbildung}: Angenommen $X$ und $Y$ sind Mengen. Eine Abbildung (oder auch Funktion) $f$
von $X$ nach $Y$ (Notation: $f: X \to Y$) ist eine Vorschrift, welche jedem Element $x \in X$ genau ein
Element $f(x) \in Y$ zuordnet (Notation: $x \mapsto f(x)$). \\

\emph{Beispiel} $f: \mathbb{R} \to \mathbb{R}, x \mapsto x^{3}$ ist die Notation zur Einführung der Funktion
$f(x) = x^{3}$. \\

\emph{Beispiel (Identität)}. Angenommen $X$ ist eine Menge. Die Abbildung $\text{id}_{x}: X \to X, x \mapsto x$
heißt Identitätsbildung oder kurz Identität. \\

\textbf{Definition kartesisches Produkt}: Das kartesische Produkt zweier Mengen $A$ und $B$ ist die Menge der
Paare

$A \times B \coloneqq \{(a,b) | a \in A, b \in B\}$

\begin{tikzpicture}
  %% Überschrift
  \node at (3,5.5) {$A = [a_{0},b_{0}], B = [a_{1}, b_{1}]$};
  %% Achsen
  \draw (-1,0) -> (6,0);
  \draw (0,-1) -> (0,5);
  %% Intervall an der vertikalen Achse
  \node[rotate=90] at (0, 1) (b0) {[};
  \node[left = .3cm of b0] {$b_{0}$};
  \node[rotate=90] at (0, 4) (b1) {]};
  \node[left = .3cm of b1] {$b_{1}$};
  %% Intervall an der horizontalen Achse
  \node at (1, 0) (a0) {[};
  \node[below = .3cm of a0] {$a_{0}$};
  \node at (5, 0) (a1) {]};
  \node[below = .3cm of a1] {$a_{1}$};

  %% Rechteck innerhalb des Diagramms mit dashed Linien
  \draw (1,1) rectangle ++(4,3);
  \draw[dashed] (1,0) -- (1,5);
  \draw[dashed] (5,0) -- (5,5);
  \draw[dashed] (0,1) -- (6,1);
  \draw[dashed] (0,4) -- (6,4);

  %% Einzelner Punkt im Rechteck
  \coordinate (dot) at (4,3);
  \draw[fill] (dot) circle (.07);
  \node[right = 0cm of dot] {$(a,b)$};

  %% Rechteck mit B und A als Beschriftung der Seiten, sowie dem Punkt (a,b)
  \draw (1,-4.5) rectangle ++(4,3);
  \node at (0.5, -3) {B};
  \node at (3,-5) {A};
  \coordinate (dot2) at (4,-2.5);
  \draw[fill] (dot2) circle (.07);
  \node[right = 0cm of dot2] {$(a,b)$};  
\end{tikzpicture}

\emph{Beispiel (Projektionsabbildung)}: Angenommen $A$ und $B$ sind Mengen. Die Projektionsabbildung vom
kartesischen Produkt auf die erste Komponente ist so definiert:

\begin{align*}
\pi_{A}: A \times B &\to A, \\
(a,b) &\mapsto a\
\end{align*}

\textbf{Definition}: Eine Abbildung $f:X \to Y$ heißt

\begin{enumerate}
\item injektiv, falls für alle $x_1 \ne x_2 \in X$ gilt: $f(x_1) \ne f(x_2)$
\item surjektiv, falls für jedes $y \in Y$ ein $x \in X$ existiert mit $f(x) = y$
\item bijektiv, falls sie injektiv und surjektiv ist
\end{enumerate}

\end{document}