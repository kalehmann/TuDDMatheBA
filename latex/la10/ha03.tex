\documentclass{article}
\usepackage{aligned-overset}
\usepackage{amsmath}
\usepackage{amssymb}
\usepackage[shortlabels]{enumitem}
\usepackage{genealogytree}
\usepackage{hyperref}
\usepackage[utf8]{inputenc}
\usepackage{mathtools}
\usepackage{MnSymbol}
\usepackage{pgfplots}
\usepackage{physics}
\usepackage{tikz}
\usetikzlibrary{positioning}
\usepackage{xcolor}
\definecolor{light-gray}{gray}{.9}
\usepackage{xfrac}

\author{Karsten Lehmann}
\date{WiSe 2020}
\title{Lineare Algebra I: Übungen - Blatt 3}

\begin{document}

\maketitle

\vfill

\newpage
\section*{Übung 9}

\begin{enumerate}[(i)]
\item
  \[
    \left(
      \begin{array}{ccc}
        1 & 1 & 1 \\
      \end{array}
    \right)
    \cdot
    \left(
      \begin{array}{ccc}
        1 & 2 & 3 \\
        2 & 3 & 4 \\
        3 & 4 & 5 \\
      \end{array}
    \right)
    =
    \left(
      \begin{array}{ccc}
        6 & 9 & 12 \\
      \end{array}
    \right)
  \]
\item
  \[
    \left(
      \begin{array}{cccc}
        \lambda_1 &           &        & 0 \\
                  & \lambda_2 \\
                  &           & \ddots \\
        0         &           &        & \lambda_n \\
      \end{array}
    \right)
    \cdot
    \left(
      \begin{array}{cccc}
        \sfrac{1}{\lambda_1} &                      &        & 0 \\
                             & \sfrac{1}{\lambda_2} \\
                             &                      & \ddots \\
        0                    &                      &        & \sfrac{1}{\lambda_n} \\
      \end{array}
    \right)
    =
    \left(
      \begin{array}{cccc}
        1 &   &        & 0 \\
          & 1 &        &   \\
          &   & \ddots &   \\
        0 &   &        & 1 \\
      \end{array}
    \right)
  \]

\item
  \[
    \left(
      \begin{array}{cccc}
        \lambda_1 &           &        & 0 \\
                  & \lambda_2 \\
                  &           & \ddots \\
        0         &           &        & \lambda_n \\
      \end{array}
    \right)
    \cdot
    \left(
      \begin{array}{cccc}
        0         &        &        & \lambda_1 \\
                  &        & \lambda_2 \\
                  & \udots \\
        \lambda_n &        &        & 0 \\
      \end{array}
    \right)
    =
    \left(
      \begin{array}{cccc}
        0           &         &             & \lambda_1^2 \\
                    &         & \lambda_2^2 &   \\
                    & \udots  &             &   \\
        \lambda_n^2 &         &             & 0 \\
      \end{array}
    \right)
  \]

\item
  \[
    \left(
      \begin{array}{cccc}
        0         &        &        & \lambda_1 \\
                  &        & \lambda_2 \\
                  & \udots \\
        \lambda_n &        &        & 0 \\
      \end{array}
    \right)
    \cdot
    \left(
      \begin{array}{cccc}
        \lambda_1 &           &        & 0 \\
                  & \lambda_2 \\
                  &           & \ddots \\
        0         &           &        & \lambda_n \\
      \end{array}
    \right)
    =
    \left(
      \begin{array}{cccc}
        \lambda_1^2 &             &        & 0 \\
                    & \lambda_2^2 &        &   \\
                    &             & \ddots &   \\
        0           &             &        & \lambda_n^2 \\
      \end{array}
    \right)
  \]

\item
  \[
    \left(
      \begin{array}{ccc}
        0 & 0 & 1 \\
        0 & 1 & 0 \\
        1 & 0 & 0 \\
      \end{array}
    \right)
    \cdot
    \left(
      \begin{array}{ccc}
        0 & 0 & 1 \\
        0 & 1 & 0 \\
        1 & 0 & 0 \\
      \end{array}
    \right)
    =
    \left(
      \begin{array}{ccc}
        1 & 0 & 0 \\
        0 & 1 & 0 \\
        0 & 0 & 1 \\
      \end{array}
    \right)
  \]
  
\end{enumerate}

\section*{Übung 10}
\begin{enumerate}[(i)]
\item
  \begin{enumerate}[a)]
  \item
    \begin{align*}
      \left(
        \begin{array}{cc}
          1 & 1 \\
          1 & 1 \\
        \end{array}
      \right)^2
      - 2 *
      \left(
        \begin{array}{cc}
          1 & 1 \\
          1 & 1 \\
        \end{array}
      \right)
      + 1
      &=
      \left(
        \begin{array}{cc}
          2 & 2 \\
          2 & 2 \\
        \end{array}
      \right)
      - 
      \left(
        \begin{array}{cc}
          2 & 2 \\
          2 & 2 \\
        \end{array}
      \right)
      + 1 \\
      &= 1\\
    \end{align*}

  \item
    \begin{align*}
      \left(
        \begin{array}{cc}
          1 & 1 \\
          0 & 1 \\
        \end{array}
      \right)^2
      - 2 *
      \left(
        \begin{array}{cc}
          1 & 1 \\
          0 & 1 \\
        \end{array}
      \right)
      + 1
      &=
      \left(
        \begin{array}{cc}
          1 & 2 \\
          0 & 1 \\
        \end{array}
      \right)
      - 
      \left(
        \begin{array}{cc}
          2 & 2 \\
          0 & 2 \\
        \end{array}
      \right)
      + 1 \\
      &=
      \left(
        \begin{array}{cc}
          -1 & 0 \\
          0 & -1 \\
        \end{array}
      \right)
      + 1 \\
      &=
      \left(
        \begin{array}{cc}
          0 & 1 \\
          1 & 0 \\
        \end{array}
      \right)  
      \\
    \end{align*}
  \end{enumerate}

\item
  \begin{align*}
    \left(
    \begin{array}{ccc}
      -1 & 1   & 1   \\
      -5 & 21  & 17  \\
      6  & -26 & -21 \\
    \end{array}
    \right)^2
    +
    \left(
    \begin{array}{ccc}
      -1 & 1   & 1   \\
      -5 & 21  & 17  \\
      6  & -26 & -21 \\
    \end{array}
    \right)
    + 1
    &=
    \left(
    \begin{array}{ccc}
      2  & -6 & -5 \\
      2  & -6 & -5 \\
      -2 & 6  & 5  \\
    \end{array}
    \right)
    +
    \left(
    \begin{array}{ccc}
      -1 & 1   & 1   \\
      -5 & 21  & 17  \\
      6  & -26 & -21 \\
    \end{array}
    \right)
    + 1 \\
    &=
    \left(
    \begin{array}{ccc}
      1  & -5  & -4  \\
      -3 & 15  & 12  \\
      4  & -20 & -16 \\
    \end{array}
    \right)
    + 1 \\
    &=
    \left(
    \begin{array}{ccc}
      2  & -4  & -3  \\
      -2 & 16  & 13  \\
      5  & -19 & -15 \\
    \end{array}
    \right)
    \end{align*}
\end{enumerate}

\section*{Übung 11}

\begin{enumerate}[(i)]
\item
  \begin{align*}
    e'_i * A                                            &= \left( a_{i1}, a_{i2}, \ldots, a_{in} \right) \\
    \left( a_{i1}, a_{i2}, \ldots, a_{in} \right) * e_j &= \left( a_{ij} \right)
  \end{align*}

\item
  $B$ ist eine $n \times 1$ Matrix und die erste Spalte der Einheitsmatrix der Größe $n$ (siehe (i)).
  \[
    B = \left(
      \begin{array}{c}
        1 \\
        0 \\
        \vdots \\
        0
      \end{array}
    \right)
  \]

  $C$ ist eine $1 \times m$ Matrix und die erste Zeile der Einheitsmatrix der Größe $n$.
  \[
    C = \left(
      \begin{array}{cccc}
        1 & 0 & \ldots & 0 \\
      \end{array}
    \right)
  \]
\end{enumerate}

\newpage
\section*{Übung 12}

\begin{enumerate}[(i)]
\item
  \begin{align*}
    (A + B)^2 &= (A + B) \cdot (A + B) \\
              &= A \cdot  (A + B) + B \cdot (A + B) \\
              &= A \cdot A + A \cdot B + B \cdot A + B \cdot B \\
              &= A^2 + A \cdot B + B \cdot A + B^2 \\
  \end{align*}
  
  Sei nun $A = \left(\begin{array}{cc}1 & 0 \\ 0 & 2 \\ \end{array}\right)$ und
  $B = \left(\begin{array}{cc}1 & 2 \\ 3 & 4 \\ \end{array}\right)$

  \begin{align*}
    A \cdot B =
    \begin{pmatrix}
      1 & 0 \\
      0 & 2 \\
    \end{pmatrix}
    \cdot
    \begin{pmatrix}
      1 & 2 \\
      3 & 4 \\
    \end{pmatrix}
    &=
    \begin{pmatrix}
      1 & 2 \\
      6 & 8 \\
    \end{pmatrix} \\
    B \cdot A =
    \begin{pmatrix}
      1 & 2 \\
      3 & 4 \\
    \end{pmatrix}
    \cdot
    \begin{pmatrix}
      1 & 0 \\
      0 & 2 \\
    \end{pmatrix}
    &=
    \begin{pmatrix}
      1 & 4 \\
      3 & 8 \\
    \end{pmatrix} \\
    A \cdot B + B \cdot A
    &=
    \begin{pmatrix}
      2 & 6 \\
      9 & 16 \\
    \end{pmatrix} \\
    2 \cdot A \cdot B
    &=
    \begin{pmatrix}
      2  & 4  \\
      12 & 16 \\
    \end{pmatrix} \\
  \end{align*}

  Somit ist in diesem Fall $2 \cdot A \cdot B \ne A \cdot B + B \cdot A$ und die Aussage stimmt nicht.

\item
  \begin{align*}
    (A + B) \cdot (A - B) &= A \cdot (A - B) + B \cdot (A - B) \\
                          &= A \cdot A - A \cdot B + B \cdot A - B \cdot B \\
                          &= A^2 - A \cdot B + B \cdot A - B^2 \\
                          &= (A^2 - B^2) + B \cdot A - A \cdot B \\
  \end{align*}

  In (i) wurde bereits ein Beispiel gezeigt, in dem $B \cdot A \ne A \cdot B$. Somit stimmt
  die Aussage $(A^2 - B^2) = (A + B) \cdot (A - B)$ ebenfalls nicht.
\end{enumerate}


\end{document}