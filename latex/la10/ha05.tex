\documentclass{article}
\usepackage{aligned-overset}
\usepackage{amsmath}
\usepackage{amssymb}
\usepackage{bm}
\usepackage[shortlabels]{enumitem}
\usepackage{hyperref}
\usepackage[utf8]{inputenc}
\usepackage{mathtools}
\usepackage{physics}
\usepackage{titling}
\usepackage{tikz}
\usetikzlibrary{calc}
\usepackage{fancyhdr}
\usepackage{xfrac}

\newcommand{\tikzmark}[1]{\tikz[overlay,remember picture]{ \node (#1) {};}}

\author{Karsten Lehmann}
\date{WiSe 2020}
\title{Hausaufgabe 05 Lineare Algebra}

\pagestyle{fancy}
\fancyhf{}
\lhead{\thetitle}
\rhead{\theauthor}
\lfoot{\thedate}
\rfoot{Seite \thepage}

\begin{document}
\section*{Übung 19}

Finden Sie jeweils die Dimension und eine Basis für die lineare Hülle folgender Vektoren in $\mathbb{R}^n$:

\begin{enumerate}[i]
\item
  \[
    u = \begin{pmatrix}
      1  \\
      -1 \\
    \end{pmatrix}_,
    v = \begin{pmatrix}
      -1 \\
      6  \\
    \end{pmatrix}_,
    w = \begin{pmatrix}
      -1 \\
      1  \\
    \end{pmatrix}
  \]
  
  $w$ ist von $u$ linear abhängig, $u = -1 \cdot w$. Weiterhin sind $u$ und $v$ linear unabhängig, da
  \begin{align*}
    1  &= t \cdot -1 \\
    -1 &= t \cdot 6  \\
  \end{align*}
  keine Lösung hat. Somit bilden $u$ und $v$ eine Basis von $span(u, v, w)$ und die Dimension ist also 2.
  
\item
  \[
    t = \begin{pmatrix}
      1 \\
      1 \\
      1 \\
      1 \\
    \end{pmatrix}_,
    u =\begin{pmatrix}
      1 \\
      1 \\
      1 \\
      3 \\
    \end{pmatrix}_,
    v = \begin{pmatrix}
      3  \\
      -5 \\
      7  \\
      2  \\
    \end{pmatrix}_,
    w = \begin{pmatrix}
      1  \\
      -7 \\
      5  \\
      -2 \\
    \end{pmatrix}
  \]  

  \begin{align*}
    \left(
    \begin{array}{cccc|c}
      1 & 1 & 3  & 1  & 0 \\
      1 & 1 & -5 & -7 & 0 \\
      1 & 1 & 7  & 5  & 0 \\
      1 & 3 & 2  & -2 & 0 \\
    \end{array}
    \right)
    &=
    \left(
    \begin{array}{cccc|c}
      1 & 1 & 3  & 1  & 0 \\
      0 & 2 & -5 & -7 & 0 \\
      0 & 0 & 1  & 1  & 0 \\
      0 & 0 & 1  & 1  & 0 \\
    \end{array}
    \right) \\    
    &=
    \left(
    \begin{array}{cccc|c}
      1 & 0 & 0 & -1 & 0 \\
      0 & 1 & 0 & -1 & 0 \\
      0 & 0 & 1 & 1  & 0 \\
      0 & 0 & 0 & 0  & 0 \\
    \end{array}
    \right) \\
  \end{align*}
  Somit gibt es nicht triviale Lösungen, zum Beispiel $x_1 = 1, x_2 = 1, x_3 = -1, x_4 = 1$

  \[
    t + u - v + w = 0
  \]
  \[
    \begin{pmatrix}
      1 \\
      1 \\
      1 \\
      1 \\
    \end{pmatrix}
    +
    \begin{pmatrix}
      1 \\
      1 \\
      1 \\
      3 \\
    \end{pmatrix}
    +
    \begin{pmatrix}
      -3 \\
      5  \\
      -7 \\
      -2 \\
    \end{pmatrix}
    + \begin{pmatrix}
      1  \\
      -7 \\
      5  \\
      -2 \\
    \end{pmatrix}
    =
    \begin{pmatrix}
      0 \\
      0 \\
      0 \\
      0 \\
    \end{pmatrix}
  \]

  Somit ist $v = t + u + w$. Damit ist $v$ von $t, u$ und $w$ linear abhängig.
  
  \begin{align*}
    \left(
    \begin{array}{ccc|c}
      1 & 1 & 1  & 0 \\
      1 & 1 & -7 & 0 \\
      1 & 1 & 5  & 0 \\
      1 & 3 & -2 & 0 \\
    \end{array}
    \right)
    &=
    \left(
    \begin{array}{ccc|c}
      1 & 1 & 1  & 0 \\
      0 & 2 & -7 & 0 \\
      0 & 0 & -8 & 0 \\
      0 & 0 & 12 & 0 \\
    \end{array}
    \right) \\
  \end{align*}
  
  Somit sind $t, u, w$ linear unabhängig,
  damit bilden $t, u$ und $w$ eine Basis von $span(t,u,v,w)$ und die Dimension ist 3.
\end{enumerate}

\section*{Übung 20}

Sei $V$ ein $\mathbb{K}$-Vektorraum und $S = \left(v_1, \ldots, v_n  \right)$ ein $n$-Tupel von Vektoren aus $V$.
Wir definieren die \emph{Aufspannabbildung} $\varphi_S$ zu diesem Tupel

\[
  \varphi_S \colon \mathbb{K}^n \to V,
\]

\[
  \begin{pmatrix} \lambda_1 \\ \vdots \\ \lambda_n \end{pmatrix} \mapsto \lambda_1 v_1 + \ldots + \lambda_n v_n
\]

Beweisen Sie folgende Eigenschaften der Aufspannabbildung:
\begin{enumerate}[(i)]
\item $\varphi_S$ ist linear \\
  Seit $\rho \ne \mu \in \mathbb{K}^n$ mit $\rho = \begin{pmatrix}\rho_1 \\ \vdots \\ \rho_n \end{pmatrix}$
  und $\mu = \begin{pmatrix}\mu_1 \\ \vdots \\ \mu_n \end{pmatrix}$.
  Weiterhin sei $\lambda \in \mathbb{K}$ beliebig.
  Die Abbildung $\varphi_S$ ist linear, wenn gilt:
  \begin{enumerate}[(i)]
  \item $\varphi_S(\rho) + \varphi_S(\mu) = \varphi(\rho + \mu)$
    \begin{align*}
      \varphi_S(\rho) + \varphi_S(\mu) &= \rho_1 v_1 + \ldots + \rho_n v_n + \mu_1 v_1 + \ldots + \mu_n v_n \\
      \overset{(V2)}&= \left(\rho_1 + \mu_1\right) v_1 + \ldots + (\rho_n + \mu_n) v_n \\
                                       &= \varphi(\rho + \mu)
    \end{align*}
  \item $\varphi_S(\lambda \cdot \rho) = \lambda \cdot \varphi_S(\rho)$
    \begin{align*}
      \varphi_S(\lambda \cdot \rho) &= \lambda \rho_1 v_1 + \ldots + \lambda \rho_n v_n \\
      \overset{(V2)}&= \lambda \cdot (\rho_1 v_1 + \ldots + \rho_n v_n) \\
                                    &= \lambda \cdot \varphi_S (\rho)
    \end{align*}
  \end{enumerate}
  
\item $\varphi_S$ ist genau dann injektiv, wenn $S$ linear unabhängig ist
  \begin{itemize}
  \item[$\Rightarrow$] $\varphi_S$ ist injektiv, falls für alle $x \ne y \in \mathbb{K}^n$ gilt:
    $\varphi_S(x) \ne \varphi_S(y)$.
    Sei nun $w = \varphi_S(y)$. Angenommen $S$ wäre nun linear Abhängig, dann existiert ein \\
    $x \ne \begin{pmatrix} 0 \\ \vdots \\ 0 \end{pmatrix}$ mit
    $\varphi_S(x) = x_1 v_1 + \cdots + x_n v_n = 0$. \\
    Somit wäre $\varphi_S(x + y) = w$, ein Widerspruch zu $\varphi_s$ ist injektiv. Damit ist
    $S$ linear unabhängig.

  \item[$\Leftarrow$] Sei $S$ linear unabhängig und $w$ als Linearkombination der Vektoren aus $S$ darstellbar,
    dann ist diese Darstellung eindeutig (Lemma 4.18 der Vorlesung). Somit existieren keine
    $x \ne y \in \mathbb{K}^n$, so dass $\varphi_S(x) = \varphi_S(y)$ und damit ist $\varphi_S$
    injektiv.
  \end{itemize}
\item $\varphi_S$ ist genau dann surjektiv, wenn $\langle S \rangle = V$
  \begin{itemize}
  \item[$\Rightarrow$] die Abbildung ist surjektiv, wenn für jedes $v \in V$ ein $k \in \mathbb{K}^n$
    existiert, so dass $\varphi_s(k) = v$. Dass heißt, man kann mit Linearkombinationen der Vektoren aus
    $S$ jeden Vektor aus $V$ darstellen. Somit ist $\langle S \rangle = V$
  \item[$\Leftarrow$] $\langle S \rangle = V$ wenn sich jeder Vektor aus $V$ durch eine Linearkombination
    der Vektoren aus $S$ darstellen lässt. Somit findet man für jedes $v \in V$ ein $\lambda_1, \ldots, \lambda_n$,
    so dass $\lambda_1 v_1 + \ldots + \lambda_n v_n = v$. Damit existiert für jedes $v \in V$ ein
    $\varphi_S\begin{pmatrix} \lambda_1 \\ \vdots \\ \lambda_n \end{pmatrix} = v$. Damit ist $\varphi_S$ surjektiv.
    
  \end{itemize}
  
\item Folgern Sie, dass $\varphi_S$ genau dann bijektiv ist, wenn $S$ eine Basis von $V$ ist.
  \begin{itemize}
  \item[$\Rightarrow$]
    $\varphi_S$ ist bijektiv genau dann, wenn $\varphi_S$ injektiv und surjektiv ist. Aus (ii) folgt dann,
    dass $S$ linear unabhängig ist und aus (iii) folgt, dass $S$ den Vektorraum $V$ aufspannt.
    Per Definition ist $S$ somit eine Basis von $V$
  \item[$\Leftarrow$]
    $S$ ist eine Basis von $V$, wenn $S$ linear unabhängig ist und $span S = V$.
    Aus der linearen Unabhängigkeit von $S$ folgt durch (ii) die Injektivität von $\varphi_S$.
    Weiterhin folgt aus ``$S$ spannt $V$ auf'' durch (iii) die Surjektivität von  $\varphi_S$.
    Somit ist $\varphi_S$ bijektiv.
  \end{itemize}
\item Folgern Sie, dass die Spalten der Fundamentalmatrix eines LGS eine Basis im Lösungsraum des LGS bilden. \\
  
\end{enumerate}

\newpage
\section*{Übung 21}

Finden Sie Dimensionen und Basen von Untervektorräumen von $\mathbb{Q}^n$, die durch folgende LGS definiert sind:
\begin{enumerate}[1)]
\item
  \[
    U = \left\{
      x \in \mathbb{Q}^2
      \middle|
    \begin{pmatrix}
      1  & -3 \\
      -2 & 6  \\
    \end{pmatrix}
    x = 0
    \right\}
  \]


  \begin{align*}
    \left(
    \begin{array}{cc}
      1  & -3 \\
      -2 & 6  \\      
    \end{array}
    \right)
    =
    \left(
    \begin{array}{cc}
      1 & \tikzmark{l1}-3 \\
      0 & 0\tikzmark{r1}  \\      
    \end{array}
    \right)    
  \end{align*}
  \tikz[overlay, remember picture]{
    \draw[red,thick] ($(l1)+(-0.2em,1em)$) rectangle ($(r1)+(.8em,-.4em)$);
  }
  \begin{align*}
    U &= \left\{
      \begin{pmatrix}
        {\color{red} B} \\
        1_1 \\
      \end{pmatrix}
      \cdot t
      \middle|
      t \in \mathbb{Q}^1
    \right\} \\
    &= \left\{
      \begin{pmatrix}
        3 \\
        1 \\
      \end{pmatrix}
      \cdot t
      \middle|
      t \in \mathbb{Q}^1
    \right\} \\
  \end{align*}

  $\left\{\begin{pmatrix} -3 \\ 1 \end{pmatrix} \right\}$ ist Basis von $U$, $\dim U = 1$ 
  
\item
  \[
    U = \left\{
      x \in \mathbb{Q}^3
      \middle|
      \begin{pmatrix}
        -3  & 1  & -2  \\
        6   & -2 & 4   \\
        -15 & 5  & -10 \\
      \end{pmatrix}
      x = 0
    \right\}
  \]

  \begin{align*}
    \begin{pmatrix}
      -3  & 1  & -2  \\
      6   & -2 & 4   \\
      -15 & 5  & -10 \\
    \end{pmatrix}
    &=
    \begin{pmatrix}
      -3  & 1  & -2  \\
      3   & -1 & 2   \\
      -3 & 1  & -2 \\
    \end{pmatrix}\\
    &=
    \begin{pmatrix}
      1 & \tikzmark{l2} -\frac{1}{3} & \frac{2}{3} \\
      0 & 0                          & 0  \\
      0 & 0                          & 0 \tikzmark{r2}  \\
    \end{pmatrix}
  \end{align*}
  \tikz[overlay, remember picture]{
    \draw[red,thick] ($(l2)+(-0.2em,1em)$) rectangle ($(r2)+(.2em,-.4em)$);
  }
    \begin{align*}
    U &= \left\{
      \begin{pmatrix}
        {\color{red} -B} \\
        1_1 \\
      \end{pmatrix}
      \cdot t
      \middle|
      t \in \mathbb{Q}^1
    \right\} \\
    &= \left\{
      \begin{pmatrix}
        \frac{1}{3} & -\frac{2}{3} \\
        1           & 0            \\
        0           & 1
      \end{pmatrix}
      \cdot t
      \middle|
      t \in \mathbb{Q}
    \right\} \\
  \end{align*}
  $\left\{      \begin{pmatrix}
        \frac{1}{3} & -\frac{2}{3} \\
        1           & 0            \\
        0           & 1
      \end{pmatrix}
   \right\}$ ist Basis von $U$, $\dim U = 2$
    

\end{enumerate}

\newpage
\section*{Übung 22}

Sei $V$ ein $\mathbb{K}$-Vektorraum. Eine Teilmenge $S \subseteq V$ heißt Erzeugendensystem von $V$, wenn
$span S = V$. Ein Erzeugendensystem $S$ heißt \emph{minimal}, wenn
keine echte Teilmenge $S \subsetneq S$ den Vektorraum $V$ aufspannt.
Beweisen Sie: eine Teilmenge $B$ ist genau dann eine Basis von $V$, wenn $B$ ein minimales
Erzeugendensystem von $V$ ist.

\begin{itemize}
\item[$\Rightarrow$]
  Laut Definition ist $B$ eine Basis von $V$, wenn
  \begin{enumerate}[1)]
  \item $B$ linear unabhängig ist
  \item $span B = V$
  \end{enumerate}
  Sei nun $b \in B$ ein Vektor, da $B \subseteq V$ gilt $b \in V$.
  Die lineare Abhängigkeit von $B$ impliziert, dass man $b$ nicht als Linearkombinationen der Vektoren aus
  $B \setminus \{ b \}$ darstellen kann, somit ist $b$ nicht in $span\left(B \setminus \{ b \}\right)$
  enthalten und keine echte Teilmenge von $B$ kann $V$ aufspannen.
  Somit ist $B$ eine minimales Erzeugendensystem von $V$.
\item[$\Leftarrow$]
  Wenn $B$ minimales Erzeugendensystem von $V$ ist, dann ist nach nach Definition des Erzeugendensystemes $span B = V$.
  Angenommen $B$ ist nun linear abhängig. Dann gibt es einen Vektor $b \in B$, welcher als Linearkombination von einem
  oder mehreren anderen Vektoren aus $B$ darstellbar ist. Dann ist $B' = B \setminus \{ b \}$ eine echte Teilmenge von
  $B$ und $span B' = V$, weil man $b$ durch eine Linearkombination der Vektoren aus $B'$ darstellen kann.

  Dass nun $B'$ als echte Teilmenge von $B$ ebenfalls $v$ aufspannen kann, ist ein Widerspruch dazu, dass
  $B$ minimales Erzeugendensystem ist.

  Also ist $B$ linear abhängig und $B$ kann per Definition des Erzeugendensystemes $V$ aufspannen, somit ist $B$
  Basis von $V$.
\end{itemize}

\end{document}