\documentclass{article}
\usepackage{aligned-overset}
\usepackage{amsmath}
\usepackage{amssymb}
\usepackage{bm}
\usepackage[shortlabels]{enumitem}
\usepackage{hyperref}
\usepackage[utf8]{inputenc}
\usepackage{mathtools}
\usepackage{titling}
\usepackage{fancyhdr}

\author{Karsten Lehmann}
\date{WiSe 2020}
\title{Lineare Algebra I: Übungen - Blatt 4}

\pagestyle{fancy}
\fancyhf{}
\lhead{\thetitle}
\rhead{\theauthor}
\lfoot{\thedate}
\rfoot{Seite \thepage}

\begin{document}
\section*{Übung 15}

Sei $R$ ein kommutativer Ring und $A \in \mathbb{M}(m \times n, R)$ eine Matrix mit Einträgen
$a_{ij}, i  = 1, \ldots, m, j = 1, \ldots, n$.
Die Transponierte $A^T$ der Matrix $A$ ist definiert als die $(n \times m)$-Matrix, deren $(i, j)$-te
Eintrag $a_{ij}^T$ gleich dem $(j,i)$-ten Eitnrag von $A$ ist: $a_{ij}^T, i = 1, \ldots, n, j = 1, \ldots, m$.

\begin{enumerate}[(i)]
\item
  Seien $a = (a_1, \ldots, a_n), b = (b_1, \ldots, b_n) \in R^{1\times n}$ zwei Zeilen gleicher Größe. Bestimmen Sie die Matrizen
  $a \cdot b^T$ und $a^T \cdot b$.

  \[
    b^T = \begin{pmatrix}b_1 \\ \vdots \\ b_n\end{pmatrix}
  \]
  \[
    a \cdot b^T = (a_1 \cdot b_1 + \ldots + a_n \cdot b_n) = \left( \sum\limits_{k=1}^n a_kb_k \right)
  \]
  \[
    a^T \cdot b =
    \begin{pmatrix}
      a_1 \cdot b_1 & \ldots & a_1 \cdot b_n \\
      \vdots        & \ddots & \vdots \\
      a_n \cdot b_1 & \ldots & a_n \cdot b_n \\
    \end{pmatrix}
  \]

\item
  Beweisen Sie, dass für $A \in \mathbb{M}(m \times n, R), B \in \mathbb{M}(n \times p,R)$ gilt: $(A \cdot B)^T = B^T \cdot A^T$
  
  Sei die Matrix $C = A \cdot B$. Nach der Definition der Multiplikation von Matrizen ist $C$  eine $(m \times p)$ Matrix
  mit den Einträgen

  $c_{ij} = \sum\limits_{k=1}^n a_{ik}b_{kj}, i = 1, \ldots, m, j = 1, \ldots, p$, oder

  \[
    C = \begin{pmatrix}
      \sum\limits_{k=1}^n a_{1k}b_{k1} & \ldots & \sum\limits_{k=1}^n a_{1k}b_{kp} \\
      \vdots                         & \ddots & \vdots \\
      \sum\limits_{k=1}^n a_{mk}b_{k1} & \ldots & \sum\limits_{k=1}^n a_{mk}b_{kp} \\
    \end{pmatrix}
  \]

  Die Transponierte $C^T$ der Matrix $C$ ist nun
  
  \[
    C^T = \begin{pmatrix}
      \sum\limits_{k=1}^n a_{1k}b_{k1} & \ldots & \sum\limits_{k=1}^n a_{mk}b_{k1} \\
      \vdots                         & \ddots & \vdots \\
      \sum\limits_{k=1}^n a_{1k}b_{kp} & \ldots & \sum\limits_{k=1}^n a_{mk}b_{kp} \\
    \end{pmatrix}
  \]

  mit den Einträgen  
  
  $c_{ij}^T = \sum\limits_{k=1}^n a_{ik}b_{kj}, i = 1, \ldots, m, j = 1, \ldots, p$

  Sei nun die Matrix $D = B^T \cdot A^T$. Die Matrix $B^T$ ist eine $(p \times n)$-Matrix und
  $A^T$ eine $(n \times m)$-Matrix. Nach der Definition der Multiplikation von Matrizen ist
  $D$ somit eine $(p \times m)$-Matrix.

  Seien $r_{ij}, i = 1, \ldots, n, j = 1, \ldots m$ und $t_{ij}, i = 1, \ldots, p, j = 1, \ldots, n$
  die Einträge von $A^T$ und $B^T$, dann ist

  \[
    D = \begin{pmatrix}
      \sum\limits_{k=1}^n t_{1k}r_{k1} & \ldots & \sum\limits_{k=1}^n t_{1k}r_{km} \\
      \vdots                         & \ddots & \vdots \\
      \sum\limits_{k=1}^n t_{pk}r_{k1} & \ldots & \sum\limits_{k=1}^n t_{pk}r_{km} \\
    \end{pmatrix}
  \]

  Per Definition der transponierten Matrix ist nun $t_{ij} = b_{ji}$ und $r_{ij} = a_{ji}$, somit

  \[
    D = \begin{pmatrix}
      \sum\limits_{k=1}^n b_{k1}a_{1k} & \ldots & \sum\limits_{k=1}^n b_{k1}a_{mk} \\
      \vdots                         & \ddots & \vdots \\
      \sum\limits_{k=1}^n b_{kp}a_{1k} & \ldots & \sum\limits_{k=1}^n b_{kp}a_{mk} \\
    \end{pmatrix}
  \]

  mit der expliziten Form

  \[
    d_{ij} = \sum\limits_{k=1}^n a_{ik}b_{kj}, i = 1, \ldots, m, j = 1, \ldots, p
  \]

  Somit ist $d_{ij} = c_{ij}^T$ und $(A \cdot B)^T = B^T \cdot A^T$.
\end{enumerate}

\newpage 
\section*{Übung 16}

Bestimmen Sie, ob die folgenden Vektoren in dem $\mathbb{K}$-Vektorraum $V$ linear unabhängig sind.

\begin{enumerate}[(i)]
\item
  \[
    \begin{pmatrix}
      4 \\
      3 \\
      1 \\
    \end{pmatrix},
    \begin{pmatrix}
      1 \\
      2 \\
      3 \\
    \end{pmatrix},
    \begin{pmatrix}
      2  \\
      -1 \\
      -5 \\
    \end{pmatrix},
    V = \mathbb{R}^3, \mathbb{K} = \mathbb{R}
  \]
  \begin{align*}
    \left(
    \begin{array}{ccc|c}
      4 & 1 & 2  & 0 \\
      3 & 2 & -1 & 0 \\
      1 & 3 & -5 & 0 \\
    \end{array}\
    \right)
    &=
    \left(
    \begin{array}{ccc|c}
      4 & 1 & 2    & 0 \\
      0 & -7 & 14  & 0 \\
      0 & 11 & -22 & 0 \\
    \end{array}\
    \right) \\
    &=
    \left(
    \begin{array}{ccc|c}
      4 & 1 & 2  & 0 \\
      0 & 1 & -2 & 0 \\
      0 & 0 & 0  & 0 \\
    \end{array}\
    \right) \\
  \end{align*}
  Es gibt unendliche viele Lösungen für dieses Gleichungssystem, somit sind diese 3 Vektoren linear abhängig.

\newpage
\item
  \[
    \begin{pmatrix}
      1 & -1 \\
      1 & -1 \\
    \end{pmatrix},
    \begin{pmatrix}
      2 & 5 \\
      1 & 3 \\
    \end{pmatrix},
    \begin{pmatrix}
      1 & 1 \\
      0 & 1 \\
    \end{pmatrix},
    \begin{pmatrix}
      3 & 4 \\
      5 & 7 \\
    \end{pmatrix},
    V = \mathbb{M}(2 \times 2, \mathbb{Q}), \mathbb{K} = \mathbb{Q}
  \]
  \begin{align*}
    \left(
    \begin{array}{cccc|c}
      1  & 2 & 1 & 3 & 0 \\
      1  & 1 & 0 & 5 & 0 \\
      -1 & 5 & 1 & 4 & 0 \\
      -1 & 3 & 1 & 7 & 0 \\
    \end{array}
    \right)
    &=
    \left(
    \begin{array}{cccc|c}
      1  & 2  & 1  & 3  & 0 \\
      0  & -1 & -1 & 2  & 0 \\
      0  & 7  & 2  & 7  & 0 \\
      0  & 5  & 2  & 10 & 0 \\
    \end{array}
    \right) \\
    &=
    \left(
    \begin{array}{cccc|c}
      1  & 2 &  1 & 3  & 0 \\
      0  & 1 &  1 & -2 & 0 \\
      0  & 0 & -5 & 21 & 0 \\
      0  & 0 & -3 & 20 & 0 \\
    \end{array}
    \right) \\
    &=
    \left(
    \begin{array}{cccc|c}
      1  & 2 & 1 & 3             & 0 \\
      0  & 1 & 1 & -2            & 0 \\
      0  & 0 & 1 & -\frac{21}{5} & 0 \\
      0  & 0 & 1 & -\frac{20}{3} & 0 \\
    \end{array}
    \right) \\
    &=
    \left(
    \begin{array}{cccc|c}
      1  & 2 & 1 & 3                              & 0 \\
      0  & 1 & 1 & -2                             & 0 \\
      0  & 0 & 1 & -\frac{21}{5}                  & 0 \\
      0  & 0 & 0 & \frac{63}{15} - \frac{100}{15} & 0 \\
    \end{array}
    \right) \\
    &=
    \left(
    \begin{array}{cccc|c}
      1  & 2 & 1 & 3             & 0 \\
      0  & 1 & 1 & -2            & 0 \\
      0  & 0 & 1 & -\frac{21}{5} & 0 \\
      0  & 0 & 0 & 1             & 0 \\
    \end{array}
    \right) \\
    &=
    \left(
    \begin{array}{cccc|c}
      1  & 0 & 0 & 0 & 0 \\
      0  & 1 & 0 & 0 & 0 \\
      0  & 0 & 1 & 0 & 0 \\
      0  & 0 & 0 & 1 & 0 \\
    \end{array}
    \right) \\
  \end{align*}
  Damit ist die einzige Lösung trivial $\lambda_1 = 0 = \lambda_2 = \lambda_3 = \lambda_4$ und die Vektoren
  sind linear unabhängig.

\newpage
\item
  \[
    1 + 2i, 1 - i, 5 - 3i, V = \mathbb{C}, \mathbb{K} = \mathbb{R}
  \]

  \[
    \lambda_1 \cdot (1 + 2i) + \lambda_2 \cdot (1 - i) + \lambda_3 \cdot (5 - 3i) = 0
  \]

  \begin{align*}
    \lambda_1 + \lambda_2 + 5 \cdot \lambda_3 &= 0 \\
    2 \cdot \lambda_1 - \lambda_2 - 3 \cdot \lambda_3 &= 0 \\
  \end{align*}

  \begin{align*}
    \left(
    \begin{array}{ccc|c}
      1 & 1  & 5  & 0 \\
      2 & -1 & -3 & 0 \\
    \end{array}
    \right)
    &=
    \left(
    \begin{array}{ccc|c}
      1 & 1  & 5   & 0 \\
      0 & -3 & -13 & 0 \\
    \end{array}
    \right) \\
    &=
    \left(
    \begin{array}{ccc|c}
      1 & 1 & 5            & 0 \\
      0 & 1 & \frac{13}{3} & 0 \\
    \end{array}
    \right) \\
    &=
    \left(
    \begin{array}{ccc|c}
      1 & 0 & \frac{2}{3}  & 0 \\
      0 & 1 & \frac{13}{3} & 0 \\
    \end{array}
    \right) \\
  \end{align*}

  \[
    L = \left\{\lambda \cdot \begin{pmatrix}-\frac{2}{3} \\ -\frac{13}{3} \\ 1 \end{pmatrix} \middle| \lambda \in \mathbb{R} \right\}
  \]

  Damit existiert eine nicht-triviale Lösung und die Vektoren sind linear abhängig.

\item
  \[
    1 + 2t + t^3, 1 + t + t^2, t - t^2 + t^3, V = \mathbb{Q}[t], \mathbb{K} = \mathbb{Q}
  \]

  \begin{align*}
    \left(
    \begin{array}{ccc|c}
      1 & 1 & 0  & 0 \\
      2 & 1 & 1  & 0 \\
      0 & 1 & -1 & 0 \\
      1 & 0 & 1  & 0 \\
    \end{array}
    \right)
    &=
    \left(
    \begin{array}{ccc|c}
      1 & 1  & 0  & 0 \\
      0 & -1 & 1  & 0 \\
      0 & 1  & -1 & 0 \\
      0 & -1 & 1  & 0 \\
    \end{array}
    \right) \\
    &=
    \left(
    \begin{array}{ccc|c}
      1 & 1 & 0 & 0 \\
      0 & 0 & 0 & 0 \\
      0 & 0 & 0 & 0 \\
      0 & 0 & 0 & 0 \\
    \end{array}
    \right) \\
  \end{align*}
  Somit existieren unendliche viele Lösungen und die Vektoren sind linear abhängig.
\end{enumerate}

Kann man den Vektor $v \in V$ als Linearkombination der angegebenen Vektoren darstellen? Wenn ja:
sind diese Darstellungen als Linearkombinationen eindeutig?

\begin{enumerate}[(i)]
\item $v = (1,1,1)^T$
  \begin{align*}
    \left(
    \begin{array}{ccc|c}
      4 & 1 & 2  & 1 \\
      3 & 2 & -1 & 1 \\
      1 & 3 & -5 & 1 \\
    \end{array}\
    \right)
    &=
    \left(
    \begin{array}{ccc|c}
      4 & 1 & 2   & 1 \\
      0 & 5 & -10 & 1 \\
      0 & 7 & -14 & 1 \\
    \end{array}\
    \right) \\
    &=
    \left(
    \begin{array}{ccc|c}
      4 & 1 & 2   & 1 \\
      0 & 35 & -70 & 7 \\
      0 & 35 & -70 & 5 \\
    \end{array}\
    \right) \\
    &=
    \left(
    \begin{array}{ccc|c}
      4 & 1  & 2   & 1 \\
      0 & 35 & -70 & 7 \\
      0 & 0  & 0   & -2 \\
    \end{array}\
    \right) \\
  \end{align*}
  Somit gibt es keine Lösung und $(1,1,1)^T$ ist nicht als Linearkombination der angegebenen Vektoren darstellbar.
\item $v = \bm{1}_2$
  \begin{align*}
    \left(
    \begin{array}{cccc|c}
      1  & 2 & 1 & 3 & 1 \\
      1  & 1 & 0 & 5 & 1 \\
      -1 & 5 & 1 & 4 & 1 \\
      -1 & 3 & 1 & 7 & 1 \\
    \end{array}
    \right)
    &=
    \left(
    \begin{array}{cccc|c}
      1  & 2 & 1 & 3  & 1 \\
      0  & 1 & 1 & -2 & 0 \\
      0  & 6 & 1 & 9  & 2 \\
      0  & 4 & 1 & 12 & 2 \\
    \end{array}
    \right) \\
    &=
    \left(
    \begin{array}{cccc|c}
      1  & 2  & 1 & 3  & 1 \\
      0  & 1  & 1 & -2 & 0 \\
      0  & 12 & 2 & 18 & 4 \\
      0  & 12 & 3 & 36 & 6 \\
    \end{array}
    \right) \\
    &=
    \left(
    \begin{array}{cccc|c}
      1  & 2 & 1   & 3  & 1 \\
      0  & 1 & 1   & -2 & 0 \\
      0  & 0 & -10 & 42 & 4 \\
      0  & 0 & 1   & 18 & 2 \\
    \end{array}
    \right) \\
    &=
    \left(
    \begin{array}{cccc|c}
      1  & 2 & 1  & 3  & 1 \\
      0  & 1 & 1  & -2 & 0 \\
      0  & 0 & -5 & 21 & 2 \\
      0  & 0 & 0  & 1  & \frac{4}{37} \\
    \end{array}
    \right) \\
    &=
    \left(
    \begin{array}{cccc|c}
      1  & 2 & 1 & 3  & 1 \\
      0  & 1 & 1 & -2 & 0 \\
      0  & 0 & 1 & 0  & \frac{2}{37} \\
      0  & 0 & 0 & 1  & \frac{4}{37} \\
    \end{array}
    \right) \\
    &=
    \left(
    \begin{array}{cccc|c}
      1  & 1 & 0 & 5 & 1 \\
      0  & 1 & 0 & 0 & \frac{6}{37} \\
      0  & 0 & 1 & 0 & \frac{2}{37} \\
      0  & 0 & 0 & 1 & \frac{4}{37} \\
    \end{array}
    \right) \\
    &=
    \left(
    \begin{array}{cccc|c}
      1  & 0 & 0 & 0 & \frac{11}{37} \\
      0  & 1 & 0 & 0 & \frac{6}{37} \\
      0  & 0 & 1 & 0 & \frac{2}{37} \\
      0  & 0 & 0 & 1 & \frac{4}{37} \\
    \end{array}
    \right) \\
  \end{align*}
  \[
    L = \left\{\begin{pmatrix}\frac{11}{37} \\ \frac{6}{37} \\ \frac{2}{37} \\ \frac{4}{37} \end{pmatrix}\right\}
  \]

  Es existiert genau eine Lösung. Damit ist $1_2$ als Linearkombination der Vektoren darstellbar und diese Darstellung ist eindeutig.
  
\item $v = i$
  \begin{align*}
    \left(
    \begin{array}{ccc|c}
      1 & 1  & 5  & 0 \\
      2 & -1 & -3 & 1 \\
    \end{array}
    \right)
    &=
    \left(
    \begin{array}{ccc|c}
      1 & 1  & 5   & 0 \\
      0 & -3 & -13 & 1 \\
    \end{array}
    \right) \\
    &=
    \left(
    \begin{array}{ccc|c}
      1 & 0 & \frac{2}{3}  & -\frac{1}{3} \\
      0 & 1 & \frac{13}{3} & \frac{1}{3} \\
    \end{array}
    \right) \\
  \end{align*}

  \[
    L = \left\{
      \begin{pmatrix}\
        -\frac{1}{3} \\
        \frac{1}{3} \\
        0
      \end{pmatrix}
      + \lambda \cdot
      \begin{pmatrix}
        -\frac{2}{3} \\
        -\frac{13}{3} \\
        1
      \end{pmatrix}
      \middle| \lambda \in \mathbb{R} \right\}
  \]

  Somit ist $i$ als Linearkombination aus den gegebenen Vektoren darstellbar. Diese Darstellung ist allerdings nicht eindeutig.
       
\item $v = 2 + 3t - 3t^2 + t^3$
  \begin{align*}
    \left(
    \begin{array}{ccc|c}
      1 & 1 & 0  & 2  \\
      2 & 1 & 1  & 3  \\
      0 & 1 & -1 & -3 \\
      1 & 0 & 1  & 1  \\
    \end{array}
    \right)
    &=
    \left(
    \begin{array}{ccc|c}
      1 & 1  & 0  & 2  \\
      0 & -1 & 1  & -1  \\
      0 & 1  & -1 & -3 \\
      0 & -1 & 1  & -1  \\
    \end{array}
    \right) \\
    &=
    \left(
    \begin{array}{ccc|c}
      1 & 1  & 0 & 2  \\
      0 & -1 & 1 & -1 \\
      0 & 0  & 0 & -4 \\
      0 & 0  & 0 & -4 \\
    \end{array}
    \right) \\
  \end{align*}
  Somit existiert keine Lösung und  $v = 2 + 3t - 3t^2 + t^3$ ist keine Lienarkombination der gegebenen Vektoren.
\end{enumerate}

\section*{Übung 17}

Sei $V_1 \coloneqq \mathbb{R}$ betrachtet als $\mathbb{R}$-Vektorraum und 
$V_2 \coloneqq \mathbb{R}$ betrachtet als $\mathbb{Q}$-Vektorraum.
Bestimmen Sie, ob die Elemente $v_1 = 1$ und $v2 = \sqrt{2}$ linear unabhängig in
$V_1$ bzw. $V_2$ sind.


Zwei Vektoren sind genau dann linear abhängig, wenn einer ein Vielfaches von dem anderen ist (Beispiel 4.14 aus der Vorlesung).

\begin{align*}
  \lambda \cdot v_1 &= v_2 \\
  \lambda \cdot 1 &= \sqrt{2} \\
  \lambda &= \sqrt{2} \\
\end{align*}

Da $\lambda \in \mathbb{R}$, sind $v_1, v_2$ in $V_1$ linear abhängig. Allerdings ist $\sqrt{2}$ irrational. Somit existiert
kein $\lambda \in \mathbb{Q}$, für welches gilt $\lambda \cdot v_1 = v_2$ und $v_1, v_2$ sind in $V_2$ linear unabhängig.

\end{document}